%THE HISTORY OF ENGLAND FROM THE ACCESSION OF JAMES II,
%
%VOLUME 1 (of 5)
%
%by Thomas Babington Macaulay.
%
%Philadelphia
%
%Porter \& Coates
%
%
%
%
%
%VOL. I.
%
%
%
%
%CONTENTS:
%
%CHAPTER I.
%
%     Introduction
%     Britain under the Romans
%     Britain under the Saxons
%     Conversion of the Saxons to Christianity
%     Danish Invasions; The Normans
%     The Norman Conquest
%     Separation of England and Normandy
%     Amalgamation of Races
%     English Conquests on the Continent
%     Wars of the Roses
%     Extinction of Villenage
%     Beneficial Operation of the Roman Catholic Religion
%     The early English Polity often misrepresented, and why?
%     Nature of the Limited Monarchies of the Middle Ages
%     Prerogatives of the early English Kings
%     Limitations of the Prerogative
%     Resistance an ordinary Check on Tyranny in the Middle Ages
%     Peculiar Character of the English Aristocracy
%     Government of the Tudors
%     Limited Monarchies of the Middle Ages generally turned into Absolute Monarchies
%     The English Monarchy a singular Exception
%     The Reformation and its Effects
%     Origin of the Church of England
%     Her peculiar Character
%     Relation in which she stood to the Crown
%     The Puritans
%     Their Republican Spirit
%     No systematic parliamentary Opposition offered to the Government of Elizabeth
%     Question of the Monopolies
%     Scotland and Ireland become Parts of the same Empire with England
%     Diminution of the Importance of England after the Accession of James I
%     Doctrine of Divine Right
%     The Separation between the Church and the Puritans becomes wider
%     Accession and Character of Charles I
%     Tactics of the Opposition in the House of Commons
%     Petition of Right
%     Petition of Right violated; Character and Designs of Wentworth
%     Character of Laud
%     Star Chamber and High Commission
%     Ship-Money
%     Resistance to the Liturgy in Scotland
%     A Parliament called and dissolved
%     The Long Parliament
%     First Appearance of the Two great English Parties
%     The Remonstrance
%     Impeachment of the Five Members
%     Departure of Charles from London
%     Commencement of the Civil War
%     Successes of the Royalists
%     Rise of the Independents
%     Oliver Cromwell
%     Selfdenying Ordinance; Victory of the Parliament
%     Domination and Character of the Army
%     Rising against the Military Government suppressed
%     Proceedings against the King
%     His Execution
%     Subjugation of Ireland and Scotland
%     Expulsion of the Long Parliament
%     The Protectorate of Oliver Cromwell
%     Oliver succeeded by Richard
%     Fall of Richard and Revival of the Long Parliament
%     Second Expulsion of the Long Parliament
%     The Army of Scotland marches into England
%     Monk declares for a Free Parliament
%     General Election of 1660
%     The Restoration
%
%CHAPTER II.
%
%     Conduct of those who restored the House of Stuart unjustly censured
%     Abolition of Tenures by Knight Service; Disbandment of the Army
%     Disputes between the Roundheads and Cavaliers renewed
%     Religious Dissension
%     Unpopularity of the Puritans
%     Character of Charles II
%     Character of the Duke of York and Earl of Clarendon
%     General Election of 1661
%     Violence of the Cavaliers in the new Parliament
%     Persecution of the Puritans
%     Zeal of the Church for Hereditary Monarchy
%     Change in the Morals of the Community
%     Profligacy of Politicians
%     State of Scotland
%     State of Ireland
%     The Government become unpopular in England
%     War with the Dutch
%     Opposition in the House of Commons
%     Fall of Clarendon
%     State of European Politics, and Ascendancy of France
%     Character of Lewis XIV
%     The Triple Alliance
%     The Country Party
%     Connection between Charles II. and France
%     Views of Lewis with respect to England
%     Treaty of Dover
%     Nature of the English Cabinet
%     The Cabal
%     Shutting of the Exchequer
%     War with the United Provinces, and their extreme Danger
%     William, Prince of Orange
%     Meeting of the Parliament; Declaration of Indulgence
%     It is cancelled, and the Test Act passed
%     The Cabal dissolved
%     Peace with the United Provinces; Administration of Danby
%     Embarrassing Situation of the Country Party
%     Dealings of that Party with the French Embassy
%     Peace of Nimeguen
%     Violent Discontents in England
%     Fall of Danby; the Popish Plot
%     Violence of the new House of Commons
%     Temple's Plan of Government
%     Character of Halifax
%     Character of Sunderland
%     Prorogation of the Parliament; Habeas Corpus Act; Second General Election of 1679
%     Popularity of Monmouth
%     Lawrence Hyde
%     Sidney Godolphin
%     Violence of Factions on the Subject of the Exclusion Bill
%     Names of Whig and Tory
%     Meeting of Parliament; The Exclusion Bill passes the Commons; \
%     Exclusion Bill rejected by the Lords
%     Execution of Stafford; General Election of 1681
%     Parliament held at Oxford, and dissolved
%     Tory Reaction
%     Persecution of the Whigs
%     Charter of the City confiscated; Whig Conspiracies
%     Detection of the Whig Conspiracies
%     Severity of the Government; Seizure of Charters
%     Influence of the Duke of York
%     He is opposed by Halifax
%     Lord Guildford
%     Policy of Lewis
%     State of Factions in the Court of Charles at the time of his Death
%
%CHAPTER III.
%
%     Great Change in the State of England since 1685
%     Population of England in 1685
%     Increase of Population greater in the North than in the South
%     Revenue in 1685
%     Military System
%     The Navy
%     The Ordnance
%     Noneffective Charge; Charge of Civil Government
%     Great Gains of Ministers and Courtiers
%     State of Agriculture
%     Mineral Wealth of the Country
%     Increase of Rent
%     The Country Gentlemen
%     The Clergy
%     The Yeomanry; Growth of the Towns; Bristol
%     Norwich
%     Other Country Towns
%     Manchester; Leeds; Sheffield
%     Birmingham
%     Liverpool
%     Watering-places; Cheltenham; Brighton; Buxton; Tunbridge Wells
%     Bath
%     London
%     The City
%     Fashionable Part of the Capital
%     Lighting of London
%     Police of London
%     Whitefriars; The Court
%     The Coffee Houses
%     Difficulty of Travelling
%     Badness of the Roads
%     Stage Coaches
%     Highwaymen
%     Inns
%     Post Office
%     Newspapers
%     News-letters
%     The Observator
%     Scarcity of Books in Country Places; Female Education
%     Literary Attainments of Gentlemen
%     Influence of French Literature
%     Immorality of the Polite Literature of England
%     State of Science in England
%     State of the Fine Arts
%     State of the Common People; Agricultural Wages
%     Wages of Manufacturers
%     Labour of Children in Factories
%     Wages of different Classes of Artisans
%     Number of Paupers
%     Benefits derived by the Common People from the Progress of
%     Civilisation
%     Delusion which leads Men to overrate the Happiness of preceding Generations
%
%CHAPTER IV.
%
%     Death of Charles II
%     Suspicions of Poison
%     Speech of James II. to the Privy Council
%     James proclaimed
%     State of the Administration
%     New Arrangements
%     Sir George Jeffreys
%     The Revenue collected without an Act of Parliament
%     A Parliament called
%     Transactions between James and the French King
%     Churchill sent Ambassador to France; His History
%     Feelings of the Continental Governments towards England
%     Policy of the Court of Rome
%     Struggle in the Mind of James; Fluctuations in his Policy
%     Public Celebration of the Roman Catholic Rites in the Palace
%     His Coronation
%     Enthusiasm of the Tories; Addresses
%     The Elections
%     Proceedings against Oates
%     Proceedings against Dangerfield
%     Proceedings against Baxter
%     Meeting of the Parliament of Scotland
%     Feeling of James towards the Puritans
%     Cruel Treatment of the Scotch Covenanters
%     Feeling of James towards the Quakers
%     William Penn
%     Peculiar Favour shown to Roman Catholics and Quakers
%     Meeting of the English Parliament; Trevor chosen Speaker;
%     Character of Seymour
%     The King's Speech to the Parliament
%     Debate in the Commons; Speech of Seymour
%     The Revenue voted; Proceedings of the Commons concerning Religion
%     Additional Taxes voted; Sir Dudley North
%     Proceedings of the Lords
%     Bill for reversing the Attainder of Stafford
%
%CHAPTER V.
%
%     Whig Refugees on the Continent
%     Their Correspondents in England
%     Characters of the leading Refugees; Ayloffe; Wade
%     Goodenough; Rumbold
%     Lord Grey
%     Monmouth
%     Ferguson
%     Scotch Refugees; Earl of Argyle
%     Sir Patrick Hume; Sir John Cochrane; Fletcher of Saltoun
%     Unreasonable Conduct of the Scotch Refugees
%     Arrangement for an Attempt on England and Scotland
%     John Locke
%     Preparations made by Government for the Defence of Scotland
%     Conversation of James with the Dutch Ambassadors; Ineffectual Attempts to prevent Argyle from sailing
%     Departure of Argyle from Holland; He lands in Scotland
%     His Disputes with his Followers
%     Temper of the Scotch Nation
%     Argyle's Forces dispersed
%     Argyle a Prisoner
%     His Execution.
%     Execution of Rumbold
%     Death of Ayloffe
%     Devastation of Argyleshire
%     Ineffectual Attempts to prevent Monmouth from leaving Holland
%     His Arrival at Lyme
%     His Declaration
%     His Popularity in the West of England
%     Encounter of the Rebels with the Militia at Bridport
%     Encounter of the Rebels with the Militia at Axminster;
%     News of the Rebellion carried to London;
%     Loyalty of the Parliament
%     Reception of Monmouth at Taunton
%     He takes the Title of King
%     His Reception at Bridgewater
%     Preparations of the Government to oppose him
%     His Design on Bristol
%     He relinquishes that Design
%     Skirmish at Philip's Norton; Despondence of Monmouth
%     He returns to Bridgewater; The Royal Army encamps at Sedgemoor
%     Battle of Sedgemoor
%     Pursuit of the Rebels
%     Military Executions; Flight of Monmouth
%     His Capture
%     His Letter to the King; He is carried to London
%     His Interview with the King
%     His Execution
%     His Memory cherished by the Common People
%     Cruelties of the Soldiers in the West; Kirke
%     Jeffreys sets out on the Western Circuit
%     Trial of Alice Lisle
%     The Bloody Assizes
%     Abraham Holmes
%     Christopher Battiseombe; The Hewlings
%     Punishment of Tutchin
%     Rebels Transported
%     Confiscation and Extortion
%     Rapacity of the Queen and her Ladies
%     Grey; Cochrane; Storey
%     Wade, Goodenough, and Ferguson
%     Jeffreys made Lord Chancellor
%     Trial and Execution of Cornish
%     Trials  and Executions of Fernley and Elizabeth Gaunt
%     Trial and Execution of Bateman
%     Persecution of the Protestant Dissenters
%
%
%
%
%
%HISTORY OF ENGLAND.




\chapter{CHAPTER I.}

I PURPOSE to write the history of England from the accession of King
James the Second down to a time which is within the memory of men still
living. I shall recount the errors which, in a few months, alienated a
loyal gentry and priesthood from the House of Stuart. I shall trace the
course of that revolution which terminated the long struggle between our
sovereigns and their parliaments, and bound up together the rights of
the people and the title of the reigning dynasty. I shall relate how the
new settlement was, during many troubled years, successfully defended
against foreign and domestic enemies; how, under that settlement,
the authority of law and the security of property were found to be
compatible with a liberty of discussion and of individual action never
before known; how, from the auspicious union of order and freedom,
sprang a prosperity of which the annals of human affairs had furnished
no example; how our country, from a state of ignominious vassalage,
rapidly rose to the place of umpire among European powers; how her
opulence and her martial glory grew together; how, by wise and resolute
good faith, was gradually established a public credit fruitful of
marvels which to the statesmen of any former age would have seemed
incredible; how a gigantic commerce gave birth to a maritime power,
compared with which every other maritime power, ancient or modern, sinks
into insignificance; how Scotland, after ages of enmity, was at length
united to England, not merely by legal bonds, but by indissoluble ties
of interest and affection; how, in America, the British colonies rapidly
became far mightier and wealthier than the realms which Cortes and
Pizarro had added to the dominions of Charles the Fifth; how in Asia,
British adventurers founded an empire not less splendid and more durable
than that of Alexander.

Nor will it be less my duty faithfully to record disasters mingled with
triumphs, and great national crimes and follies far more humiliating
than any disaster. It will be seen that even what we justly account our
chief blessings were not without alloy. It will be seen that the system
which effectually secured our liberties against the encroachments of
kingly power gave birth to a new class of abuses from which absolute
monarchies are exempt. It will be seen that, in consequence partly
of unwise interference, and partly of unwise neglect, the increase of
wealth and the extension of trade produced, together with immense good,
some evils from which poor and rude societies are free. It will be seen
how, in two important dependencies of the crown, wrong was followed
by just retribution; how imprudence and obstinacy broke the ties which
bound the North American colonies to the parent state; how Ireland,
cursed by the domination of race over race, and of religion over
religion, remained indeed a member of the empire, but a withered
and distorted member, adding no strength to the body politic, and
reproachfully pointed at by all who feared or envied the greatness of
England.

Yet, unless I greatly deceive myself, the general effect of this
chequered narrative will be to excite thankfulness in all religious
minds, and hope in the breasts of all patriots. For the history of our
country during the last hundred and sixty years is eminently the history
of physical, of moral, and of intellectual improvement. Those who
compare the age on which their lot has fallen with a golden age which
exists only in their imagination may talk of degeneracy and decay: but
no man who is correctly informed as to the past will be disposed to take
a morose or desponding view of the present.

I should very imperfectly execute the task which I have undertaken if
I were merely to treat of battles and sieges, of the rise and fall
of administrations, of intrigues in the palace, and of debates in the
parliament. It will be my endeavour to relate the history of the people
as well as the history of the government, to trace the progress of
useful and ornamental arts, to describe the rise of religious sects
and the changes of literary taste, to portray the manners of successive
generations and not to pass by with neglect even the revolutions which
have taken place in dress, furniture, repasts, and public amusements. I
shall cheerfully bear the reproach of having descended below the dignity
of history, if I can succeed in placing before the English of the
nineteenth century a true picture of the life of their ancestors.

The events which I propose to relate form only a single act of a great
and eventful drama extending through ages, and must be very imperfectly
understood unless the plot of the preceding acts be well known. I shall
therefore introduce my narrative by a slight sketch of the history of
our country from the earliest times. I shall pass very rapidly over many
centuries: but I shall dwell at some length on the vicissitudes of that
contest which the administration of King James the Second brought to a
decisive crisis. 
%[1]
\footnote{ In this, and in the next chapter, I have very seldom
thought it necessary to cite authorities: for, in these chapters, I have
not detailed events minutely, or used recondite materials; and the facts
which I mention are for the most part such that a person tolerably well
read in English history, if not already apprised of them, will at least
know where to look for evidence of them. In the subsequent chapters I
shall carefully indicate the sources of my information.}


Nothing in the early existence of Britain indicated the greatness which
she was destined to attain. Her inhabitants when first they became
known to the Tyrian mariners, were little superior to the natives of the
Sandwich Islands. She was subjugated by the Roman arms; but she
received only a faint tincture of Roman arts and letters. Of the western
provinces which obeyed the Caesars, she was the last that was conquered,
and the first that was flung away. No magnificent remains of Latin
porches and aqueducts are to be found in Britain. No writer of British
birth is reckoned among the masters of Latin poetry and eloquence. It is
not probable that the islanders were at any time generally familiar with
the tongue of their Italian rulers. From the Atlantic to the vicinity
of the Rhine the Latin has, during many centuries, been predominant. It
drove out the Celtic; it was not driven out by the Teutonic; and it is
at this day the basis of the French, Spanish and Portuguese languages.
In our island the Latin appears never to have superseded the old Gaelic
speech, and could not stand its ground against the German.

The scanty and superficial civilisation which the Britons had derived
from their southern masters was effaced by the calamities of the fifth
century. In the continental kingdoms into which the Roman empire was
then dissolved, the conquerors learned much from the conquered race. In
Britain the conquered race became as barbarous as the conquerors.

All the chiefs who founded Teutonic dynasties in the continental
provinces of the Roman empire, Alaric, Theodoric, Clovis, Alboin, were
zealous Christians. The followers of Ida and Cerdic, on the other hand,
brought to their settlements in Britain all the superstitions of the
Elbe. While the German princes who reigned at Paris, Toledo, Arles, and
Ravenna listened with reverence to the instructions of bishops, adored
the relics of martyrs, and took part eagerly in disputes touching the
Nicene theology, the rulers of Wessex and Mercia were still performing
savage rites in the temples of Thor and Woden.

The continental kingdoms which had risen on the ruins of the Western
Empire kept up some intercourse with those eastern provinces where the
ancient civilisation, though slowly fading away under the influence of
misgovernment, might still astonish and instruct barbarians, where the
court still exhibited the splendour of Diocletian and Constantine,
where the public buildings were still adorned with the sculptures of
Polycletus and the paintings of Apelles, and where laborious pedants,
themselves destitute of taste, sense, and spirit, could still read and
interpret the masterpieces of Sophocles, of Demosthenes, and of Plato.
From this communion Britain was cut off. Her shores were, to the
polished race which dwelt by the Bosphorus, objects of a mysterious
horror, such as that with which the Ionians of the age of Homer had
regarded the Straits of Scylla and the city of the Laestrygonian
cannibals. There was one province of our island in which, as Procopius
had been told, the ground was covered with serpents, and the air was
such that no man could inhale it and live. To this desolate region the
spirits of the departed were ferried over from the land of the Franks at
midnight. A strange race of fishermen performed the ghastly office. The
speech of the dead was distinctly heard by the boatmen, their weight
made the keel sink deep in the water; but their forms were invisible
to mortal eye. Such were the marvels which an able historian, the
contemporary of Belisarius, of Simplicius, and of Tribonian, gravely
related in the rich and polite Constantinople, touching the country in
which the founder of Constantinople had assumed the imperial purple.
Concerning all the other provinces of the Western Empire we have
continuous information. It is only in Britain that an age of fable
completely separates two ages of truth. Odoacer and Totila, Euric and
Thrasimund, Clovis, Fredegunda, and Brunechild, are historical men and
women. But Hengist and Horsa, Vortigern and Rowena, Arthur and Mordred
are mythical persons, whose very existence may be questioned, and whose
adventures must be classed with those of Hercules and Romulus.

At length the darkness begins to break; and the country which had been
lost to view as Britain reappears as England. The conversion of the
Saxon colonists to Christianity was the first of a long series of
salutary revolutions. It is true that the Church had been deeply
corrupted both by that superstition and by that philosophy against which
she had long contended, and over which she had at last triumphed. She
had given a too easy admission to doctrines borrowed from the ancient
schools, and to rites borrowed from the ancient temples. Roman policy
and Gothic ignorance, Grecian ingenuity and Syrian asceticism, had
contributed to deprave her. Yet she retained enough of the sublime
theology and benevolent morality of her earlier days to elevate many
intellects, and to purify many hearts. Some things also which at a later
period were justly regarded as among her chief blemishes were, in the
seventh century, and long afterwards, among her chief merits. That
the sacerdotal order should encroach on the functions of the civil
magistrate would, in our time, be a great evil. But that which in an age
of good government is an evil may, in an ago of grossly bad government,
be a blessing. It is better that mankind should be governed by wise
laws well administered, and by an enlightened public opinion, than by
priestcraft: but it is better that men should be governed by priestcraft
than by brute violence, by such a prelate as Dunstan than by such
a warrior as Penda. A society sunk in ignorance, and ruled by mere
physical force, has great reason to rejoice when a class, of which the
influence is intellectual and moral, rises to ascendancy. Such a class
will doubtless abuse its power: but mental power, even when abused,
is still a nobler and better power than that which consists merely in
corporeal strength. We read in our Saxon chronicles of tyrants, who,
when at the height of greatness, were smitten with remorse, who abhorred
the pleasures and dignities which they had purchased by guilt, who
abdicated their crowns, and who sought to atone for their offences by
cruel penances and incessant prayers. These stories have drawn forth
bitter expressions of contempt from some writers who, while they boasted
of liberality, were in truth as narrow-minded as any monk of the dark
ages, and whose habit was to apply to all events in the history of the
world the standard received in the Parisian society of the eighteenth
century. Yet surely a system which, however deformed by superstition,
introduced strong moral restraints into communities previously governed
only by vigour of muscle and by audacity of spirit, a system which
taught the fiercest and mightiest ruler that he was, like his meanest
bondman, a responsible being, might have seemed to deserve a more
respectful mention from philosophers and philanthropists.

The same observations will apply to the contempt with which, in the
last century, it was fashionable to speak of the pilgrimages, the
sanctuaries, the crusades, and the monastic institutions of the middle
ages. In times when men were scarcely ever induced to travel by liberal
curiosity, or by the pursuit of gain, it was better that the rude
inhabitant of the North should visit Italy and the East as a pilgrim,
than that he should never see anything but those squalid cabins and
uncleared woods amidst which he was born. In times when life and when
female honour were exposed to daily risk from tyrants and marauders,
it was better that the precinct of a shrine should be regarded with
an irrational awe, than that there should be no refuge inaccessible to
cruelty and licentiousness. In times when statesmen were incapable
of forming extensive political combinations, it was better that the
Christian nations should be roused and united for the recovery of the
Holy Sepulchre, than that they should, one by one, be overwhelmed by
the Mahometan power. Whatever reproach may, at a later period, have been
justly thrown on the indolence and luxury of religious orders, it was
surely good that, in an age of ignorance and violence, there should be
quiet cloisters and gardens, in which the arts of peace could be safely
cultivated, in which gentle and contemplative natures could find an
asylum, in which one brother could employ himself in transcribing the
Æneid of Virgil, and another in meditating the Analytics of Aristotle,
in which he who had a genius for art might illuminate a martyrology or
carve a crucifix, and in which he who had a turn for natural philosophy
might make experiments on the properties of plants and minerals. Had
not such retreats been scattered here and there, among the huts of
a miserable peasantry, and the castles of a ferocious aristocracy,
European society would have consisted merely of beasts of burden and
beasts of prey. The Church has many times been compared by divines
to the ark of which we read in the Book of Genesis: but never was the
resemblance more perfect than during that evil time when she alone rode,
amidst darkness and tempest, on the deluge beneath which all the great
works of ancient power and wisdom lay entombed, bearing within her that
feeble germ from which a Second and more glorious civilisation was to
spring.

Even the spiritual supremacy arrogated by the Pope was, in the dark
ages, productive of far more good than evil. Its effect was to unite the
nations of Western Europe in one great commonwealth. What the Olympian
chariot course and the Pythian oracle were to all the Greek cities, from
Trebizond to Marseilles, Rome and her Bishop were to all Christians
of the Latin communion, from Calabria to the Hebrides. Thus grew up
sentiments of enlarged benevolence. Races separated from each other by
seas and mountains acknowledged a fraternal tie and a common code of
public law. Even in war, the cruelty of the conqueror was not seldom
mitigated by the recollection that he and his vanquished enemies were
all members of one great federation.

Into this federation our Saxon ancestors were now admitted. A regular
communication was opened between our shores and that part of Europe in
which the traces of ancient power and policy were yet discernible.
Many noble monuments which have since been destroyed or defaced still
retained their pristine magnificence; and travellers, to whom Livy and
Sallust were unintelligible, might gain from the Roman aqueducts and
temples some faint notion of Roman history. The dome of Agrippa, still
glittering with bronze, the mausoleum of Adrian, not yet deprived of its
columns and statues, the Flavian amphitheatre, not yet degraded into a
quarry, told to the rude English pilgrims some part of the story of that
great civilised world which had passed away. The islanders returned,
with awe deeply impressed on their half opened minds, and told the
wondering inhabitants of the hovels of London and York that, near the
grave of Saint Peter, a mighty race, now extinct, had piled up buildings
which would never be dissolved till the judgment day. Learning followed
in the train of Christianity. The poetry and eloquence of the Augustan
age was assiduously studied in Mercian and Northumbrian monasteries. The
names of Bede and Alcuin were justly celebrated throughout Europe. Such
was the state of our country when, in the ninth century, began the last
great migration of the northern barbarians.

During many years Denmark and Scandinavia continued to pour forth
innumerable pirates, distinguished by strength, by valour, by merciless
ferocity, and by hatred of the Christian name. No country suffered so
much from these invaders as England. Her coast lay near to the ports
whence they sailed; nor was any shire so far distant from the sea as
to be secure from attack. The same atrocities which had attended the
victory of the Saxon over the Celt were now, after the lapse of ages,
suffered by the Saxon at the hand of the Dane. Civilization,--just as
it began to rise, was met by this blow, and sank down once more. Large
colonies of adventurers from the Baltic established themselves on the
eastern shores of our island, spread gradually westward, and, supported
by constant reinforcements from beyond the sea, aspired to the dominion
of the whole realm. The struggle between the two fierce Teutonic breeds
lasted through six generations. Each was alternately paramount. Cruel
massacres followed by cruel retribution, provinces wasted, convents
plundered, and cities rased to the ground, make up the greater part of
the history of those evil days. At length the North ceased to send forth
a constant stream of fresh depredators; and from that time the mutual
aversion of the races began to subside. Intermarriage became frequent.
The Danes learned the religion of the Saxons; and thus one cause
of deadly animosity was removed. The Danish and Saxon tongues, both
dialects of one widespread language, were blended together. But the
distinction between the two nations was by no means effaced, when
an event took place which prostrated both, in common slavery and
degradation, at the feet of a third people.

The Normans were then the foremost race of Christendom. Their valour and
ferocity had made them conspicuous among the rovers whom Scandinavia had
sent forth to ravage Western Europe. Their sails were long the terror of
both coasts of the Channel. Their arms were repeatedly carried far into
the heart of: the Carlovingian empire, and were victorious under the
walls of Maestricht and Paris. At length one of the feeble heirs of
Charlemagne ceded to the strangers a fertile province, watered by
a noble river, and contiguous to the sea which was their favourite
element. In that province they founded a mighty state, which gradually
extended its influence over the neighbouring principalities of Britanny
and Maine. Without laying aside that dauntless valour which had been the
terror of every land from the Elbe to the Pyrenees, the Normans rapidly
acquired all, and more than all, the knowledge and refinement which they
found in the country where they settled. Their courage secured their
territory against foreign invasion. They established internal order,
such as had long been unknown in the Frank empire. They embraced
Christianity; and with Christianity they learned a great part of what
the clergy had to teach. They abandoned their native speech, and adopted
the French tongue, in which the Latin was the predominant element. They
speedily raised their new language to a dignity and importance which it
had never before possessed. They found it a barbarous jargon; they fixed
it in writing; and they employed it in legislation, in poetry, and in
romance. They renounced that brutal intemperance to which all the other
branches of the great German family were too much inclined. The polite
luxury of the Norman presented a striking contrast to the coarse
voracity and drunkenness of his Saxon and Danish neighbours. He loved
to display his magnificence, not in huge piles of food and hogsheads of
strong drink, but in large and stately edifices, rich armour, gallant
horses, choice falcons, well ordered tournaments, banquets delicate
rather than abundant, and wines remarkable rather for their exquisite
flavour than for their intoxicating power. That chivalrous spirit, which
has exercised so powerful an influence on the politics, morals, and
manners of all the European nations, was found in the highest exaltation
among the Norman nobles. Those nobles were distinguished by their
graceful bearing and insinuating address. They were distinguished also
by their skill in negotiation, and by a natural eloquence which they
assiduously cultivated. It was the boast of one of their historians that
the Norman gentlemen were orators from the cradle. But their chief
fame was derived from their military exploits. Every country, from
the Atlantic Ocean to the Dead Sea, witnessed the prodigies of their
discipline and valour. One Norman knight, at the head of a handful of
warriors, scattered the Celts of Connaught. Another founded the monarchy
of the Two Sicilies, and saw the emperors both of the East and of the
West fly before his arms. A third, the Ulysses of the first crusade, was
invested by his fellow soldiers with the sovereignty of Antioch; and
a fourth, the Tancred whose name lives in the great poem of Tasso, was
celebrated through Christendom as the bravest and most generous of the
deliverers of the Holy Sepulchre.

The vicinity of so remarkable a people early began to produce an effect
on the public mind of England. Before the Conquest, English princes
received their education in Normandy. English sees and English estates
were bestowed on Normans. The French of Normandy was familiarly spoken
in the palace of Westminster. The court of Rouen seems to have been
to the court of Edward the Confessor what the court of Versailles long
afterwards was to the court of Charles the Second.

The battle of Hastings, and the events which followed it, not only
placed a Duke of Normandy on the English throne, but gave up the whole
population of England to the tyranny of the Norman race. The subjugation
of a nation by a nation has seldom, even in Asia, been more complete.
The country was portioned out among the captains of the invaders.
Strong military institutions, closely connected with the institution of
property, enabled the foreign conquerors to oppress the children of the
soil. A cruel penal code, cruelly enforced, guarded the privileges,
and even the sports, of the alien tyrants. Yet the subject race, though
beaten down and trodden underfoot, still made its sting felt. Some bold
men, the favourite heroes of our oldest ballads, betook themselves to
the woods, and there, in defiance of curfew laws and forest laws, waged
a predatory war against their oppressors. Assassination was an event of
daily occurrence. Many Normans suddenly disappeared leaving no trace.
The corpses of many were found bearing the marks of violence. Death by
torture was denounced against the murderers, and strict search was
made for them, but generally in vain; for the whole nation was in a
conspiracy to screen them. It was at length thought necessary to lay
a heavy fine on every Hundred in which a person of French extraction
should be found slain; and this regulation was followed up by another
regulation, providing that every person who was found slain should be
supposed to be a Frenchman, unless he was proved to be a Saxon.

During the century and a half which followed the Conquest, there is, to
speak strictly, no English history. The French Kings of England
rose, indeed, to an eminence which was the wonder and dread of all
neighbouring nations. They conquered Ireland. They received the homage
of Scotland. By their valour, by their policy, by their fortunate
matrimonial alliances, they became far more popular on the Continent
than their liege lords the Kings of France. Asia, as well as Europe,
was dazzled by the power and glory of our tyrants. Arabian chroniclers
recorded with unwilling admiration the fall of Acre, the defence of
Joppa, and the victorious march to Ascalon; and Arabian mothers
long awed their infants to silence with the name of the lionhearted
Plantagenet. At one time it seemed that the line of Hugh Capet was about
to end as the Merovingian and Carlovingian lines had ended, and that a
single great monarchy would spread from the Orkneys to the Pyrenees. So
strong an association is established in most minds between the greatness
of a sovereign and the greatness of the nation which he rules, that
almost every historian of England has expatiated with a sentiment of
exultation on the power and splendour of her foreign masters, and has
lamented the decay of that power and splendour as a calamity to our
country. This is, in truth, as absurd as it would be in a Haytian negro
of our time to dwell with national pride on the greatness of Lewis the
Fourteenth, and to speak of Blenheim and Ramilies with patriotic regret
and shame. The Conqueror and his descendants to the fourth generation
were not Englishmen: most of them were born in France: they spent the
greater part of their lives in France: their ordinary speech was French:
almost every high office in their gift was filled by a Frenchman: every
acquisition which they made on the Continent estranged them more and
more from the population of our island. One of the ablest among them
indeed attempted to win the hearts of his English subjects by espousing
an English princess. But, by many of his barons, this marriage was
regarded as a marriage between a white planter and a quadroon girl would
now be regarded in Virginia. In history he is known by the honourable
surname of Beauclerc; but, in his own time, his own countrymen
called him by a Saxon nickname, in contemptuous allusion to his Saxon
connection.

Had the Plantagenets, as at one time seemed likely, succeeded in uniting
all France under their government, it is probable that England would
never have had an independent existence. Her princes, her lords, her
prelates, would have been men differing in race and language from
the artisans and the tillers of the earth. The revenues of her great
proprietors would have been spent in festivities and diversions on the
banks of the Seine. The noble language of Milton and Burke would have
remained a rustic dialect, without a literature, a fixed grammar, or a
fixed orthography, and would have been contemptuously abandoned to the
use of boors. No man of English extraction would have risen to eminence,
except by becoming in speech and habits a Frenchman.

England owes her escape from such calamities to an event which her
historians have generally represented as disastrous. Her interest was so
directly opposed to the interests of her rulers that she had no hope but
in their errors and misfortunes. The talents and even the virtues of her
first six French Kings were a curse to her. The follies and vices of the
seventh were her salvation. Had John inherited the great qualities of
his father, of Henry Beauclerc, or of the Conqueror, nay, had he even
possessed the martial courage of Stephen or of Richard, and had the King
of France at the same time been as incapable as all the other successors
of Hugh Capet had been, the House of Plantagenet must have risen to
unrivalled ascendancy in Europe. But, just at this conjuncture, France,
for the first time since the death of Charlemagne, was governed by a
prince of great firmness and ability. On the other hand England,
which, since the battle of Hastings, had been ruled generally by wise
statesmen, always by brave soldiers, fell under the dominion of a
trifler and a coward. From that moment her prospects brightened. John
was driven from Normandy. The Norman nobles were compelled to make their
election between the island and the continent. Shut up by the sea with
the people whom they had hitherto oppressed and despised, they gradually
came to regard England as their country, and the English as their
countrymen. The two races, so long hostile, soon found that they had
common interests and common enemies. Both were alike aggrieved by the
tyranny of a bad king. Both were alike indignant at the favour shown by
the court to the natives of Poitou and Aquitaine. The great grandsons of
those who had fought under William and the great grandsons of those who
had fought under Harold began to draw near to each other in friendship;
and the first pledge of their reconciliation was the Great Charter, won
by their united exertions, and framed for their common benefit.

Here commences the history of the English nation. The history of the
preceding events is the history of wrongs inflicted and sustained by
various tribes, which indeed all dwelt on English ground, but which
regarded each other with aversion such as has scarcely ever existed
between communities separated by physical barriers. For even the mutual
animosity of countries at war with each other is languid when compared
with the animosity of nations which, morally separated, are yet locally
intermingled. In no country has the enmity of race been carried farther
than in England. In no country has that enmity been more completely
effaced. The stages of the process by which the hostile elements were
melted down into one homogeneous mass are not accurately known to us.
But it is certain that, when John became King, the distinction between
Saxons and Normans was strongly marked, and that before the end of the
reign of his grandson it had almost disappeared. In the time of Richard
the First, the ordinary imprecation of a Norman gentleman was "May I
become an Englishman!" His ordinary form of indignant denial was "Do you
take me for an Englishman?" The descendant of such a gentleman a hundred
years later was proud of the English name.

The sources of the noblest rivers which spread fertility over
continents, and bear richly laden fleets to the sea, are to be sought
in wild and barren mountain tracts, incorrectly laid down in maps,
and rarely explored by travellers. To such a tract the history of our
country during the thirteenth century may not unaptly be compared.
Sterile and obscure as is that portion of our annals, it is there that
we must seek for the origin of our freedom, our prosperity, and our
glory. Then it was that the great English people was formed, that the
national character began to exhibit those peculiarities which it has
ever since retained, and that our fathers became emphatically islanders,
islanders not merely in geographical position, but in their politics,
their feelings, and their manners. Then first appeared with distinctness
that constitution which has ever since, through all changes,
preserved its identity; that constitution of which all the other free
constitutions in the world are copies, and which, in spite of some
defects, deserves to be regarded as the best under which any great
society has ever yet existed during many ages. Then it was that the
House of Commons, the archetype of all the representative assemblies
which now meet, either in the old or in the new world, held its first
sittings. Then it was that the common law rose to the dignity of
a science, and rapidly became a not unworthy rival of the imperial
jurisprudence. Then it was that the courage of those sailors who manned
the rude barks of the Cinque Ports first made the flag of England
terrible on the seas. Then it was that the most ancient colleges which
still exist at both the great national seats of learning were founded.
Then was formed that language, less musical indeed than the languages
of the south, but in force, in richness, in aptitude for all the highest
purposes of the poet, the philosopher, and the orator, inferior to the
tongue of Greece alone. Then too appeared the first faint dawn of that
noble literature, the most splendid and the most durable of the many
glories of England.

Early in the fourteenth century the amalgamation of the races was
all but complete; and it was soon made manifest, by signs not to be
mistaken, that a people inferior to none existing in the world had been
formed by the mixture of three branches of the great Teutonic family
with each other, and with the aboriginal Britons. There was, indeed,
scarcely anything in common between the England to which John had been
chased by Philip Augustus, and the England from which the armies of
Edward the Third went forth to conquer France.

A period of more than a hundred years followed, during which the chief
object of the English was to establish, by force of arms, a great empire
on the Continent. The claim of Edward to the inheritance occupied by
the House of Valois was a claim in which it might seem that his subjects
were little interested. But the passion for conquest spread fast from
the prince to the people. The war differed widely from the wars
which the Plantagenets of the twelfth century had waged against the
descendants of Hugh Capet. For the success of Henry the Second, or of
Richard the First, would have made England a province of France. The
effect of the successes of Edward the Third and Henry the Fifth was to
make France, for a time, a province of England. The disdain with which,
in the twelfth century, the conquerors from the Continent had regarded
the islanders, was now retorted by the islanders on the people of the
Continent. Every yeoman from Kent to Northumberland valued himself as
one of a race born for victory and dominion, and looked down with
scorn on the nation before which his ancestors had trembled. Even those
knights of Gascony and Guienne who had fought gallantly under the Black
Prince were regarded by the English as men of an inferior breed, and
were contemptuously excluded from honourable and lucrative commands. In
no long time our ancestors altogether lost sight of the original
ground of quarrel. They began to consider the crown of France as a
mere appendage to the crown of England; and, when in violation of the
ordinary law of succession, they transferred the crown of England to the
House of Lancaster, they seem to have thought that the right of Richard
the Second to the crown of France passed, as of course, to that house.
The zeal and vigour which they displayed present a remarkable contrast
to the torpor of the French, who were far more deeply interested in
the event of the struggle. The most splendid victories recorded in the
history of the middle ages were gained at this time, against great odds,
by the English armies. Victories indeed they were of which a nation may
justly be proud; for they are to be attributed to the moral superiority
of the victors, a superiority which was most striking in the lowest
ranks. The knights of England found worthy rivals in the knights of
France. Chandos encountered an equal foe in Du Guesclin. But France had
no infantry that dared to face the English bows and bills. A French King
was brought prisoner to London. An English King was crowned at Paris.
The banner of St. George was carried far beyond the Pyrenees and the
Alps. On the south of the Ebro the English won a great battle, which for
a time decided the fate of Leon and Castile; and the English Companies
obtained a terrible preeminence among the bands of warriors who let out
their weapons for hire to the princes and commonwealths of Italy.

Nor were the arts of peace neglected by our fathers during that stirring
period. While France was wasted by war, till she at length found in
her own desolation a miserable defence against invaders, the English
gathered in their harvests, adorned their cities, pleaded, traded, and
studied in security. Many of our noblest architectural monuments belong
to that age. Then rose the fair chapels of New College and of Saint
George, the nave of Winchester and the choir of York, the spire of
Salisbury and the majestic towers of Lincoln. A copious and forcible
language, formed by an infusion of French into German, was now the
common property of the aristocracy and of the people. Nor was it long
before genius began to apply that admirable machine to worthy purposes.
While English warriors, leaving behind them the devastated provinces of
France, entered Valladolid in triumph, and spread terror to the gates of
Florence, English poets depicted in vivid tints all the wide variety
of human manners and fortunes, and English thinkers aspired to know, or
dared to doubt, where bigots had been content to wonder and to believe.
The same age which produced the Black Prince and Derby, Chandos and
Hawkwood, produced also Geoffrey Chaucer and John Wycliffe.

In so splendid and imperial a manner did the English people, properly
so called, first take place among the nations of the world. Yet while
we contemplate with pleasure the high and commanding qualities which
our forefathers displayed, we cannot but admit that the end which they
pursued was an end condemned both by humanity and by enlightened policy,
and that the reverses which compelled them, after a long and bloody
struggle, to relinquish the hope of establishing a great continental
empire, were really blessings in the guise of disasters. The spirit of
the French was at last aroused: they began to oppose a vigorous national
resistance to the foreign conquerors; and from that time the skill
of the English captains and the courage of the English soldiers were,
happily for mankind, exerted in vain. After many desperate struggles,
and with many bitter regrets, our ancestors gave up the contest. Since
that age no British government has ever seriously and steadily pursued
the design of making great conquests on the Continent. The people,
indeed, continued to cherish with pride the recollection of Cressy, of
Poitiers, and of Agincourt. Even after the lapse of many years it was
easy to fire their blood and to draw forth their subsidies by promising
them an expedition for the conquest of France. But happily the energies
of our country have been directed to better objects; and she now
occupies in the history of mankind a place far more glorious than if
she had, as at one time seemed not improbable, acquired by the sword
an ascendancy similar to that which formerly belonged to the Roman
republic.

Cooped up once more within the limits of the island, the warlike people
employed in civil strife those arms which had been the terror of Europe.
The means of profuse expenditure had long been drawn by the English
barons from the oppressed provinces of France. That source of supply
was gone: but the ostentatious and luxurious habits which prosperity had
engendered still remained; and the great lords, unable to gratify their
tastes by plundering the French, were eager to plunder each other.
The realm to which they were now confined would not, in the phrase of
Comines, the most judicious observer of that time, suffice for them all.
Two aristocratical factions, headed by two branches of the royal family,
engaged in a long and fierce struggle for supremacy. As the animosity
of those factions did not really arise from the dispute about the
succession it lasted long after all ground of dispute about the
succession was removed. The party of the Red Rose survived the last
prince who claimed the crown in right of Henry the Fourth. The party
of the White Rose survived the marriage of Richmond and Elizabeth.
Left without chiefs who had any decent show of right, the adherents of
Lancaster rallied round a line of bastards, and the adherents of York
set up a succession of impostors. When, at length, many aspiring nobles
had perished on the field of battle or by the hands of the executioner,
when many illustrious houses had disappeared forever from history, when
those great families which remained had been exhausted and sobered by
calamities, it was universally acknowledged that the claims of all the
contending Plantagenets were united in the house of Tudor.

Meanwhile a change was proceeding infinitely more momentous than the
acquisition or loss of any province, than the rise or fall of
any dynasty. Slavery and the evils by which slavery is everywhere
accompanied were fast disappearing.

It is remarkable that the two greatest and most salutary social
revolutions which have taken place in England, that revolution which, in
the thirteenth century, put an end to the tyranny of nation over nation,
and that revolution which, a few generations later, put an end to the
property of man in man, were silently and imperceptibly effected. They
struck contemporary observers with no surprise, and have received from
historians a very scanty measure of attention. They were brought about
neither by legislative regulations nor by physical force. Moral causes
noiselessly effaced first the distinction between Norman and Saxon, and
then the distinction between master and slave. None can venture to fix
the precise moment at which either distinction ceased. Some faint traces
of the old Norman feeling might perhaps have been found late in the
fourteenth century. Some faint traces of the institution of villenage
were detected by the curious so late as the days of the Stuarts; nor has
that institution ever, to this hour, been abolished by statute.

It would be most unjust not to acknowledge that the chief agent in
these two great deliverances was religion; and it may perhaps be doubted
whether a purer religion might not have been found a less efficient
agent. The benevolent spirit of the Christian morality is undoubtedly
adverse to distinctions of caste. But to the Church of Rome such
distinctions are peculiarly odious; for they are incompatible with other
distinctions which are essential to her system. She ascribes to every
priest a mysterious dignity which entitles him to the reverence of every
layman; and she does not consider any man as disqualified, by reason
of his nation or of his family, for the priesthood. Her doctrines
respecting the sacerdotal character, however erroneous they may be, have
repeatedly mitigated some of the worst evils which can afflict society.
That superstition cannot be regarded as unmixedly noxious which, in
regions cursed by the tyranny of race over race, creates an aristocracy
altogether independent of race, inverts the relation between the
oppressor and the oppressed, and compels the hereditary master to kneel
before the spiritual tribunal of the hereditary bondman. To this day,
in some countries where negro slavery exists, Popery appears in
advantageous contrast to other forms of Christianity. It is notorious
that the antipathy between the European and African races is by no
means so strong at Rio Janerio as at Washington. In our own country this
peculiarity of the Roman Catholic system produced, during the middle
ages, many salutary effects. It is true that, shortly after the battle
of Hastings, Saxon prelates and abbots were violently deposed, and that
ecclesiastical adventurers from the Continent were intruded by hundreds
into lucrative benefices. Yet even then pious divines of Norman blood
raised their voices against such a violation of the constitution of the
Church, refused to accept mitres from the hands of William, and charged
him, on the peril of his soul, not to forget that the vanquished
islanders were his fellow Christians. The first protector whom the
English found among the dominant caste was Archbishop Anselm. At a
time when the English name was a reproach, and when all the civil and
military dignities of the kingdom were supposed to belong exclusively
to the countrymen of the Conqueror, the despised race learned, with
transports of delight, that one of themselves, Nicholas Breakspear,
had been elevated to the papal throne, and had held out his foot to be
kissed by ambassadors sprung from the noblest houses of Normandy. It was
a national as well as a religious feeling that drew great multitudes to
the shrine of Becket, whom they regarded as the enemy of their enemies.
Whether he was a Norman or a Saxon may be doubted: but there is no doubt
that he perished by Norman hands, and that the Saxons cherished his
memory with peculiar tenderness and veneration, and, in their popular
poetry, represented him as one of their own race. A successor of Becket
was foremost among the refractory magnates who obtained that charter
which secured the privileges both of the Norman barons and of the Saxon
yeomanry. How great a part the Roman Catholic ecclesiastics subsequently
had in the abolition of villenage we learn from the unexceptionable
testimony of Sir Thomas Smith, one of the ablest Protestant counsellors
of Elizabeth. When the dying slaveholder asked for the last sacraments,
his spiritual attendants regularly adjured him, as he loved his soul,
to emancipate his brethren for whom Christ had died. So successfully had
the Church used her formidable machinery that, before the Reformation
came, she had enfranchised almost all the bondmen in the kingdom
except her own, who, to do her justice, seem to have been very tenderly
treated.

There can be no doubt that, when these two great revolutions had been
effected, our forefathers were by far the best governed people in
Europe. During three hundred years the social system had been in a
constant course of improvement. Under the first Plantagenets there had
been barons able to bid defiance to the sovereign, and peasants degraded
to the level of the swine and oxen which they tended. The exorbitant
power of the baron had been gradually reduced. The condition of the
peasant had been gradually elevated. Between the aristocracy and
the working people had sprung up a middle class, agricultural and
commercial. There was still, it may be, more inequality than is
favourable to the happiness and virtue of our species: but no man was
altogether above the restraints of law; and no man was altogether below
its protection.

That the political institutions of England were, at this early period,
regarded by the English with pride and affection, and by the most
enlightened men of neighbouring nations with admiration and envy,
is proved by the clearest evidence. But touching the nature of these
institutions there has been much dishonest and acrimonious controversy.

The historical literature of England has indeed suffered grievously from
a circumstance which has not a little contributed to her prosperity. The
change, great as it is, which her polity has undergone during the
last six centuries, has been the effect of gradual development, not of
demolition and reconstruction. The present constitution of our country
is, to the constitution under which she flourished five hundred years
ago, what the tree is to the sapling, what the man is to the boy. The
alteration has been great. Yet there never was a moment at which the
chief part of what existed was not old. A polity thus formed must abound
in anomalies. But for the evils arising from mere anomalies we have
ample compensation. Other societies possess written constitutions
more symmetrical. But no other society has yet succeeded in uniting
revolution with prescription, progress with stability, the energy of
youth with the majesty of immemorial antiquity.

This great blessing, however, has its drawbacks: and one of those
drawbacks is that every source of information as to our early history
has been poisoned by party spirit. As there is no country where
statesmen have been so much under the influence of the past, so there is
no country where historians have been so much under the influence of
the present. Between these two things, indeed, there is a natural
connection. Where history is regarded merely as a picture of life and
manners, or as a collection of experiments from which general maxims
of civil wisdom may be drawn, a writer lies under no very pressing
temptation to misrepresent transactions of ancient date. But where
history is regarded as a repository of titledeeds, on which the rights
of governments and nations depend, the motive to falsification becomes
almost irresistible. A Frenchman is not now impelled by any strong
interest either to exaggerate or to underrate the power of the Kings of
the house of Valois. The privileges of the States General, of the States
of Britanny, of the States of Burgundy, are to him matters of as little
practical importance as the constitution of the Jewish Sanhedrim or of
the Amphictyonic Council. The gulph of a great revolution completely
separates the new from the old system. No such chasm divides the
existence of the English nation into two distinct parts. Our laws and
customs have never been lost in general and irreparable ruin. With us
the precedents of the middle ages are still valid precedents, and are
still cited, on the gravest occasions, by the most eminent Statesmen.
For example, when King George the Third was attacked by the malady which
made him incapable of performing his regal functions, and when the most
distinguished lawyers and politicians differed widely as to the course
which ought, in such circumstances, to be pursued, the Houses of
Parliament would not proceed to discuss any plan of regency till all
the precedents which were to be found in our annals, from the earliest
times, had been collected and arranged. Committees were appointed to
examine the ancient records of the realm. The first case reported was
that of the year 1217: much importance was attached to the cases of
1326, of 1377, and of 1422: but the case which was justly considered
as most in point was that of 1455. Thus in our country the dearest
interests of parties have frequently been on the results of the
researches of antiquaries. The inevitable consequence was that our
antiquaries conducted their researches in the spirit of partisans.

It is therefore not surprising that those who have written, concerning
the limits of prerogative and liberty in the old polity of England
should generally have shown the temper, not of judges, but of angry and
uncandid advocates. For they were discussing, not a speculative matter,
but a matter which had a direct and practical connection with the most
momentous and exciting disputes of their own day. From the commencement
of the long contest between the Parliament and the Stuarts down to the
time when the pretensions of the Stuarts ceased to be formidable, few
questions were practically more important than the question whether the
administration of that family had or had not been in accordance with the
ancient constitution of the kingdom. This question could be decided only
by reference to the records of preceding reigns. Bracton and Fleta, the
Mirror of Justice and the Rolls of Parliament, were ransacked to find
pretexts for the excesses of the Star Chamber on one side, and of the
High Court of Justice on the other. During a long course of years every
Whig historian was anxious to prove that the old English government was
all but republican, every Tory historian to prove that it was all but
despotic.

With such feelings, both parties looked into the chronicles of the
middle ages. Both readily found what they sought; and both obstinately
refused to see anything but what they sought. The champions of the
Stuarts could easily point out instances of oppression exercised on
the subject. The defenders of the Roundheads could as easily produce
instances of determined and successful resistance offered to the Crown.
The Tories quoted, from ancient writings, expressions almost as servile
as were heard from the pulpit of Mainwaring. The Whigs discovered
expressions as bold and severe as any that resounded from the judgment
seat of Bradshaw. One set of writers adduced numerous instances in which
Kings had extorted money without the authority of Parliament. Another
set cited cases in which the Parliament had assumed to itself the power
of inflicting punishment on Kings. Those who saw only one half of the
evidence would have concluded that the Plantagenets were as absolute
as the Sultans of Turkey: those who saw only the other half would have
concluded that the Plantagenets had as little real power as the Doges
of Venice; and both conclusions would have been equally remote from the
truth.

The old English government was one of a class of limited monarchies
which sprang up in Western Europe during the middle ages, and which,
notwithstanding many diversities, bore to one another a strong family
likeness. That there should have been such a likeness is not strange The
countries in which those monarchies arose had been provinces of the same
great civilised empire, and had been overrun and conquered, about the
same time, by tribes of the same rude and warlike nation. They were
members of the same great coalition against Islam. They were in
communion with the same superb and ambitious Church. Their polity
naturally took the same form. They had institutions derived partly from
imperial Rome, partly from papal Rome, partly from the old Germany.
All had Kings; and in all the kingly office became by degrees strictly
hereditary. All had nobles bearing titles which had originally indicated
military rank. The dignity of knighthood, the rules of heraldry, were
common to all. All had richly endowed ecclesiastical establishments,
municipal corporations enjoying large franchises, and senates whose
consent was necessary to the validity of some public acts.

Of these kindred constitutions the English was, from an early period,
justly reputed the best. The prerogatives of the sovereign were
undoubtedly extensive. The spirit of religion and the spirit of chivalry
concurred to exalt his dignity. The sacred oil had been poured on his
head. It was no disparagement to the bravest and noblest knights to
kneel at his feet. His person was inviolable. He alone was entitled to
convoke the Estates of the realm: he could at his pleasure dismiss them;
and his assent was necessary to all their legislative acts. He was the
chief of the executive administration, the sole organ of communication
with foreign powers, the captain of the military and naval forces of the
state, the fountain of justice, of mercy, and of honour. He had large
powers for the regulation of trade. It was by him that money was
coined, that weights and measures were fixed, that marts and havens
were appointed. His ecclesiastical patronage was immense. His hereditary
revenues, economically administered, sufficed to meet the ordinary
charges of government. His own domains were of vast extent. He was also
feudal lord paramount of the whole soil of his kingdom, and, in that
capacity, possessed many lucrative and many formidable rights, which
enabled him to annoy and depress those who thwarted him, and to enrich
and aggrandise, without any cost to himself, those who enjoyed his
favour.

But his power, though ample, was limited by three great constitutional
principles, so ancient that none can say when they began to exist,
so potent that their natural development, continued through many
generations, has produced the order of things under which we now live.

First, the King could not legislate without the consent of his
Parliament. Secondly, he could impose no tax without the consent of
his Parliament. Thirdly, he was bound to conduct the executive
administration according to the laws of the land, and, if he broke those
laws, his advisers and his agents were responsible.

No candid Tory will deny that these principles had, five hundred years
ago, acquired the authority of fundamental rules. On the other hand,
no candid Whig will affirm that they were, till a later period, cleared
from all ambiguity, or followed out to all their consequences. A
constitution of the middle ages was not, like a constitution of the
eighteenth or nineteenth century, created entire by a single act,
and fully set forth in a single document. It is only in a refined
and speculative age that a polity is constructed on system. In rude
societies the progress of government resembles the progress of language
and of versification. Rude societies have language, and often copious
and energetic language: but they have no scientific grammar, no
definitions of nouns and verbs, no names for declensions, moods, tenses,
and voices. Rude societies have versification, and often versification
of great power and sweetness: but they have no metrical canons; and the
minstrel whose numbers, regulated solely by his ear, are the delight
of his audience, would himself be unable to say of how many dactyls and
trochees each of his lines consists. As eloquence exists before syntax,
and song before prosody, so government may exist in a high degree
of excellence long before the limits of legislative, executive, and
judicial power have been traced with precision.

It was thus in our country. The line which bounded the royal
prerogative, though in general sufficiently clear, had not everywhere
been drawn with accuracy and distinctness. There was, therefore, near
the border some debatable ground on which incursions and reprisals
continued to take place, till, after ages of strife, plain and durable
landmarks were at length set up. It may be instructive to note in what
way, and to what extent, our ancient sovereigns were in the habit of
violating the three great principles by which the liberties of the
nation were protected.

No English King has ever laid claim to the general legislative power.
The most violent and imperious Plantagenet never fancied himself
competent to enact, without the consent of his great council, that a
jury should consist of ten persons instead of twelve, that a widow's
dower should be a fourth part instead of a third, that perjury should
be a felony, or that the custom of gavelkind should be introduced into
Yorkshire. 
%[2]
\footnote{ This is excellently put by Mr. Hallam in the first chapter
of his Constitutional History.}
 But the King had the power of pardoning offenders; and
there is one point at which the power of pardoning and the power of
legislating seem to fade into each other, and may easily, at least in a
simple age, be confounded. A penal statute is virtually annulled if the
penalties which it imposes are regularly remitted as often as they are
incurred. The sovereign was undoubtedly competent to remit penalties
without limit. He was therefore competent to annul virtually a penal
statute. It might seem that there could be no serious objection to his
doing formally what he might do virtually. Thus, with the help of subtle
and courtly lawyers, grew up, on the doubtful frontier which separates
executive from legislative functions, that great anomaly known as the
dispensing power.

That the King could not impose taxes without the consent of Parliament
is admitted to have been, from time immemorial, a fundamental law of
England. It was among the articles which John was compelled by the
Barons to sign. Edward the First ventured to break through the rule:
but, able, powerful, and popular as he was, he encountered an opposition
to which he found it expedient to yield. He covenanted accordingly in
express terms, for himself and his heirs, that they would never again
levy any aid without the assent and goodwill of the Estates of the
realm. His powerful and victorious grandson attempted to violate this
solemn compact: but the attempt was strenuously withstood. At length the
Plantagenets gave up the point in despair: but, though they ceased to
infringe the law openly, they occasionally contrived, by evading it,
to procure an extraordinary supply for a temporary purpose. They were
interdicted from taxing; but they claimed the right of begging and
borrowing. They therefore sometimes begged in a tone not easily to be
distinguished from that of command, and sometimes borrowed with small
thought of repaying. But the fact that they thought it necessary to
disguise their exactions under the names of benevolences and loans
sufficiently proves that the authority of the great constitutional rule
was universally recognised.

The principle that the King of England was bound to conduct the
administration according to law, and that, if he did anything against
law, his advisers and agents were answerable, was established at a very
early period, as the severe judgments pronounced and executed on many
royal favourites sufficiently prove. It is, however, certain that the
rights of individuals were often violated by the Plantagenets, and that
the injured parties were often unable to obtain redress. According to
law no Englishman could be arrested or detained in confinement merely
by the mandate of the sovereign. In fact, persons obnoxious to the
government were frequently imprisoned without any other authority than
a royal order. According to law, torture, the disgrace of the Roman
jurisprudence, could not, in any circumstances, be inflicted on an
English subject. Nevertheless, during the troubles of the fifteenth
century, a rack was introduced into the Tower, and was occasionally used
under the plea of political necessity. But it would be a great error to
infer from such irregularities that the English monarchs were, either in
theory or in practice, absolute. We live in a highly civilised society,
through which intelligence is so rapidly diffused by means of the press
and of the post office that any gross act of oppression committed in
any part of our island is, in a few hours, discussed by millions. If the
sovereign were now to immure a subject in defiance of the writ of Habeas
Corpus, or to put a conspirator to the torture, the whole nation would
be instantly electrified by the news. In the middle ages the state of
society was widely different. Rarely and with great difficulty did the
wrongs of individuals come to the knowledge of the public. A man might
be illegally confined during many months in the castle of Carlisle or
Norwich; and no whisper of the transaction might reach London. It is
highly probable that the rack had been many years in use before the
great majority of the nation had the least suspicion that it was ever
employed. Nor were our ancestors by any means so much alive as we are to
the importance of maintaining great general rules. We have been taught
by long experience that we cannot without danger suffer any breach of
the constitution to pass unnoticed. It is therefore now universally held
that a government which unnecessarily exceeds its powers ought to be
visited with severe parliamentary censure, and that a government which,
under the pressure of a great exigency, and with pure intentions, has
exceeded its powers, ought without delay to apply to Parliament for an
act of indemnity. But such were not the feelings of the Englishmen of
the fourteenth and fifteenth centuries. They were little disposed to
contend for a principle merely as a principle, or to cry out against an
irregularity which was not also felt to be a grievance. As long as the
general spirit of the administration was mild and popular, they
were willing to allow some latitude to their sovereign. If, for ends
generally acknowledged to be good, he exerted a vigour beyond the
law, they not only forgave, but applauded him, and while they enjoyed
security and prosperity under his rule, were but too ready to believe
that whoever had incurred his displeasure had deserved it. But to this
indulgence there was a limit; nor was that King wise who presumed far on
the forbearance of the English people. They might sometimes allow him to
overstep the constitutional line: but they also claimed the privilege
of overstepping that line themselves, whenever his encroachments were so
serious as to excite alarm. If, not content with occasionally oppressing
individuals, he cared to oppress great masses, his subjects promptly
appealed to the laws, and, that appeal failing, appealed as promptly to
the God of battles.

Our forefathers might indeed safely tolerate a king in a few excesses;
for they had in reserve a check which soon brought the fiercest and
proudest king to reason, the check of physical force. It is difficult
for an Englishman of the nineteenth century to imagine to himself the
facility and rapidity with which, four hundred years ago, this check was
applied. The people have long unlearned the use of arms. The art of
war has been carried to a perfection unknown to former ages; and the
knowledge of that art is confined to a particular class. A hundred
thousand soldiers, well disciplined and commanded, will keep down ten
millions of ploughmen and artisans. A few regiments of household troops
are sufficient to overawe all the discontented spirits of a large
capital. In the meantime the effect of the constant progress of wealth
has been to make insurrection far more terrible to thinking men than
maladministration. Immense sums have been expended on works which, if
a rebellion broke out, might perish in a few hours. The mass of movable
wealth collected in the shops and warehouses of London alone exceeds
five hundredfold that which the whole island contained in the days of
the Plantagenets; and, if the government were subverted by physical
force, all this movable wealth would be exposed to imminent risk of
spoliation and destruction. Still greater would be the risk to public
credit, on which thousands of families directly depend for subsistence,
and with which the credit of the whole commercial world is inseparably
connected. It is no exaggeration to say that a civil war of a week on
English ground would now produce disasters which would be felt from the
Hoang-ho to the Missouri, and of which the traces would be discernible
at the distance of a century. In such a state of society resistance must
be regarded as a cure more desperate than almost any malady which can
afflict the state. In the middle ages, on the contrary, resistance was
an ordinary remedy for political distempers, a remedy which was always
at hand, and which, though doubtless sharp at the moment, produced no
deep or lasting ill effects. If a popular chief raised his standard in
a popular cause, an irregular army could be assembled in a day. Regular
army there was none. Every man had a slight tincture of soldiership,
and scarcely any man more than a slight tincture. The national wealth
consisted chiefly in flocks and herds, in the harvest of the year, and
in the simple buildings inhabited by the people. All the furniture, the
stock of shops, the machinery which could be found in the realm was of
less value than the property which some single parishes now contain.
Manufactures were rude; credit was almost unknown. Society, therefore,
recovered from the shock as soon as the actual conflict was over. The
calamities of civil war were confined to the slaughter on the field of
battle, and to a few subsequent executions and confiscations. In a week
the peasant was driving his team and the esquire flying his hawks over
the field of Towton or of Bosworth, as if no extraordinary event had
interrupted the regular course of human life.

More than a hundred and sixty years have now elapsed since the English
people have by force subverted a government. During the hundred and
sixty years which preceded the union of the Roses, nine Kings reigned in
England. Six of these nine Kings were deposed. Five lost their lives
as well as their crowns. It is evident, therefore, that any comparison
between our ancient and our modern polity must lead to most erroneous
conclusions, unless large allowance be made for the effect of that
restraint which resistance and the fear of resistance constantly
imposed on the Plantagenets. As our ancestors had against tyranny a most
important security which we want, they might safely dispense with some
securities to which we justly attach the highest importance. As we
cannot, without the risk of evils from which the imagination recoils,
employ physical force as a check on misgovernment, it is evidently our
wisdom to keep all the constitutional checks on misgovernment in the
highest state of efficiency, to watch with jealousy the first beginnings
of encroachment, and never to suffer irregularities, even when harmless
in themselves, to pass unchallenged, lest they acquire the force of
precedents. Four hundred years ago such minute vigilance might well seem
unnecessary. A nation of hardy archers and spearmen might, with small
risk to its liberties, connive at some illegal acts on the part of a
prince whose general administration was good, and whose throne was not
defended by a single company of regular soldiers.

Under this system, rude as it may appear when compared with those
elaborate constitutions of which the last seventy years have been
fruitful, the English long enjoyed a large measure of freedom and
happiness. Though, during the feeble reign of Henry the Sixth, the state
was torn, first by factions, and at length by civil war; though Edward
the Fourth was a prince of dissolute and imperious character; though
Richard the Third has generally been represented as a monster of
depravity; though the exactions of Henry the Seventh caused great
repining; it is certain that our ancestors, under those Kings, were far
better governed than the Belgians under Philip, surnamed the Good, or
the French under that Lewis who was styled the Father of his people.
Even while the wars of the Roses were actually raging, our country
appears to have been in a happier condition than the neighbouring realms
during years of profound peace. Comines was one of the most enlightened
statesmen of his time. He had seen all the richest and most highly
civilised parts of the Continent. He had lived in the opulent towns of
Flanders, the Manchesters and Liverpools of the fifteenth century. He
had visited Florence, recently adorned by the magnificence of Lorenzo,
and Venice, not yet bumbled by the Confederates of Cambray. This eminent
man deliberately pronounced England to be the best governed country of
which he had any knowledge. Her constitution he emphatically designated
as a just and holy thing, which, while it protected the people, really
strengthened the hands of a prince who respected it. In no other country
were men so effectually secured from wrong. The calamities produced by
our intestine wars seemed to him to be confined to the nobles and the
fighting men, and to leave no traces such as he had been accustomed to
see elsewhere, no ruined dwellings, no depopulated cities.

It was not only by the efficiency of the restraints imposed on the royal
prerogative that England was advantageously distinguished from most of
the neighbouring countries. A: peculiarity equally important, though
less noticed, was the relation in which the nobility stood here to the
commonalty. There was a strong hereditary aristocracy: but it was of all
hereditary aristocracies the least insolent and exclusive. It had none
of the invidious character of a caste. It was constantly receiving
members from the people, and constantly sending down members to mingle
with the people. Any gentleman might become a peer. The younger son of a
peer was but a gentleman. Grandsons of peers yielded precedence to newly
made knights. The dignity of knighthood was not beyond the reach of
any man who could by diligence and thrift realise a good estate, or
who could attract notice by his valour in a battle or a siege. It was
regarded as no disparagement for the daughter of a Duke, nay of a royal
Duke, to espouse a distinguished commoner. Thus, Sir John Howard married
the daughter of Thomas Mowbray Duke of Norfolk. Sir Richard Pole married
the Countess of Salisbury, daughter of George, Duke of Clarence. Good
blood was indeed held in high respect: but between good blood and the
privileges of peerage there was, most fortunately for our country, no
necessary connection. Pedigrees as long, and scutcheons as old, were to
be found out of the House of Lords as in it. There were new men who bore
the highest titles. There were untitled men well known to be descended
from knights who had broken the Saxon ranks at Hastings, and scaled the
walls of Jerusalem. There were Bohuns, Mowbrays, DeVeres, nay, kinsmen
of the House of Plantagenet, with no higher addition than that of
Esquire, and with no civil privileges beyond those enjoyed by every
farmer and shopkeeper. There was therefore here no line like that which
in some other countries divided the patrician from the plebeian. The
yeoman was not inclined to murmur at dignities to which his own children
might rise. The grandee was not inclined to insult a class into which
his own children must descend.

After the wars of York and Lancaster, the links which connected the
nobility and commonalty became closer and more numerous than ever. The
extent of destruction which had fallen on the old aristocracy may be
inferred from a single circumstance. In the year 1451 Henry the Sixth
summoned fifty-three temporal Lords to parliament. The temporal Lords
summoned by Henry the Seventh to the parliament of 1485 were only
twenty-nine, and of these several had recently been elevated to the
peerage. During the following century the ranks of the nobility were
largely recruited from among the gentry. The constitution of the House
of Commons tended greatly to promote the salutary intermixture of
classes. The knight of the shire was the connecting link between
the baron and the shopkeeper. On the same benches on which sate the
goldsmiths, drapers, and grocers, who had been returned to parliament by
the commercial towns, sate also members who, in any other country, would
have been called noblemen, hereditary lords of manors, entitled to hold
courts and to bear coat armour, and able to trace back an honourable
descent through many generations. Some of them were younger sons and
brothers of lords. Others could boast of even royal blood. At length the
eldest son of an Earl of Bedford, called in courtesy by the second title
of his father, offered himself as candidate for a seat in the House of
Commons, and his example was followed by others. Seated in that house,
the heirs of the great peers naturally became as zealous for its
privileges as any of the humble burgesses with whom they were mingled.
Thus our democracy was, from an early period, the most aristocratic, and
our aristocracy the most democratic in the world; a peculiarity
which has lasted down to the present day, and which has produced many
important moral and political effects.

The government of Henry the Seventh, of his son, and of his
grandchildren was, on the whole, more arbitrary than that of the
Plantagenets. Personal character may in some degree explain the
difference; for courage and force of will were common to all the men and
women of the House of Tudor. They exercised their power during a period
of a hundred and twenty years, always with vigour, often with violence,
sometimes with cruelty. They, in imitation of the dynasty which
had preceded them, occasionally invaded the rights of the subject,
occasionally exacted taxes under the name of loans and gifts, and
occasionally dispensed with penal statutes: nay, though they never
presumed to enact any permanent law by their own authority, they
occasionally took upon themselves, when Parliament was not sitting,
to meet temporary exigencies by temporary edicts. It was, however,
impossible for the Tudors to carry oppression beyond a certain point:
for they had no armed force, and they were surrounded by an armed
people. Their palace was guarded by a few domestics, whom the array of
a single shire, or of a single ward of London, could with ease have
overpowered. These haughty princes were therefore under a restraint
stronger than any that mere law can impose, under a restraint which did
not, indeed, prevent them from sometimes treating an individual in an
arbitrary and even in a barbarous manner, but which effectually secured
the nation against general and long continued oppression. They might
safely be tyrants, within the precinct of the court: but it was
necessary for them to watch with constant anxiety the temper of the
country. Henry the Eighth, for example, encountered no opposition when
he wished to send Buckingham and Surrey, Anne Boleyn and Lady Salisbury,
to the scaffold. But when, without the consent of Parliament, he
demanded of his subjects a contribution amounting to one sixth of their
goods, he soon found it necessary to retract. The cry of hundreds of
thousands was that they were English and not French, freemen and not
slaves. In Kent the royal commissioners fled for their lives. In Suffolk
four thousand men appeared in arms. The King's lieutenants in that
county vainly exerted themselves to raise an army. Those who did not
join in the insurrection declared that they would not fight against
their brethren in such a quarrel. Henry, proud and selfwilled as he was,
shrank, not without reason from a conflict with the roused spirit of
the nation. He had before his eyes the fate of his predecessors who
had perished at Berkeley and Pomfret. He not only cancelled his
illegal commissions; he not only granted a general pardon to all the
malecontents; but he publicly and solemnly apologised for his infraction
of the laws.

His conduct, on this occasion, well illustrates the whole policy of his
house. The temper of the princes of that line was hot, and their
spirits high, but they understood the character of the nation that they
governed, and never once, like some of their predecessors, and some of
their successors, carried obstinacy to a fatal point. The discretion of
the Tudors was such, that their power, though it was often resisted,
was never subverted. The reign of every one of them was disturbed by
formidable discontents: but the government was always able either to
soothe the mutineers or to conquer and punish them. Sometimes, by timely
concessions, it succeeded in averting civil hostilities; but in general
it stood firm, and called for help on the nation. The nation obeyed
the call, rallied round the sovereign, and enabled him to quell the
disaffected minority.

Thus, from the age of Henry the Third to the age of Elizabeth, England
grew and flourished under a polity which contained the germ of our
present institutions, and which, though not very exactly defined, or
very exactly observed, was yet effectually prevented from degenerating
into despotism, by the awe in which the governors stood of the spirit
and strength of the governed.

But such a polity is suited only to a particular stage in the progress
of society. The same causes which produce a division of labour in the
peaceful arts must at length make war a distinct science and a distinct
trade. A time arrives when the use of arms begins to occupy the entire
attention of a separate class. It soon appears that peasants and
burghers, however brave, are unable to stand their ground against
veteran soldiers, whose whole life is a preparation for the day of
battle, whose nerves have been braced by long familiarity with danger,
and whose movements have all the precision of clockwork. It is found
that the defence of nations can no longer be safely entrusted to
warriors taken from the plough or the loom for a campaign of forty
days. If any state forms a great regular army, the bordering states
must imitate the example, or must submit to a foreign yoke. But, where
a great regular army exists, limited monarchy, such as it was in the
middle ages, can exist no longer. The sovereign is at once emancipated
from what had been the chief restraint on his power; and he inevitably
becomes absolute, unless he is subjected to checks such as would be
superfluous in a society where all are soldiers occasionally, and none
permanently.

With the danger came also the means of escape. In the monarchies of the
middle ages the power of the sword belonged to the prince; but the power
of the purse belonged to the nation; and the progress of civilisation,
as it made the sword of the prince more and more formidable to the
nation, made the purse of the nation more and more necessary to the
prince. His hereditary revenues would no longer suffice, even for the
expenses of civil government. It was utterly impossible that, without
a regular and extensive system of taxation, he could keep in constant
efficiency a great body of disciplined troops. The policy which the
parliamentary assemblies of Europe ought to have adopted was to take
their stand firmly on their constitutional right to give or withhold
money, and resolutely to refuse funds for the support of armies, till
ample securities had been provided against despotism.

This wise policy was followed in our country alone. In the neighbouring
kingdoms great military establishments were formed; no new safeguards
for public liberty were devised; and the consequence was, that the old
parliamentary institutions everywhere ceased to exist. In France, where
they had always been feeble, they languished, and at length died of
mere weakness. In Spain, where they had been as strong as in any part
of Europe, they struggled fiercely for life, but struggled too late. The
mechanics of Toledo and Valladolid vainly defended the privileges of the
Castilian Cortes against the veteran battalions of Charles the Fifth. As
vainly, in the next generation, did the citizens of Saragossa stand up
against Philip the Second, for the old constitution of Aragon. One after
another, the great national councils of the continental monarchies,
councils once scarcely less proud and powerful than those which sate
at Westminster, sank into utter insignificance. If they met, they met
merely as our Convocation now meets, to go through some venerable forms.

In England events took a different course. This singular felicity she
owed chiefly to her insular situation. Before the end of the fifteenth
century great military establishments were indispensable to the dignity,
and even to the safety, of the French and Castilian monarchies. If
either of those two powers had disarmed, it would soon have been
compelled to submit to the dictation of the other. But England,
protected by the sea against invasion, and rarely engaged in warlike
operations on the Continent, was not, as yet, under the necessity
of employing regular troops. The sixteenth century, the seventeenth
century, found her still without a standing army. At the commencement
of the seventeenth century political science had made considerable
progress. The fate of the Spanish Cortes and of the French States
General had given solemn warning to our Parliaments; and our
Parliaments, fully aware of the nature and magnitude of the danger,
adopted, in good time, a system of tactics which, after a contest
protracted through three generations, was at length successful.

Almost every writer who has treated of that contest has been desirous to
show that his own party was the party which was struggling to preserve
the old constitution unaltered. The truth however is that the old
constitution could not be preserved unaltered. A law, beyond the control
of human wisdom, had decreed that there should no longer be governments
of that peculiar class which, in the fourteenth and fifteenth centuries,
had been common throughout Europe. The question, therefore, was not
whether our polity should undergo a change, but what the nature of
the change should be. The introduction of a new and mighty force had
disturbed the old equilibrium, and had turned one limited monarchy after
another into an absolute monarchy. What had happened elsewhere would
assuredly have happened here, unless the balance had been redressed by
a great transfer of power from the crown to the parliament. Our princes
were about to have at their command means of coercion such as no
Plantagenet or Tudor had ever possessed. They must inevitably have
become despots, unless they had been, at the same time, placed under
restraints to which no Plantagenet or Tudor had ever been subject.

It seems certain, therefore, that, had none but political causes been
at work, the seventeenth century would not have passed away without
a fierce conflict between our Kings and their Parliaments. But other
causes of perhaps greater potency contributed to produce the same
effect. While the government of the Tudors was in its highest vigour
an event took place which has coloured the destinies of all Christian
nations, and in an especial manner the destinies of England. Twice
during the middle ages the mind of Europe had risen up against the
domination of Rome. The first insurrection broke out in the south of
France. The energy of Innocent the Third, the zeal of the young orders
of Francis and Dominic, and the ferocity of the Crusaders whom the
priesthood let loose on an unwarlike population, crushed the Albigensian
churches. The second reformation had its origin in England, and spread
to Bohemia. The Council of Constance, by removing some ecclesiastical
disorders which had given scandal to Christendom, and the princes
of Europe, by unsparingly using fire and sword against the heretics,
succeeded in arresting and turning back the movement. Nor is this
much to be lamented. The sympathies of a Protestant, it is true, will
naturally be on the side of the Albigensians and of the Lollards. Yet an
enlightened and temperate Protestant will perhaps be disposed to doubt
whether the success, either of the Albigensians or of the Lollards,
would, on the whole, have promoted the happiness and virtue of mankind.
Corrupt as the Church of Rome was, there is reason to believe that, if
that Church had been overthrown in the twelfth or even in the fourteenth
century, the vacant space would have been occupied by some system more
corrupt still. There was then, through the greater part of Europe, very
little knowledge; and that little was confined to the clergy. Not one
man in five hundred could have spelled his way through a psalm. Books
were few and costly. The art of printing was unknown. Copies of the
Bible, inferior in beauty and clearness to those which every cottager
may now command, sold for prices which many priests could not afford
to give. It was obviously impossible that the laity should search the
Scriptures for themselves. It is probable therefore, that, as soon as
they had put off one spiritual yoke, they would have put on another,
and that the power lately exercised by the clergy of the Church of
Rome would have passed to a far worse class of teachers. The sixteenth
century was comparatively a time of light. Yet even in the sixteenth
century a considerable number of those who quitted the old religion
followed the first confident and plausible guide who offered himself,
and were soon led into errors far more serious than those which they had
renounced. Thus Matthias and Kniperdoling, apostles of lust, robbery,
and murder, were able for a time to rule great cities. In a darker age
such false prophets might have founded empires; and Christianity might
have been distorted into a cruel and licentious superstition, more
noxious, not only than Popery, but even than Islamism.

About a hundred years after the rising of the Council of Constance, that
great change emphatically called the Reformation began. The fulness
of time was now come. The clergy were no longer the sole or the chief
depositories of knowledge The invention of printing had furnished the
assailants of the Church with a mighty weapon which had been wanting
to their predecessors. The study of the ancient writers, the rapid
development of the powers of the modern languages, the unprecedented
activity which was displayed in every department of literature, the
political state of Europe, the vices of the Roman court, the exactions
of the Roman chancery, the jealousy with which the wealth and privileges
of the clergy were naturally regarded by laymen, the jealousy with which
the Italian ascendency was naturally regarded by men born on our side of
the Alps, all these things gave to the teachers of the new theology an
advantage which they perfectly understood how to use.

Those who hold that the influence of the Church of Rome in the dark
ages was, on the whole, beneficial to mankind, may yet with perfect
consistency regard the Reformation as an inestimable blessing. The
leading strings, which preserve and uphold the infant, would impede the
fullgrown man. And so the very means by which the human mind is, in one
stage of its progress, supported and propelled, may, in another stage,
be mere hindrances. There is a season in the life both of an individual
and of a society, at which submission and faith, such as at a later
period would be justly called servility and credulity, are useful
qualities. The child who teachably and undoubtingly listens to the
instructions of his elders is likely to improve rapidly. But the man who
should receive with childlike docility every assertion and dogma uttered
by another man no wiser than himself would become contemptible. It is
the same with communities. The childhood of the European nations
was passed under the tutelage of the clergy. The ascendancy of the
sacerdotal order was long the ascendancy which naturally and properly
belongs to intellectual superiority. The priests, with all their faults,
were by far the wisest portion of society. It was, therefore, on the
whole, good that they should be respected and obeyed. The encroachments
of the ecclesiastical power on the province of the civil power produced
much more happiness than misery, while the ecclesiastical power was in
the hands of the only class that had studied history, philosophy, and
public law, and while the civil power was in the hands of savage chiefs,
who could not read their own grants and edicts. But a change took place.
Knowledge gradually spread among laymen. At the commencement of the
sixteenth century many of them were in every intellectual attainment
fully equal to the most enlightened of their spiritual pastors.
Thenceforward that dominion, which, during the dark ages, had been, in
spite of many abuses, a legitimate and salutary guardianship, became an
unjust and noxious tyranny.

From the time when the barbarians overran the Western Empire to the time
of the revival of letters, the influence of the Church of Rome had been
generally favourable to science to civilisation, and to good government.
But, during the last three centuries, to stunt the growth of the human
mind has been her chief object. Throughout Christendom, whatever advance
has been made in knowledge, in freedom, in wealth, and in the arts of
life, has been made in spite of her, and has everywhere been in inverse
proportion to her power. The loveliest and most fertile provinces
of Europe have, under her rule, been sunk in poverty, in political
servitude, and in intellectual torpor, while Protestant countries, once
proverbial for sterility and barbarism, have been turned by skill
and industry into gardens, and can boast of a long list of heroes and
statesmen, philosophers and poets. Whoever, knowing what Italy and
Scotland naturally are, and what, four hundred years ago, they actually
were, shall now compare the country round Rome with the country round
Edinburgh, will be able to form some judgment as to the tendency of
Papal domination. The descent of Spain, once the first among monarchies,
to the lowest depths of degradation, the elevation of Holland, in spite
of many natural disadvantages, to a position such as no commonwealth so
small has ever reached, teach the same lesson. Whoever passes in Germany
from a Roman Catholic to a Protestant principality, in Switzerland
from a Roman Catholic to a Protestant canton, in Ireland from a Roman
Catholic to a Protestant county, finds that he has passed from a lower
to a higher grade of civilisation. On the other side of the Atlantic the
same law prevails. The Protestants of the United States have left far
behind them the Roman Catholics of Mexico, Peru, and Brazil. The Roman
Catholics of Lower Canada remain inert, while the whole continent round
them is in a ferment with Protestant activity and enterprise. The French
have doubtless shown an energy and an intelligence which, even when
misdirected, have justly entitled them to be called a great people. But
this apparent exception, when examined, will be found to confirm the
rule; for in no country that is called Roman Catholic, has the Roman
Catholic Church, during several generations, possessed so little
authority as in France. The literature of France is justly held in high
esteem throughout the world. But if we deduct from that literature all
that belongs to four parties which have been, on different grounds,
in rebellion against the Papal domination, all that belongs to
the Protestants, all that belongs to the assertors of the Gallican
liberties, all that belongs to the Jansenists, and all that belongs to
the philosophers, how much will be left?

It is difficult to say whether England owes more to the Roman Catholic
religion or to the Reformation. For the amalgamation of races and for
the abolition of villenage, she is chiefly indebted to the influence
which the priesthood in the middle ages exercised over the laity. For
political and intellectual freedom, and for all the blessings which
political and intellectual freedom have brought in their train, she
is chiefly indebted to the great rebellion of the laity against the
priesthood.

The struggle between the old and the new theology in our country was
long, and the event sometimes seemed doubtful. There were two extreme
parties, prepared to act with violence or to suffer with stubborn
resolution. Between them lay, during a considerable time, a middle
party, which blended, very illogically, but by no means unnaturally,
lessons learned in the nursery with the sermons of the modern
evangelists, and, while clinging with fondness to all observances, yet
detested abuses with which those observances were closely connected. Men
in such a frame of mind were willing to obey, almost with thankfulness,
the dictation of an able ruler who spared them the trouble of judging
for themselves, and, raising a firm and commanding voice above the
uproar of controversy, told them how to worship and what to believe.
It is not strange, therefore, that the Tudors should have been able to
exercise a great influence on ecclesiastical affairs; nor is it strange
that their influence should, for the most part, have been exercised with
a view to their own interest.

Henry the Eighth attempted to constitute an Anglican Church differing
from the Roman Catholic Church on the point of the supremacy, and on
that point alone. His success in this attempt was extraordinary. The
force of his character, the singularly favourable situation in which
he stood with respect to foreign powers, the immense wealth which the
spoliation of the abbeys placed at his disposal, and the support of
that class which still halted between two Opinions, enabled him to bid
defiance to both the extreme parties, to burn as heretics those who
avowed the tenets of the Reformers, and to hang as traitors those who
owned the authority of the Pope. But Henry's system died with him. Had
his life been prolonged, he would have found it difficult to maintain a
position assailed with equal fury by all who were zealous either for
the new or for the old opinions. The ministers who held the royal
prerogatives in trust for his infant son could not venture to persist in
so hazardous a policy; nor could Elizabeth venture to return to it. It
was necessary to make a choice. The government must either submit to
Rome, or must obtain the aid of the Protestants. The government and the
Protestants had only one thing in common, hatred of the Papal power.
The English Reformers were eager to go as far as their brethren on the
Continent. They unanimously condemned as Antichristian numerous dogmas
and practices to which Henry had stubbornly adhered, and which Elizabeth
reluctantly abandoned. Many felt a strong repugnance even to things
indifferent which had formed part of the polity or ritual of the
mystical Babylon. Thus Bishop Hooper, who died manfully at Gloucester
for his religion, long refused to wear the episcopal vestments. Bishop
Ridley, a martyr of still greater renown, pulled down the ancient altars
of his diocese, and ordered the Eucharist to be administered in the
middle of churches, at tables which the Papists irreverently termed
oyster boards. Bishop Jewel pronounced the clerical garb to be a stage
dress, a fool's coat, a relique of the Amorites, and promised that
he would spare no labour to extirpate such degrading absurdities.
Archbishop Grindal long hesitated about accepting a mitre from dislike
of what he regarded as the mummery of consecration. Bishop Parkhurst
uttered a fervent prayer that the Church of England would propose to
herself the Church of Zurich as the absolute pattern of a Christian
community. Bishop Ponet was of opinion that the word Bishop should be
abandoned to the Papists, and that the chief officers of the purified
church should be called Superintendents. When it is considered that
none of these prelates belonged to the extreme section of the Protestant
party, it cannot be doubted that, if the general sense of that party
had been followed, the work of reform would have been carried on as
unsparingly in England as in Scotland.

But, as the government needed the support of the protestants, so the
Protestants needed the protection of the government. Much was therefore
given up on both sides: an union was effected; and the fruit of that
union was the Church of England.

To the peculiarities of this great institution, and to the strong
passions which it has called forth in the minds both of friends and of
enemies, are to be attributed many of the most important events which
have, since the Reformation, taken place in our country; nor can the
secular history of England be at all understood by us, unless we study
it in constant connection with the history of her ecclesiastical polity.

The man who took the chief part in settling the condition, of the
alliance which produced the Anglican Church was Archbishop Cranmer. He
was the representative of both the parties which, at that time, needed
each other's assistance. He was at once a divine and a courtier. In his
character of divine he was perfectly ready to go as far in the way of
change as any Swiss or Scottish Reformer. In his character of courtier
he was desirous to preserve that organisation which had, during many
ages, admirably served the purposes of the Bishops of Rome, and might be
expected now to serve equally well the purposes of the English Kings and
of their ministers. His temper and his understanding, eminently fitted
him to act as mediator. Saintly in his professions, unscrupulous in
his dealings, zealous for nothing, bold in speculation, a coward and a
timeserver in action, a placable enemy and a lukewarm friend, he was in
every way qualified to arrange the terms of the coalition between the
religious and the worldly enemies of Popery.

To this day the constitution, the doctrines, and the services of the
Church, retain the visible marks of the compromise from which she
sprang. She occupies a middle position between the Churches of Rome
and Geneva. Her doctrinal confessions and discourses, composed by
Protestants, set forth principles of theology in which Calvin or
Knox would have found scarcely a word to disapprove. Her prayers and
thanksgivings, derived from the ancient Breviaries, are very generally
such that Cardinal Fisher or Cardinal Pole might have heartily joined in
them. A controversialist who puts an Arminian sense on her Articles and
Homilies will be pronounced by candid men to be as unreasonable as a
controversialist who denies that the doctrine of baptismal regeneration
can be discovered in her Liturgy.

The Church of Rome held that episcopacy was of divine institution, and
that certain supernatural graces of a high order had been transmitted by
the imposition of hands through fifty generations, from the Eleven who
received their commission on the Galilean mount, to the bishops who
met at Trent. A large body of Protestants, on the other hand, regarded
prelacy as positively unlawful, and persuaded themselves that they
found a very different form of ecclesiastical government prescribed in
Scripture. The founders of the Anglican Church took a middle course.
They retained episcopacy; but they did not declare it to be an
institution essential to the welfare of a Christian society, or to the
efficacy of the sacraments. Cranmer, indeed, on one important occasion,
plainly avowed his conviction that, in the primitive times, there was no
distinction between bishops and priests, and that the laying on of hands
was altogether superfluous.

Among the Presbyterians the conduct of public worship is, to a great
extent, left to the minister. Their prayers, therefore, are not exactly
the same in any two assemblies on the same day, or on any two days in
the same assembly. In one parish they are fervent, eloquent, and full of
meaning. In the next parish they may be languid or absurd. The priests
of the Roman Catholic Church, on the other hand, have, during many
generations, daily chanted the same ancient confessions, supplications,
and thanksgivings, in India and Lithuania, in Ireland and Peru. The
service, being in a dead language, is intelligible only to the learned;
and the great majority of the congregation may be said to assist as
spectators rather than as auditors. Here, again, the Church of England
took a middle course. She copied the Roman Catholic forms of prayer,
but translated them into the vulgar tongue, and invited the illiterate
multitude to join its voice to that of the minister.

In every part of her system the same policy may be traced. Utterly
rejecting the doctrine of transubstantiation, and condemning as
idolatrous all adoration paid to the sacramental bread and wine, she
yet, to the disgust of the Puritan, required her children to receive the
memorials of divine love, meekly kneeling upon their knees. Discarding
many rich vestments which surrounded the altars of the ancient faith,
she yet retained, to the horror of weak minds, a robe of white linen,
typical of the purity which belonged to her as the mystical spouse of
Christ. Discarding a crowd of pantomimic gestures which, in the Roman
Catholic worship, are substituted for intelligible words, she yet
shocked many rigid Protestants by marking the infant just sprinkled from
the font with the sign of the cross. The Roman Catholic addressed his
prayers to a multitude of Saints, among whom were numbered many men
of doubtful, and some of hateful, character. The Puritan refused the
addition of Saint even to the apostle of the Gentiles, and to the
disciple whom Jesus loved. The Church of England, though she asked
for the intercession of no created being, still set apart days for the
commemoration of some who had done and suffered great things for the
faith. She retained confirmation and ordination as edifying rites; but
she degraded them from the rank of sacraments. Shrift was no part of her
system. Yet she gently invited the dying penitent to confess his sins to
a divine, and empowered her ministers to soothe the departing soul by
an absolution which breathes the very spirit of the old religion. In
general it may be said that she appeals more to the understanding, and
less to the senses and the imagination, than the Church of Rome, and
that she appeals less to the understanding, and more to the senses
and imagination, than the Protestant Churches of Scotland, France, and
Switzerland.

Nothing, however, so strongly distinguished the Church of England from
other Churches as the relation in which she stood to the monarchy. The
King was her head. The limits of the authority which he possessed,
as such, were not traced, and indeed have never yet been traced with
precision. The laws which declared him supreme in ecclesiastical
matters were drawn rudely and in general terms. If, for the purpose of
ascertaining the sense of those laws, we examine the books and lives of
those who founded the English Church, our perplexity will be increased.
For the founders of the English Church wrote and acted in an age of
violent intellectual fermentation, and of constant action and reaction.
They therefore often contradicted each other and sometimes contradicted
themselves. That the King was, under Christ, sole head of the Church was
a doctrine which they all with one voice affirmed: but those words had
very different significations in different mouths, and in the same
mouth at different conjunctures. Sometimes an authority which would have
satisfied Hildebrand was ascribed to the sovereign: then it dwindled
down to an authority little more than that which had been claimed by
many ancient English princes who had been in constant communion with the
Church of Rome. What Henry and his favourite counsellors meant, at one
time, by the supremacy, was certainly nothing less than the whole power
of the keys. The King was to be the Pope of his kingdom, the vicar
of God, the expositor of Catholic verity, the channel of sacramental
graces. He arrogated to himself the right of deciding dogmatically what
was orthodox doctrine and what was heresy, of drawing up and imposing
confessions of faith, and of giving religious instruction to his people.
He proclaimed that all jurisdiction, spiritual as well as temporal, was
derived from him alone, and that it was in his power to confer episcopal
authority, and to take it away. He actually ordered his seal to be put
to commissions by which bishops were appointed, who were to exercise
their functions as his deputies, and during his pleasure. According to
this system, as expounded by Cranmer, the King was the spiritual as well
as the temporal chief of the nation. In both capacities His Highness
must have lieutenants. As he appointed civil officers to keep his seal,
to collect his revenues, and to dispense justice in his name, so
he appointed divines of various ranks to preach the gospel, and to
administer the sacraments. It was unnecessary that there should be any
imposition of hands. The King,--such was the opinion of Cranmer given in
the plainest words,--might in virtue of authority derived from God, make
a priest; and the priest so made needed no ordination whatever. These
opinions the Archbishop, in spite of the opposition of less courtly
divines, followed out to every legitimate consequence. He held that his
own spiritual functions, like the secular functions of the Chancellor
and Treasurer, were at once determined by a demise of the crown. When
Henry died, therefore, the Primate and his suffragans took out fresh
commissions, empowering them to ordain and to govern the Church till the
new sovereign should think fit to order otherwise. When it was objected
that a power to bind and to loose, altogether distinct from temporal
power, had been given by our Lord to his apostles, some theologians of
this school replied that the power to bind and to loose had descended,
not to the clergy, but to the whole body of Christian men, and ought
to be exercised by the chief magistrate as the representative of the
society. When it was objected that Saint Paul had spoken of certain
persons whom the Holy Ghost had made overseers and shepherds of the
faithful, it was answered that King Henry was the very overseer,
the very shepherd whom the Holy Ghost had appointed, and to whom the
expressions of Saint Paul applied. 
%[3]
\footnote{ See a very curious paper which Strype believed to be in
Gardiner's handwriting. Ecclesiastical Memorials, Book 1., Chap. xvii.}


These high pretensions gave scandal to Protestants as well as to
Catholics; and the scandal was greatly increased when the supremacy,
which Mary had resigned back to the Pope, was again annexed to the
crown, on the accession of Elizabeth. It seemed monstrous that a woman
should be the chief bishop of a Church in which an apostle had forbidden
her even to let her voice be heard. The Queen, therefore, found it
necessary expressly to disclaim that sacerdotal character which
her father had assumed, and which, according to Cranmer, had been
inseparably joined, by divine ordinance, to the regal function. When the
Anglican confession of faith was revised in her reign, the supremacy
was explained in a manner somewhat different from that which had been
fashionable at the court of Henry. Cranmer had declared, in emphatic
terms, that God had immediately committed to Christian princes the whole
cure of all their subjects, as well concerning the administration of
God's word for the cure of souls, as concerning the administration of
things political. 
%[4]
\footnote{ These are Cranmer's own words. See the Appendix to Burnet's
History of the Reformation, Part 1. Book III. No. 21. Question 9.}
 The thirty-seventh article of religion, framed
under Elizabeth, declares, in terms as emphatic, that the ministering
of God's word does not belong to princes. The Queen, however, still had
over the Church a visitatorial power of vast and undefined extent. She
was entrusted by Parliament with the office of restraining and punishing
heresy and every sort of ecclesiastical abuse, and was permitted to
delegate her authority to commissioners. The Bishops were little more
than her ministers. Rather than grant to the civil magistrate the
absolute power of nominating spiritual pastors, the Church of Rome, in
the eleventh century, set all Europe on fire. Rather than grant to the
civil magistrate the absolute power of nominating spiritual pastors, the
ministers of the Church of Scotland, in our time, resigned their livings
by hundreds. The Church of England had no such scruples. By the royal
authority alone her prelates were appointed. By the royal authority
alone her Convocations were summoned, regulated, prorogued, and
dissolved. Without the royal sanction her canons had no force. One
of the articles of her faith was that without the royal consent no
ecclesiastical council could lawfully assemble. From all her judicatures
an appeal lay, in the last resort, to the sovereign, even when the
question was whether an opinion ought to be accounted heretical, or
whether the administration of a sacrament had been valid. Nor did the
Church grudge this extensive power to our princes. By them she had been
called into existence, nursed through a feeble infancy, guarded from
Papists on one side and from Puritans on the other, protected against
Parliaments which bore her no good will, and avenged on literary
assailants whom she found it hard to answer. Thus gratitude, hope, fear,
common attachments, common enmities, bound her to the throne. All her
traditions, all her tastes, were monarchical. Loyalty became a point
of professional honour among her clergy, the peculiar badge which
distinguished them at once from Calvinists and from Papists. Both the
Calvinists and the Papists, widely as they differed in other respects,
regarded with extreme jealousy all encroachments of the temporal power
on the domain of the spiritual power. Both Calvinists and Papists
maintained that subjects might justifiably draw the sword against
ungodly rulers. In France Calvinists resisted Charles the Ninth: Papists
resisted Henry the Fourth: both Papists and Calvinists resisted Henry
the Third. In Scotland Calvinists led Mary captive. On the north of
the Trent Papists took arms against the English throne. The Church
of England meantime condemned both Calvinists and Papists, and loudly
boasted that no duty was more constantly or earnestly inculcated by her
than that of submission to princes.

The advantages which the crown derived from this close alliance with
the Established Church were great; but they were not without serious
drawbacks. The compromise arranged by Cranmer had from the first been
considered by a large body of Protestants as a scheme for serving two
masters, as an attempt to unite the worship of the Lord with the worship
of Baal. In the days of Edward the Sixth the scruples of this party had
repeatedly thrown great difficulties in the way of the government. When
Elizabeth came to the throne, those difficulties were much increased.
Violence naturally engenders violence. The spirit of Protestantism was
therefore far fiercer and more intolerant after the cruelties of Mary
than before them. Many persons who were warmly attached to the new
opinions had, during the evil days, taken refuge in Switzerland and
Germany. They had been hospitably received by their brethren in the
faith, had sate at the feet of the great doctors of Strasburg, Zurich,
and Geneva, and had been, during some years, accustomed to a more simple
worship, and to a more democratical form of church government, than
England had yet seen. These men returned to their country convinced that
the reform which had been effected under King Edward had been far less
searching and extensive than the interests of pure religion required.
But it was in vain that they attempted to obtain any concession from
Elizabeth. Indeed her system, wherever it differed from her brother's,
seemed to them to differ for the worse. They were little disposed to
submit, in matters of faith, to any human authority. They had recently,
in reliance on their own interpretation of Scripture, risen up against a
Church strong in immemorial antiquity and catholic consent. It was by no
common exertion of intellectual energy that they had thrown off the yoke
of that gorgeous and imperial superstition; and it was vain to expect
that, immediately after such an emancipation, they would patiently
submit to a new spiritual tyranny. Long accustomed, when the priest
lifted up the host, to bow down with their faces to the earth, as before
a present God, they had learned to treat the mass as an idolatrous
mummery. Long accustomed to regard the Pope as the successor of the
chief of the apostles, as the bearer of the keys of earth and heaven,
they had learned to regard him as the Beast, the Antichrist, the Man of
Sin. It was not to be expected that they would immediately transfer
to an upstart authority the homage which they had withdrawn from the
Vatican; that they would submit their private judgment to the authority
of a Church founded on private judgment alone; that they would be afraid
to dissent from teachers who themselves dissented from what had lately
been the universal faith of western Christendom. It is easy to conceive
the indignation which must have been felt by bold and inquisitive
spirits, glorying in newly acquired freedom, when an institution younger
by many years than themselves, an institution which had, under their own
eyes, gradually received its form from the passions and interest of a
court, began to mimic the lofty style of Rome.

Since these men could not be convinced, it was determined that they
should be persecuted. Persecution produced its natural effect on them.
It found them a sect: it made them a faction. To their hatred of the
Church was now added hatred of the Crown. The two sentiments were
intermingled; and each embittered the other. The opinions of the Puritan
concerning the relation of ruler and subject were widely different from
those which were inculcated in the Homilies. His favourite divines had,
both by precept and by example, encouraged resistance to tyrants
and persecutors. His fellow Calvinists in France, in Holland, and
in Scotland, were in arms against idolatrous and cruel princes. His
notions, too, respecting, the government of the state took a tinge
from his notions respecting the government of the Church. Some of the
sarcasms which were popularly thrown on episcopacy might, without much
difficulty, be turned against royalty; and many of the arguments which
were used to prove that spiritual power was best lodged in a synod
seemed to lead to the conclusion that temporal power was best lodged in
a parliament.

Thus, as the priest of the Established Church was, from interest, from
principle, and from passion, zealous for the royal prerogatives, the
Puritan was, from interest, from principle, and from passion, hostile to
them. The power of the discontented sectaries was great. They were found
in every rank; but they were strongest among the mercantile classes in
the towns, and among the small proprietors in the country. Early in
the reign of Elizabeth they began to return a majority of the House of
Commons. And doubtless had our ancestors been then at liberty to fix
their attention entirely on domestic questions, the strife between the
Crown and the Parliament would instantly have commenced. But that was
no season for internal dissensions. It might, indeed, well be doubted
whether the firmest union among all the orders of the state could avert
the common danger by which all were threatened. Roman Catholic Europe
and reformed Europe were struggling for death or life. France divided
against herself, had, for a time, ceased to be of any account in
Christendom. The English Government was at the head of the Protestant
interest, and, while persecuting Presbyterians at home, extended a
powerful protection to Presbyterian Churches abroad. At the head of the
opposite party was the mightiest prince of the age, a prince who ruled
Spain, Portugal, Italy, the East and the West Indies, whose armies
repeatedly marched to Paris, and whose fleets kept the coasts of
Devonshire and Sussex in alarm. It long seemed probable that Englishmen
would have to fight desperately on English ground for their religion and
independence. Nor were they ever for a moment free from apprehensions
of some great treason at home. For in that age it had become a point of
conscience and of honour with many men of generous natures to sacrifice
their country to their religion. A succession of dark plots, formed by
Roman Catholics against the life of the Queen and the existence of the
nation, kept society in constant alarm. Whatever might be the faults of
Elizabeth, it was plain that, to speak humanly, the fate of the realm
and of all reformed Churches was staked on the security of her person
and on the success of her administration. To strengthen her hands was,
therefore, the first duty of a patriot and a Protestant; and that duty
was well performed. The Puritans, even in the depths of the prisons to
which she had sent them, prayed, and with no simulated fervour, that she
might be kept from the dagger of the assassin, that rebellion might be
put down under her feet, and that her arms might be victorious by sea
and land. One of the most stubborn of the stubborn sect, immediately
after his hand had been lopped off for an offence into which he had been
hurried by his intemperate zeal, waved his hat with the hand which was
still left him, and shouted "God save the Queen!" The sentiment with
which these men regarded her has descended to their posterity. The
Nonconformists, rigorously as she treated them, have, as a body, always
venerated her memory. 
%[5]
\footnote{ The Puritan historian, Neal, after censuring the cruelty
with which she treated the sect to which he belonged, concludes thus:
"However, notwithstanding all these blemishes, Queen Elizabeth stands
upon record as a wise and politic princess, for delivering her kingdom
from the difficulties in which it was involved at her accession, for
preserving the Protestant reformation against the potent attempts of the
Pope, the Emperor, and King of Spain abroad, and the Queen of Scots and
her Popish subjects at home.... She was the glory of the age in which
she lived, and will be the admiration of posterity."--History of the
Puritans, Part I. Chap. viii.}


During the greater part of her reign, therefore, the Puritans in the
House of Commons, though sometimes mutinous, felt no disposition to
array themselves in systematic opposition to the government. But,
when the defeat of the Armada, the successful resistance of the United
Provinces to the Spanish power, the firm establishment of Henry the
Fourth on the throne of France, and the death of Philip the Second,
had secured the State and the Church against all danger from abroad,
an obstinate struggle, destined to last during several generations,
instantly began at home.

It was in the Parliament of 1601 that the opposition which had, during
forty years, been silently gathering and husbanding strength, fought
its first great battle and won its first victory. The ground was well
chosen. The English Sovereigns had always been entrusted with the
supreme direction of commercial police. It was their undoubted
prerogative to regulate coin, weights, and measures, and to appoint
fairs, markets, and ports. The line which bounded their authority over
trade had, as usual, been but loosely drawn. They therefore, as usual,
encroached on the province which rightfully belonged to the legislature.
The encroachment was, as usual, patiently borne, till it became serious.
But at length the Queen took upon herself to grant patents of monopoly
by scores. There was scarcely a family in the realm which did not
feel itself aggrieved by the oppression and extortion which this abuse
naturally caused. Iron, oil, vinegar, coal, saltpetre, lead, starch,
yarn, skins, leather, glass, could be bought only at exorbitant prices.
The House of Commons met in an angry and determined mood. It was in vain
that a courtly minority blamed the Speaker for suffering the acts of
the Queen's Highness to be called in question. The language of the
discontented party was high and menacing, and was echoed by the voice
of the whole nation. The coach of the chief minister of the crown was
surrounded by an indignant populace, who cursed the monopolies, and
exclaimed that the prerogative should not be suffered to touch the old
liberties of England. There seemed for a moment to be some danger that
the long and glorious reign of Elizabeth would have a shameful and
disastrous end. She, however, with admirable judgment and temper,
declined the contest, put herself at the head of the reforming party,
redressed the grievance, thanked the Commons, in touching and dignified
language, for their tender care of the general weal, brought back to
herself the hearts of the people, and left to her successors a memorable
example of the way in which it behoves a ruler to deal with public
movements which he has not the means of resisting.

In the year 1603 the great Queen died. That year is, on many accounts,
one of the most important epochs in our history. It was then that both
Scotland and Ireland became parts of the same empire with England. Both
Scotland and Ireland, indeed, had been subjugated by the Plantagenets;
but neither country had been patient under the yoke. Scotland had, with
heroic energy, vindicated her independence, had, from the time of Robert
Bruce, been a separate kingdom, and was now joined to the southern
part of the island in a manner which rather gratified than wounded her
national pride. Ireland had never, since the days of Henry the Second,
been able to expel the foreign invaders; but she had struggled against
them long and fiercely. During the fourteenth and fifteenth centuries
the English power in that island was constantly declining, and in the
days of Henry the Seventh, sank to the lowest point. The Irish dominions
of that prince consisted only of the counties of Dublin and Louth, of
some parts of Meath and Kildare, and of a few seaports scattered along
the coast. A large portion even of Leinster was not yet divided into
counties. Munster, Ulster, and Connaught were ruled by petty sovereigns,
partly Celts, and partly degenerate Normans, who had forgotten their
origin and had adopted the Celtic language and manners. But during the
sixteenth century, the English power had made great progress. The half
savage chieftains who reigned beyond the pale had submitted one after
another to the lieutenants of the Tudors. At length, a few weeks before
the death of Elizabeth, the conquest, which had been begun more than
four hundred years before by Strongbow, was completed by Mountjoy.
Scarcely had James the First mounted the English throne when the last
O'Donnel and O'Neil who have held the rank of independent princes kissed
his hand at Whitehall. Thenceforward his writs ran and his judges held
assizes in every part of Ireland; and the English law superseded the
customs which had prevailed among the aboriginal tribes.

In extent Scotland and Ireland were nearly equal to each other, and were
together nearly equal to England, but were much less thickly
peopled than England, and were very far behind England in wealth and
civilisation. Scotland had been kept back by the sterility of her soil;
and, in the midst of light, the thick darkness of the middle ages still
rested on Ireland.

The population of Scotland, with the exception of the Celtic tribes
which were thinly scattered over the Hebrides and over the mountainous
parts of the northern shires, was of the same blood with the population
of England, and spoke a tongue which did not differ from the purest
English more than the dialects of Somersetshire and Lancashire differed
from each other. In Ireland, on the contrary, the population, with the
exception of the small English colony near the coast, was Celtic, and
still kept the Celtic speech and manners.

In natural courage and intelligence both the nations which now became
connected with England ranked high. In perseverance, in selfcommand, in
forethought, in all the virtues which conduce to success in life, the
Scots have never been surpassed. The Irish, on the other hand, were
distinguished by qualities which tend to make men interesting rather
than prosperous. They were an ardent and impetuous race, easily moved
to tears or to laughter, to fury or to love. Alone among the nations of
northern Europe they had the susceptibility, the vivacity, the natural
turn for acting and rhetoric, which are indigenous on the shores of the
Mediterranean Sea. In mental cultivation Scotland had an indisputable
superiority. Though that kingdom was then the poorest in Christendom,
it already vied in every branch of learning with the most favoured
countries. Scotsmen, whose dwellings and whose food were as wretched as
those of the Icelanders of our time, wrote Latin verse with more than
the delicacy of Vida, and made discoveries in science which would have
added to the renown of Galileo. Ireland could boast of no Buchanan or
Napier. The genius, with which her aboriginal inhabitants were largely
endowed' showed itself as yet only in ballads which wild and rugged as
they were, seemed to the judging eye of Spenser to contain a portion of
the pure gold of poetry.

Scotland, in becoming part of the British monarchy, preserved her
dignity. Having, during many generations, courageously withstood the
English arms, she was now joined to her stronger neighbour on the most
honourable terms. She gave a King instead of receiving one. She retained
her own constitution and laws. Her tribunals and parliaments remained
entirely independent of the tribunals and parliaments which sate at
Westminster. The administration of Scotland was in Scottish hands; for
no Englishman had any motive to emigrate northward, and to contend with
the shrewdest and most pertinacious of all races for what was to be
scraped together in the poorest of all treasuries. Nevertheless Scotland
by no means escaped the fate ordained for every country which is
connected, but not incorporated, with another country of greater
resources. Though in name an independent kingdom, she was, during more
than a century, really treated, in many respects, as a subject province.

Ireland was undisguisedly governed as a dependency won by the sword. Her
rude national institutions had perished. The English colonists submitted
to the dictation of the mother country, without whose support they could
not exist, and indemnified themselves by trampling on the people among
whom they had settled. The parliaments which met at Dublin could pass no
law which had not been previously approved by the English Privy Council.
The authority of the English legislature extended over Ireland. The
executive administration was entrusted to men taken either from England
or from the English pale, and, in either case, regarded as foreigners,
and even as enemies, by the Celtic population.

But the circumstance which, more than any other, has made Ireland to
differ from Scotland remains to be noticed. Scotland was Protestant. In
no part of Europe had the movement of the popular mind against the Roman
Catholic Church been so rapid and violent. The Reformers had vanquished,
deposed, and imprisoned their idolatrous sovereign. They would not
endure even such a compromise as had been effected in England. They had
established the Calvinistic doctrine, discipline, and worship; and they
made little distinction between Popery and Prelacy, between the Mass and
the Book of Common Prayer. Unfortunately for Scotland, the prince whom
she sent to govern a fairer inheritance had been so much annoyed by
the pertinacity with which her theologians had asserted against him the
privileges of the synod and the pulpit that he hated the ecclesiastical
polity to which she was fondly attached as much as it was in his
effeminate nature to hate anything, and had no sooner mounted the
English throne than he began to show an intolerant zeal for the
government and ritual of the English Church.

The Irish were the only people of northern Europe who had remained true
to the old religion. This is to be partly ascribed to the circumstance
that they were some centuries behind their neighbours in knowledge. But
other causes had cooperated. The Reformation had been a national as well
as a moral revolt. It had been, not only an insurrection of the laity
against the clergy, but also an insurrection of all the branches of the
great German race against an alien domination. It is a most significant
circumstance that no large society of which the tongue is not Teutonic
has ever turned Protestant, and that, wherever a language derived from
that of ancient Rome is spoken, the religion of modern Rome to this day
prevails. The patriotism of the Irish had taken a peculiar direction.
The object of their animosity was not Rome, but England; and they had
especial reason to abhor those English sovereigns who had been the
chiefs of the great schism, Henry the Eighth and Elizabeth. During
the vain struggle which two generations of Milesian princes maintained
against the Tudors, religious enthusiasm and national enthusiasm became
inseparably blended in the minds of the vanquished race. The new feud
of Protestant and Papist inflamed the old feud of Saxon and Celt.
The English conquerors meanwhile, neglected all legitimate means of
conversion. No care was taken to provide the vanquished nation with
instructors capable of making themselves understood. No translation of
the Bible was put forth in the Irish language. The government contented
itself with setting up a vast hierarchy of Protestant archbishops,
bishops, and rectors, who did nothing, and who, for doing nothing, were
paid out of the spoils of a Church loved and revered by the great body
of the people.

There was much in the state both of Scotland and of Ireland which might
well excite the painful apprehensions of a farsighted statesman. As yet,
however, there was the appearance of tranquillity. For the first time
all the British isles were peaceably united under one sceptre.

It should seem that the weight of England among European nations ought,
from this epoch, to have greatly increased. The territory which her new
King governed was, in extent, nearly double that which Elizabeth had
inherited. His empire was the most complete within itself and the most
secure from attack that was to be found in the world. The Plantagenets
and Tudors had been repeatedly under the necessity of defending
themselves against Scotland while they were engaged in continental war.
The long conflict in Ireland had been a severe and perpetual drain on
their resources. Yet even under such disadvantages those sovereigns had
been highly considered throughout Christendom. It might, therefore, not
unreasonably be expected that England, Scotland, and Ireland combined
would form a state second to none that then existed.

All such expectations were strangely disappointed. On the day of the
accession of James the First, England descended from the rank which she
had hitherto held, and began to be regarded as a power hardly of the
second order. During many years the great British monarchy, under four
successive princes of the House of Stuart, was scarcely a more important
member of the European system than the little kingdom of Scotland had
previously been. This, however, is little to be regretted. Of James the
First, as of John, it may be said that, if his administration had been
able and splendid, it would probably have been fatal to our country,
and that we owe more to his weakness and meanness than to the wisdom and
courage of much better sovereigns. He came to the throne at a critical
moment. The time was fast approaching when either the King must
become absolute, or the parliament must control the whole executive
administration. Had James been, like Henry the Fourth, like Maurice of
Nassau, or like Gustavus Adolphus, a valiant, active, and politic ruler,
had he put himself at the head of the Protestants of Europe, had
he gained great victories over Tilly and Spinola, had he adorned
Westminster with the spoils of Bavarian monasteries and Flemish
cathedrals, had he hung Austrian and Castilian banners in Saint Paul's,
and had he found himself, after great achievements, at the head of fifty
thousand troops, brave, well disciplined, and devotedly attached to his
person, the English Parliament would soon have been nothing more than
a name. Happily he was not a man to play such a part. He began his
administration by putting an end to the war which had raged during
many years between England and Spain; and from that time he shunned
hostilities with a caution which was proof against the insults of his
neighbours and the clamours of his subjects. Not till the last year of
his life could the influence of his son, his favourite, his Parliament,
and his people combined, induce him to strike one feeble blow in
defence of his family and of his religion. It was well for those whom he
governed that he in this matter disregarded their wishes. The effect of
his pacific policy was that, in his time, no regular troops were needed,
and that, while France, Spain, Italy, Belgium, and Germany swarmed with
mercenary soldiers, the defence of our island was still confided to the
militia.

As the King had no standing army, and did not even attempt to form one,
it would have been wise in him to avoid any conflict with his people.
But such was his indiscretion that, while he altogether neglected the
means which alone could make him really absolute, he constantly put
forward, in the most offensive form, claims of which none of his
predecessors had ever dreamed. It was at this time that those strange
theories which Filmer afterwards formed into a system and which became
the badge of the most violent class of Tories and high churchmen, first
emerged into notice. It was gravely maintained that the Supreme Being
regarded hereditary monarchy, as opposed to other forms of government,
with peculiar favour; that the rule of succession in order of
primogeniture was a divine institution, anterior to the Christian, and
even to the Mosaic dispensation; that no human power, not even that
of the whole legislature, no length of adverse possession, though it
extended to ten centuries, could deprive a legitimate prince of his
rights, that the authority of such a prince was necessarily always
despotic; that the laws, by which, in England and in other countries,
the prerogative was limited, were to be regarded merely as concessions
which the sovereign had freely made and might at his pleasure resume;
and that any treaty which a king might conclude with his people was
merely a declaration of his present intentions, and not a contract of
which the performance could be demanded. It is evident that this theory,
though intended to strengthen the foundations of government, altogether
unsettles them. Does the divine and immutable law of primogeniture admit
females, or exclude them? On either supposition half the sovereigns of
Europe must be usurpers, reigning in defiance of the law of God, and
liable to be dispossessed by the rightful heirs. The doctrine that
kingly government is peculiarly favoured by Heaven receives no
countenance from the Old Testament; for in the Old Testament we read
that the chosen people were blamed and punished for desiring a king, and
that they were afterwards commanded to withdraw their allegiance
from him. Their whole history, far from countenancing the notion that
succession in order of primogeniture is of divine institution, would
rather seem to indicate that younger brothers are under the especial
protection of heaven. Isaac was not the eldest son of Abraham, nor Jacob
of Isaac, nor Judah of Jacob, nor David of Jesse nor Solomon of David
Nor does the system of Filmer receive any countenance from those
passages of the New Testament which describe government as an ordinance
of God: for the government under which the writers of the New Testament
lived was not a hereditary monarchy. The Roman Emperors were republican
magistrates, named by the senate. None of them pretended to rule by
right of birth; and, in fact, both Tiberius, to whom Christ commanded
that tribute should be given, and Nero, whom Paul directed the Romans to
obey, were, according to the patriarchal theory of government, usurpers.
In the middle ages the doctrine of indefeasible hereditary right would
have been regarded as heretical: for it was altogether incompatible with
the high pretensions of the Church of Rome. It was a doctrine unknown
to the founders of the Church of England. The Homily on Wilful Rebellion
had strongly, and indeed too strongly, inculcated submission to
constituted authority, but had made no distinction between hereditary
end elective monarchies, or between monarchies and republics. Indeed
most of the predecessors of James would, from personal motives, have
regarded the patriarchal theory of government with aversion. William
Rufus, Henry the First, Stephen, John, Henry the Fourth, Henry the
Fifth, Henry the Sixth, Richard the Third, and Henry the Seventh, had
all reigned in defiance of the strict rule of descent. A grave
doubt hung over the legitimacy both of Mary and of Elizabeth. It was
impossible that both Catharine of Aragon and Anne Boleyn could have been
lawfully married to Henry the Eighth; and the highest authority in
the realm had pronounced that neither was so. The Tudors, far from
considering the law of succession as a divine and unchangeable
institution, were constantly tampering with it. Henry the Eighth
obtained an act of parliament, giving him power to leave the crown by
will, and actually made a will to the prejudice of the royal family
of Scotland. Edward the Sixth, unauthorised by Parliament, assumed a
similar power, with the full approbation of the most eminent Reformers.
Elizabeth, conscious that her own title was open to grave objection, and
unwilling to admit even a reversionary right in her rival and enemy
the Queen of Scots, induced the Parliament to pass a law, enacting that
whoever should deny the competency of the reigning sovereign, with the
assent of the Estates of the realm, to alter the succession, should
suffer death as a traitor: But the situation of James was widely
different from that of Elizabeth. Far inferior to her in abilities and
in popularity, regarded by the English as an alien, and excluded from
the throne by the testament of Henry the Eighth, the King of Scots was
yet the undoubted heir of William the Conqueror and of Egbert. He had,
therefore, an obvious interest in inculcating the superstitions notion
that birth confers rights anterior to law, and unalterable by law. It
was a notion, moreover, well suited to his intellect and temper. It soon
found many advocates among those who aspired to his favour, and made
rapid progress among the clergy of the Established Church.

Thus, at the very moment at which a republican spirit began to manifest
itself strongly in the Parliament and in the country, the claims of the
monarch took a monstrous form which would have disgusted the proudest
and most arbitrary of those who had preceded him on the throne.

James was always boasting of his skill in what he called kingcraft; and
yet it is hardly possible even to imagine a course more directly opposed
to all the rules of kingcraft, than that which he followed. The policy
of wise rulers has always been to disguise strong acts under popular
forms. It was thus that Augustus and Napoleon established absolute
monarchies, while the public regarded them merely as eminent citizens
invested with temporary magistracies. The policy of James was the direct
reverse of theirs. He enraged and alarmed his Parliament by constantly
telling them that they held their privileges merely during his pleasure
and that they had no more business to inquire what he might lawfully
do than what the Deity might lawfully do. Yet he quailed before them,
abandoned minister after minister to their vengeance, and suffered them
to tease him into acts directly opposed to his strongest inclinations.
Thus the indignation excited by his claims and the scorn excited by
his concessions went on growing together. By his fondness for worthless
minions, and by the sanction which he gave to their tyranny and
rapacity, he kept discontent constantly alive. His cowardice, his
childishness, his pedantry, his ungainly person, his provincial accent,
made him an object of derision. Even in his virtues and accomplishments
there was something eminently unkingly. Throughout the whole course of
his reign, all the venerable associations by which the throng had long
been fenced were gradually losing their strength. During two hundred
years all the sovereigns who had ruled England, with the exception of
Henry the Sixth, had been strongminded, highspirited, courageous, and of
princely bearing. Almost all had possessed abilities above the ordinary
level. It was no light thing that on the very eve of the decisive
struggle between our Kings and their Parliaments, royalty should be
exhibited to the world stammering, slobbering, shedding unmanly tears,
trembling at a drawn sword, and talking in the style alternately of a
buffoon and of a pedagogue.

In the meantime the religious dissensions, by which, from the days of
Edward the Sixth, the Protestant body had been distracted, had become
more formidable than ever. The interval which had separated the first
generation of Puritans from Cranmer and Jewel was small indeed when
compared with the interval which separated the third generation of
Puritans from Laud and Hammond. While the recollection of Mary's
cruelties was still fresh, while the powers of the Roman Catholic party
still inspired apprehension, while Spain still retained ascendency and
aspired to universal dominion, all the reformed sects knew that they had
a strong common interest and a deadly common enemy. The animosity
which they felt towards each other was languid when compared with
the animosity which they all felt towards Rome. Conformists and
Nonconformists had heartily joined in enacting penal laws of extreme
severity against the Papists. But when more than half a century of
undisturbed possession had given confidence to the Established Church,
when nine tenths of the nation had become heartily Protestant, when
England was at peace with all the world, when there was no danger that
Popery would be forced by foreign arms on the nation, when the last
confessors who had stood before Bonner had passed away, a change took
place in the feeling of the Anglican clergy. Their hostility to the
Roman Catholic doctrine and discipline was considerably mitigated.
Their dislike of the Puritans, on the other hand, increased daily. The
controversies which had from the beginning divided the Protestant party
took such a form as made reconciliation hopeless; and new controversies
of still greater importance were added to the old subjects of dispute.

The founders of the Anglican Church had retained episcopacy as an
ancient, a decent, and a convenient ecclesiastical polity, but had not
declared that form of church government to be of divine institution. We
have already seen how low an estimate Cranmer had formed of the office
of a Bishop. In the reign of Elizabeth, Jewel, Cooper, Whitgift, and
other eminent doctors defended prelacy, as innocent, as useful, as what
the state might lawfully establish, as what, when established by the
state, was entitled to the respect of every citizen. But they never
denied that a Christian community without a Bishop might be a pure
Church. 
%[6]
\footnote{ On this subject, Bishop Cooper's language is remarkably
clear and strong. He maintains, in his Answer to Martin Marprelate,
printed in 1589, "that no form of church government is divinely
ordained; that Protestant communities, in establishing different forms,
have only made a legitimate use of their Christian liberty; and
that episcopacy is peculiarly suited to England, because the English
constitution is monarchical." "All those Churches," says the Bishop,
"in which the Gospell, in these daies, after great darknesse, was first
renewed, and the learned men whom God sent to instruct them, I doubt not
but have been directed by the Spirite of God to retaine this liberty,
that, in external government and other outward orders; they might choose
such as they thought in wisedome and godlinesse to be most convenient
for the state of their countrey and disposition of their people. Why
then should this liberty that other countreys have used under anie
colour be wrested from us? I think it therefore great presumption and
boldnesse that some of our nation, and those, whatever they may think
of themselves, not of the greatest wisedome and skill, should take upon
them to controlle the whole realme, and to binde both prince and people
in respect of conscience to alter the present state, and tie themselves
to a certain platforme devised by some of our neighbours, which, in the
judgment of many wise and godly persons, is most unfit for the state of
a Kingdome."}
 On the contrary, they regarded the Protestants of the
Continent as of the same household of faith with themselves. Englishmen
in England were indeed bound to acknowledge the authority of the Bishop,
as they were bound to acknowledge the authority of the Sheriff and of
the Coroner: but the obligation was purely local. An English churchman,
nay even an English prelate, if he went to Holland, conformed without
scruple to the established religion of Holland. Abroad the ambassadors
of Elizabeth and James went in state to the very worship which Elizabeth
and James persecuted at home, and carefully abstained from decorating
their private chapels after the Anglican fashion, lest scandal should
be given to weaker brethren. An instrument is still extant by which the
Primate of all England, in the year 1582, authorised a Scotch minister,
ordained, according to the laudable forms of the Scotch Church, by the
Synod of East Lothian, to preach and administer the sacraments in
any part of the province of Canterbury. 
%[7]
\footnote{ Strype's Life of Grindal, Appendix to Book II. No. xvii.}
 In the year 1603, the
Convocation solemnly recognised the Church of Scotland, a Church in
which episcopal control and episcopal ordination were then unknown, as a
branch of the Holy Catholic Church of Christ. 
%[8]
\footnote{ Canon 55, of 1603.}
 It was even held that
Presbyterian ministers were entitled to place and voice in oecumenical
councils. When the States General of the United Provinces convoked at
Dort a synod of doctors not episcopally ordained, an English Bishop and
an English Dean, commissioned by the head of the English Church, sate
with those doctors, preached to them, and voted with them on the gravest
questions of theology. 
%[9]
\footnote{ Joseph Hall, then dean of Worcester, and afterwards bishop
of Norwich, was one of the commissioners. In his life of himself, he
says: "My unworthiness was named for one of the assistants of that
honourable, grave, and reverend meeting." To high churchmen this
humility will seem not a little out of place.}
 Nay, many English benefices were held by
divines who had been admitted to the ministry in the Calvinistic form
used on the Continent; nor was reordination by a Bishop in such cases
then thought necessary, or even lawful. 
%[10]
\footnote{ It was by the Act of Uniformity, passed after the
Restoration, that persons not episcopally ordained were, for the first
time, made incapable of holding benefices. No man was more zealous for
this law than Clarendon. Yet he says: "This was new; for there had been
many, and at present there were some, who possessed benefices with cure
of souls and other ecclesiastical promotions, who had never received
orders but in France or Holland; and these men must now receive new
ordination, which had been always held unlawful in the Church, or by
this act of parliament must be deprived of their livelihood which they
enjoyed in the most flourishing and peaceable time of the Church."}


But a new race of divines was already rising in the Church of England.
In their view the episcopal office was essential to the welfare of a
Christian society and to the efficacy of the most solemn ordinances of
religion. To that office belonged certain high and sacred privileges,
which no human power could give or take away. A church might as well be
without the doctrine of the Trinity, or the doctrine of the Incarnation,
as without the apostolical orders; and the Church of Rome, which, in the
midst of all her corruptions, had retained the apostolical orders,
was nearer to primitive purity than those reformed societies which had
rashly set up, in opposition to the divine model, a system invented by
men.

In the days of Edward the Sixth and of Elizabeth, the defenders of the
Anglican ritual had generally contented themselves with saying that it
might be used without sin, and that, therefore, none but a perverse and
undutiful subject would refuse to use it when enjoined to do so by the
magistrate. Now, however, that rising party which claimed for the polity
of the Church a celestial origin began to ascribe to her services a new
dignity and importance. It was hinted that, if the established worship
had any fault, that fault was extreme simplicity, and that the Reformers
had, in the heat of their quarrel with Rome, abolished many ancient
ceremonies which might with advantage have been retained. Days and
places were again held in mysterious veneration. Some practices which
had long been disused, and which were commonly regarded as superstitious
mummeries, were revived. Paintings and carvings, which had escaped the
fury of the first generation of Protestants, became the objects of a
respect such as to many seemed idolatrous.

No part of the system of the old Church had been more detested by the
Reformers than the honour paid to celibacy. They held that the doctrine
of Rome on this subject had been prophetically condemned by the apostle
Paul, as a doctrine of devils; and they dwelt much on the crimes and
scandals which seemed to prove the justice of this awful denunciation.
Luther had evinced his own opinion in the clearest manner, by espousing
a nun. Some of the most illustrious bishops and priests who had died by
fire during the reign of Mary had left wives and children. Now, however,
it began to be rumoured that the old monastic spirit had reappeared
in the Church of England; that there was in high quarters a prejudice
against married priests; that even laymen, who called themselves
Protestants, had made resolutions of celibacy which almost amounted
to vows; nay, that a minister of the established religion had set up a
nunnery, in which the psalms were chaunted at midnight, by a company of
virgins dedicated to God. 
%[11]
\footnote{ Peckard's Life of Ferrar; The Arminian Nunnery, or a Brief
Description of the late erected monastical Place called the Arminian
Nunnery, at Little Gidding in Huntingdonshire, 1641.}


Nor was this all. A class of questions, as to which the founders of the
Anglican Church and the first generation of Puritans had differed
little or not at all, began to furnish matter for fierce disputes. The
controversies which had divided the Protestant body in its infancy had
related almost exclusively to Church government and to ceremonies. There
had been no serious quarrel between the contending parties on points of
metaphysical theology. The doctrines held by the chiefs of the hierarchy
touching original sin, faith, grace, predestination, and election,
were those which are popularly called Calvinistic. Towards the close of
Elizabeth's reign her favourite prelate, Archbishop Whitgift, drew
up, in concert with the Bishop of London and other theologians, the
celebrated instrument known by the name of the Lambeth Articles. In that
instrument the most startling of the Calvinistic doctrines are affirmed
with a distinctness which would shock many who, in our age, are reputed
Calvinists. One clergyman, who took the opposite side, and spoke harshly
of Calvin, was arraigned for his presumption by the University of
Cambridge, and escaped punishment only by expressing his firm belief in
the tenets of reprobation and final perseverance, and his sorrow for
the offence which he had given to pious men by reflecting on the great
French reformer. The school of divinity of which Hooker was the chief
occupies a middle place between the school of Cranmer and the school of
Laud; and Hooker has, in modern times, been claimed by the Arminians
as an ally. Yet Hooker pronounced Calvin to have been a man superior
in wisdom to any other divine that France had produced, a man to whom
thousands were indebted for the knowledge of divine truth, but who was
himself indebted to God alone. When the Arminian controversy arose
in Holland, the English government and the English Church lent strong
support to the Calvinistic party; nor is the English name altogether
free from the stain which has been left on that party by the
imprisonment of Grocius and the judicial murder of Barneveldt.

But, even before the meeting of the Dutch synod, that part of the
Anglican clergy which was peculiarly hostile to the Calvinistic Church
government and to the Calvinistic worship had begun to regard with
dislike the Calvinistic metaphysics; and this feeling was very naturally
strengthened by the gross injustice, insolence, and cruelty of the party
which was prevalent at Dort. The Arminian doctrine, a doctrine less
austerely logical than that of the early Reformers, but more agreeable
to the popular notions of the divine justice and benevolence, spread
fast and wide. The infection soon reached the court. Opinions which
at the time of the accession of James, no clergyman could have avowed
without imminent risk of being stripped of his gown, were now the best
title to preferment. A divine of that age, who was asked by a simple
country gentleman what the Arminians held, answered, with as much truth
as wit, that they held all the best bishoprics and deaneries in England.

While the majority of the Anglican clergy quitted, in one direction, the
position which they had originally occupied, the majority of the
Puritan body departed, in a direction diametrically opposite, from the
principles and practices of their fathers. The persecution which the
separatists had undergone had been severe enough to irritate, but not
severe enough to destroy. They had been, not tamed into submission, but
baited into savageness and stubborness. After the fashion of oppressed
sects, they mistook their own vindictive feelings for emotions of piety,
encouraged in themselves by reading and meditation, a disposition to
brood over their wrongs, and, when they had worked themselves up into
hating their enemies, imagined that they were only hating the enemies
of heaven. In the New Testament there was little indeed which, even when
perverted by the most disingenuous exposition, could seem to countenance
the indulgence of malevolent passions. But the Old Testament contained
the history of a race selected by God to be witnesses of his unity and
ministers of his vengeance, and specially commanded by him to do many
things which, if done without his special command, would have been
atrocious crimes. In such a history it was not difficult for fierce
and gloomy spirits to find much that might be distorted to suit their
wishes. The extreme Puritans therefore began to feel for the Old
Testament a preference, which, perhaps, they did not distinctly avow
even to themselves; but which showed itself in all their sentiments and
habits. They paid to the Hebrew language a respect which they refused
to that tongue in which the discourses of Jesus and the epistles of Paul
have come down to us. They baptized their children by the names, not of
Christian saints, but of Hebrew patriarchs and warriors. In defiance
of the express and reiterated declarations of Luther and Calvin, they
turned the weekly festival by which the Church had, from the primitive
times, commemorated the resurrection of her Lord, into a Jewish Sabbath.
They sought for principles of jurisprudence in the Mosaic law, and for
precedents to guide their ordinary conduct in the books of Judges
and Kings. Their thoughts and discourse ran much on acts which were
assuredly not recorded as examples for our imitation. The prophet who
hewed in pieces a captive king, the rebel general who gave the blood of
a queen to the dogs, the matron who, in defiance of plighted faith, and
of the laws of eastern hospitality, drove the nail into the brain of the
fugitive ally who had just fed at her board, and who was sleeping under
the shadow of her tent, were proposed as models to Christians suffering
under the tyranny of princes and prelates. Morals and manners were
subjected to a code resembling that of the synagogue, when the synagogue
was in its worst state. The dress, the deportment, the language, the
studies, the amusements of the rigid sect were regulated on principles
not unlike those of the Pharisees who, proud of their washed hands and
broad phylacteries, taunted the Redeemer as a sabbath-breaker and a
winebibber. It was a sin to hang garlands on a Maypole, to drink a
friend's health, to fly a hawk, to hunt a stag, to play at chess, to
wear love-locks, to put starch into a ruff, to touch the virginals,
to read the Fairy Queen. Rules such as these, rules which would have
appeared insupportable to the free and joyous spirit of Luther, and
contemptible to the serene and philosophical intellect of Zwingle, threw
over all life a more than monastic gloom. The learning and eloquence by
which the great Reformers had been eminently distinguished, and to which
they had been, in no small measure, indebted for their success, were
regarded by the new school of Protestants with suspicion, if not with
aversion. Some precisians had scruples about teaching the Latin grammar,
because the names of Mars, Bacchus, and Apollo occurred in it. The
fine arts were all but proscribed. The solemn peal of the organ was
superstitious. The light music of Ben Jonson's masques was dissolute.
Half the fine paintings in England were idolatrous, and the other half
indecent. The extreme Puritan was at once known from other men by his
gait, his garb, his lank hair, the sour solemnity of his face, the
upturned white of his eyes, the nasal twang with which he spoke, and
above all, by his peculiar dialect. He employed, on every occasion, the
imagery and style of Scripture. Hebraisms violently introduced into the
English language, and metaphors borrowed from the boldest lyric poetry
of a remote age and country, and applied to the common concerns of
English life, were the most striking peculiarities of this cant,
which moved, not without cause, the derision both of Prelatists and
libertines.

Thus the political and religious schism which had originated in the
sixteenth century was, during the first quarter of the seventeenth
century, constantly widening. Theories tending to Turkish despotism
were in fashion at Whitehall. Theories tending to republicanism were
in favour with a large portion of the House of Commons. The violent
Prelatists who were, to a man, zealous for prerogative, and the violent
Puritans who were, to a man, zealous for the privileges of Parliament,
regarded each other with animosity more intense than that which, in the
preceding generation, had existed between Catholics and Protestants.

While the minds of men were in this state, the country, after a peace
of many years, at length engaged in a war which required strenuous
exertions. This war hastened the approach of the great constitutional
crisis. It was necessary that the King should have a large military
force. He could not have such a force without money. He could not
legally raise money without the consent of Parliament. It followed,
therefore, that he either must administer the government in conformity
with the sense of the House of Commons, or must venture on such a
violation of the fundamental laws of the land as had been unknown during
several centuries. The Plantagenets and the Tudors had, it is true,
occasionally supplied a deficiency in their revenue by a benevolence or
a forced loan: but these expedients were always of a temporary nature.
To meet the regular charge of a long war by regular taxation, imposed
without the consent of the Estates of the realm, was a course which
Henry the Eighth himself would not have dared to take. It seemed,
therefore, that the decisive hour was approaching, and that the English
Parliament would soon either share the fate of the senates of the
Continent, or obtain supreme ascendency in the state.

Just at this conjuncture James died. Charles the First succeeded to the
throne. He had received from nature a far better understanding, a far
stronger will, and a far keener and firmer temper than his father's.
He had inherited his father's political theories, and was much more
disposed than his father to carry them into practice. He was, like his
father, a zealous Episcopalian. He was, moreover, what his father had
never been, a zealous Arminian, and, though no Papist, liked a Papist
much better than a Puritan. It would be unjust to deny that Charles had
some of the qualities of a good, and even of a great prince. He wrote
and spoke, not, like his father, with the exactness of a professor, but
after the fashion of intelligent and well educated gentlemen. His taste
in literature and art was excellent, his manner dignified, though not
gracious, his domestic life without blemish. Faithlessness was the chief
cause of his disasters, and is the chief stain on his memory. He was, in
truth, impelled by an incurable propensity to dark and crooked ways.
It may seem strange that his conscience, which, on occasions of little
moment, was sufficiently sensitive, should never have reproached
him with this great vice. But there is reason to believe that he was
perfidious, not only from constitution and from habit, but also on
principle. He seems to have learned from the theologians whom he most
esteemed that between him and his subjects there could be nothing of the
nature of mutual contract; that he could not, even if he would, divest
himself of his despotic authority; and that, in every promise which he
made, there was an implied reservation that such promise might be broken
in case of necessity, and that of the necessity he was the sole judge.

And now began that hazardous game on which were staked the destinies of
the English people. It was played on the side of the House of Commons
with keenness, but with admirable dexterity, coolness, and perseverance.
Great statesmen who looked far behind them and far before them were at
the head of that assembly. They were resolved to place the King in such
a situation that he must either conduct the administration in conformity
with the wishes of his Parliament, or make outrageous attacks on the
most sacred principles of the constitution. They accordingly doled out
supplies to him very sparingly. He found that he must govern either in
harmony with the House of Commons or in defiance of all law. His choice
was soon made. He dissolved his first Parliament, and levied taxes by
his own authority. He convoked a second Parliament, and found it more
intractable than the first. He again resorted to the expedient of
dissolution, raised fresh taxes without any show of legal right, and
threw the chiefs of the opposition into prison At the same time a new
grievance, which the peculiar feelings and habits of the English nation
made insupportably painful, and which seemed to all discerning men to
be of fearful augury, excited general discontent and alarm. Companies
of soldiers were billeted on the people; and martial law was, in some
places, substituted for the ancient jurisprudence of the realm.

The King called a third Parliament, and soon perceived that the
opposition was stronger and fiercer than ever. He now determined on a
change of tactics. Instead of opposing an inflexible resistance to the
demands of the Commons, he, after much altercation and many evasions,
agreed to a compromise which, if he had faithfully adhered to it, would
have averted a long series of calamities. The Parliament granted
an ample supply. The King ratified, in the most solemn manner, that
celebrated law, which is known by the name of the Petition of Right,
and which is the second Great Charter of the liberties of England. By
ratifying that law he bound himself never again to raise money without
the consent of the Houses, never again to imprison any person, except
in due course of law, and never again to subject his people to the
jurisdiction of courts martial.

The day on which the royal sanction was, after many delays, solemnly
given to this great Act, was a day of joy and hope. The Commons,
who crowded the bar of the House of Lords, broke forth into loud
acclamations as soon as the clerk had pronounced the ancient form of
words by which our princes have, during many ages, signified their
assent to the wishes of the Estates of the realm. Those acclamations
were reechoed by the voice of the capital and of the nation; but
within three weeks it became manifest that Charles had no intention of
observing the compact into which he had entered. The supply given by the
representatives of the nation was collected. The promise by which that
supply had been obtained was broken. A violent contest followed. The
Parliament was dissolved with every mark of royal displeasure. Some of
the most distinguished members were imprisoned; and one of them, Sir
John Eliot, after years of suffering, died in confinement.

Charles, however, could not venture to raise, by his own authority,
taxes sufficient for carrying on war. He accordingly hastened to make
peace with his neighbours, and thenceforth gave his whole mind to
British politics.

Now commenced a new era. Many English Kings had occasionally committed
unconstitutional acts: but none had ever systematically attempted to
make himself a despot, and to reduce the Parliament to a nullity. Such
was the end which Charles distinctly proposed to himself. From March
1629 to April 1640, the Houses were not convoked. Never in our history
had there been an interval of eleven years between Parliament and
Parliament. Only once had there been an interval of even half that
length. This fact alone is sufficient to refute those who represent
Charles as having merely trodden in the footsteps of the Plantagenets
and Tudors.

It is proved, by the testimony of the King's most strenuous supporters,
that, during this part of his reign, the provisions of the Petition of
Right were violated by him, not occasionally, but constantly, and on
system; that a large part of the revenue was raised without any legal
authority; and that persons obnoxious to the government languished for
years in prison, without being ever called upon to plead before any
tribunal.

For these things history must hold the King himself chiefly responsible.
From the time of his third Parliament he was his own prime minister.
Several persons, however, whose temper and talents were suited to
his purposes, were at the head of different departments of the
administration.

Thomas Wentworth, successively created Lord Wentworth and Earl of
Strafford, a man of great abilities, eloquence, and courage, but of a
cruel and imperious nature, was the counsellor most trusted in political
and military affairs. He had been one of the most distinguished members
of the opposition, and felt towards those whom he had deserted that
peculiar malignity which has, in all ages, been characteristic of
apostates. He perfectly understood the feelings, the resources, and the
policy of the party to which he had lately belonged, and had formed a
vast and deeply meditated scheme which very nearly confounded even the
able tactics of the statesmen by whom the House of Commons had been
directed. To this scheme, in his confidential correspondence, he gave
the expressive name of Thorough. His object was to do in England all,
and more than all, that Richelieu was doing in France; to make Charles a
monarch as absolute as any on the Continent; to put the estates and the
personal liberty of the whole people at the disposal of the crown; to
deprive the courts of law of all independent authority, even in ordinary
questions of civil right between man and man; and to punish with
merciless rigour all who murmured at the acts of the government, or who
applied, even in the most decent and regular manner, to any tribunal for
relief against those acts. 
%[12]
\footnote{ The correspondence of Wentworth seems to me fully to bear
out what I have said in the text. To transcribe all the passages
which have led me to the conclusion at which I have arrived, would be
impossible, nor would it be easy to make a better selection than has
already been made by Mr. Hallam. I may, however direct the attention of
the reader particularly to the very able paper which Wentworth drew up
respecting the affairs of the Palatinate. The date is March 31, 1637.}


This was his end; and he distinctly saw in what manner alone this
end could be attained. There was, in truth, about all his notions a
clearness, a coherence, a precision, which, if he had not been pursuing
an object pernicious to his country and to his kind, would have justly
entitled him to high admiration. He saw that there was one instrument,
and only one, by which his vast and daring projects could be carried
into execution. That instrument was a standing army. To the forming of
such an army, therefore, he directed all the energy of his strong mind.
In Ireland, where he was viceroy, he actually succeeded in establishing
a military despotism, not only over the aboriginal population, but also
over the English colonists, and was able to boast that, in that island,
the King was as absolute as any prince in the whole world could be. 
%[13]
\footnote{ These are Wentworth's own words. See his letter to Laud,
dated Dec. 16, 1634.}


The ecclesiastical administration was, in the meantime, principally
directed by William Laud, Archbishop of Canterbury. Of all the prelates
of the Anglican Church, Laud had departed farthest from the principles
of the Reformation, and had drawn nearest to Rome. His theology was more
remote than even that of the Dutch Arminians from the theology of the
Calvinists. His passion for ceremonies, his reverence for holidays,
vigils, and sacred places, his ill concealed dislike of the marriage
of ecclesiastics, the ardent and not altogether disinterested zeal
with which he asserted the claims of the clergy to the reverence of the
laity, would have made him an object of aversion to the Puritans, even
if he had used only legal and gentle means for the attainment of his
ends. But his understanding was narrow; and his commerce with the world
had been small. He was by nature rash, irritable, quick to feel for his
own dignity, slow to sympathise with the sufferings of others, and prone
to the error, common in superstitious men, of mistaking his own peevish
and malignant moods for emotions of pious zeal. Under his direction
every corner of the realm was subjected to a constant and minute
inspection. Every little congregation of separatists was tracked out and
broken up. Even the devotions of private families could not escape the
vigilance of his spies. Such fear did his rigour inspire that the
deadly hatred of the Church, which festered in innumerable bosoms, was
generally disguised under an outward show of conformity. On the very eve
of troubles, fatal to himself and to his order, the Bishops of several
extensive dioceses were able to report to him that not a single
dissenter was to be found within their jurisdiction. 
%[14]
\footnote{ See his report to Charles for the year 1639.}


The tribunals afforded no protection to the subject against the civil
and ecclesiastical tyranny of that period. The judges of the common
law, holding their situations during the pleasure of the King, were
scandalously obsequious. Yet, obsequious as they were, they were less
ready and less efficient instruments of arbitrary power than a class of
courts, the memory of which is still, after the lapse of more than two
centuries, held in deep abhorrence by the nation. Foremost among
these courts in power and in infamy were the Star Chamber and the High
Commission, the former a political, the latter a religious inquisition.
Neither was a part of the old constitution of England. The Star Chamber
had been remodelled, and the High Commission created, by the Tudors. The
power which these boards had possessed before the accession of Charles
had been extensive and formidable, but had been small indeed when
compared with that which they now usurped. Guided chiefly by the violent
spirit of the primate, and free from the control of Parliament, they
displayed a rapacity, a violence, a malignant energy, which had been
unknown to any former age. The government was able through their
instrumentality, to fine, imprison, pillory, and mutilate without
restraint. A separate council which sate at York, under the presidency
of Wentworth, was armed, in defiance of law, by a pure act of
prerogative, with almost boundless power over the northern counties. All
these tribunals insulted and defied the authority of Westminster Hall,
and daily committed excesses which the most distinguished Royalists have
warmly condemned. We are informed by Clarendon that there was hardly
a man of note in the realm who had not personal experience of the
harshness and greediness of the Star Chamber, that the High Commission
had so conducted itself that it had scarce a friend left in the kingdom,
and that the tyranny of the Council of York had made the Great Charter a
dead letter on the north of the Trent.

The government of England was now, in all points but one, as despotic as
that of France. But that one point was all important. There was still no
standing army. There was therefore, no security that the whole fabric
of tyranny might not be subverted in a single day; and, if taxes were
imposed by the royal authority for the support of an army, it was
probable that there would be an immediate and irresistible explosion.
This was the difficulty which more than any other perplexed Wentworth.
The Lord Keeper Finch, in concert with other lawyers who were employed
by the government, recommended an expedient which was eagerly adopted.
The ancient princes of England, as they called on the inhabitants of the
counties near Scotland to arm and array themselves for the defence of
the border, had sometimes called on the maritime counties to furnish
ships for the defence of the coast. In the room of ships money had
sometimes been accepted. This old practice it was now determined, after
a long interval, not only to revive but to extend. Former princes had
raised shipmoney only in time of war: it was now exacted in a time of
profound peace. Former princes, even in the most perilous wars, had
raised shipmoney only along the coasts: it was now exacted from the
inland shires. Former princes had raised shipmoney only for the maritime
defence of the country: It was now exacted, by the admission of the
Royalists themselves. With the object, not of maintaining a navy, but
of furnishing the King with supplies which might be increased at
his discretion to any amount, and expended at his discretion for any
purpose.

The whole nation was alarmed and incensed. John Hampden, an opulent and
well born gentleman of Buckinghamshire, highly considered in his own
neighbourhood, but as yet little known to the kingdom generally, had the
courage to step forward, to confront the whole power of the government,
and take on himself the cost and the risk of disputing the prerogative
to which the King laid claim. The case was argued before the judges
in the Exchequer Chamber. So strong were the arguments against the
pretensions of the crown that, dependent and servile as the judges were,
the majority against Hampden was the smallest possible. Still there was
a majority. The interpreters of the law had pronounced that one great
and productive tax might be imposed by the royal authority. Wentworth
justly observed that it was impossible to vindicate their judgment
except by reasons directly leading to a conclusion which they had not
ventured to draw. If money might legally be raised without the consent
of Parliament for the support of a fleet, it was not easy to deny that
money might, without consent of Parliament, be legally raised for the
support of an army.

The decision of the judges increased the irritation of the people. A
century earlier, irritation less serious would have produced a general
rising. But discontent did not now so readily as in an earlier age take
the form of rebellion. The nation had been long steadily advancing in
wealth and in civilisation. Since the great northern Earls took up arms
against Elizabeth seventy years had elapsed; and during those seventy
years there had been no civil war. Never, during the whole existence
of the English nation, had so long a period passed without intestine
hostilities. Men had become accustomed to the pursuits of peaceful
industry, and, exasperated as they were, hesitated long before they drew
the sword.

This was the conjuncture at which the liberties of the nation were in
the greatest peril. The opponents of the government began to despair of
the destiny of their country; and many looked to the American wilderness
as the only asylum in which they could enjoy civil and spiritual
freedom. There a few resolute Puritans, who, in the cause of their
religion, feared neither the rage of the ocean nor the hardships of
uncivilised life, neither the fangs of savage beasts nor the tomahawks
of more savage men, had built, amidst the primeval forests, villages
which are now great and opulent cities, but which have, through
every change, retained some trace of the character derived from their
founders. The government regarded these infant colonies with aversion,
and attempted violently to stop the stream of emigration, but could not
prevent the population of New England from being largely recruited by
stouthearted and Godfearing men from every part of the old England.
And now Wentworth exulted in the near prospect of Thorough. A few years
might probably suffice for the execution of his great design. If
strict economy were observed, if all collision with foreign powers were
carefully avoided, the debts of the crown would be cleared off: there
would be funds available for the support of a large military force; and
that force would soon break the refractory spirit of the nation.

At this crisis an act of insane bigotry suddenly changed the whole
face of public affairs. Had the King been wise, he would have pursued a
cautious and soothing policy towards Scotland till he was master in the
South. For Scotland was of all his kingdoms that in which there was the
greatest risk that a spark might produce a flame, and that a flame might
become a conflagration. Constitutional opposition, indeed, such as he
had encountered at Westminster, he had not to apprehend at Edinburgh.
The Parliament of his northern kingdom was a very different body from
that which bore the same name in England. It was ill constituted: it was
little considered; and it had never imposed any serious restraint on
any of his predecessors. The three Estates sate in one house. The
commissioners of the burghs were considered merely as retainers of the
great nobles. No act could be introduced till it had been approved by
the Lords of Articles, a committee which was really, though not in
form, nominated by the crown. But, though the Scottish Parliament was
obsequious, the Scottish people had always been singularly turbulent and
ungovernable. They had butchered their first James in his bedchamber:
they had repeatedly arrayed themselves in arms against James the
Second; they had slain James the Third on the field of battle: their
disobedience had broken the heart of James the Fifth: they had deposed
and imprisoned Mary: they had led her son captive; and their temper was
still as intractable as ever. Their habits were rude and martial. All
along the southern border, and all along the line between the highlands
and the lowlands, raged an incessant predatory war. In every part of the
country men were accustomed to redress their wrongs by the strong hand.
Whatever loyalty the nation had anciently felt to the Stuarts had cooled
during their long absence. The supreme influence over the public mind
was divided between two classes of malecontents, the lords of the soil
and the preachers; lords animated by the same spirit which had often
impelled the old Douglasses to withstand the royal house, and preachers
who had inherited the republican opinions and the unconquerable spirit
of Knox. Both the national and religious feelings of the population
had been wounded. All orders of men complained that their country, that
country which had, with so much glory, defended her independence against
the ablest and bravest Plantagenets, had, through the instrumentality of
her native princes, become in effect, though not in name, a province
of England. In no part of Europe had the Calvinistic doctrine and
discipline taken so strong a hold on the public mind. The Church of Rome
was regarded by the great body of the people with a hatred which might
justly be called ferocious; and the Church of England, which seemed
to be every day becoming more and more like the Church of Rome, was an
object of scarcely less aversion.

The government had long wished to extend the Anglican system over the
whole island, and had already, with this view, made several changes
highly distasteful to every Presbyterian. One innovation, however, the
most hazardous of all, because it was directly cognisable by the senses
of the common people, had not yet been attempted. The public worship
of God was still conducted in the manner acceptable to the nation. Now,
however, Charles and Laud determined to force on the Scots the English
liturgy, or rather a liturgy which, wherever it differed from that of
England, differed, in the judgment of all rigid Protestants, for the
worse.

To this step, taken in the mere wantonness of tyranny, and in criminal
ignorance or more criminal contempt of public feeling, our country owes
her freedom. The first performance of the foreign ceremonies produced
a riot. The riot rapidly became a revolution. Ambition, patriotism,
fanaticism, were mingled in one headlong torrent. The whole nation was
in arms. The power of England was indeed, as appeared some years later,
sufficient to coerce Scotland: but a large part of the English people
sympathised with the religious feelings of the insurgents; and many
Englishmen who had no scruple about antiphonies and genuflexions, altars
and surplices, saw with pleasure the progress of a rebellion which
seemed likely to confound the arbitrary projects of the court, and to
make the calling of a Parliament necessary.

For the senseless freak which had produced these effects Wentworth
is not responsible. 
%[15]
\footnote{ See his letter to the Earl of Northumberland, dated July
30, 1638.}
 It had, in fact, thrown all his plans into
confusion. To counsel submission, however, was not in his nature. An
attempt was made to put down the insurrection by the sword: but the
King's military means and military talents were unequal to the task.
To impose fresh taxes on England in defiance of law, would, at this
conjuncture, have been madness. No resource was left but a Parliament;
and in the spring of 1640 a Parliament was convoked.

The nation had been put into good humour by the prospect of seeing
constitutional government restored, and grievances redressed. The new
House of Commons was more temperate and more respectful to the throne
than any which had sate since the death of Elizabeth. The moderation
of this assembly has been highly extolled by the most distinguished
Royalists and seems to have caused no small vexation and disappointment
to the chiefs of the opposition: but it was the uniform practice of
Charles, a practice equally impolitic and ungenerous, to refuse all
compliance with the desires of his people, till those desires
were expressed in a menacing tone. As soon as the Commons showed a
disposition to take into consideration the grievances under which
the country had suffered during eleven years, the King dissolved the
Parliament with every mark of displeasure.

Between the dissolution of this shortlived assembly and the meeting
of that ever memorable body known by the name of the Long Parliament,
intervened a few months, during which the yoke was pressed down more
severely than ever on the nation, while the spirit of the nation rose up
more angrily than ever against the yoke. Members of the House of Commons
were questioned by the Privy Council touching their parliamentary
conduct, and thrown into prison for refusing to reply. Shipmoney was
levied with increased rigour. The Lord Mayor and the Sheriffs of London
were threatened with imprisonment for remissness in collecting the
payments. Soldiers were enlisted by force. Money for their support was
exacted from their counties. Torture, which had always been illegal, and
which had recently been declared illegal even by the servile judges of
that age, was inflicted for the last time in England in the month of
May, 1610.

Everything now depended on the event of the King's military operations
against the Scots. Among his troops there was little of that feeling
which separates professional soldiers from the mass of a nation, and
attaches them to their leaders. His army, composed for the most part of
recruits, who regretted the plough from which they had been violently
taken, and who were imbued with the religious and political sentiments
then prevalent throughout the country, was more formidable to himself
than to the enemy. The Scots, encouraged by the heads of the English
opposition, and feebly resisted by the English forces, marched across
the Tweed and the Tyne, and encamped on the borders of Yorkshire.
And now the murmurs of discontent swelled into an uproar by which all
spirits save one were overawed.

But the voice of Strafford was still for Thorough; and he even, in this
extremity, showed a nature so cruel and despotic, that his own pikemen
were ready to tear him in pieces.

There was yet one last expedient which, as the King flattered himself,
might save him from the misery of facing another House of Commons. To
the House of Lords he was less averse. The Bishops were devoted to
him; and though the temporal peers were generally dissatisfied with
his administration, they were, as a class, so deeply interested in the
maintenance of order, and in the stability of ancient institutions, that
they were not likely to call for extensive reforms. Departing from
the uninterrupted practice of centuries, he called a Great Council
consisting of Lords alone. But the Lords were too prudent to assume the
unconstitutional functions with which he wished to invest them. Without
money, without credit, without authority even in his own camp, he
yielded to the pressure of necessity. The Houses were convoked; and the
elections proved that, since the spring, the distrust and hatred with
which the government was regarded had made fearful progress.

In November, 1640, met that renowned Parliament which, in spite of many
errors and disasters, is justly entitled to the reverence and
gratitude of all who, in any part of the world enjoy the blessings of
constitutional government.

During the year which followed, no very important division of opinion
appeared in the Houses. The civil and ecclesiastical administration
had, through a period of nearly twelve years, been so oppressive and so
unconstitutional that even those classes of which the inclinations
are generally on the side of order and authority were eager to promote
popular reforms and to bring the instruments of tyranny to justice. It
was enacted that no interval of more than three years should ever elapse
between Parliament and Parliament, and that, if writs under the Great
Seal were not issued at the proper time, the returning officers should,
without such writs, call the constituent bodies together for the choice
of representatives. The Star Chamber, the High Commission, the Council
of York were swept away. Men who, after suffering cruel mutilations, had
been confined in remote dungeons, regained their liberty. On the chief
ministers of the crown the vengeance of the nation was unsparingly
wreaked. The Lord Keeper, the Primate, the Lord Lieutenant were
impeached. Finch saved himself by flight. Laud was flung into the Tower.
Strafford was put to death by act of attainder. On the day on which this
act passed, the King gave his assent to a law by which he bound himself
not to adjourn, prorogue, or dissolve the existing Parliament without
its own consent.

After ten months of assiduous toil, the Houses, in September 1641,
adjourned for a short vacation; and the King visited Scotland. He with
difficulty pacified that kingdom by consenting, not only to relinquish
his plans of ecclesiastical reform, but even to pass, with a very bad
grace, an act declaring that episcopacy was contrary to the word of God.

The recess of the English Parliament lasted six weeks. The day on
which the Houses met again is one of the most remarkable epochs in our
history. From that day dates the corporate existence of the two great
parties which have ever since alternately governed the country. In one
sense, indeed, the distinction which then became obvious had always
existed, and always must exist. For it has its origin in diversities
of temper, of understanding, and of interest, which are found in all
societies, and which will be found till the human mind ceases to be
drawn in opposite directions by the charm of habit and by the charm of
novelty. Not only in politics but in literature, in art, in science,
in surgery and mechanics, in navigation and agriculture, nay, even in
mathematics, we find this distinction. Everywhere there is a class of
men who cling with fondness to whatever is ancient, and who, even when
convinced by overpowering reasons that innovation would be beneficial,
consent to it with many misgivings and forebodings. We find also
everywhere another class of men, sanguine in hope, bold in speculation,
always pressing forward, quick to discern the imperfections of whatever
exists, disposed to think lightly of the risks and inconveniences which
attend improvements and disposed to give every change credit for being
an improvement. In the sentiments of both classes there is something to
approve. But of both the best specimens will be found not far from the
common frontier. The extreme section of one class consists of bigoted
dotards: the extreme section of the other consists of shallow and
reckless empirics.

There can be no doubt that in our very first Parliaments might have been
discerned a body of members anxious to preserve, and a body eager to
reform. But, while the sessions of the legislature were short, these
bodies did not take definite and permanent forms, array themselves under
recognised leaders, or assume distinguishing names, badges, and war
cries. During the first months of the Long Parliament, the indignation
excited by many years of lawless oppression was so strong and
general that the House of Commons acted as one man. Abuse after
abuse disappeared without a struggle. If a small minority of the
representative body wished to retain the Star Chamber and the High
Commission, that minority, overawed by the enthusiasm and by the
numerical superiority of the reformers, contented itself with secretly
regretting institutions which could not, with any hope of success, be
openly defended. At a later period the Royalists found it convenient to
antedate the separation between themselves and their opponents, and
to attribute the Act which restrained the King from dissolving or
proroguing the Parliament, the Triennial Act, the impeachment of
the ministers, and the attainder of Strafford, to the faction which
afterwards made war on the King. But no artifice could be more
disingenuous. Every one of those strong measures was actively promoted
by the men who were afterward foremost among the Cavaliers. No
republican spoke of the long misgovernment of Charles more severely than
Colepepper. The most remarkable speech in favour of the Triennial Bill
was made by Digby. The impeachment of the Lord Keeper was moved by
Falkland. The demand that the Lord Lieutenant should be kept close
prisoner was made at the bar of the Lords by Hyde. Not till the law
attainting Strafford was proposed did the signs of serious disunion
become visible. Even against that law, a law which nothing but extreme
necessity could justify, only about sixty members of the House of
Commons voted. It is certain that Hyde was not in the minority, and that
Falkland not only voted with the majority, but spoke strongly for the
bill. Even the few who entertained a scruple about inflicting death by
a retrospective enactment thought it necessary to express the utmost
abhorrence of Strafford's character and administration.

But under this apparent concord a great schism was latent; and when,
in October, 1641, the Parliament reassembled after a short recess, two
hostile parties, essentially the same with those which, under different
names, have ever since contended, and are still contending, for the
direction of public affairs, appeared confronting each other. During
some years they were designated as Cavaliers and Roundheads. They
were subsequently called Tories and Whigs; nor does it seem that these
appellations are likely soon to become obsolete.

It would not be difficult to compose a lampoon or panegyric on either
of these renowned factions. For no man not utterly destitute of judgment
and candor will deny that there are many deep stains on the fame of the
party to which he belongs, or that the party to which he is opposed may
justly boast of many illustrious names, of many heroic actions, and of
many great services rendered to the state. The truth is that, though
both parties have often seriously erred, England could have spared
neither. If, in her institutions, freedom and order, the advantages
arising from innovation and the advantages arising from prescription,
have been combined to an extent elsewhere unknown, we may attribute this
happy peculiarity to the strenuous conflicts and alternate victories
of two rival confederacies of statesmen, a confederacy zealous for
authority and antiquity, and a confederacy zealous for liberty and
progress.

It ought to be remembered that the difference between the two great
sections of English politicians has always been a difference rather of
degree than of principle. There were certain limits on the right and on
the left, which were very rarely overstepped. A few enthusiasts on one
side were ready to lay all our laws and franchises at the feet of
our Kings. A few enthusiasts on the other side were bent on pursuing,
through endless civil troubles, their darling phantom of a republic.
But the great majority of those who fought for the crown were averse
to despotism; and the great majority of the champions of popular rights
were averse to anarchy. Twice, in the course of the seventeenth century,
the two parties suspended their dissensions, and united their strength
in a common cause. Their first coalition restored hereditary monarchy.
Their second coalition rescued constitutional freedom.

It is also to be noted that these two parties have never been the whole
nation, nay, that they have never, taken together, made up a majority
of the nation. Between them has always been a great mass, which has
not steadfastly adhered to either, which has sometimes remained inertly
neutral, and which has sometimes oscillated to and fro. That mass has
more than once passed in a few years from one extreme to the other, and
back again. Sometimes it has changed sides, merely because it was tired
of supporting the same men, sometimes because it was dismayed by its
own excesses, sometimes because it had expected impossibilities, and had
been disappointed. But whenever it has leaned with its whole weight in
either direction, that weight has, for the time, been irresistible.

When the rival parties first appeared in a distinct form, they seemed
to be not unequally matched. On the side of the government was a
large majority of the nobles, and of those opulent and well descended
gentlemen to whom nothing was wanting of nobility but the name. These,
with the dependents whose support they could command, were no small
power in the state. On the same side were the great body of the clergy,
both the Universities, and all those laymen who were strongly attached
to episcopal government and to the Anglican ritual. These respectable
classes found themselves in the company of some allies much less
decorous than themselves. The Puritan austerity drove to the king's
faction all who made pleasure their business, who affected gallantry,
splendour of dress, or taste in the higher arts. With these went all who
live by amusing the leisure of others, from the painter and the comic
poet, down to the ropedancer and the Merry Andrew. For these artists
well knew that they might thrive under a superb and luxurious despotism,
but must starve under the rigid rule of the precisians. In the same
interest were the Roman Catholics to a man. The Queen, a daughter of
France, was of their own faith. Her husband was known to be strongly
attached to her, and not a little in awe of her. Though undoubtedly a
Protestant on conviction, he regarded the professors of the old religion
with no ill-will, and would gladly have granted them a much larger
toleration than he was disposed to concede to the Presbyterians. If the
opposition obtained the mastery, it was probable that the sanguinary
laws enacted against Papists in the reign of Elizabeth, would be
severely enforced. The Roman Catholics were therefore induced by the
strongest motives to espouse the cause of the court. They in general
acted with a caution which brought on them the reproach of cowardice
and lukewarmness; but it is probable that, in maintaining great reserve,
they consulted the King's interest as well as their own. It was not for
his service that they should be conspicuous among his friends.

The main strength of the opposition lay among the small freeholders in
the country, and among the merchants and shopkeepers of the towns.
But these were headed by a formidable minority of the aristocracy, a
minority which included the rich and powerful Earls of Northumberland,
Bedford, Warwick, Stamford, and Essex, and several other Lords of great
wealth and influence. In the same ranks was found the whole body of
Protestant Nonconformists, and most of those members of the Established
Church who still adhered to the Calvinistic opinions which, forty
years before, had been generally held by the prelates and clergy. The
municipal corporations took, with few exceptions, the same side. In the
House of Commons the opposition preponderated, but not very decidedly.

Neither party wanted strong arguments for the course which it was
disposed to take. The reasonings of the most enlightened Royalists may
be summed up thus:--"It is true that great abuses have existed; but they
have been redressed. It is true that precious rights have been invaded;
but they have been vindicated and surrounded with new securities. The
sittings of the Estates of the realm have been, in defiance of all
precedent and of the spirit of the constitution, intermitted during
eleven years; but it has now been provided that henceforth three years
shall never elapse without a Parliament. The Star Chamber the High
Commission, the Council of York, oppressed end plundered us; but those
hateful courts have now ceased to exist. The Lord Lieutenant aimed at
establishing military despotism; but he has answered for his treason
with his head. The Primate tainted our worship with Popish rites and
punished our scruples with Popish cruelty; but he is awaiting in the
Tower the judgment of his peers. The Lord Keeper sanctioned a plan by
which the property of every man in England was placed at the mercy of
the Crown; but he has been disgraced, ruined, and compelled to take
refuge in a foreign land. The ministers of tyranny have expiated
their crimes. The victims of tyranny have been compensated for their
sufferings. It would therefore be most unwise to persevere further in
that course which was justifiable and necessary when we first met, after
a long interval, and found the whole administration one mass of abuses.
It is time to take heed that we do not so pursue our victory over
despotism as to run into anarchy. It was not in our power to overturn
the bad institutions which lately afflicted our country, without shocks
which have loosened the foundations of government. Now that those
institutions have fallen, we must hasten to prop the edifice which it
was lately our duty to batter. Henceforth it will be our wisdom to look
with jealousy on schemes of innovation, and to guard from encroachment
all the prerogatives with which the law has, for the public good, armed
the sovereign."

Such were the views of those men of whom the excellent Falkland may be
regarded as the leader. It was contended on the other side with not less
force, by men of not less ability and virtue, that the safety which the
liberties of the English people enjoyed was rather apparent than real,
and that the arbitrary projects of the court would be resumed as soon
as the vigilance of the Commons was relaxed. True it was,--such was the
reasoning of Pym, of Hollis, and of Hampden--that many good laws had
been passed: but, if good laws had been sufficient to restrain the
King, his subjects would have had little reason ever to complain of his
administration. The recent statutes were surely not of more authority
than the Great Charter or the Petition of Right. Yet neither the Great
Charter, hallowed by the veneration of four centuries, nor the Petition
of Right, sanctioned, after mature reflection, and for valuable
consideration, by Charles himself, had been found effectual for the
protection of the people. If once the check of fear were withdrawn,
if once the spirit of opposition were suffered to slumber, all the
securities for English freedom resolved themselves into a single one,
the royal word; and it had been proved by a long and severe experience
that the royal word could not be trusted.

The two parties were still regarding each other with cautious hostility,
and had not yet measured their strength, when news arrived which
inflamed the passions and confirmed the opinions of both. The great
chieftains of Ulster, who, at the time of the accession of James, had,
after a long struggle, submitted to the royal authority, had not long
brooked the humiliation of dependence. They had conspired against the
English government, and had been attainted of treason. Their immense
domains had been forfeited to the crown, and had soon been peopled by
thousands of English and Scotch emigrants. The new settlers were, in
civilisation and intelligence, far superior to the native population,
and sometimes abused their superiority. The animosity produced by
difference of race was increased by difference of religion. Under the
iron rule of Wentworth, scarcely a murmur was heard: but, when that
strong pressure was withdrawn, when Scotland had set the example of
successful resistance, when England was distracted by internal quarrels,
the smothered rage of the Irish broke forth into acts of fearful
violence. On a sudden, the aboriginal population rose on the colonists.
A war, to which national and theological hatred gave a character of
peculiar ferocity, desolated Ulster, and spread to the neighbouring
provinces. The castle of Dublin was scarcely thought secure. Every post
brought to London exaggerated accounts of outrages which, without any
exaggeration were sufficient to move pity end horror. These evil tidings
roused to the height the zeal of both the great parties which were
marshalled against each other at Westminster. The Royalists maintained
that it was the first duty of every good Englishman and Protestant,
at such a crisis, to strengthen the hands of the sovereign. To the
opposition it seemed that there were now stronger reasons than ever for
thwarting and restraining him. That the commonwealth was in danger
was undoubtedly a good reason for giving large powers to a trustworthy
magistrate: but it was a good reason for taking away powers from a
magistrate who was at heart a public enemy. To raise a great army had
always been the King's first object. A great army must now be raised.
It was to be feared that, unless some new securities were devised, the
forces levied for the reduction of Ireland would be employed against
the liberties of England. Nor was this all. A horrible suspicion, unjust
indeed, but not altogether unnatural, had arisen in many minds. The
Queen was an avowed Roman Catholic: the King was not regarded by the
Puritans, whom he had mercilessly persecuted, as a sincere Protestant;
and so notorious was his duplicity, that there was no treachery of which
his subjects might not, with some show of reason, believe him capable.
It was soon whispered that the rebellion of the Roman Catholics of
Ulster was part of a vast work of darkness which had been planned at
Whitehall.

After some weeks of prelude, the first great parliamentary conflict
between the parties, which have ever since contended, and are still
contending, for the government of the nation, took place on the
twenty-second of November, 1641. It was moved by the opposition,
that the House of Commons should present to the King a remonstrance,
enumerating the faults of his administration from the time of his
accession, and expressing the distrust with which his policy was still
regarded by his people. That assembly, which a few months before had
been unanimous in calling for the reform of abuses, was now divided
into two fierce and eager factions of nearly equal strength. After a hot
debate of many hours, the remonstrance was carried by only eleven votes.

The result of this struggle was highly favourable to the conservative
party. It could not be doubted that only some great indiscretion could
prevent them from shortly obtaining the predominance in the Lower House.
The Upper House was already their own. Nothing was wanting to ensure
their success, but that the King should, in all his conduct, show
respect for the laws and scrupulous good faith towards his subjects.

His first measures promised well. He had, it seemed, at last discovered
that an entire change of system was necessary, and had wisely made
up his mind to what could no longer be avoided. He declared his
determination to govern in harmony with the Commons, and, for that end,
to call to his councils men in whose talents and character the Commons
might place confidence. Nor was the selection ill made. Falkland, Hyde,
and Colepepper, all three distinguished by the part which they had taken
in reforming abuses and in punishing evil ministers, were invited to
become the confidential advisers of the Crown, and were solemnly assured
by Charles that he would take no step in any way affecting the Lower
House of Parliament without their privity.

Had he kept this promise, it cannot be doubted that the reaction which
was already in progress would very soon have become quite as strong as
the most respectable Royalists would have desired. Already the violent
members of the opposition had begun to despair of the fortunes of their
party, to tremble for their own safety, and to talk of selling their
estates and emigrating to America. That the fair prospects which had
begun to open before the King were suddenly overcast, that his life was
darkened by adversity, and at length shortened by violence, is to be
attributed to his own faithlessness and contempt of law.

The truth seems to be that he detested both the parties into which the
House of Commons was divided: nor is this strange; for in both those
parties the love of liberty and the love of order were mingled, though
in different proportions. The advisers whom necessity had compelled him
to call round him were by no means after his own heart. They had joined
in condemning his tyranny, in abridging his power, and in punishing his
instruments. They were now indeed prepared to defend in a strictly legal
way his strictly legal prerogative; but they would have recoiled with
horror from the thought of reviving Wentworth's projects of Thorough.
They were, therefore, in the King's opinion, traitors, who differed only
in the degree of their seditious malignity from Pym and Hampden.

He accordingly, a few days after he had promised the chiefs of the
constitutional Royalists that no step of importance should be taken
without their knowledge, formed a resolution the most momentous of his
whole life, carefully concealed that resolution from them, and executed
it in a manner which overwhelmed them with shame and dismay. He sent the
Attorney General to impeach Pym, Hollis, Hampden, and other members of
the House of Commons of high treason at the bar of the House of Lords.
Not content with this flagrant violation of the Great Charter and of the
uninterrupted practice of centuries, he went in person, accompanied by
armed men, to seize the leaders of the opposition within the walls of
Parliament.

The attempt failed. The accused members had left the House a short time
before Charles entered it. A sudden and violent revulsion of feeling,
both in the Parliament and in the country, followed. The most favourable
view that has ever been taken of the King's conduct on this occasion by
his most partial advocates is that he had weakly suffered himself to be
hurried into a gross indiscretion by the evil counsels of his wife and
of his courtiers. But the general voice loudly charged him with far
deeper guilt. At the very moment at which his subjects, after a long
estrangement produced by his maladministration, were returning to him
with feelings of confidence and affection, he had aimed a deadly blow at
all their dearest rights, at the privileges of Parliament, at the very
principle of trial by jury. He had shown that he considered opposition
to his arbitrary designs as a crime to be expiated only by blood. He had
broken faith, not only with his Great Council and with his people,
but with his own adherents. He had done what, but for an unforeseen
accident, would probably have produced a bloody conflict round the
Speaker's chair. Those who had the chief sway in the Lower House now
felt that not only their power and popularity, but their lands and
their necks, were staked on the event of the struggle in which they were
engaged. The flagging zeal of the party opposed to the court revived in
an instant. During the night which followed the outrage the whole city
of London was in arms. In a few hours the roads leading to the capital
were covered with multitudes of yeomen spurring hard to Westminster with
the badges of the parliamentary cause in their hats. In the House of
Commons the opposition became at once irresistible, and carried, by more
than two votes to one, resolutions of unprecedented violence. Strong
bodies of the trainbands, regularly relieved, mounted guard round
Westminster Hall. The gates of the King's palace were daily besieged by
a furious multitude whose taunts and execrations were heard even in
the presence chamber, and who could scarcely be kept out of the royal
apartments by the gentlemen of the household. Had Charles remained much
longer in his stormy capital, it is probable that the Commons would have
found a plea for making him, under outward forms of respect, a state
prisoner.

He quitted London, never to return till the day of a terrible and
memorable reckoning had arrived. A negotiation began which occupied
many months. Accusations and recriminations passed backward and forward
between the contending parties. All accommodation had become impossible.
The sure punishment which waits on habitual perfidy had at length
overtaken the King. It was to no purpose that he now pawned his royal
word, and invoked heaven to witness the sincerity of his professions.
The distrust with which his adversaries regarded him was not to be
removed by oaths or treaties. They were convinced that they could be
safe only when he was utterly helpless. Their demand, therefore, was,
that he should surrender, not only those prerogatives which he had
usurped in violation of ancient laws and of his own recent promises, but
also other prerogatives which the English Kings had always possessed,
and continue to possess at the present day. No minister must be
appointed, no peer created, without the consent of the Houses. Above
all, the sovereign must resign that supreme military authority which,
from time beyond all memory, had appertained to the regal office.

That Charles would comply with such demands while he had any means of
resistance, was not to be expected. Yet it will be difficult to show
that the Houses could safely have exacted less. They were truly in a
most embarrassing position. The great majority of the nation was firmly
attached to hereditary monarchy. Those who held republican opinions
were as yet few, and did not venture to speak out. It was therefore
impossible to abolish kingly government. Yet it was plain that no
confidence could be placed in the King. It would have been absurd in
those who knew, by recent proof, that he was bent on destroying them, to
content themselves with presenting to him another Petition of Right,
and receiving from him fresh promises similar to those which he
had repeatedly made and broken. Nothing but the want of an army had
prevented him from entirely subverting the old constitution of the
realm. It was now necessary to levy a great regular army for the
conquest of Ireland; and it would therefore have been mere insanity to
leave him in possession of that plenitude of military authority which
his ancestors had enjoyed.

When a country is in the situation in which England then was, when the
kingly office is regarded with love and veneration, but the person
who fills that office is hated and distrusted, it should seem that the
course which ought to be taken is obvious. The dignity of the office
should be preserved: the person should be discarded. Thus our ancestors
acted in 1399 and in 1689. Had there been, in 1642, any man occupying a
position similar to that which Henry of Lancaster occupied at the time
of the deposition of Richard the Second, and which William of Orange
occupied at the time of the deposition of James the Second, it is
probable that the Houses would have changed the dynasty, and would have
made no formal change in the constitution. The new King, called to the
throne by their choice, and dependent on their support, would have been
under the necessity of governing in conformity with their wishes
and opinions. But there was no prince of the blood royal in the
parliamentary party; and, though that party contained many men of high
rank and many men of eminent ability, there was none who towered so
conspicuously above the rest that he could be proposed as a candidate
for the crown. As there was to be a King, and as no new King could be
found, it was necessary to leave the regal title to Charles. Only one
course, therefore, was left: and that was to disjoin the regal title
from the regal prerogatives.

The change which the Houses proposed to make in our institutions,
though it seems exorbitant, when distinctly set forth and digested into
articles of capitulation, really amounts to little more than the change
which, in the next generation, was effected by the Revolution. It is
true that, at the Revolution, the sovereign was not deprived by law of
the power of naming his ministers: but it is equally true that, since
the Revolution, no minister has been able to retain office six months
in opposition to the sense of the House of Commons. It is true that
the sovereign still possesses the power of creating peers, and the
more important power of the sword: but it is equally true that in the
exercise of these powers the sovereign has, ever since the
Revolution, been guided by advisers who possess the confidence of the
representatives of the nation. In fact, the leaders of the Roundhead
party in 1642, and the statesmen who, about half a century later,
effected the Revolution, had exactly the same object in view.
That object was to terminate the contest between the Crown and the
Parliament, by giving to the Parliament a supreme control over the
executive administration. The statesmen of the Revolution effected this
indirectly by changing the dynasty. The Roundheads of 1642, being unable
to change the dynasty, were compelled to take a direct course towards
their end.

We cannot, however, wonder that the demands of the opposition, importing
as they did a complete and formal transfer to the Parliament of powers
which had always belonged to the Crown, should have shocked that great
party of which the characteristics are respect for constitutional
authority and dread of violent innovation. That party had recently been
in hopes of obtaining by peaceable means the ascendency in the House of
Commons; but every such hope had been blighted. The duplicity of Charles
had made his old enemies irreconcileable, had driven back into the ranks
of the disaffected a crowd of moderate men who were in the very act of
coming over to his side, and had so cruelly mortified his best friends
that they had for a time stood aloof in silent shame and resentment.
Now, however, the constitutional Royalists were forced to make their
choice between two dangers; and they thought it their duty rather to
rally round a prince whose past conduct they condemned, and whose word
inspired them with little confidence, than to suffer the regal office to
be degraded, and the polity of the realm to be entirely remodelled.
With such feelings, many men whose virtues and abilities would have done
honour to any cause, ranged themselves on the side of the King.

In August 1642 the sword was at length drawn; and soon, in almost every
shire of the kingdom, two hostile factions appeared in arms against
each other. It is not easy to say which of the contending parties was at
first the more formidable. The Houses commanded London and the counties
round London, the fleet, the navigation of the Thames, and most of the
large towns and seaports. They had at their disposal almost all the
military stores of the kingdom, and were able to raise duties, both on
goods imported from foreign countries, and on some important products
of domestic industry. The King was ill provided with artillery and
ammunition. The taxes which he laid on the rural districts occupied by
his troops produced, it is probable, a sum far less than that which
the Parliament drew from the city of London alone. He relied, indeed,
chiefly, for pecuniary aid, on the munificence of his opulent adherents.
Many of these mortgaged their land, pawned their jewels, and broke up
their silver chargers and christening bowls, in order to assist him.
But experience has fully proved that the voluntary liberality of
individuals, even in times of the greatest excitement, is a poor
financial resource when compared with severe and methodical taxation,
which presses on the willing and unwilling alike.

Charles, however, had one advantage, which, if he had used it well,
would have more than compensated for the want of stores and money, and
which, notwithstanding his mismanagement, gave him, during some months,
a superiority in the war. His troops at first fought much better than
those of the Parliament. Both armies, it is true, were almost entirely
composed of men who had never seen a field of battle. Nevertheless, the
difference was great. The Parliamentary ranks were filled with hirelings
whom want and idleness had induced to enlist. Hampden's regiment was
regarded as one of the best; and even Hampden's regiment was described
by Cromwell as a mere rabble of tapsters and serving men out of place.
The royal army, on the other hand, consisted in great part of gentlemen,
high spirited, ardent, accustomed to consider dishonour as more terrible
than death, accustomed to fencing, to the use of fire arms, to bold
riding, and to manly and perilous sport, which has been well called the
image of war. Such gentlemen, mounted on their favourite horses, and
commanding little bands composed of their younger brothers, grooms,
gamekeepers, and huntsmen, were, from the very first day on which they
took the field, qualified to play their part with credit in a skirmish.
The steadiness, the prompt obedience, the mechanical precision of
movement, which are characteristic of the regular soldier, these gallant
volunteers never attained. But they were at first opposed to enemies as
undisciplined as themselves, and far less active, athletic, and daring.
For a time, therefore, the Cavaliers were successful in almost every
encounter.

The Houses had also been unfortunate in the choice of a general. The
rank and wealth of the Earl of Essex made him one of the most important
members of the parliamentary party. He had borne arms on the Continent
with credit, and, when the war began, had as high a military reputation
as any man in the country. But it soon appeared that he was unfit for
the post of Commander in Chief. He had little energy and no originality.
The methodical tactics which he had learned in the war of the Palatinate
did not save him from the disgrace of being surprised and baffled by
such a Captain as Rupert, who could claim no higher fame than that of an
enterprising partisan.

Nor were the officers who held the chief commissions under Essex
qualified to supply what was wanting in him. For this, indeed, the
Houses are scarcely to be blamed. In a country which had not, within the
memory of the oldest person living, made war on a great scale by
land, generals of tried skill and valour were not to be found. It was
necessary, therefore, in the first instance, to trust untried men; and
the preference was naturally given to men distinguished either by their
station, or by the abilities which they had displayed in Parliament.
In scarcely a single instance, however, was the selection fortunate.
Neither the grandees nor the orators proved good soldiers. The Earl
of Stamford, one of the greatest nobles of England, was routed by
the Royalists at Stratton. Nathaniel Fiennes, inferior to none of his
contemporaries in talents for civil business, disgraced himself by the
pusillanimous surrender of Bristol. Indeed, of all the statesmen who at
this juncture accepted high military commands, Hampden alone appears to
have carried into the camp the capacity and strength of mind which had
made him eminent in politics.

When the war had lasted a year, the advantage was decidedly with the
Royalists. They were victorious, both in the western and in the northern
counties. They had wrested Bristol, the second city in the kingdom, from
the Parliament. They had won several battles, and had not sustained a
single serious or ignominious defeat. Among the Roundheads adversity had
begun to produce dissension and discontent. The Parliament was kept
in alarm, sometimes by plots, and sometimes by riots. It was thought
necessary to fortify London against the royal army, and to hang
some disaffected citizens at their own doors. Several of the most
distinguished peers who had hitherto remained at Westminster fled to the
court at Oxford; nor can it be doubted that, if the operations of the
Cavaliers had, at this season, been directed by a sagacious and powerful
mind, Charles would soon have marched in triumph to Whitehall.

But the King suffered the auspicious moment to pass away; and it never
returned. In August 1643 he sate down before the city of Gloucester.
That city was defended by the inhabitants and by the garrison, with a
determination such as had not, since the commencement of the war, been
shown by the adherents of the Parliament. The emulation of London was
excited. The trainbands of the City volunteered to march wherever their
services might be required. A great force was speedily collected,
and began to move westward. The siege of Gloucester was raised: the
Royalists in every part of the kingdom were disheartened: the spirit of
the parliamentary party revived: and the apostate Lords, who had
lately fled from Westminster to Oxford, hastened back from Oxford to
Westminster.

And now a new and alarming class of symptoms began to appear in the
distempered body politic. There had been, from the first, in the
parliamentary party, some men whose minds were set on objects from which
the majority of that party would have shrunk with horror. These men
were, in religion, Independents. They conceived that every Christian
congregation had, under Christ, supreme jurisdiction in things
spiritual; that appeals to provincial and national synods were scarcely
less unscriptural than appeals to the Court of Arches, or to the
Vatican; and that Popery, Prelacy, and Presbyterianism were merely three
forms of one great apostasy. In politics, the Independents were, to use
the phrase of their time, root and branch men, or, to use the kindred
phrase of our own time, radicals. Not content with limiting the power of
the monarch, they were desirous to erect a commonwealth on the ruins of
the old English polity. At first they had been inconsiderable, both
in numbers and in weight; but before the war had lasted two years they
became, not indeed the largest, but the most powerful faction in the
country. Some of the old parliamentary leaders had been removed by
death; and others had forfeited the public confidence. Pym had been
borne, with princely honours, to a grave among the Plantagenets. Hampden
had fallen, as became him, while vainly endeavouring, by his heroic
example, to inspire his followers with courage to face the fiery cavalry
of Rupert. Bedford had been untrue to the cause. Northumberland was
known to be lukewarm. Essex and his lieutenants had shown little vigour
and ability in the conduct of military operations. At such a conjuncture
it was that the Independent party, ardent, resolute, and uncompromising,
began to raise its head, both in the camp and in the House of Commons.

The soul of that party was Oliver Cromwell. Bred to peaceful
occupations, he had, at more than forty years of age, accepted a
commission in the parliamentary army. No sooner had he become a soldier
than he discerned, with the keen glance of genius, what Essex, and men
like Essex, with all their experience, were unable to perceive. He saw
precisely where the strength of the Royalists lay, and by what means
alone that strength could be overpowered. He saw that it was necessary
to reconstruct the army of the Parliament. He saw also that there were
abundant and excellent materials for the purpose, materials less showy,
indeed, but more solid, than those of which the gallant squadrons of the
King were composed. It was necessary to look for recruits who were not
mere mercenaries, for recruits of decent station and grave character,
fearing God and zealous for public liberty. With such men he filled his
own regiment, and, while he subjected them to a discipline more rigid
than had ever before been known in England, he administered to their
intellectual and moral nature stimulants of fearful potency.

The events of the year 1644 fully proved the superiority of his
abilities. In the south, where Essex held the command, the parliamentary
forces underwent a succession of shameful disasters; but in the north
the victory of Marston Moor fully compensated for all that had been lost
elsewhere. That victory was not a more serious blow to the Royalists
than to the party which had hitherto been dominant at Westminster, for
it was notorious that the day, disgracefully lost by the Presbyterians,
had been retrieved by the energy of Cromwell, and by the steady valour
of the warriors whom he had trained.

These events produced the Selfdenying Ordinance and the new model of the
army. Under decorous pretexts, and with every mark of respect, Essex and
most of those who had held high posts under him were removed; and the
conduct of the war was intrusted to very different hands. Fairfax, a
brave soldier, but of mean understanding and irresolute temper, was the
nominal Lord General of the forces; but Cromwell was their real head.

Cromwell made haste to organise the whole army on the same principles
on which he had organised his own regiment. As soon as this process was
complete, the event of the war was decided. The Cavaliers had now to
encounter natural courage equal to their own, enthusiasm stronger than
their own, and discipline such as was utterly wanting to them. It soon
became a proverb that the soldiers of Fairfax and Cromwell were men of
a different breed from the soldiers of Essex. At Naseby took place the
first great encounter between the Royalists and the remodelled army of
the Houses. The victory of the Roundheads was complete and decisive. It
was followed by other triumphs in rapid succession. In a few months
the authority of the Parliament was fully established over the whole
kingdom. Charles fled to the Scots, and was by them, in a manner which
did not much exalt their national character, delivered up to his English
subjects.

While the event of the war was still doubtful, the Houses had put the
Primate to death, had interdicted, within the sphere of their authority,
the use of the Liturgy, and had required all men to subscribe that
renowned instrument known by the name of the Solemn League and Covenant.
Covenanting work, as it was called, went on fast. Hundreds of thousands
affixed their names to the rolls, and, with hands lifted up towards
heaven, swore to endeavour, without respect of persons, the extirpation
of Popery and Prelacy, heresy and schism, and to bring to public
trial and condign punishment all who should hinder the reformation of
religion. When the struggle was over, the work of innovation and revenge
was pushed on with increased ardour. The ecclesiastical polity of the
kingdom was remodelled. Most of the old clergy were ejected from their
benefices. Fines, often of ruinous amount, were laid on the Royalists,
already impoverished by large aids furnished to the King. Many estates
were confiscated. Many proscribed Cavaliers found it expedient to
purchase, at an enormous cost, the protection of eminent members of the
victorious party. Large domains, belonging to the crown, to the bishops,
and to the chapters, were seized, and either granted away or put up to
auction. In consequence of these spoliations, a great part of the soil
of England was at once offered for sale. As money was scarce, as the
market was glutted, as the title was insecure and as the awe inspired
by powerful bidders prevented free competition, the prices were often
merely nominal. Thus many old and honourable families disappeared and
were heard of no more; and many new men rose rapidly to affluence.

But, while the Houses were employing their authority thus, it suddenly
passed out of their hands. It had been obtained by calling into
existence a power which could not be controlled. In the summer of
1647, about twelve months after the last fortress of the Cavaliers had
submitted to the Parliament, the Parliament was compelled to submit to
its own soldiers.

Thirteen years followed, during which England was, under various names
and forms, really governed by the sword. Never before that time,
or since that time, was the civil power in our country subjected to
military dictation.

The army which now became supreme in the state was an army very
different from any that has since been seen among us. At present the
pay of the common soldier is not such as can seduce any but the
humblest class of English labourers from their calling. A barrier
almost impassable separates him from the commissioned officer. The great
majority of those who rise high in the service rise by purchase. So
numerous and extensive are the remote dependencies of England, that
every man who enlists in the line must expect to pass many years in
exile, and some years in climates unfavourable to the health and vigour
of the European race. The army of the Long Parliament was raised for
home service. The pay of the private soldier was much above the wages
earned by the great body of the people; and, if he distinguished himself
by intelligence and courage, he might hope to attain high commands.
The ranks were accordingly composed of persons superior in station and
education to the multitude. These persons, sober, moral, diligent, and
accustomed to reflect, had been induced to take up arms, not by the
pressure of want, not by the love of novelty and license, not by the
arts of recruiting officers, but by religious and political zeal,
mingled with the desire of distinction and promotion. The boast of the
soldiers, as we find it recorded in their solemn resolutions, was that
they had not been forced into the service, nor had enlisted chiefly
for the sake of lucre. That they were no janissaries, but freeborn
Englishmen, who had, of their own accord, put their lives in jeopardy
for the liberties and religion of England, and whose right and duty it
was to watch over the welfare of the nation which they had saved.

A force thus composed might, without injury to its efficiency, be
indulged in some liberties which, if allowed to any other troops, would
have proved subversive of all discipline. In general, soldiers who
should form themselves into political clubs, elect delegates, and pass
resolutions on high questions of state, would soon break loose from all
control, would cease to form an army, and would become the worst and
most dangerous of mobs. Nor would it be safe, in our time, to tolerate
in any regiment religious meetings, at which a corporal versed in
Scripture should lead the devotions of his less gifted colonel, and
admonish a backsliding major. But such was the intelligence, the
gravity, and the selfcommand of the warriors whom Cromwell had trained,
that in their camp a political organisation and a religious organisation
could exist without destroying military organisation. The same men,
who, off duty, were noted as demagogues and field preachers, were
distinguished by steadiness, by the spirit of order, and by prompt
obedience on watch, on drill, and on the field of battle.

In war this strange force was irresistible. The stubborn courage
characteristic of the English people was, by the system of Cromwell, at
once regulated and stimulated. Other leaders have maintained orders as
strict. Other leaders have inspired their followers with zeal as ardent.
But in his camp alone the most rigid discipline was found in company
with the fiercest enthusiasm. His troops moved to victory with the
precision of machines, while burning with the wildest fanaticism of
Crusaders. From the time when the army was remodelled to the time when
it was disbanded, it never found, either in the British islands or on
the Continent, an enemy who could stand its onset. In England,
Scotland, Ireland, Flanders, the Puritan warriors, often surrounded
by difficulties, sometimes contending against threefold odds, not only
never failed to conquer, but never failed to destroy and break in pieces
whatever force was opposed to them. They at length came to regard the
day of battle as a day of certain triumph, and marched against the most
renowned battalions of Europe with disdainful confidence. Turenne was
startled by the shout of stern exultation with which his English allies
advanced to the combat, and expressed the delight of a true soldier,
when he learned that it was ever the fashion of Cromwell's pikemen to
rejoice greatly when they beheld the enemy; and the banished Cavaliers
felt an emotion of national pride, when they saw a brigade of their
countrymen, outnumbered by foes and abandoned by friends, drive before
it in headlong rout the finest infantry of Spain, and force a passage
into a counterscarp which had just been pronounced impregnable by the
ablest of the Marshals of France.

But that which chiefly distinguished the army of Cromwell from other
armies was the austere morality and the fear of God which pervaded all
ranks. It is acknowledged by the most zealous Royalists that, in that
singular camp, no oath was heard, no drunkenness or gambling was seen,
and that, during the long dominion of the soldiery, the property of the
peaceable citizen and the honour of woman were held sacred. If outrages
were committed, they were outrages of a very different kind from
those of which a victorious army is generally guilty. No servant girl
complained of the rough gallantry of the redcoats. Not an ounce of plate
was taken from the shops of the goldsmiths. But a Pelagian sermon, or
a window on which the Virgin and Child were painted, produced in the
Puritan ranks an excitement which it required the utmost exertions
of the officers to quell. One of Cromwell's chief difficulties was to
restrain his musketeers and dragoons from invading by main force the
pulpits of ministers whose discourses, to use the language of that time,
were not savoury; and too many of our cathedrals still bear the marks
of the hatred with which those stern spirits regarded every vestige of
Popery.

To keep down the English people was no light task even for that army. No
sooner was the first pressure of military tyranny felt, than the nation,
unbroken to such servitude, began to struggle fiercely. Insurrections
broke out even in those counties which, during the recent war, had been
the most submissive to the Parliament. Indeed, the Parliament itself
abhorred its old defenders more than its old enemies, and was desirous
to come to terms of accommodation with Charles at the expense of the
troops. In Scotland at the same time, a coalition was formed between the
Royalists and a large body of Presbyterians who regarded the doctrines
of the Independents with detestation. At length the storm burst. There
were risings in Norfolk, Suffolk, Essex, Kent, Wales. The fleet in the
Thames suddenly hoisted the royal colours, stood out to sea, and menaced
the southern coast. A great Scottish force crossed the frontier
and advanced into Lancashire. It might well be suspected that these
movements were contemplated with secret complacency by a majority both
of the Lords and of the Commons.

But the yoke of the army was not to be so shaken off. While Fairfax
suppressed the risings in the neighbourhood of the capital, Oliver
routed the Welsh insurgents, and, leaving their castles in ruins,
marched against the Scots. His troops were few, when compared with the
invaders; but he was little in the habit of counting his enemies. The
Scottish army was utterly destroyed. A change in the Scottish government
followed. An administration, hostile to the King, was formed at
Edinburgh; and Cromwell, more than ever the darling of his soldiers,
returned in triumph to London.

And now a design, to which, at the commencement of the civil war, no man
would have dared to allude, and which was not less inconsistent with the
Solemn League and Covenant than with the old law of England, began to
take a distinct form. The austere warriors who ruled the nation had,
during some months, meditated a fearful vengeance on the captive King.
When and how the scheme originated; whether it spread from the general
to the ranks, or from the ranks to the general; whether it is to be
ascribed to policy using fanaticism as a tool, or to fanaticism bearing
down policy with headlong impulse, are questions which, even at this
day, cannot be answered with perfect confidence. It seems, however,
on the whole, probable that he who seemed to lead was really forced to
follow, and that, on this occasion, as on another great occasion a few
years later, he sacrificed his own judgment and his own inclinations to
the wishes of the army. For the power which he had called into existence
was a power which even he could not always control; and, that he might
ordinarily command, it was necessary that he should sometimes obey. He
publicly protested that he was no mover in the matter, that the first
steps had been taken without his privity, that he could not advise the
Parliament to strike the blow, but that he submitted his own feelings to
the force of circumstances which seemed to him to indicate the purposes
of Providence. It has been the fashion to consider these professions as
instances of the hypocrisy which is vulgarly imputed to him. But even
those who pronounce him a hypocrite will scarcely venture to call him a
fool. They are therefore bound to show that he had some purpose to serve
by secretly stimulating the army to take that course which he did not
venture openly to recommend. It would be absurd to suppose that he who
was never by his respectable enemies represented as wantonly cruel or
implacably vindictive, would have taken the most important step of his
life under the influence of mere malevolence. He was far too wise a man
not to know, when he consented to shed that august blood, that he was
doing a deed which was inexpiable, and which would move the grief and
horror, not only of the Royalists, but of nine tenths of those who had
stood by the Parliament. Whatever visions may have deluded others, he
was assuredly dreaming neither of a republic on the antique pattern,
nor of the millennial reign of the Saints. If he already aspired to
be himself the founder of a new dynasty, it was plain that Charles the
First was a less formidable competitor than Charles the Second would
be. At the moment of the death of Charles the First the loyalty of
every Cavalier would be transferred, unimpaired, to Charles the Second.
Charles the First was a captive: Charles the Second would be at liberty.
Charles the First was an object of suspicion and dislike to a large
proportion of those who yet shuddered at the thought of slaying him:
Charles the Second would excite all the interest which belongs to
distressed youth and innocence. It is impossible to believe that
considerations so obvious, and so important, escaped the most profound
politician of that age. The truth is that Cromwell had, at one
time, meant to mediate between the throne and the Parliament, and to
reorganise the distracted State by the power of the sword, under the
sanction of the royal name. In this design he persisted till he was
compelled to abandon it by the refractory temper of the soldiers, and
by the incurable duplicity of the King. A party in the camp began to
clamour for the head of the traitor, who was for treating with Agag.
Conspiracies were formed. Threats of impeachment were loudly uttered.
A mutiny broke out, which all the vigour and resolution of Oliver
could hardly quell. And though, by a judicious mixture of severity and
kindness, he succeeded in restoring order, he saw that it would be in
the highest degree difficult and perilous to contend against the rage of
warriors, who regarded the fallen tyrant as their foe, and as the foe
of their God. At the same time it became more evident than ever that the
King could not be trusted. The vices of Charles had grown upon him. They
were, indeed, vices which difficulties and perplexities generally bring
out in the strongest light. Cunning is the natural defence of the weak.
A prince, therefore, who is habitually a deceiver when at the height of
power, is not likely to learn frankness in the midst of embarrassments
and distresses. Charles was not only a most unscrupulous but a most
unlucky dissembler. There never was a politician to whom so many frauds
and falsehoods were brought home by undeniable evidence. He publicly
recognised the Houses at Westminster as a legal Parliament, and, at the
same time, made a private minute in council declaring the recognition
null. He publicly disclaimed all thought of calling in foreign aid
against his people: he privately solicited aid from France, from
Denmark, and from Lorraine. He publicly denied that he employed Papists:
at the same time he privately sent to his generals directions to employ
every Papist that would serve. He publicly took the sacrament at Oxford,
as a pledge that he never would even connive at Popery. He privately
assured his wife, that he intended to tolerate Popery in England; and he
authorised Lord Glamorgan to promise that Popery should be established
in Ireland. Then he attempted to clear himself at his agent's expense.
Glamorgan received, in the Royal handwriting, reprimands intended to be
read by others, and eulogies which were to be seen only by himself. To
such an extent, indeed, had insincerity now tainted the King's whole
nature, that his most devoted friends could not refrain from complaining
to each other, with bitter grief and shame, of his crooked politics. His
defeats, they said, gave them less pain than his intrigues. Since he had
been a prisoner, there was no section of the victorious party which had
not been the object both of his flatteries and of his machinations; but
never was he more unfortunate than when he attempted at once to cajole
and to undermine Cromwell.

Cromwell had to determine whether he would put to hazard the attachment
of his party, the attachment of his army, his own greatness, nay his
own life, in an attempt which would probably have been vain, to save
a prince whom no engagement could bind. With many struggles and
misgivings, and probably not without many prayers, the decision was
made. Charles was left to his fate. The military saints resolved that,
in defiance of the old laws of the realm, and of the almost universal
sentiment of the nation, the King should expiate his crimes with
his blood. He for a time expected a death like that of his unhappy
predecessors, Edward the Second and Richard the Second. But he was in
no danger of such treason. Those who had him in their gripe were not
midnight stabbers. What they did they did in order that it might be a
spectacle to heaven and earth, and that it might be held in everlasting
remembrance. They enjoyed keenly the very scandal which they gave. That
the ancient constitution and the public opinion of England were directly
opposed to regicide made regicide seem strangely fascinating to a party
bent on effecting a complete political and social revolution. In order
to accomplish their purpose, it was necessary that they should first
break in pieces every part of the machinery of the government; and this
necessity was rather agreeable than painful to them. The Commons passed
a vote tending to accommodation with the King. The soldiers excluded the
majority by force. The Lords unanimously rejected the proposition that
the King should be brought to trial. Their house was instantly closed.
No court, known to the law, would take on itself the office of judging
the fountain of justice. A revolutionary tribunal was created. That
tribunal pronounced Charles a tyrant, a traitor, a murderer, and a
public enemy; and his head was severed from his shoulders, before
thousands of spectators, in front of the banqueting hall of his own
palace.

In no long time it became manifest that those political and religious
zealots, to whom this deed is to be ascribed, had committed, not only a
crime, but an error. They had given to a prince, hitherto known to his
people chiefly by his faults, an opportunity of displaying, on a great
theatre, before the eyes of all nations and all ages, some qualities
which irresistibly call forth the admiration and love of mankind, the
high spirit of a gallant gentleman, the patience and meekness of a
penitent Christian. Nay, they had so contrived their revenge that the
very man whose life had been a series of attacks on the liberties of
England now seemed to die a martyr in the cause of those liberties. No
demagogue ever produced such an impression on the public mind as the
captive King, who, retaining in that extremity all his regal dignity,
and confronting death with dauntless courage, gave utterance to the
feelings of his oppressed people, manfully refused to plead before
a court unknown to the law, appealed from military violence to the
principles of the constitution, asked by what right the House of Commons
had been purged of its most respectable members and the House of Lords
deprived of its legislative functions, and told his weeping hearers
that he was defending, not only his own cause, but theirs. His long
misgovernment, his innumerable perfidies, were forgotten. His memory
was, in the minds of the great majority of his subjects, associated with
those free institutions which he had, during many years, laboured to
destroy: for those free institutions had perished with him, and, amidst
the mournful silence of a community kept down by arms, had been defended
by his voice alone. From that day began a reaction in favour of monarchy
and of the exiled house, reaction which never ceased till the throne had
again been set up in all its old dignity.

At first, however, the slayers of the King seemed to have derived new
energy from that sacrament of blood by which they had bound themselves
closely together, and separated themselves for ever from the great body
of their countrymen. England was declared a commonwealth. The House of
Commons, reduced to a small number of members, was nominally the supreme
power in the state. In fact, the army and its great chief governed
everything. Oliver had made his choice. He had kept the hearts of his
soldiers, and had broken with almost every other class of his fellow
citizens. Beyond the limits of his camps and fortresses he could
scarcely be said to have a party. Those elements of force which, when
the civil war broke out, had appeared arrayed against each other, were
combined against him; all the Cavaliers, the great majority of the
Roundheads, the Anglican Church, the Presbyterian Church, the Roman
Catholic Church, England, Scotland, Ireland. Yet such, was his genius
and resolution that he was able to overpower and crush everything that
crossed his path, to make himself more absolute master of his country
than any of her legitimate Kings had been, and to make his country more
dreaded and respected than she had been during many generations under
the rule of her legitimate Kings.

England had already ceased to struggle. But the two other kingdoms which
had been governed by the Stuarts were hostile to the new republic. The
Independent party was equally odious to the Roman Catholics of Ireland
and to the Presbyterians of Scotland. Both those countries, lately in
rebellion against Charles the First, now acknowledged the authority of
Charles the Second.

But everything yielded to the vigour and ability of Cromwell. In a
few months he subjugated Ireland, as Ireland had never been subjugated
during the five centuries of slaughter which had elapsed since the
landing of the first Norman settlers. He resolved to put an end to that
conflict of races and religions which had so long distracted the island,
by making the English and Protestant population decidedly predominant.
For this end he gave the rein to the fierce enthusiasm of his followers,
waged war resembling that which Israel waged on the Canaanites, smote
the idolaters with the edge of the sword, so that great cities were left
without inhabitants, drove many thousands to the Continent, shipped off
many thousands to the West Indies, and supplied the void thus made by
pouring in numerous colonists, of Saxon blood, and of Calvinistic faith.
Strange to say, under that iron rule, the conquered country began to
wear an outward face of prosperity. Districts, which had recently been
as wild as those where the first white settlers of Connecticut were
contending with the red men, were in a few years transformed into the
likeness of Kent and Norfolk. New buildings, roads, and plantations were
everywhere seen. The rent of estates rose fast; and soon the English
landowners began to complain that they were met in every market by the
products of Ireland, and to clamour for protecting laws.

From Ireland the victorious chief, who was now in name, as he had long
been in reality, Lord General of the armies of the Commonwealth, turned
to Scotland. The Young King was there. He had consented to profess
himself a Presbyterian, and to subscribe the Covenant; and, in return
for these concessions, the austere Puritans who bore sway at Edinburgh
had permitted him to assume the crown, and to hold, under their
inspection and control, a solemn and melancholy court. This mock royalty
was of short duration. In two great battles Cromwell annihilated the
military force of Scotland. Charles fled for his life, and, with extreme
difficulty, escaped the fate of his father. The ancient kingdom of the
Stuarts was reduced, for the first time, to profound submission. Of that
independence, so manfully defended against the mightiest and ablest of
the Plantagenets, no vestige was left. The English Parliament made
laws for Scotland. English judges held assizes in Scotland. Even that
stubborn Church, which has held its own against so many governments,
scarce dared to utter an audible murmur.

Thus far there had been at least the semblance of harmony between the
warriors who had subjugated Ireland and Scotland and the politicians who
sate at Westminster: but the alliance which had been cemented by danger
was dissolved by victory. The Parliament forgot that it was but the
creature of the army. The army was less disposed than ever to submit to
the dictation of the Parliament. Indeed the few members who made up what
was contemptuously called the Rump of the House of Commons had no more
claim than the military chiefs to be esteemed the representatives of
the nation. The dispute was soon brought to a decisive issue. Cromwell
filled the House with armed men. The Speaker was pulled out of his
chair, the mace taken from the table, the room cleared, and the door
locked. The nation, which loved neither of the contending parties,
but which was forced, in its own despite, to respect the capacity
and resolution of the General, looked on with patience, if not with
complacency.

King, Lords, and Commons, had now in turn been vanquished and destroyed;
and Cromwell seemed to be left the sole heir of the powers of all three.
Yet were certain limitations still imposed on him by the very army to
which he owed his immense authority. That singular body of men was, for
the most part, composed of zealous republicans. In the act of enslaving
their country, they had deceived themselves into the belief that they
were emancipating her. The book which they venerated furnished them with
a precedent which was frequently in their mouths. It was true that the
ignorant and ungrateful nation murmured against its deliverers. Even so
had another chosen nation murmured against the leader who brought it, by
painful and dreary paths, from the house of bondage to the land flowing
with milk and honey. Yet had that leader rescued his brethren in spite
of themselves; nor had he shrunk from making terrible examples of those
who contemned the proffered freedom, and pined for the fleshpots, the
taskmasters, and the idolatries of Egypt. The object of the warlike
saints who surrounded Cromwell was the settlement of a free and pious
commonwealth. For that end they were ready to employ, without scruple,
any means, however violent and lawless. It was not impossible,
therefore, to establish by their aid a dictatorship such as no King
had ever exercised: but it was probable that their aid would be at once
withdrawn from a ruler who, even under strict constitutional restraints,
should venture to assume the kingly name and dignity.

The sentiments of Cromwell were widely different. He was not what he had
been; nor would it be just to consider the change which his views had
undergone as the effect merely of selfish ambition. He had, when he
came up to the Long Parliament, brought with him from his rural retreat
little knowledge of books, no experience of great affairs, and a temper
galled by the long tyranny of the government and of the hierarchy. He
had, during the thirteen years which followed, gone through a political
education of no common kind. He had been a chief actor in a succession
of revolutions. He had been long the soul, and at last the head, of
a party. He had commanded armies, won battles, negotiated treaties,
subdued, pacified, and regulated kingdoms. It would have been strange
indeed if his notions had been still the same as in the days when his
mind was principally occupied by his fields and his religion, and when
the greatest events which diversified the course of his life were a
cattle fair or a prayer meeting at Huntingdon. He saw that some schemes
of innovation for which he had once been zealous, whether good or bad
in themselves, were opposed to the general feeling of the country, and
that, if he persevered in those schemes, he had nothing before him but
constant troubles, which must be suppressed by the constant use of the
sword. He therefore wished to restore, in all essentials, that ancient
constitution which the majority of the people had always loved, and for
which they now pined. The course afterwards taken by Monk was not open
to Cromwell. The memory of one terrible day separated the great regicide
for ever from the House of Stuart. What remained was that he should
mount the ancient English throne, and reign according to the ancient
English polity. If he could effect this, he might hope that the wounds
of the lacerated State would heal fast. Great numbers of honest
and quiet men would speedily rally round him. Those Royalists whose
attachment was rather to institutions than to persons, to the kingly
office than to King Charles the First or King Charles the Second, would
soon kiss the hand of King Oliver. The peers, who now remained sullenly
at their country houses, and refused to take any part in public affairs,
would, when summoned to their House by the writ of a King in possession,
gladly resume their ancient functions. Northumberland and Bedford,
Manchester and Pembroke, would be proud to bear the crown and the
spurs, the sceptre and the globe, before the restorer of aristocracy. A
sentiment of loyalty would gradually bind the people to the new dynasty;
and, on the decease of the founder of that dynasty, the royal dignity
might descend with general acquiescence to his posterity.

The ablest Royalists were of opinion that these views were correct, and
that, if Cromwell had been permitted to follow his own judgment, the
exiled line would never have been restored. But his plan was directly
opposed to the feelings of the only class which he dared not offend.
The name of King was hateful to the soldiers. Some of them were indeed
unwilling to see the administration in the hands of any single person.
The great majority, however, were disposed to support their general, as
elective first magistrate of a commonwealth, against all factions which
might resist his authority: but they would not consent that he should
assume the regal title, or that the dignity, which was the just reward
of his personal merit, should be declared hereditary in his family. All
that was left to him was to give to the new republic a constitution as
like the constitution of the old monarchy as the army would bear. That
his elevation to power might not seem to be merely his own act, he
convoked a council, composed partly of persons on whose support he could
depend, and partly of persons whose opposition he might safely defy.
This assembly, which he called a Parliament, and which the populace
nicknamed, from one of the most conspicuous members, Barebonesa's
Parliament, after exposing itself during a short time to the public
contempt, surrendered back to the General the powers which it
had received from him, and left him at liberty to frame a plan of
government.

His plan bore, from the first, a considerable resemblance to the old
English constitution: but, in a few years, he thought it safe to proceed
further, and to restore almost every part of the ancient system under
hew names and forms. The title of King was not revived; but the kingly
prerogatives were intrusted to a Lord High Protector. The sovereign
was called not His Majesty, but His Highness. He was not crowned and
anointed in Westminster Abbey, but was solemnly enthroned, girt with
a sword of state, clad in a robe of purple, and presented with a rich
Bible, in Westminster Hall. His office was not declared hereditary: but
he was permitted to name his successor; and none could doubt that he
would name his Son.

A House of Commons was a necessary part of the new polity. In
constituting this body, the Protector showed a wisdom and a public
spirit which were not duly appreciated by his contemporaries. The vices
of the old representative system, though by no means so serious as they
afterwards became, had already been remarked by farsighted men. Cromwell
reformed that system on the same principles on which Mr. Pitt, a hundred
and thirty years later, attempted to reform it, and on which it was at
length reformed in our own times. Small boroughs were disfranchised
even more unsparingly than in 1832; and the number of county members
was greatly increased. Very few unrepresented towns had yet grown into
importance. Of those towns the most considerable were Manchester, Leeds,
and Halifax. Representatives were given to all three. An addition
was made to the number of the members for the capital. The elective
franchise was placed on such a footing that every man of substance,
whether possessed of freehold estates in land or not, had a vote for
the county in which he resided. A few Scotchmen and a few of the English
colonists settled in Ireland were summoned to the assembly which was to
legislate, at Westminster, for every part of the British isles.

To create a House of Lords was a less easy task. Democracy does not
require the support of prescription. Monarchy has often stood without
that support. But a patrician order is the work of time. Oliver found
already existing a nobility, opulent, highly considered, and as popular
with the commonalty as any nobility has ever been. Had he, as King of
England, commanded the peers to meet him in Parliament according to the
old usage of the realm, many of them would undoubtedly have obeyed the
call. This he could not do; and it was to no purpose that he offered
to the chiefs of illustrious families seats in his new senate. They
conceived that they could not accept a nomination to an upstart assembly
without renouncing their birthright and betraying their order. The
Protector was, therefore, under the necessity of filling his Upper House
with new men who, during the late stirring times, had made themselves
conspicuous. This was the least happy of his contrivances, and
displeased all parties. The Levellers were angry with him for
instituting a privileged class. The multitude, which felt respect and
fondness for the great historical names of the land, laughed without
restraint at a House of Lords, in which lucky draymen and shoemakers
were seated, to which few of the old nobles were invited, and from which
almost all those old nobles who were invited turned disdainfully away.

How Oliver's Parliaments were constituted, however, was practically
of little moment: for he possessed the means of conducting the
administration without their support, and in defiance of their
opposition. His wish seems to have been to govern constitutionally, and
to substitute the empire of the laws for that of the sword. But he soon
found that, hated as he was, both by Royalists and Presbyterians, he
could be safe only by being absolute. The first House of Commons which
the people elected by his command, questioned his authority, and was
dissolved without having passed a single act. His second House of
Commons, though it recognised him as Protector, and would gladly have
made him King, obstinately refused to acknowledge his new Lords. He had
no course left but to dissolve the Parliament. "God," he exclaimed, at
parting, "be judge between you and me!"

Yet was the energy of the Protector's administration in nowise relaxed
by these dissensions. Those soldiers who would not suffer him to assume
the kingly title stood by him when he ventured on acts of power, as
high as any English King has ever attempted. The government, therefore,
though in form a republic, was in truth a despotism, moderated only by
the wisdom, the sobriety, and the magnanimity of the despot. The country
was divided into military districts. Those districts were placed under
the command of Major Generals. Every insurrectionary movement was
promptly put down and punished. The fear inspired by the power of the
sword, in so strong, steady, and expert a hand, quelled the spirit both
of Cavaliers and Levellers. The loyal gentry declared that they were
still as ready as ever to risk their lives for the old government and
the old dynasty, if there were the slightest hope of success: but to
rush, at the head of their serving men and tenants, on the pikes of
brigades victorious in a hundred battles and sieges, would be a frantic
waste of innocent and honourable blood. Both Royalists and Republicans,
having no hope in open resistance, began to revolve dark schemes of
assassination: but the Protector's intelligence was good: his vigilance
was unremitting; and, whenever he moved beyond the walls of his palace,
the drawn swords and cuirasses of his trusty bodyguards encompassed him
thick on every side.

Had he been a cruel, licentious, and rapacious prince, the nation might
have found courage in despair, and might have made a convulsive effort
to free itself from military domination. But the grievances which the
country suffered, though such as excited serious discontent, were by
no means such as impel great masses of men to stake their lives, their
fortunes, and the welfare of their families against fearful odds. The
taxation, though heavier than it had been under the Stuarts, was not
heavy when compared with that of the neighbouring states and with
the resources of England. Property was secure. Even the Cavalier, who
refrained from giving disturbance to the new settlement, enjoyed in
peace whatever the civil troubles had left hem. The laws were violated
only in cases where the safety of the Protector's person and government
was concerned. Justice was administered between man and man with an
exactness and purity not before known. Under no English government since
the Reformation, had there been so little religious persecution. The
unfortunate Roman Catholics, indeed, were held to be scarcely within the
pale of Christian charity. But the clergy of the fallen Anglican Church
were suffered to celebrate their worship on condition that they would
abstain from preaching about politics. Even the Jews, whose public
worship had, ever since the thirteenth century, been interdicted, were,
in spite of the strong opposition of jealous traders and fanatical
theologians, permitted to build a synagogue in London.

The Protector's foreign policy at the same time extorted the ungracious
approbation of those who most detested him. The Cavaliers could scarcely
refrain from wishing that one who had done so much to raise the fame of
the nation had been a legitimate King; and the Republicans were forced
to own that the tyrant suffered none but himself to wrong his country,
and that, if he had robbed her of liberty, he had at least given her
glory in exchange. After half a century during which England had been of
scarcely more weight in European politics than Venice or Saxony, she at
once became the most formidable power in the world, dictated terms
of peace to the United Provinces, avenged the common injuries of
Christendom on the pirates of Barbary, vanquished the Spaniards by land
and sea, seized one of the finest West Indian islands, and acquired on
the Flemish coast a fortress which consoled the national pride for the
loss of Calais. She was supreme on the ocean. She was the head of the
Protestant interest. All the reformed Churches scattered over Roman
Catholic kingdoms acknowledged Cromwell as their guardian. The Huguenots
of Languedoc, the shepherds who, in the hamlets of the Alps professed a
Protestantism older than that of Augsburg, were secured from oppression
by the mere terror of his great name The Pope himself was forced to
preach humanity and moderation to Popish princes. For a voice which
seldom threatened in vain had declared that, unless favour were shown
to the people of God, the English guns should be heard in the Castle of
Saint Angelo. In truth, there was nothing which Cromwell had, for his
own sake and that of his family, so much reason to desire as a general
religious war in Europe. In such a war he must have been the captain of
the Protestant armies. The heart of England would have been with him.
His victories would have been hailed with an unanimous enthusiasm
unknown in the country since the rout of the Armada, and would have
effaced the stain which one act, condemned by the general voice of
the nation, has left on his splendid fame. Unhappily for him he had no
opportunity of displaying his admirable military talents, except against
the inhabitants of the British isles.

While he lived his power stood firm, an object of mingled aversion,
admiration, and dread to his subjects. Few indeed loved his government;
but those who hated it most hated it less than they feared it. Had it
been a worse government, it might perhaps have been overthrown in spite
of all its strength. Had it been a weaker government, it would certainly
have been overthrown in spite of all its merits. But it had moderation
enough to abstain from those oppressions which drive men mad; and it
had a force and energy which none but men driven mad by oppression would
venture to encounter.

It has often been affirmed, but with little reason, that Oliver died
at a time fortunate for his renown, and that, if his life had been
prolonged, it would probably have closed amidst disgraces and disasters.
It is certain that he was, to the last, honoured by his soldiers, obeyed
by the whole population of the British islands, and dreaded by all
foreign powers, that he was laid among the ancient sovereigns of England
with funeral pomp such as London had never before seen, and that he
was succeeded by his son Richard as quietly as any King had ever been
succeeded by any Prince of Wales.

During five months, the administration of Richard Cromwell went on
so tranquilly and regularly that all Europe believed him to be firmly
established on the chair of state. In truth his situation was in some
respects much more advantageous than that of his father. The young
man had made no enemy. His hands were unstained by civil blood.
The Cavaliers themselves allowed him to be an honest, good-natured
gentleman. The Presbyterian party, powerful both in numbers and in
wealth, had been at deadly feud with the late Protector, but was
disposed to regard the present Protector with favour. That party had
always been desirous to see the old civil polity of the realm restored
with some clearer definitions and some stronger safeguards for public
liberty, but had many reasons for dreading the restoration of the old
family. Richard was the very man for politicians of this description.
His humanity, ingenuousness, and modesty, the mediocrity of his
abilities, and the docility with which he submitted to the guidance of
persons wiser than himself, admirably qualified him to be the head of a
limited monarchy.

For a time it seemed highly probable that he would, under the direction
of able advisers, effect what his father had attempted in vain. A
Parliament was called, and the writs were directed after the old
fashion. The small boroughs which had recently been disfranchised
regained their lost privilege: Manchester, Leeds, and Halifax ceased to
return members; and the county of York was again limited to two knights.
It may seem strange to a generation which has been excited almost to
madness by the question of parliamentary reform that great shires and
towns should have submitted with patience and even with complacency, to
this change: but though speculative men might, even in that age, discern
the vices of the old representative system, and predict that those vices
would, sooner or later, produce serious practical evil, the practical
evil had not yet been felt. Oliver's representative system, on the other
hand, though constructed on sound principles, was not popular. Both the
events in which it originated, and the effects which it had produced,
prejudiced men against it. It had sprung from military violence. It
had been fruitful of nothing but disputes. The whole nation was sick
of government by the sword, and pined for government by the law. The
restoration, therefore, even of anomalies and abuses, which were in
strict conformity with the law, and which had been destroyed by the
sword, gave general satisfaction.

Among the Commons there was a strong opposition, consisting partly of
avowed Republicans, and partly of concealed Royalists: but a large and
steady majority appeared to be favourable to the plan of reviving
the old civil constitution under a new dynasty. Richard was solemnly
recognised as first magistrate. The Commons not only consented to
transact business with Oliver's Lords, but passed a vote acknowledging
the right of those nobles who had, in the late troubles, taken the side
of public liberty, to sit in the Upper House of Parliament without any
new creation.

Thus far the statesmen by whose advice Richard acted had been
successful. Almost all the parts of the government were now constituted
as they had been constituted at the commencement of the civil war. Had
the Protector and the Parliament been suffered to proceed undisturbed,
there can be little doubt that an order of things similar to that which
was afterwards established under the House of Hanover would have been
established under the House of Cromwell. But there was in the state
a power more than sufficient to deal with Protector and Parliament
together. Over the soldiers Richard had no authority except that which
he derived from the great name which he had inherited. He had never led
them to victory. He had never even borne arms. All his tastes and habits
were pacific. Nor were his opinions and feelings on religious subjects
approved by the military saints. That he was a good man he evinced by
proofs more satisfactory than deep groans or long sermons, by humility
and suavity when he was at the height of human greatness, and by
cheerful resignation under cruel wrongs and misfortunes: but the cant
then common in every guardroom gave him a disgust which he had not
always the prudence to conceal. The officers who had the principal
influence among the troops stationed near London were not his friends.
They were men distinguished by valour and conduct in the field, but
destitute of the wisdom and civil courage which had been conspicuous
in their deceased leader. Some of them were honest, but fanatical,
Independents and Republicans. Of this class Fleetwood was the
representative. Others were impatient to be what Oliver had been. His
rapid elevation, his prosperity and glory, his inauguration in the Hall,
and his gorgeous obsequies in the Abbey, had inflamed their imagination.
They were as well born as he, and as well educated: they could not
understand why they were not as worthy to wear the purple robe, and to
wield the sword of state; and they pursued the objects of their wild
ambition, not, like him, with patience, vigilance, sagacity, and
determination, but with the restlessness and irresolution characteristic
of aspiring mediocrity. Among these feeble copies of a great original
the most conspicuous was Lambert.

On the very day of Richard's accession the officers began to conspire
against their new master. The good understanding which existed between
him and his Parliament hastened the crisis. Alarm and resentment spread
through the camp. Both the religious and the professional feelings of
the army were deeply wounded. It seemed that the Independents were to be
subjected to the Presbyterians, and that the men of the sword were to
be subjected to the men of the gown. A coalition was formed between
the military malecontents and the republican minority of the House of
Commons. It may well be doubted whether Richard could have triumphed
over that coalition, even if he had inherited his father's clear
judgment and iron courage. It is certain that simplicity and meekness
like his were not the qualities which the conjuncture required. He fell
ingloriously, and without a struggle. He was used by the army as an
instrument for the purpose of dissolving the Parliament, and was then
contemptuously thrown aside. The officers gratified their republican
allies by declaring that the expulsion of the Rump had been illegal, and
by inviting that assembly to resume its functions. The old Speaker and a
quorum of the old members came together, and were proclaimed, amidst
the scarcely stifled derision and execration of the whole nation, the
supreme power in the commonwealth. It was at the same time expressly
declared that there should be no first magistrate, and no House of
Lords.

But this state of things could not last. On the day on which the long
Parliament revived, revived also its old quarrel with the army. Again
the Rump forgot that it owed its existence to the pleasure of the
soldiers, and began to treat them as subjects. Again the doors of the
House of Commons were closed by military violence; and a provisional
government, named by the officers, assumed the direction of affairs.

Meanwhile the sense of great evils, and the strong apprehension of still
greater evils close at hand, had at length produced an alliance between
the Cavaliers and the Presbyterians. Some Presbyterians had, indeed,
been disposed to such an alliance even before the death of Charles the
First: but it was not till after the fall of Richard Cromwell that the
whole party became eager for the restoration of the royal house. There
was no longer any reasonable hope that the old constitution could be
reestablished under a new dynasty. One choice only was left, the Stuarts
or the army. The banished family had committed great faults; but it had
dearly expiated those faults, and had undergone a long, and, it might be
hoped, a salutary training in the school of adversity. It was probable
that Charles the Second would take warning by the fate of Charles
the First. But, be this as it might, the dangers which threatened the
country were such that, in order to avert them, some opinions might well
be compromised, and some risks might well be incurred. It seemed but too
likely that England would fall under the most odious and degrading of
all kinds of government, under a government uniting all the evils of
despotism to all the evils of anarchy. Anything was preferable to the
yoke of a succession of incapable and inglorious tyrants, raised to
power, like the Deys of Barbary, by military revolutions recurring at
short intervals. Lambert seemed likely to be the first of these rulers;
but within a year Lambert might give place to Desborough, and Desborough
to Harrison. As often as the truncheon was transferred from one feeble
hand to another, the nation would be pillaged for the purpose of
bestowing a fresh donative on the troops. If the Presbyterians
obstinately stood aloof from the Royalists, the state was lost; and men
might well doubt whether, by the combined exertions of Presbyterians and
Royalists, it could be saved. For the dread of that invincible army was
on all the inhabitants of the island; and the Cavaliers, taught by
a hundred disastrous fields how little numbers can effect against
discipline, were even more completely cowed than the Roundheads.

While the soldiers remained united, all the plots and risings of the
malecontents were ineffectual. But a few days after the second expulsion
of the Rump, came tidings which gladdened the hearts of all who were
attached either to monarchy or to liberty: That mighty force which had,
during many years, acted as one man, and which, while so acting, had
been found irresistible, was at length divided against itself. The army
of Scotland had done good service to the Commonwealth, and was in
the highest state of efficiency. It had borne no part in the late
revolutions, and had seen them with indignation resembling the
indignation which the Roman legions posted on the Danube and the
Euphrates felt, when they learned that the empire had been put up to
sale by the Praetorian Guards. It was intolerable that certain regiments
should, merely because they happened to be quartered near Westminster,
take on themselves to make and unmake several governments in the course
of half a year. If it were fit that the state should be regulated by the
soldiers, those soldiers who upheld the English ascendency on the north
of the Tweed were as well entitled to a voice as those who garrisoned
the Tower of London. There appears to have been less fanaticism among
the troops stationed in Scotland than in any other part of the army; and
their general, George Monk, was himself the very opposite of a zealot.
He had at the commencement of the civil war, borne arms for the King,
had been made prisoner by the Roundheads, had then accepted a commission
from the Parliament, and, with very slender pretensions to saintship,
had raised himself to high commands by his courage and professional
skill. He had been an useful servant to both the Protectors, and had
quietly acquiesced when the officers at Westminster had pulled down
Richard and restored the Long Parliament, and would perhaps have
acquiesced as quietly in the second expulsion of the Long Parliament,
if the provisional government had abstained from giving him cause of
offence and apprehension. For his nature was cautious and somewhat
sluggish; nor was he at all disposed to hazard sure and moderate
advantages for the chalice of obtaining even the most splendid
success. He seems to have been impelled to attack the new rulers of
the Commonwealth less by the hope that, if he overthrew them, he should
become great, than by the fear that, if he submitted to them, he should
not even be secure. Whatever were his motives, he declared himself
the champion of the oppressed civil power, refused to acknowledge the
usurped authority of the provisional government, and, at the head of
seven thousand veterans, marched into England.

This step was the signal for a general explosion. The people everywhere
refused to pay taxes. The apprentices of the City assembled by thousands
and clamoured for a free Parliament. The fleet sailed up the Thames, and
declared against the tyranny of the soldiers. The soldiers, no longer
under the control of one commanding mind, separated into factions. Every
regiment, afraid lest it should be left alone a mark for the vengeance
of the oppressed nation, hastened to make a separate peace. Lambert, who
had hastened northward to encounter the army of Scotland, was abandoned
by his troops, and became a prisoner. During thirteen years the civil
power had, in every conflict, been compelled to yield to the military
power. The military power now humbled itself before the civil power.
The Rump, generally hated and despised, but still the only body in the
country which had any show of legal authority, returned again to the
house from which it had been twice ignominiously expelled.

In the mean time Monk was advancing towards London. Wherever he came,
the gentry flocked round him, imploring him to use his power for the
purpose of restoring peace and liberty to the distracted nation.
The General, coldblooded, taciturn, zealous for no polity and for no
religion, maintained an impenetrable reserve. What were at this time
his plans, and whether he had any plan, may well be doubted. His great
object, apparently, was to keep himself, as long as possible, free to
choose between several lines of action. Such, indeed, is commonly the
policy of men who are, like him, distinguished rather by wariness than
by farsightedness. It was probably not till he had been some days in the
capital that he had made up his mind. The cry of the whole people was
for a free Parliament; and there could be no doubt that a Parliament
really free would instantly restore the exiled family. The Rump and the
soldiers were still hostile to the House of Stuart. But the Rump was
universally detested and despised. The power of the soldiers was indeed
still formidable, but had been greatly diminished by discord. They had
no head. They had recently been, in many parts of the country, arrayed
against each other. On the very day before Monk reached London, there
was a fight in the Strand between the cavalry and the infantry. An
united army had long kept down a divided nation; but the nation was now
united, and the army was divided.

During a short time the dissimulation or irresolution of Monk kept all
parties in a state of painful suspense. At length he broke silence, and
declared for a free Parliament.

As soon as his declaration was known, the whole nation was wild with
delight. Wherever he appeared thousands thronged round him, shouting and
blessing his name. The bells of all England rang joyously: the gutters
ran with ale; and, night after night, the sky five miles round London
was reddened by innumerable bonfires. Those Presbyterian members of the
House of Commons who had many years before been expelled by the army,
returned to their seats, and were hailed with acclamations by great
multitudes, which filled Westminster Hall and Palace Yard. The
Independent leaders no longer dared to show their faces in the streets,
and were scarcely safe within their own dwellings. Temporary provision
was made for the government: writs were issued for a general election;
and then that memorable Parliament, which had, in the course of
twenty eventful years, experienced every variety of fortune, which had
triumphed over its sovereign, which had been enslaved and degraded by
its servants, which had been twice ejected and twice restored, solemnly
decreed its own dissolution.

The result of the elections was such as might have been expected from
the temper of the nation. The new House of Commons consisted, with few
exceptions, of persons friendly to the royal family. The Presbyterians
formed the majority.

That there would be a restoration now seemed almost certain; but whether
there would be a peaceable restoration was matter of painful doubt. The
soldiers were in a gloomy and savage mood. They hated the title of King.
They hated the name of Stuart. They hated Presbyterianism much, and
Prelacy more. They saw with bitter indignation that the close of their
long domination was approaching, and that a life of inglorious toil
and penury was before them. They attributed their ill fortune to the
weakness of some generals, and to the treason of others. One hour
of their beloved Oliver might even now restore the glory which had
departed. Betrayed, disunited, and left without any chief in whom they
could confide, they were yet to be dreaded. It was no light thing to
encounter the rage and despair of fifty thousand fighting men, whose
backs no enemy had ever seen. Monk, and those with whom he acted, were
well aware that the crisis was most perilous. They employed every art
to soothe and to divide the discontented warriors. At the same time
vigorous preparation was made for a conflict. The army of Scotland, now
quartered in London, was kept in good humour by bribes, praises, and
promises. The wealthy citizens grudged nothing to a redcoat, and were
indeed so liberal of their best wine, that warlike saints were sometimes
seen in a condition not very honourable either to their religious or
to their military character. Some refractory regiments Monk ventured
to disband. In the mean time the greatest exertions were made by the
provisional government, with the strenuous aid of the whole body of
the gentry and magistracy, to organise the militia. In every county the
trainbands were held ready to march; and this force cannot be estimated
at less than a hundred and twenty thousand men. In Hyde Park twenty
thousand citizens, well armed and accoutred, passed in review, and
showed a spirit which justified the hope that, in case of need, they
would fight manfully for their shops and firesides. The fleet was
heartily with the nation. It was a stirring time, a time of anxiety, yet
of hope. The prevailing opinion was that England would be delivered, but
not without a desperate and bloody struggle, and that the class which
had so long ruled by the sword would perish by the sword.

Happily the dangers of a conflict were averted. There was indeed one
moment of extreme peril. Lambert escaped from his confinement, and
called his comrades to arms. The flame of civil war was actually
rekindled; but by prompt and vigorous exertion it was trodden out before
it had time to spread. The luckless imitator of Cromwell was again
a prisoner. The failure of his enterprise damped the spirit of the
soldiers; and they sullenly resigned themselves to their fate.

The new Parliament, which, having been called without the royal writ, is
more accurately described as a Convention, met at Westminster. The
Lords repaired to the hall, from which they had, during more than eleven
years, been excluded by force. Both Houses instantly invited the King to
return to his country. He was proclaimed with pomp never before known.
A gallant fleet convoyed him from Holland to the coast of Kent. When he
landed, the cliffs of Dover were covered by thousands of gazers, among
whom scarcely one could be found who was not weeping with delight. The
journey to London was a continued triumph. The whole road from Rochester
was bordered by booths and tents, and looked like an interminable fair.
Everywhere flags were flying, bells and music sounding, wine and ale
flowing in rivers to the health of him whose return was the return of
peace, of law, and of freedom. But in the midst of the general joy, one
spot presented a dark and threatening aspect. On Blackheath the army was
drawn up to welcome the sovereign. He smiled, bowed, and extended his
hand graciously to the lips of the colonels and majors. But all his
courtesy was vain. The countenances of the soldiers were sad and
lowering; and had they given way to their feelings, the festive pageant
of which they reluctantly made a part would have had a mournful and
bloody end. But there was no concert among them. Discord and defection
had left them no confidence in their chiefs or in each other. The
whole array of the City of London was under arms. Numerous companies of
militia had assembled from various parts of the realm, under the command
of loyal noblemen and gentlemen, to welcome the King. That great day
closed in peace; and the restored wanderer reposed safe in the palace of
his ancestors.




\chapter{CHAPTER II.}

THE history of England, during the seventeenth century, is the history
of the transformation of a limited monarchy, constituted after the
fashion of the middle ages, into a limited monarchy suited to that more
advanced state of society in which the public charges can no longer be
borne by the estates of the crown, and in which the public defence
can no longer be entrusted to a feudal militia. We have seen that the
politicians who were at the head of the Long Parliament made, in 1642,
a great effort to accomplish this change by transferring, directly
and formally, to the estates of the realm the choice of ministers, the
command of the army, and the superintendence of the whole executive
administration. This scheme was, perhaps, the best that could then be
contrived: but it was completely disconcerted by the course which the
civil war took. The Houses triumphed, it is true; but not till after
such a struggle as made it necessary for them to call into existence
a power which they could not control, and which soon began to domineer
over all orders and all parties: During a few years, the evils
inseparable from military government were, in some degree, mitigated
by the wisdom and magnanimity of the great man who held the supreme
command. But, when the sword, which he had wielded, with energy indeed,
but with energy always guided by good sense and generally tempered by
good nature, had passed to captains who possessed neither his abilities
nor his virtues. It seemed too probable that order and liberty would
perish in one ignominious ruin.

That ruin was happily averted. It has been too much the practice of
writers zealous for freedom to represent the Restoration as a disastrous
event, and to condemn the folly or baseness of that Convention, which
recalled the royal family without exacting new securities against
maladministration. Those who hold this language do not comprehend the
real nature of the crisis which followed the deposition of Richard
Cromwell. England was in imminent danger of falling under the tyranny of
a succession of small men raised up and pulled down by military caprice.
To deliver the country from the domination of the soldiers was the first
object of every enlightened patriot: but it was an object which, while
the soldiers were united, the most sanguine could scarcely expect to
attain. On a sudden a gleam of hope appeared. General was opposed to
general, army to army. On the use which might be made of one auspicious
moment depended the future destiny of the nation. Our ancestors used
that moment well. They forgot old injuries, waved petty scruples,
adjourned to a more convenient season all dispute about the reforms
which our institutions needed, and stood together, Cavaliers and
Roundheads, Episcopalians and Presbyterians, in firm union, for the
old laws of the land against military despotism. The exact partition of
power among King, Lords, and Commons might well be postponed till it
had been decided whether England should be governed by King, Lords,
and Commons, or by cuirassiers and pikemen. Had the statesmen of the
Convention taken a different course, had they held long debates on the
principles of government, had they drawn up a new constitution and sent
it to Charles, had conferences been opened, had couriers been passing
and repassing during some weeks between Westminster and the Netherlands,
with projects and counterprojects, replies by Hyde and rejoinders by
Prynne, the coalition on which the public safety depended would have
been dissolved: the Presbyterians and Royalists would certainly have
quarrelled: the military factions might possibly have been reconciled;
and the misjudging friends of liberty might long have regretted, under
a rule worse than that of the worst Stuart, the golden opportunity which
had been suffered to escape.

The old civil polity was, therefore, by the general consent of both the
great parties, reestablished. It was again exactly what it had been when
Charles the First, eighteen years before, withdrew from his capital. All
those acts of the Long Parliament which had received the royal assent
were admitted to be still in full force. One fresh concession, a
concession in which the Cavaliers were even more deeply interested than
the Roundheads, was easily obtained from the restored King. The military
tenure of land had been originally created as a means of national
defence. But in the course of ages whatever was useful in the
institution had disappeared; and nothing was left but ceremonies and
grievances. A landed proprietor who held an estate under the crown by
knight service,--and it was thus that most of the soil of England was
held,--had to pay a large fine on coming to his property. He could not
alienate one acre without purchasing a license. When he died, if his
domains descended to an infant, the sovereign was guardian, and was not
only entitled to great part of the rents during the minority, but could
require the ward, under heavy penalties, to marry any person of suitable
rank. The chief bait which attracted a needy sycophant to the court was
the hope of obtaining as the reward of servility and flattery, a royal
letter to an heiress. These abuses had perished with the monarchy. That
they should not revive with it was the wish of every landed gentleman in
the kingdom. They were, therefore, solemnly abolished by statute; and
no relic of the ancient tenures in chivalry was allowed to remain except
those honorary services which are still, at a coronation, rendered to
the person of the sovereign by some lords of manors.

The troops were now to be disbanded. Fifty thousand men, accustomed to
the profession of arms, were at once thrown on the world: and experience
seemed to warrant the belief that this change would produce much misery
and crime, that the discharged veterans would be seen begging in every
street, or that they would be driven by hunger to pillage. But no such
result followed. In a few months there remained not a trace indicating
that the most formidable army in the world had just been absorbed into
the mass of the community. The Royalists themselves confessed that, in
every department of honest industry the discarded warriors prospered
beyond other men, that none was charged with any theft or robbery,
that none was heard to ask an alms, and that, if a baker, a mason, or a
waggoner attracted notice by his diligence and sobriety, he was in all
probability one of Oliver's old soldiers.

The military tyranny had passed away; but it had left deep and enduring
traces in the public mind. The name of standing army was long held in
abhorrence: and it is remarkable that this feeling was even stronger
among the Cavaliers than among the Roundheads. It ought to be considered
as a most fortunate circumstance that, when our country was, for the
first and last time, ruled by the sword, the sword was in the hands,
not of legitimate princes, but of those rebels who slew the King and
demolished the Church. Had a prince with a title as good as that of
Charles, commanded an army as good as that of Cromwell, there would
have been little hope indeed for the liberties of England. Happily that
instrument by which alone the monarchy could be made absolute became an
object of peculiar horror and disgust to the monarchical party, and long
continued to be inseparably associated in the imagination of Royalists
and Prelatists with regicide and field preaching. A century after the
death of Cromwell, the Tories still continued to clamour against every
augmentation of the regular soldiery, and to sound the praise of a
national militia. So late as the year 1786, a minister who enjoyed no
common measure of their confidence found it impossible to overcome their
aversion to his scheme of fortifying the coast: nor did they ever look
with entire complacency on the standing army, till the French Revolution
gave a new direction to their apprehensions.

The coalition which had restored the King terminated with the danger
from which it had sprung; and two hostile parties again appeared
ready for conflict. Both, indeed, were agreed as to the propriety of
inflicting punishment on some unhappy men who were, at that moment,
objects of almost universal hatred. Cromwell was no more; and those who
had fled before him were forced to content themselves with the miserable
satisfaction of digging up, hanging, quartering, and burning the remains
of the greatest prince that has ever ruled England.

Other objects of vengeance, few indeed, yet too many, were found among
the republican chiefs. Soon, however, the conquerors, glutted with the
blood of the regicides, turned against each other. The Roundheads,
while admitting the virtues of the late King, and while condemning the
sentence passed upon him by an illegal tribunal, yet maintained that his
administration had been, in many things, unconstitutional, and that
the Houses had taken arms against him from good motives and on strong
grounds. The monarchy, these politicians conceived, had no worse enemy
than the flatterer who exalted prerogative above the law, who condemned
all opposition to regal encroachments, and who reviled, not only
Cromwell and Harrison, but Pym and Hampden, as traitors. If the King
wished for a quiet and prosperous reign, he must confide in those who,
though they had drawn the sword in defence of the invaded privileges of
Parliament, had yet exposed themselves to the rage of the soldiers in
order to save his father, and had taken the chief part in bringing back
the royal family.

The feeling of the Cavaliers was widely different. During eighteen years
they had, through all vicissitudes, been faithful to the Crown. Having
shared the distress of their prince, were they not to share his triumph?
Was no distinction to be made between them and the disloyal subject who
had fought against his rightful sovereign, who had adhered to Richard
Cromwell, and who had never concurred in the restoration of the Stuarts,
till it appeared that nothing else could save the nation from the
tyranny of the army? Grant that such a man had, by his recent services,
fairly earned his pardon. Yet were his services, rendered at the
eleventh hour, to be put in comparison with the toils and sufferings of
those who had borne the burden and heat of the day? Was he to be ranked
with men who had no need of the royal clemency, with men who had, in
every part of their lives, merited the royal gratitude? Above all, was
he to be suffered to retain a fortune raised out of the substance of the
ruined defenders of the throne? Was it not enough that his head and his
patrimonial estate, a hundred times forfeited to justice, were secure,
and that he shared, with the rest of the nation, in the blessings of
that mild government of which he had long been the foe? Was it necessary
that he should be rewarded for his treason at the expense of men whose
only crime was the fidelity with which they had observed their oath of
allegiance. And what interest had the King in gorging his old enemies
with prey torn from his old friends? What confidence could be placed in
men who had opposed their sovereign, made war on him, imprisoned him,
and who, even now, instead of hanging down their heads in shame and
contrition, vindicated all that they had done, and seemed to think that
they had given an illustrious proof of loyalty by just stopping short of
regicide? It was true they had lately assisted to set up the throne: but
it was not less true that they had previously pulled it down, and that
they still avowed principles which might impel them to pull it down
again. Undoubtedly it might be fit that marks of royal approbation
should be bestowed on some converts who had been eminently useful: but
policy, as well as justice and gratitude, enjoined the King to give the
highest place in his regard to those who, from first to last, through
good and evil, had stood by his house. On these grounds the Cavaliers
very naturally demanded indemnity for all that they had suffered, and
preference in the distribution of the favours of the Crown. Some violent
members of the party went further, and clamoured for large categories of
proscription.

The political feud was, as usual, exasperated by a religious feud.
The King found the Church in a singular state. A short time before the
commencement of the civil war, his father had given a reluctant assent
to a bill, strongly supported by Falkland, which deprived the Bishops
of their seats in the House of Lords: but Episcopacy and the Liturgy had
never been abolished by law. The Long Parliament, however, had passed
ordinances which had made a complete revolution in Church government
and in public worship. The new system was, in principle, scarcely less
Erastian than that which it displaced. The Houses, guided chiefly by
the counsels of the accomplished Selden, had determined to keep the
spiritual power strictly subordinate to the temporal power. They had
refused to declare that any form of ecclesiastical polity was of divine
origin; and they had provided that, from all the Church courts, an
appeal should lie in the last resort to Parliament. With this highly
important reservation, it had been resolved to set up in England a
hierarchy closely resembling that which now exists in Scotland. The
authority of councils, rising one above another in regular gradation,
was substituted for the authority of Bishops and Archbishops. The
Liturgy gave place to the Presbyterian Directory. But scarcely had
the new regulations been framed, when the Independents rose to supreme
influence in the state. The Independents had no disposition to enforce
the ordinances touching classical, provincial, and national synods.
Those ordinances, therefore, were never carried into full execution. The
Presbyterian system was fully established nowhere but in Middlesex and
Lancashire. In the other fifty counties almost every parish seems to
have been unconnected with the neighbouring parishes. In some districts,
indeed, the ministers formed themselves into voluntary associations, for
the purpose of mutual help and counsel; but these associations had no
coercive power. The patrons of livings, being now checked by neither
Bishop nor Presbytery, would have been at liberty to confide the cure
of souls to the most scandalous of mankind, but for the arbitrary
intervention of Oliver. He established, by his own authority, a board
of commissioners, called Triers. Most of these persons were Independent
divines; but a few Presbyterian ministers and a few laymen had seats.
The certificate of the Triers stood in the place both of institution
and of induction; and without such a certificate no person could hold a
benefice. This was undoubtedly one of the most despotic acts ever done
by any English ruler. Yet, as it was generally felt that, without some
such precaution, the country would be overrun by ignorant and drunken
reprobates, bearing the name and receiving the pay of ministers,
some highly respectable persons, who were not in general friendly
to Cromwell, allowed that, on this occasion, he had been a public
benefactor. The presentees whom the Triers had approved took possession
of the rectories, cultivated the glebe lands, collected the tithes,
prayed without book or surplice, and administered the Eucharist to
communicants seated at long tables.

Thus the ecclesiastical polity of the realm was in inextricable
confusion. Episcopacy was the form of government prescribed by the old
law which was still unrepealed. The form of government prescribed by
parliamentary ordinance was Presbyterian. But neither the old law
nor the parliamentary ordinance was practically in force. The Church
actually established may be described as an irregular body made up of a
few Presbyteries and many Independent congregations, which were all held
down and held together by the authority of the government.

Of those who had been active in bringing back the King, many were
zealous for Synods and for the Directory, and many were desirous to
terminate by a compromise the religious dissensions which had long
agitated England. Between the bigoted followers of Laud and the bigoted
followers of Knox there could be neither peace nor truce: but it did
not seem impossible to effect an accommodation between the moderate
Episcopalians of the school of Usher and the moderate Presbyterians
of the school of Baxter. The moderate Episcopalians would admit that
a Bishop might lawfully be assisted by a council. The moderate
Presbyterians would not deny that each provincial assembly might
lawfully have a permanent president, and that this president might
lawfully be called a Bishop. There might be a revised Liturgy which
should not exclude extemporaneous prayer, a baptismal service in
which the sign of the cross might be used or omitted at discretion, a
communion service at which the faithful might sit if their conscience
forbade them to kneel. But to no such plan could the great bodies of the
Cavaliers listen with patience. The religious members of that party were
conscientiously attached to the whole system of their Church. She had
been dear to their murdered King. She had consoled them in defeat and
penury. Her service, so often whispered in an inner chamber during the
season of trial, had such a charm for them that they were unwilling to
part with a single response. Other Royalists, who made little presence
to piety, yet loved the episcopal church because she was the foe of
their foes. They valued a prayer or a ceremony, not on account of the
comfort which it conveyed to themselves, but on account of the vexation
which it gave to the Roundheads, and were so far from being disposed to
purchase union by concession that they objected to concession chiefly
because it tended to produce union.

Such feelings, though blamable, were natural, and not wholly
inexcusable. The Puritans had undoubtedly, in the day of their power,
given cruel provocation. They ought to have learned, if from nothing
else, yet from their own discontents, from their own struggles, from
their own victory, from the fall of that proud hierarchy by which they
had been so heavily oppressed, that, in England, and in the seventeenth
century, it was not in the power of the civil magistrate to drill the
minds of men into conformity with his own system of theology. They
proved, however, as intolerant and as meddling as ever Laud had been.
They interdicted under heavy penalties the use of the Book of Common
Prayer, not only in churches, but even in private houses. It was a
crime in a child to read by the bedside of a sick parent one of those
beautiful collects which had soothed the griefs of forty generations
of Christians. Severe punishments were denounced against such as
should presume to blame the Calvinistic mode of worship. Clergymen of
respectable character were not only ejected from their benefices by
thousands, but were frequently exposed to the outrages of a fanatical
rabble. Churches and sepulchres, fine works of art and curious remains
of antiquity, were brutally defaced. The Parliament resolved that all
pictures in the royal collection which contained representations of
Jesus or of the Virgin Mother should be burned. Sculpture fared as
ill as painting. Nymphs and Graces, the work of Ionian chisels, were
delivered over to Puritan stonemasons to be made decent. Against the
lighter vices the ruling faction waged war with a zeal little tempered
by humanity or by common sense. Sharp laws were passed against betting.
It was enacted that adultery should be punished with death. The illicit
intercourse of the sexes, even where neither violence nor seduction was
imputed, where no public scandal was given, where no conjugal right was
violated, was made a misdemeanour. Public amusements, from the masques
which were exhibited at the mansions of the great down to the wrestling
matches and grinning matches on village greens, were vigorously
attacked. One ordinance directed that all the Maypoles in England should
forthwith be hewn down. Another proscribed all theatrical diversions.
The playhouses were to be dismantled, the spectators fined, the
actors whipped at the cart's tail. Rope-dancing, puppet-shows, bowls,
horse-racing, were regarded with no friendly eye. But bearbaiting, then
a favourite diversion of high and low, was the abomination which
most strongly stirred the wrath of the austere sectaries. It is to be
remarked that their antipathy to this sport had nothing in common with
the feeling which has, in our own time, induced the legislature to
interfere for the purpose of protecting beasts against the wanton
cruelty of men. The Puritan hated bearbaiting, not because it gave pain
to the bear, but because it gave pleasure to the spectators. Indeed,
he generally contrived to enjoy the double pleasure of tormenting both
spectators and bear. 
%[16]
\footnote{ How little compassion for the bear had to do with the
matter is sufficiently proved by the following extract from a paper
entitled A perfect Diurnal of some Passages of Parliament, and from
other Parts of the Kingdom, from Monday July 24th, to Monday July 31st,
1643. "Upon the Queen's coming from Holland, she brought with her,
besides a company of savage-like ruffians, a company of savage bears,
to what purpose you may judge by the sequel. Those bears were left about
Newark, and were brought into country towns constantly on the Lord's
day to be baited, such is the religion those here related would settle
amongst us; and, if any went about to hinder or but speak against their
damnable profanations, they were presently noted as Roundheads
and Puritans, and sure to be plundered for it. But some of Colonel
Cromwell's forces coming by accident into Uppingham town, in Rutland,
on the Lord's day, found these bears playing there in the usual manner,
and, in the height of their sport, caused them to be seized upon, tied
to a tree and shot." This was by no means a solitary instance. Colonel
Pride, when Sheriff of Surrey, ordered the beasts in the bear garden
of Southwark to be killed. He is represented by a loyal satirist as
defending the act thus: "The first thing that is upon my spirits is the
killing of the bears, for which the people hate me, and call me all the
names in the rainbow. But did not David kill a bear? Did not the Lord
Deputy Ireton kill a bear? Did not another lord of ours kill five
bears?"-Last Speech and Dying Words of Thomas pride.}


Perhaps no single circumstance more strongly illustrates the temper of
the precisians than their conduct respecting Christmas day. Christmas
had been, from time immemorial, the season of joy and domestic
affection, the season when families assembled, when children came home
from school, when quarrels were made up, when carols were heard in every
street, when every house was decorated with evergreens, and every
table was loaded with good cheer. At that season all hearts not utterly
destitute of kindness were enlarged and softened. At that season the
poor were admitted to partake largely of the overflowings of the wealth
of the rich, whose bounty was peculiarly acceptable on account of
the shortness of the days and of the severity of the weather. At that
season, the interval between landlord and tenant, master and servant,
was less marked than through the rest of the year. Where there is much
enjoyment there will be some excess: yet, on the whole, the spirit in
which the holiday was kept was not unworthy of a Christian festival. The
long Parliament gave orders, in 1644, that the twenty-fifth of December
should be strictly observed as a fast, and that all men should pass it
in humbly bemoaning the great national sin which they and their fathers
had so often committed on that day by romping under the mistletoe,
eating boar's head, and drinking ale flavored with roasted apples. No
public act of that time seems to have irritated the common people more.
On the next anniversary of the festival formidable riots broke out in
many places. The constables were resisted, the magistrates insulted, the
houses of noted zealots attacked, and the prescribed service of the day
openly read in the churches.

Such was the spirit of the extreme Puritans, both Presbyterian and
Independent. Oliver, indeed, was little disposed to be either
a persecutor or a meddler. But Oliver, the head of a party, and
consequently, to a great extent, the slave of a party, could not
govern altogether according to his own inclinations. Even under his
administration many magistrates, within their own jurisdiction, made
themselves as odious as Sir Hudibras, interfered with all the pleasures
of the neighbourhood, dispersed festive meetings, and put fiddlers in
the stocks. Still more formidable was the zeal of the soldiers. In every
village where they appeared there was an end of dancing, bellringing,
and hockey. In London they several times interrupted theatrical
performances at which the Protector had the judgment and good nature to
connive.

With the fear and hatred inspired by such a tyranny contempt was largely
mingled. The peculiarities of the Puritan, his look, his dress,
his dialect, his strange scruples, had been, ever since the time of
Elizabeth, favourite subjects with mockers. But these peculiarities
appeared far more grotesque in a faction which ruled a great empire
than in obscure and persecuted congregations. The cant, which had moved
laughter when it was heard on the stage from Tribulation Wholesome and
Zeal-of-the-Land Busy, was still more laughable when it proceeded from
the lips of Generals and Councillors of State. It is also to be noticed
that during the civil troubles several sects had sprung into existence,
whose eccentricities surpassed anything that had before been seen in
England. A mad tailor, named Lodowick Muggleton, wandered from pothouse
to pothouse, tippling ale, and denouncing eternal torments against those
who refused to believe, on his testimony, that the Supreme Being was
only six feet high, and that the sun was just four miles from the earth.

%[17]
\footnote{ See Penn's New Witnesses proved Old Heretics, and
Muggleton's works, passim.}
 George Fox had raised a tempest of derision by proclaiming that it
was a violation of Christian sincerity to designate a single person by a
plural pronoun, and that it was an idolatrous homage to Janus and Woden
to talk about January and Wednesday. His doctrine, a few years later,
was embraced by some eminent men, and rose greatly in the public
estimation. But at the time of the Restoration the Quakers were
popularly regarded as the most despicable of fanatics. By the Puritans
they were treated with severity here, and were persecuted to the
death in New England. Nevertheless the public, which seldom makes nice
distinctions, often confounded the Puritan with the Quaker. Both were
schismatics. Both hated episcopacy and the Liturgy. Both had what seemed
extravagant whimsies about dress, diversions and postures. Widely as the
two differed in opinion, they were popularly classed together as canting
schismatics; and whatever was ridiculous or odious in either increased
the scorn and aversion which the multitude felt for both.

Before the civil wars, even those who most disliked the opinions and
manners of the Puritan were forced to admit that his moral conduct was
generally, in essentials, blameless; but this praise was now no longer
bestowed, and, unfortunately, was no longer deserved. The general fate
of sects is to obtain a high reputation for sanctity while they are
oppressed, and to lose it as soon as they become powerful: and the
reason is obvious. It is seldom that a man enrolls himself in a
proscribed body from any but conscientious motives. Such a body,
therefore, is composed, with scarcely an exception, of sincere persons.
The most rigid discipline that can be enforced within a religious
society is a very feeble instrument of purification, when compared with
a little sharp persecution from without. We may be certain that very few
persons, not seriously impressed by religious convictions, applied for
baptism while Diocletian was vexing the Church, or joined themselves
to Protestant congregations at the risk of being burned by Bonner. But,
when a sect becomes powerful, when its favour is the road to riches and
dignities, worldly and ambitious men crowd into it, talk its language,
conform strictly to its ritual, mimic its peculiarities, and frequently
go beyond its honest members in all the outward indications of zeal. No
discernment, no watchfulness, on the part of ecclesiastical rulers, can
prevent the intrusion of such false brethren. The tares and wheat must
grow together. Soon the world begins to find out that the godly are
not better than other men, and argues, with some justice, that, if not
better, they must be much worse. In no long time all those signs which
were formerly regarded as characteristic of a saint are regarded as
characteristic of a knave.

Thus it was with the English Nonconformists. They had been oppressed;
and oppression had kept them a pure body. They then became supreme in
the state. No man could hope to rise to eminence and command but by
their favour. Their favour was to be gained only by exchanging with
them the signs and passwords of spiritual fraternity. One of the
first resolutions adopted by Barebone's Parliament, the most intensely
Puritanical of all our political assemblies, was that no person should
be admitted into the public service till the House should be satisfied
of his real godliness. What were then considered as the signs of real
godliness, the sadcoloured dress, the sour look, the straight hair,
the nasal whine, the speech interspersed with quaint texts, the Sunday,
gloomy as a Pharisaical Sabbath, were easily imitated by men to whom all
religions were the same. The sincere Puritans soon found themselves lost
in a multitude, not merely of men of the world, but of the very worst
sort of men of the world. For the most notorious libertine who had
fought under the royal standard might justly be thought virtuous
when compared with some of those who, while they talked about sweet
experiences and comfortable scriptures, lived in the constant practice
of fraud, rapacity, and secret debauchery. The people, with a rashness
which we may justly lament, but at which we cannot wonder, formed their
estimate of the whole body from these hypocrites. The theology, the
manners, the dialect of the Puritan were thus associated in the public
mind with the darkest and meanest vices. As soon as the Restoration
had made it safe to avow enmity to the party which had so long been
predominant, a general outcry against Puritanism rose from every corner
of the kingdom, and was often swollen by the voices of those very
dissemblers whose villany had brought disgrace on the Puritan name.

Thus the two great parties, which, after a long contest, had for a
moment concurred in restoring monarchy, were, both in politics and in
religion, again opposed to each other. The great body of the nation
leaned to the Royalists. The crimes of Strafford and Laud, the excesses
of the Star Chamber and of the High Commission, the great services
which the Long Parliament had, during the first year of its existence,
rendered to the state, had faded from the minds of men. The execution of
Charles the First, the sullen tyranny of the Rump, the violence of the
army, were remembered with loathing; and the multitude was inclined to
hold all who had withstood the late King responsible for his death and
for the subsequent disasters.

The House of Commons, having been elected while the Presbyterians were
dominant, by no means represented the general sense of the people. Most
of the members, while execrating Cromwell and Bradshaw, reverenced the
memory of Essex and of Pym. One sturdy Cavalier, who ventured to declare
that all who had drawn the sword against Charles the First were as much
traitors as those who kind cut off his head, was called to order, placed
at the bar, and reprimanded by the Speaker. The general wish of the
House undoubtedly was to settle the ecclesiastical disputes in a manner
satisfactory to the moderate Puritans. But to such a settlement both the
court and the nation were averse.

The restored King was at this time more loved by the people than any of
his predecessors had ever been. The calamities of his house, the heroic
death of his father, his own long sufferings and romantic adventures,
made him an object of tender interest. His return had delivered the
country from an intolerable bondage. Recalled by the voice of both the
contending factions, he was in a position which enabled him to arbitrate
between them; and in some respects he was well qualified for the task.
He had received from nature excellent parts and a happy temper. His
education had been such as might have been expected to develope his
understanding, and to form him to the practice of every public and
private virtue. He had passed through all varieties of fortune, and had
seen both sides of human nature. He had, while very young, been driven
forth from a palace to a life of exile. penury, and danger. He had, at
the age when the mind and body are in their highest perfection, and when
the first effervescence of boyish passions should have subsided, been
recalled from his wanderings to wear a crown. He had been taught by
bitter experience how much baseness, perfidy, and ingratitude may lie
hid under the obsequious demeanor of courtiers. He had found, on the
other hand, in the huts of the poorest, true nobility of soul. When
wealth was offered to any who would betray him, when death was denounced
against all who should shelter him, cottagers and serving men had kept
his secret truly, and had kissed his hand under his mean disguises with
as much reverence as if he had been seated on his ancestral throne. From
such a school it might have been expected that a young man who wanted
neither abilities nor amiable qualities would have come forth a great
and good King. Charles came forth from that school with social habits,
with polite and engaging manners, and with some talent for lively
conversation, addicted beyond measure to sensual indulgence, fond of
sauntering and of frivolous amusements, incapable of selfdenial and of
exertion, without faith in human virtue or in human attachment without
desire of renown, and without sensibility to reproach. According to him,
every person was to be bought: but some people haggled more about their
price than others; and when this haggling was very obstinate and very
skilful it was called by some fine name. The chief trick by which clever
men kept up the price of their abilities was called integrity. The chief
trick by which handsome women kept up the price of their beauty was
called modesty. The love of God, the love of country, the love of
family, the love of friends, were phrases of the same sort, delicate
and convenient synonymes for the love of self. Thinking thus of mankind,
Charles naturally cared very little what they thought of him. Honour and
shame were scarcely more to him than light and darkness to the blind.
His contempt of flattery has been highly commended, but seems, when
viewed in connection with the rest of his character, to deserve no
commendation. It is possible to be below flattery as well as above it.
One who trusts nobody will not trust sycophants. One who does not value
real glory will not value its counterfeit.

It is creditable to Charles's temper that, ill as he thought of his
species, he never became a misanthrope. He saw little in men but what
was hateful. Yet he did not hate them. Nay, he was so far humane that it
was highly disagreeable to him to see their sufferings or to hear their
complaints. This, however, is a sort of humanity which, though amiable
and laudable in a private man whose power to help or hurt is bounded by
a narrow circle, has in princes often been rather a vice than a virtue.
More than one well disposed ruler has given up whole provinces to rapine
and oppression, merely from a wish to see none but happy faces round his
own board and in his own walks. No man is fit to govern great societies
who hesitates about disobliging the few who have access to him, for the
sake of the many whom he will never see. The facility of Charles was
such as has perhaps never been found in any man of equal sense. He was a
slave without being a dupe. Worthless men and women, to the very bottom
of whose hearts he saw, and whom he knew to be destitute of affection
for him and undeserving of his confidence, could easily wheedle him out
of titles, places, domains, state secrets and pardons. He bestowed
much; yet he neither enjoyed the pleasure nor acquired the fame of
beneficence. He never gave spontaneously; but it was painful to him to
refuse. The consequence was that his bounty generally went, not to those
who deserved it best, nor even to those whom he liked best, but to the
most shameless and importunate suitor who could obtain an audience.

The motives which governed the political conduct of Charles the Second
differed widely from those by which his predecessor and his successor
were actuated. He was not a man to be imposed upon by the patriarchal
theory of government and the doctrine of divine right. He was utterly
without ambition. He detested business, and would sooner have abdicated
his crown than have undergone the trouble of really directing the
administration. Such was his aversion to toil, and such his ignorance of
affairs, that the very clerks who attended him when he sate in council
could not refrain from sneering at his frivolous remarks, and at his
childish impatience. Neither gratitude nor revenge had any share
in determining his course; for never was there a mind on which both
services and injuries left such faint and transitory impressions.
He wished merely to be a King such as Lewis the Fifteenth of France
afterwards was; a King who could draw without limit on the treasury for
the gratification of his private tastes, who could hire with wealth and
honours persons capable of assisting him to kill the time, and who,
even when the state was brought by maladministration to the depths of
humiliation and to the brink of ruin, could still exclude unwelcome
truth from the purlieus of his own seraglio, and refuse to see and hear
whatever might disturb his luxurious repose. For these ends, and for
these ends alone, he wished to obtain arbitrary power, if it could
be obtained without risk or trouble. In the religious disputes
which divided his Protestant subjects his conscience was not at all
interested. For his opinions oscillated in contented suspense between
infidelity and Popery. But, though his conscience was neutral in the
quarrel between the Episcopalians and the Presbyterians, his taste was
by no means so. His favourite vices were precisely those to which the
Puritans were least indulgent. He could not get through one day without
the help of diversions which the Puritans regarded as sinful. As a man
eminently well bred, and keenly sensible of the ridiculous, he was moved
to contemptuous mirth by the Puritan oddities. He had indeed some reason
to dislike the rigid sect. He had, at the age when the passions are
most impetuous and when levity is most pardonable, spent some months in
Scotland, a King in name, but in fact a state prisoner in the hands
of austere Presbyterians. Not content with requiring him to conform to
their worship and to subscribe their Covenant, they had watched all
his motions, and lectured him on all his youthful follies. He had been
compelled to give reluctant attendance at endless prayers and sermons,
and might think himself fortunate when he was not insolently reminded
from the pulpit of his own frailties, of his father's tyranny, and of
his mother's idolatry. Indeed he had been so miserable during this part
of his life that the defeat which made him again a wanderer might be
regarded as a deliverance rather than as a calamity. Under the influence
of such feelings as these Charles was desirous to depress the party
which had resisted his father.

The King's brother, James Duke of York, took the same side. Though a
libertine, James was diligent, methodical, and fond of authority and
business. His understanding was singularly slow and narrow, and his
temper obstinate, harsh, and unforgiving. That such a prince should have
looked with no good will on the free institutions of England, and on the
party which was peculiarly zealous for those institutions, can excite
no surprise. As yet the Duke professed himself a member of the Anglican
Church but he had already shown inclinations which had seriously alarmed
good Protestants.

The person on whom devolved at this time the greatest part of the labour
of governing was Edward Hyde, Chancellor of the realm, who was soon
created Earl of Clarendon. The respect which we justly feel for
Clarendon as a writer must not blind us to the faults which he committed
as a statesman. Some of those faults, however, are explained and excused
by the unfortunate position in which he stood. He had, during the first
year of the Long Parliament, been honourably distinguished among the
senators who laboured to redress the grievances of the nation. One
of the most odious of those grievances, the Council of York, had been
removed in consequence chiefly of his exertions. When the great schism
took place, when the reforming party and the conservative party first
appeared marshalled against each other, he, with many wise and good men,
took the conservative side. He thenceforward followed the fortunes of
the court, enjoyed as large a share of the confidence of Charles the
First as the reserved nature and tortuous policy of that prince allowed
to any minister, and subsequently shared the exile and directed the
political conduct of Charles the Second. At the Restoration Hyde became
chief minister. In a few months it was announced that he was closely
related by affinity to the royal house. His daughter had become, by a
secret marriage, Duchess of York. His grandchildren might perhaps wear
the crown. He was raised by this illustrious connection over the heads
of the old nobility of the land, and was for a time supposed to be
allpowerful. In some respects he was well fitted for his great place. No
man wrote abler state papers. No man spoke with more weight and dignity
in Council and in Parliament. No man was better acquainted with general
maxims of statecraft. No man observed the varieties of character with a
more discriminating eye. It must be added that he had a strong sense of
moral and religious obligation, a sincere reverence for the laws of his
country, and a conscientious regard for the honour and interest of the
Crown. But his temper was sour, arrogant, and impatient of opposition.
Above all, he had been long an exile; and this circumstance alone would
have completely disqualified him for the supreme direction of affairs.
It is scarcely possible that a politician, who has been compelled by
civil troubles to go into banishment, and to pass many of the best years
of his life abroad, can be fit, on the day on which he returns to his
native land, to be at the head of the government. Clarendon was no
exception to this rule. He had left England with a mind heated by a
fierce conflict which had ended in the downfall of his party and of his
own fortunes. From 1646 to 1660 he had lived beyond sea, looking on all
that passed at home from a great distance, and through a false medium.
His notions of public affairs were necessarily derived from the
reports of plotters, many of whom were ruined and desperate men. Events
naturally seemed to him auspicious, not in proportion as they increased
the prosperity and glory of the nation, but in proportion as they tended
to hasten the hour of his own return. His wish, a wish which he has not
disguised, was that, till his countrymen brought back the old line, they
might never enjoy quiet or freedom. At length he returned; and, without
having a single week to look about him, to mix with society, to note
the changes which fourteen eventful years had produced in the national
character and feelings, he was at once set to rule the state. In such
circumstances, a minister of the greatest tact and docility would
probably have fallen into serious errors. But tact and docility made no
part of the character of Clarendon. To him England was still the
England of his youth; and he sternly frowned down every theory and every
practice which had sprung up during his own exile. Though he was far
from meditating any attack on the ancient and undoubted power of the
House of Commons, he saw with extreme uneasiness the growth of that
power. The royal prerogative, for which he had long suffered, and by
which he had at length been raised to wealth and dignity, was sacred
in his eyes. The Roundheads he regarded both with political and with
personal aversion. To the Anglican Church he had always been strongly
attached, and had repeatedly, where her interests were concerned,
separated himself with regret from his dearest friends. His zeal for
Episcopacy and for the Book of Common Prayer was now more ardent than
ever, and was mingled with a vindictive hatred of the Puritans, which
did him little honour either as a statesman or as a Christian.

While the House of Commons which had recalled the royal family was
sitting, it was impossible to effect the re-establishment of the
old ecclesiastical system. Not only were the intentions of the court
strictly concealed, but assurances which quieted the minds of the
moderate Presbyterians were given by the King in the most solemn manner.
He had promised, before his restoration, that he would grant liberty of
conscience to his subjects. He now repeated that promise, and added
a promise to use his best endeavours for the purpose of effecting a
compromise between the contending sects. He wished, he said, to see the
spiritual jurisdiction divided between bishops and synods. The Liturgy
should be revised by a body of learned divines, one-half of whom should
be Presbyterians. The questions respecting the surplice, the posture at
the Eucharist, and the sign of the cross in baptism, should be settled
in a way which would set tender consciences at ease. When the King
had thus laid asleep the vigilance of those whom he most feared, he
dissolved the Parliament. He had already given his assent to an act by
which an amnesty was granted, with few exceptions, to all who, during
the late troubles, had been guilty of political offences. He had also
obtained from the Commons a grant for life of taxes, the annual product
of which was estimated at twelve hundred thousand pounds. The actual
income, indeed, during some years, amounted to little more than a
million: but this sum, together with the hereditary revenue of the
crown, was then sufficient to defray the expenses of the government in
time of peace. Nothing was allowed for a standing army. The nation was
sick of the very name; and the least mention of such a force would have
incensed and alarmed all parties.

Early in 1661 took place a general election. The people were mad with
loyal enthusiasm. The capital was excited by preparations for the most
splendid coronation that had ever been known. The result was that a body
of representatives was returned, such as England had never yet seen. A
large proportion of the successful candidates were men who had fought
for the Crown and the Church, and whose minds had been exasperated by
many injuries and insults suffered at the hands of the Roundheads. When
the members met, the passions which animated each individually acquired
new strength from sympathy. The House of Commons was, during some years,
more zealous for royalty than the King, more zealous for episcopacy
than the Bishops. Charles and Clarendon were almost terrified at the
completeness of their own success. They found themselves in a situation
not unlike that in which Lewis the Eighteenth and the Duke of Richelieu
were placed while the Chamber of 1815 was sitting. Even if the King
had been desirous to fulfill the promises which he had made to the
Presbyterians, it would have been out of his power to do so. It was
indeed only by the strong exertion of his influence that he could
prevent the victorious Cavaliers from rescinding the act of indemnity,
and retaliating without mercy all that they had suffered.

The Commons began by resolving that every member should, on pain of
expulsion, take the sacrament according to the form prescribed by the
old Liturgy, and that the Covenant should be burned by the hangman in
Palace Yard. An act was passed, which not only acknowledged the power
of the sword to be solely in the King, but declared that in no extremity
whatever could the two Houses be justified in withstanding him by force.
Another act was passed which required every officer of a corporation to
receive the Eucharist according to the rites of the Church of England,
and to swear that he held resistance to the King's authority to be in
all cases unlawful. A few hotheaded men wished to bring in a bill, which
should at once annul all the statutes passed by the Long Parliament,
and should restore the Star Chamber and the High Commission; but the
reaction, violent as it was, did not proceed quite to this length. It
still continued to be the law that a Parliament should be held every
three years: but the stringent clauses which directed the returning
officers to proceed to election at the proper time, even without the
royal writ, were repealed. The Bishops were restored to their seats in
the Upper House. The old ecclesiastical polity and the old Liturgy were
revived without any modification which had any tendency to conciliate
even the most reasonable Presbyterians. Episcopal ordination was now,
for the first time, made an indispensable qualification for church
preferment. About two thousand ministers of religion, whose conscience
did not suffer them to conform, were driven from their benefices in one
day. The dominant party exultingly reminded the sufferers that the Long
Parliament, when at the height of power, had turned out a still greater
number of Royalist divines. The reproach was but too well founded: but
the Long Parliament had at least allowed to the divines whom it ejected
a provision sufficient to keep them from starving; and this example the
Cavaliers, intoxicated with animosity, had not the justice and humanity
to follow.

Then came penal statutes against Nonconformists, statutes for which
precedents might too easily be found in the Puritan legislation, but to
which the King could not give his assent without a breach of promises
publicly made, in the most important crisis of his life, to those on
whom his fate depended. The Presbyterians, in extreme distress and
terror, fled to the foot of the throne, and pleaded their recent
services and the royal faith solemnly and repeatedly plighted. The King
wavered. He could not deny his own hand and seal. He could not but be
conscious that he owed much to the petitioners. He was little in the
habit of resisting importunate solicitation. His temper was not that of
a persecutor. He disliked the Puritans indeed; but in him dislike was a
languid feeling, very little resembling the energetic hatred which had
burned in the heart of Laud. He was, moreover, partial to the Roman
Catholic religion; and he knew that it would be impossible to grant
liberty of worship to the professors of that religion without extending
the same indulgence to Protestant dissenters. He therefore made a feeble
attempt to restrain the intolerant zeal of the House of Commons; but
that House was under the influence of far deeper convictions and far
stronger passions than his own. After a faint struggle he yielded, and
passed, with the show of alacrity, a series of odious acts against
the separatists. It was made a crime to attend a dissenting place of
worship. A single justice of the peace might convict without a jury, and
might, for the third offence, pass sentence of transportation beyond sea
for seven years. With refined cruelty it was provided that the offender
should not be transported to New England, where he was likely to find
sympathising friends. If he returned to his own country before the
expiration of his term of exile, he was liable to capital punishment.
A new and most unreasonable test was imposed on divines who had been
deprived of their benefices for nonconformity; and all who refused to
take that test were prohibited from coming within five miles of any town
which was governed by a corporation, of any town which was represented
in Parliament, or of any town where they had themselves resided as
ministers. The magistrates, by whom these rigorous statutes were to
be enforced, were in general men inflamed by party spirit and by the
remembrance of wrongs suffered in the time of the commonwealth. The
gaols were therefore soon crowded with dissenters, and, among the
sufferers, were some of whose genius and virtue any Christian society
might well be proud.

The Church of England was not ungrateful for the protection which she
received from the government. From the first day of her existence, she
had been attached to monarchy. But, during the quarter of a century
which followed the Restoration, her zeal for royal authority and
hereditary right passed all bounds. She had suffered with the House of
Stuart. She had been restored with that House. She was connected with
it by common interests, friendships, and enmities. It seemed impossible
that a day could ever come when the ties which bound her to the children
of her august martyr would be sundered, and when the loyalty in which
she gloried would cease to be a pleasing and profitable duty. She
accordingly magnified in fulsome phrase that prerogative which was
constantly employed to defend and to aggrandise her, and reprobated,
much at her ease, the depravity of those whom oppression, from which
she was exempt, had goaded to rebellion. Her favourite theme was
the doctrine of non-resistance. That doctrine she taught without any
qualification, and followed out to all its extreme consequences. Her
disciples were never weary of repeating that in no conceivable case, not
even if England were cursed with a King resembling Busiris or Phalaris,
with a King who, in defiance of law, and without the presence of
justice, should daily doom hundreds of innocent victims to torture
and death, would all the Estates of the realm united be justified in
withstanding his tyranny by physical force. Happily the principles of
human nature afford abundant security that such theories will never be
more than theories. The day of trial came; and the very men who had most
loudly and most sincerely professed this extravagant loyalty were, in
every county of England arrayed in arms against the throne.

Property all over the kingdom was now again changing hands. The national
sales, not having been confirmed by Act of Parliament, were regarded by
the tribunals as nullities. The bishops, the deans, the chapters, the
Royalist nobility and gentry, reentered on their confiscated estates,
and ejected even purchasers who had given fair prices. The losses which
the Cavaliers had sustained during the ascendency of their opponents
were thus in part repaired; but in part only. All actions for mesne
profits were effectually barred by the general amnesty; and the
numerous Royalists, who, in order to discharge fines imposed by the Long
Parliament, or in order to purchase the favour of powerful Roundheads,
had sold lands for much less than the real value, were not relieved from
the legal consequences of their own acts.

While these changes were in progress, a change still more important took
place in the morals and manners of the community. Those passions
and tastes which, under the rule of the Puritans, had been sternly
repressed, and, if gratified at all, had been gratified by stealth,
broke forth with ungovernable violence as soon as the check was
withdrawn. Men flew to frivolous amusements and to criminal pleasures
with the greediness which long and enforced abstinence naturally
produces. Little restraint was imposed by public opinion. For the
nation, nauseated with cant, suspicious of all pretensions to sanctity
and still smarting from the recent tyranny of rulers austere in life and
powerful in prayer, looked for a time with complacency on the softer and
gayer vices. Still less restraint was imposed by the government.
Indeed there was no excess which was not encouraged by the ostentatious
profligacy of the King and of his favourite courtiers. A few counsellors
of Charles the First, who were now no longer young, retained the
decorous gravity which had been thirty years before in fashion at
Whitehall. Such were Clarendon himself, and his friends, Thomas
Wriothesley, Earl of Southampton, Lord Treasurer, and James Butler, Duke
of Ormond, who, having through many vicissitudes struggled gallantly
for the royal cause in Ireland, now governed that kingdom as Lord
Lieutenant. But neither the memory of the services of these men, nor
their great power in the state, could protect them from the sarcasms
which modish vice loves to dart at obsolete virtue. The praise of
politeness and vivacity could now scarcely be obtained except by some
violation of decorum. Talents great and various assisted to spread the
contagion. Ethical philosophy had recently taken a form well suited
to please a generation equally devoted to monarchy and to vice. Thomas
Hobbes had, in language more precise and luminous than has ever been
employed by any other metaphysical writer, maintained that the will of
the prince was the standard of right and wrong, and that every subject
ought to be ready to profess Popery, Mahometanism, or Paganism, at the
royal command. Thousands who were incompetent to appreciate what was
really valuable in his speculations, eagerly welcomed a theory which,
while it exalted the kingly office, relaxed the obligations of morality,
and degraded religion into a mere affair of state. Hobbism soon became
an almost essential part of the character of the fine gentleman. All
the lighter kinds of literature were deeply tainted by the prevailing
licentiousness. Poetry stooped to be the pandar of every low desire.
Ridicule, instead of putting guilt and error to the blush, turned her
formidable shafts against innocence and truth. The restored Church
contended indeed against the prevailing immorality, but contended
feebly, and with half a heart. It was necessary to the decorum of
her character that she should admonish her erring children: but her
admonitions were given in a somewhat perfunctory manner. Her attention
was elsewhere engaged. Her whole soul was in the work of crushing the
Puritans, and of teaching her disciples to give unto Caesar the things
which were Caesar's. She had been pillaged and oppressed by the party
which preached an austere morality. She had been restored to opulence
and honour by libertines. Little as the men of mirth and fashion were
disposed to shape their lives according to her precepts, they were yet
ready to fight knee deep in blood for her cathedrals and places, for
every line of her rubric and every thread of her vestments. If the
debauched Cavalier haunted brothels and gambling houses, he at least
avoided conventicles. If he never spoke without uttering ribaldry and
blasphemy, he made some amends by his eagerness to send Baxter and Howe
to gaol for preaching and praying. Thus the clergy, for a time, made war
on schism with so much vigour that they had little leisure to make war
on vice. The ribaldry of Etherege and Wycherley was, in the presence and
under the special sanction of the head of the Church, publicly recited
by female lips in female ears, while the author of the Pilgrim's
Progress languished in a dungeon for the crime of proclaiming the gospel
to the poor. It is an unquestionable and a most instructive fact that
the years during which the political power of the Anglican hierarchy was
in the zenith were precisely the years during which national virtue was
at the lowest point.

Scarcely any rank or profession escaped the infection of the prevailing
immorality; but those persons who made politics their business were
perhaps the most corrupt part of the corrupt society. For they were
exposed, not only to the same noxious influences which affected the
nation generally, but also to a taint of a peculiar and of a most
malignant kind. Their character had been formed amidst frequent and
violent revolutions and counterrevolutions. In the course of a few
years they had seen the ecclesiastical and civil polity of their country
repeatedly changed. They had seen an Episcopal Church persecuting
Puritans, a Puritan Church persecuting Episcopalians, and an Episcopal
Church persecuting Puritans again. They had seen hereditary monarchy
abolished and restored. They had seen the Long Parliament thrice supreme
in the state, and thrice dissolved amidst the curses and laughter of
millions. They had seen a new dynasty rapidly rising to the height of
power and glory, and then on a sudden hurled down from the chair of
state without a struggle. They had seen a new representative system
devised, tried and abandoned. They had seen a new House of Lords
created and scattered. They had seen great masses of property violently
transferred from Cavaliers to Roundheads, and from Roundheads back to
Cavaliers. During these events no man could be a stirring and thriving
politician who was not prepared to change with every change of fortune.
It was only in retirement that any person could long keep the character
either of a steady Royalist or of a steady Republican. One who, in
such an age, is determined to attain civil greatness must renounce all
thoughts of consistency. Instead of affecting immutability in the midst
of endless mutation, he must be always on the watch for the indications
of a coming reaction. He must seize the exact moment for deserting
a falling cause. Having gone all lengths with a faction while it
was uppermost, he must suddenly extricate himself from it when its
difficulties begin, must assail it, must persecute it, must enter on a
new career of power and prosperity in company with new associates. His
situation naturally developes in him to the highest degree a peculiar
class of abilities and a peculiar class of vices. He becomes quick of
observation and fertile of resource. He catches without effort the tone
of any sect or party with which he chances to mingle. He discerns
the signs of the times with a sagacity which to the multitude appears
miraculous, with a sagacity resembling that with which a veteran police
officer pursues the faintest indications of crime, or with which a
Mohawk warrior follows a track through the woods. But we shell seldom
find, in a statesman so trained, integrity, constancy, any of the
virtues of the noble family of Truth. He has no faith in any doctrine,
no zeal for any cause. He has seen so many old institutions swept away,
that he has no reverence for prescription. He has seen so many
new institutions, from which much had been expected, produce mere
disappointment, that he has no hope of improvement. He sneers alike at
those who are anxious to preserve and at those who are eager to reform.
There is nothing in the state which he could not, without a scruple or
a blush, join in defending or in destroying. Fidelity to opinions and
to friends seems to him mere dulness and wrongheadedness. Politics
he regards, not as a science of which the object is the happiness of
mankind, but as an exciting game of mixed chance and skill, at which
a dexterous and lucky player may win an estate, a coronet, perhaps a
crown, and at which one rash move may lead to the loss of fortune and
of life. Ambition, which, in good times, and in good minds, is half a
virtue, now, disjoined from every elevated and philanthropic sentiment,
becomes a selfish cupidity scarcely less ignoble than avarice. Among
those politicians who, from the Restoration to the accession of the
House of Hanover, were at the head of the great parties in the state,
very few can be named whose reputation is not stained by what, in our
age, would be called gross perfidy and corruption. It is scarcely an
exaggeration to say that the most unprincipled public men who have taken
part in affairs within our memory would, if tried by the standard
which was in fashion during the latter part of the seventeenth century,
deserve to be regarded as scrupulous and disinterested.

While these political, religious, and moral changes were taking place in
England, the Royal authority had been without difficulty reestablished
in every other part of the British islands. In Scotland the restoration
of the Stuarts had been hailed with delight; for it was regarded as
the restoration of national independence. And true it was that the yoke
which Cromwell had imposed was, in appearance, taken away, that the
Scottish Estates again met in their old hall at Edinburgh, and that the
Senators of the College of Justice again administered the Scottish
law according to the old forms. Yet was the independence of the little
kingdom necessarily rather nominal than real; for, as long as the King
had England on his side, he had nothing to apprehend from disaffection
in his other dominions. He was now in such a situation that he could
renew the attempt which had proved destructive to his father without any
danger of his father's fate. Charles the First had tried to force his
own religion by his regal power on the Scots at a moment when both his
religion and his regal power were unpopular in England; and he had not
only failed, but had raised troubles which had ultimately cost him
his crown and his head. Times had now changed: England was zealous for
monarchy and prelacy; and therefore the scheme which had formerly been
in the highest degree imprudent might be resumed with little risk to
the throne. The government resolved to set up a prelatical church in
Scotland. The design was disapproved by every Scotchman whose judgment
was entitled to respect. Some Scottish statesmen who were zealous
for the King's prerogative had been bred Presbyterians. Though little
troubled with scruples, they retained a preference for the religion of
their childhood; and they well knew how strong a hold that religion had
on the hearts of their countrymen. They remonstrated strongly: but, when
they found that they remonstrated in vain, they had not virtue enough to
persist in an opposition which would have given offence to their
master; and several of them stooped to the wickedness and baseness of
persecuting what in their consciences they believed to be the purest
form of Christianity. The Scottish Parliament was so constituted that
it had scarcely ever offered any serious opposition even to Kings much
weaker than Charles then was. Episcopacy, therefore, was established
by law. As to the form of worship, a large discretion was left to the
clergy. In some churches the English Liturgy was used. In others, the
ministers selected from that Liturgy such prayers and thanksgivings
as were likely to be least offensive to the people. But in general the
doxology was sung at the close of public worship; and the Apostles'
Creed was recited when baptism was administered. By the great body of
the Scottish nation the new Church was detested both as superstitious
and as foreign; as tainted with the corruptions of Rome, and as a
mark of the predominance of England. There was, however, no general
insurrection. The country was not what it had been twenty-two years
before. Disastrous war and alien domination had tamed the spirit of the
people. The aristocracy, which was held in great honour by the middle
class and by the populace, had put itself at the head of the movement
against Charles the First, but proved obsequious to Charles the Second.
From the English Puritans no aid was now to be expected. They were a
feeble party, proscribed both by law and by public opinion. The bulk
of the Scottish nation, therefore, sullenly submitted, and, with many
misgivings of conscience, attended the ministrations of the Episcopal
clergy, or of Presbyterian divines who had consented to accept from the
government a half toleration, known by the name of the Indulgence.
But there were, particularly in the western lowlands, many fierce and
resolute men who held that the obligation to observe the Covenant was
paramount to the obligation to obey the magistrate. These people, in
defiance of the law, persisted in meeting to worship God after their own
fashion. The Indulgence they regarded, not as a partial reparation of
the wrongs inflicted by the State on the Church, but as a new wrong, the
more odious because it was disguised under the appearance of a benefit.
Persecution, they said, could only kill the body; but the black
Indulgence was deadly to the soul. Driven from the towns, they assembled
on heaths and mountains. Attacked by the civil power, they without
scruple repelled force by force. At every conventicle they mustered in
arms. They repeatedly broke out into open rebellion. They were easily
defeated, and mercilessly punished: but neither defeat nor punishment
could subdue their spirit. Hunted down like wild beasts, tortured till
their bones were beaten flat, imprisoned by hundreds, hanged by scores,
exposed at one time to the license of soldiers from England, abandoned
at another time to the mercy of troops of marauders from the Highlands,
they still stood at bay in a mood so savage that the boldest and
mightiest oppressor could not but dread the audacity of their despair.

Such was, during the reign of Charles the Second, the state of Scotland.
Ireland was not less distracted. In that island existed feuds, compared
with which the hottest animosities of English politicians were lukewarm.
The enmity between the Irish Cavaliers and the Irish Roundheads was
almost forgotten in the fiercer enmity which raged between the English
and the Celtic races. The interval between the Episcopalian and the
Presbyterian seemed to vanish, when compared with the interval which
separated both from the Papist. During the late civil troubles the
greater part of the Irish soil had been transferred from the vanquished
nation to the victors. To the favour of the Crown few either of the
old or of the new occupants had any pretensions. The despoilers and the
despoiled had, for the most part, been rebels alike. The government
was soon perplexed and wearied by the conflicting claims and mutual
accusations of the two incensed factions. Those colonists among
whom Cromwell had portioned out the conquered territory, and whose
descendants are still called Cromwellians, asserted that the aboriginal
inhabitants were deadly enemies of the English nation under every
dynasty, and of the Protestant religion in every form. They described
and exaggerated the atrocities which had disgraced the insurrection of
Ulster: they urged the King to follow up with resolution the policy of
the Protector; and they were not ashamed to hint that there would never
be peace in Ireland till the old Irish race should be extirpated.
The Roman Catholics extenuated their offense as they best might, and
expatiated in piteous language on the severity of their punishment,
which, in truth, had not been lenient. They implored Charles not to
confound the innocent with the guilty, and reminded him that many of the
guilty had atoned for their fault by returning to their allegiance, and
by defending his rights against the murderers of his father. The court,
sick of the importunities of two parties, neither of which it had any
reason to love, at length relieved itself from trouble by dictating
a compromise. That system, cruel, but most complete and energetic, by
which Oliver had proposed to make the island thoroughly English, was
abandoned. The Cromwellians were induced to relinquish a third part of
their acquisitions. The land thus surrendered was capriciously divided
among claimants whom the government chose to favour. But great numbers
who protested that they were innocent of all disloyalty, and some
persons who boasted that their loyalty had been signally displayed,
obtained neither restitution nor compensation, and filled France and
Spain with outcries against the injustice and ingratitude of the House
of Stuart.

Meantime the government had, even in England, ceased to be popular. The
Royalists had begun to quarrel with the court and with each other; and
the party which had been vanquished, trampled down, and, as it seemed,
annihilated, but which had still retained a strong principle of life,
again raised its head, and renewed the interminable war.

Had the administration been faultless, the enthusiasm with which the
return of the King and the termination of the military tyranny had been
hailed could not have been permanent. For it is the law of our nature
that such fits of excitement shall always be followed by remissions. The
manner in which the court abused its victory made the remission speedy
and complete. Every moderate man was shocked by the insolence, cruelty,
and perfidy with which the Nonconformists were treated. The penal laws
had effectually purged the oppressed party of those insincere members
whose vices had disgraced it, and had made it again an honest and
pious body of men. The Puritan, a conqueror, a ruler, a persecutor,
a sequestrator, had been detested. The Puritan, betrayed and evil
entreated, deserted by all the timeservers who, in his prosperity, had
claimed brotherhood with him, hunted from his home, forbidden under
severe penalties to pray or receive the sacrament according to his
conscience, yet still firm in his resolution to obey God rather than
man, was, in spite of some unpleasing recollections, an object of pity
and respect to well constituted minds. These feelings became stronger
when it was noised abroad that the court was not disposed to treat
Papists with the same rigour which had been shown to Presbyterians. A
vague suspicion that the King and the Duke were not sincere Protestants
sprang up and gathered strength. Many persons too who had been disgusted
by the austerity and hypocrisy of the Saints of the Commonwealth began
to be still more disgusted by the open profligacy of the court and of
the Cavaliers, and were disposed to doubt whether the sullen preciseness
of Praise God Barebone might not be preferable to the outrageous
profaneness and licentiousness of the Buckinghams and Sedleys. Even
immoral men, who were not utterly destitute of sense and public spirit,
complained that the government treated the most serious matters as
trifles, and made trifles its serious business. A King might be
pardoned for amusing his leisure with wine, wit, and beauty. But it was
intolerable that he should sink into a mere lounger and voluptuary, that
the gravest affairs of state should be neglected, and that the public
service should be starved and the finances deranged in order that
harlots and parasites might grow rich.

A large body of Royalists joined in these complaints, and added many
sharp reflections on the King's ingratitude. His whole revenue, indeed,
would not have sufficed to reward them all in proportion to their own
consciousness of desert. For to every distressed gentleman who
had fought under Rupert or Derby his own services seemed eminently
meritorious, and his own sufferings eminently severe. Every one had
flattered himself that, whatever became of the rest, he should be
largely recompensed for all that he had lost during the civil troubles,
and that the restoration of the monarchy would be followed by the
restoration of his own dilapidated fortunes. None of these expectants
could restrain his indignation, when he found that he was as poor under
the King as he had been under the Rump or the Protector. The negligence
and extravagance of the court excited the bitter indignation of these
loyal veterans. They justly said that one half of what His Majesty
squandered on concubines and buffoons would gladden the hearts of
hundreds of old Cavaliers who, after cutting down their oaks and melting
their plate to help his father, now wandered about in threadbare suits,
and did not know where to turn for a meal.

At the same time a sudden fall of rents took place. The income of every
landed proprietor was diminished by five shillings in the pound. The cry
of agricultural distress rose from every shire in the kingdom; and
for that distress the government was, as usual, held accountable. The
gentry, compelled to retrench their expenses for a period, saw with
indignation the increasing splendour and profusion of Whitehall, and
were immovably fixed in the belief that the money which ought to have
supported their households had, by some inexplicable process, gone to
the favourites of the King.

The minds of men were now in such a temper that every public act excited
discontent. Charles had taken to wife Catharine Princess of Portugal.
The marriage was generally disliked; and the murmurs became loud when it
appeared that the King was not likely to have any legitimate posterity.
Dunkirk, won by Oliver from Spain, was sold to Lewis the Fourteenth,
King of France. This bargain excited general indignation. Englishmen
were already beginning to observe with uneasiness the progress of the
French power, and to regard the House of Bourbon with the same feeling
with which their grandfathers had regarded the House of Austria. Was it
wise, men asked, at such a time, to make any addition to the strength of
a monarchy already too formidable? Dunkirk was, moreover, prized by
the people, not merely as a place of arms, and as a key to the Low
Countries, but also as a trophy of English valour. It was to the
subjects of Charles what Calais had been to an earlier generation, and
what the rock of Gibraltar, so manfully defended, through disastrous and
perilous years, against the fleets and armies of a mighty coalition, is
to ourselves. The plea of economy might have had some weight, if it had
been urged by an economical government. But it was notorious that the
charges of Dunkirk fell far short of the sums which were wasted at court
in vice and folly. It seemed insupportable that a sovereign, profuse
beyond example in all that regarded his own pleasures, should be
niggardly in all that regarded the safety and honour of the state.

The public discontent was heightened, when it was found that, while
Dunkirk was abandoned on the plea of economy, the fortress of Tangier,
which was part of the dower of Queen Catharine, was repaired and kept up
at an enormous charge. That place was associated with no recollections
gratifying to the national pride: it could in no way promote the
national interests: it involved us in inglorious, unprofitable, and
interminable wars with tribes of half savage Mussulmans and it was
situated in a climate singularly unfavourable to the health and vigour
of the English race.

But the murmurs excited by these errors were faint, when compared with
the clamours which soon broke forth. The government engaged in war with
the United Provinces. The House of Commons readily voted sums unexampled
in our history, sums exceeding those which had supported the fleets and
armies of Cromwell at the time when his power was the terror of all
the world. But such was the extravagance, dishonesty, and incapacity of
those who had succeeded to his authority, that this liberality proved
worse than useless. The sycophants of the court, ill qualified to
contend against the great men who then directed the arms of Holland,
against such a statesman as De Witt, and such a commander as De Ruyter,
made fortunes rapidly, while the sailors mutinied from very hunger,
while the dockyards were unguarded, while the ships were leaky and
without rigging. It was at length determined to abandon all schemes of
offensive war; and it soon appeared that even a defensive war was a task
too hard for that administration. The Dutch fleet sailed up the Thames,
and burned the ships of war which lay at Chatham. It was said that, on
the very day of that great humiliation, the King feasted with the ladies
of his seraglio, and amused himself with hunting a moth about the supper
room. Then, at length, tardy justice was done to the memory of Oliver.
Everywhere men magnified his valour, genius, and patriotism. Everywhere
it was remembered how, when he ruled, all foreign powers had trembled
at the name of England, how the States General, now so haughty, had
crouched at his feet, and how, when it was known that he was no more,
Amsterdam was lighted up as for a great deliverance, and children
ran along the canals, shouting for joy that the Devil was dead. Even
Royalists exclaimed that the state could be saved only by calling the
old soldiers of the Commonwealth to arms. Soon the capital began to feel
the miseries of a blockade. Fuel was scarcely to be procured. Tilbury
Fort, the place where Elizabeth had, with manly spirit, hurled foul
scorn at Parma and Spain, was insulted by the invaders. The roar of
foreign guns was heard, for the first time, by the citizens of London.
In the Council it was seriously proposed that, if the enemy advanced,
the Tower should be abandoned. Great multitudes of people assembled in
the streets crying out that England was bought and sold. The houses and
carriages of the ministers were attacked by the populace; and it seemed
likely that the government would have to deal at once with an invasion
and with an insurrection. The extreme danger, it is true, soon passed
by. A treaty was concluded, very different from the treaties which
Oliver had been in the habit of signing; and the nation was once more
at peace, but was in a mood scarcely less fierce and sullen than in the
days of shipmoney.

The discontent engendered by maladministration was heightened by
calamities which the best administration could not have averted. While
the ignominious war with Holland was raging, London suffered two great
disasters, such as never, in so short a space of time, befel one city.
A pestilence, surpassing in horror any that during three centuries
had visited the island, swept away, in six mouths, more than a hundred
thousand human beings. And scarcely had the dead cart ceased to go its
rounds, when a fire, such as had not been known in Europe since the
conflagration of Rome under Nero, laid in ruins the whole city, from the
Tower to the Temple, and from the river to the purlieus of Smithfield.

Had there been a general election while the nation was smarting under so
many disgraces and misfortunes, it is probable that the Roundheads would
have regained ascendency in the state. But the Parliament was still
the Cavalier Parliament, chosen in the transport of loyalty which had
followed the Restoration. Nevertheless it soon became evident that no
English legislature, however loyal, would now consent to be merely what
the legislature had been under the Tudors. From the death of Elizabeth
to the eve of the civil war, the Puritans, who predominated in the
representative body, had been constantly, by a dexterous use of the
power of the purse, encroaching on the province of the executive
government. The gentlemen who, after the Restoration, filled the Lower
House, though they abhorred the Puritan name, were well pleased to
inherit the fruit of the Puritan policy. They were indeed most willing
to employ the power which they possessed in the state for the purpose of
making their King mighty and honoured, both at home and abroad: but
with the power itself they were resolved not to part. The great English
revolution of the seventeenth century, that is to say, the transfer of
the supreme control of the executive administration from the crown to
the House of Commons, was, through the whole long existence of this
Parliament, proceeding noiselessly, but rapidly and steadily. Charles,
kept poor by his follies and vices, wanted money. The Commons alone
could legally grant him money. They could not be prevented from putting
their own price on their grants. The price which they put on their
grants was this, that they should be allowed to interfere with every one
of the King's prerogatives, to wring from him his consent to laws which
he disliked, to break up cabinets, to dictate the course of foreign
policy, and even to direct the administration of war. To the royal
office, and the royal person, they loudly and sincerely professed the
strongest attachment. But to Clarendon they owed no allegiance; and they
fell on him as furiously as their predecessors had fallen on Strafford.
The minister's virtues and vices alike contributed to his ruin. He
was the ostensible head of the administration, and was therefore held
responsible even for those acts which he had strongly, but vainly,
opposed in Council. He was regarded by the Puritans, and by all who
pitied them, as an implacable bigot, a second Laud, with much more than
Laud's understanding. He had on all occasions maintained that the Act of
indemnity ought to be strictly observed; and this part of his conduct,
though highly honourable to him, made him hateful to all those Royalists
who wished to repair their ruined fortunes by suing the Roundheads for
damages and mesne profits. The Presbyterians of Scotland attributed to
him the downfall of their Church. The Papists of Ireland attributed to
him the loss of their lands. As father of the Duchess of York, he had
an obvious motive for wishing that there might be a barren Queen; and he
was therefore suspected of having purposely recommended one. The sale
of Dunkirk was justly imputed to him. For the war with Holland, he
was, with less justice, held accountable. His hot temper, his arrogant
deportment, the indelicate eagerness with which he grasped at riches,
the ostentation with which he squandered them, his picture gallery,
filled with masterpieces of Vandyke which had once been the property of
ruined Cavaliers, his palace, which reared its long and stately front
right opposite to the humbler residence of our Kings, drew on him much
deserved, and some undeserved, censure. When the Dutch fleet was in the
Thames, it was against the Chancellor that the rage of the populace was
chiefly directed. His windows were broken; the trees of his garden were
cut down; and a gibbet was set up before his door. But nowhere was he
more detested than in the House of Commons. He was unable to perceive
that the time was fast approaching when that House, if it continued to
exist at all, must be supreme in the state, when the management of that
House would be the most important department of politics, and when,
without the help of men possessing the ear of that House, it would
be impossible to carry on the government. He obstinately persisted in
considering the Parliament as a body in no respect differing from the
Parliament which had been sitting when, forty years before, he first
began to study law at the Temple. He did not wish to deprive the
legislature of those powers which were inherent in it by the old
constitution of the realm: but the new development of those powers,
though a development natural, inevitable, and to be prevented only by
utterly destroying the powers themselves, disgusted and alarmed him.
Nothing would have induced him to put the great seal to a writ for
raising shipmoney, or to give his voice in Council for committing a
member of Parliament to the Tower, on account of words spoken in debate:
but, when the Commons began to inquire in what manner the money voted
for the war had been wasted, and to examine into the maladministration
of the navy, he flamed with indignation. Such inquiry, according to him,
was out of their province. He admitted that the House was a most loyal
assembly, that it had done good service to the crown, and that its
intentions were excellent. But, both in public and in the closet, he,
on every occasion, expressed his concern that gentlemen so sincerely
attached to monarchy should unadvisedly encroach on the prerogative of
the monarch. Widely as they differed in spirit from the members of the
Long Parliament, they yet, he said, imitated that Parliament in meddling
with matters which lay beyond the sphere of the Estates of the realm,
and which were subject to the authority of the crown alone. The country,
he maintained, would never be well governed till the knights of shires
and the burgesses were content to be what their predecessors had been
in the days of Elizabeth. All the plans which men more observant
than himself of the signs of that time proposed, for the purpose of
maintaining a good understanding between the Court and the Commons,
he disdainfully rejected as crude projects, inconsistent with the
old polity of England. Towards the young orators, who were rising
to distinction and authority in the Lower House, his deportment was
ungracious: and he succeeded in making them, with scarcely an exception,
his deadly enemies. Indeed one of his most serious faults was
an inordinate contempt for youth: and this contempt was the more
unjustifiable, because his own experience in English politics was by no
means proportioned to his age. For so great a part of his life had been
passed abroad that he knew less of that world in which he found himself
on his return than many who might have been his sons.

For these reasons he was disliked by the Commons. For very different
reasons he was equally disliked by the Court. His morals as well as his
polities were those of an earlier generation. Even when he was a young
law student, living much with men of wit and pleasure, his natural
gravity and his religious principles had to a great extent preserved
him from the contagion of fashionable debauchery; and he was by no means
likely, in advanced years and in declining health, to turn libertine.
On the vices of the young and gay he looked with an aversion almost as
bitter and contemptuous as that which he felt for the theological errors
of the sectaries. He missed no opportunity of showing his scorn of
the mimics, revellers, and courtesans who crowded the palace; and the
admonitions which he addressed to the King himself were very sharp,
and, what Charles disliked still more, very long. Scarcely any voice was
raised in favour of a minister loaded with the double odium of faults
which roused the fury of the people, and of virtues which annoyed and
importuned the sovereign. Southampton was no more. Ormond performed
the duties of friendship manfully and faithfully, but in vain. The
Chancellor fell with a great ruin. The seal was taken from him: the
Commons impeached him: his head was not safe: he fled from the country:
an act was passed which doomed him to perpetual exile; and those who had
assailed and undermined him began to struggle for the fragments of his
power.

The sacrifice of Clarendon in some degree took off the edge of the
public appetite for revenge. Yet was the anger excited by the profusion
and negligence of the government, and by the miscarriages of the late
war, by no means extinguished. The counsellors of Charles, with the fate
of the Chancellor before their eyes, were anxious for their own safety.
They accordingly advised their master to soothe the irritation which
prevailed both in the Parliament and throughout the country, and for
that end, to take a step which has no parallel in the history of the
House of Stuart, and which was worthy of the prudence and magnanimity of
Oliver.

We have now reached a point at which the history of the great English
revolution begins to be complicated with the history of foreign
politics. The power of Spain had, during many years, been declining.
She still, it is true held in Europe the Milanese and the two Sicilies,
Belgium, and Franche Comte. In America her dominions still spread, on
both sides of the equator, far beyond the limits of the torrid zone. But
this great body had been smitten with palsy, and was not only
incapable of giving molestation to other states, but could not, without
assistance, repel aggression. France was now, beyond all doubt,
the greatest power in Europe. Her resources have, since those days,
absolutely increased, but have not increased so fast as the resources
of England. It must also be remembered that, a hundred and eighty years
ago, the empire of Russia, now a monarchy of the first class, was as
entirely out of the system of European politics as Abyssinia or Siam,
that the House of Brandenburg was then hardly more powerful than the
House of Saxony, and that the republic of the United States had not
then begun to exist. The weight of France, therefore, though still very
considerable, has relatively diminished. Her territory was not in the
days of Lewis the Fourteenth quite so extensive as at present: but
it was large, compact, fertile, well placed both for attack and for
defence, situated in a happy climate, and inhabited by a brave, active,
and ingenious people. The state implicitly obeyed the direction of a
single mind. The great fiefs which, three hundred years before, had
been, in all but name, independent principalities, had been annexed to
the crown. Only a few old men could remember the last meeting of the
States General. The resistance which the Huguenots, the nobles, and the
parliaments had offered to the kingly power, had been put down by the
two great Cardinals who had ruled the nation during forty years. The
government was now a despotism, but, at least in its dealings with the
upper classes, a mild and generous despotism, tempered by courteous
manners and chivalrous sentiments. The means at the disposal of the
sovereign were, for that age, truly formidable. His revenue, raised, it
is true, by a severe and unequal taxation which pressed heavily on the
cultivators of the soil, far exceeded that of any other potentate. His
army, excellently disciplined, and commanded by the greatest generals
then living, already consisted of more than a hundred and twenty
thousand men. Such an array of regular troops had not been seen in
Europe since the downfall of the Roman empire. Of maritime powers France
was not the first. But, though she had rivals on the sea, she had not
yet a superior. Such was her strength during the last forty years of the
seventeenth century, that no enemy could singly withstand her, and that
two great coalitions, in which half Christendom was united against her,
failed of success.

The personal qualities of the French King added to the respect inspired
by the power and importance of his kingdom. No sovereign has ever
represented the majesty of a great state with more dignity and grace. He
was his own prime minister, and performed the duties of a prime minister
with an ability and industry which could not be reasonably expected from
one who had in infancy succeeded to a crown, and who had been surrounded
by flatterers before he could speak. He had shown, in an eminent degree,
two talents invaluable to a prince, the talent of choosing his servants
well, and the talent of appropriating to himself the chief part of the
credit of their acts. In his dealings with foreign powers he had some
generosity, but no justice. To unhappy allies who threw themselves
at his feet, and had no hope but in his compassion, he extended his
protection with a romantic disinterestedness, which seemed better suited
to a knight errant than to a statesman. But he broke through the most
sacred ties of public faith without scruple or shame, whenever they
interfered with his interest, or with what he called his glory. His
perfidy and violence, however, excited less enmity than the insolence
with which he constantly reminded his neighbours of his own greatness
and of their littleness. He did not at this time profess the austere
devotion which, at a later period, gave to his court the aspect of a
monastery. On the contrary, he was as licentious, though by no means as
frivolous and indolent, as his brother of England. But he was a sincere
Roman Catholic; and both his conscience and his vanity impelled him to
use his power for the defence and propagation of the true faith, after
the example of his renowned predecessors, Clovis, Charlemagne, and Saint
Lewis.

Our ancestors naturally looked with serious alarm on the growing power
of France. This feeling, in itself perfectly reasonable, was mingled
with other feelings less praiseworthy. France was our old enemy. It was
against France that the most glorious battles recorded in our annals
had been fought. The conquest of France had been twice effected by the
Plantagenets. The loss of France had been long remembered as a great
national disaster. The title of King of France was still borne by our
sovereigns. The lilies of France still appeared, mingled with our own
lions, on the shield of the House of Stuart. In the sixteenth century
the dread inspired by Spain had suspended the animosity of which France
had anciently been the object. But the dread inspired by Spain had given
place to contemptuous compassion; and France was again regarded as our
national foe. The sale of Dunkirk to France had been the most generally
unpopular act of the restored King. Attachment to France had been
prominent among the crimes imputed by the Commons to CIarendon. Even in
trifles the public feeling showed itself. When a brawl took place in the
streets of Westminster between the retinues of the French and Spanish
embassies, the populace, though forcibly prevented from interfering,
had given unequivocal proofs that the old antipathy to France was not
extinct.

France and Spain were now engaged in a more serious contest. One of the
chief objects of the policy of Lewis throughout his life was to extend
his dominions towards the Rhine. For this end he had engaged in war
with Spain, and he was now in the full career of conquest. The United
Provinces saw with anxiety the progress of his arms. That renowned
federation had reached the height of power, prosperity, and glory. The
Batavian territory, conquered from the waves and defended against them
by human art, was in extent little superior to the principality of
Wales. But all that narrow space was a busy and populous hive, in which
new wealth was every day created, and in which vast masses of old
wealth were hoarded. The aspect of Holland, the rich cultivation, the
innumerable canals, the ever whirling mills, the endless fleets of
barges, the quick succession of great towns, the ports bristling with
thousands of masts, the large and stately mansions, the trim villas, the
richly furnished apartments, the picture galleries, the summer rouses,
the tulip beds, produced on English travellers in that age an effect
similar to the effect which the first sight of England now produces on a
Norwegian or a Canadian. The States General had been compelled to humble
themselves before Cromwell. But after the Restoration they had taken
their revenge, had waged war with success against Charles, and had
concluded peace on honourable terms. Rich, however, as the Republic
was, and highly considered in Europe, she was no match for the power of
Lewis. She apprehended, not without good cause, that his kingdom
might soon be extended to her frontiers; and she might well dread
the immediate vicinity of a monarch so great, so ambitious, and so
unscrupulous. Yet it was not easy to devise any expedient which might
avert the danger. The Dutch alone could not turn the scale against
France. On the side of the Rhine no help was to be expected. Several
German princes had been gained by Lewis; and the Emperor himself was
embarrassed by the discontents of Hungary. England was separated from
the United Provinces by the recollection of cruel injuries recently
inflicted and endured; and her policy had, since the restoration, been
so devoid of wisdom and spirit, that it was scarcely possible to expect
from her any valuable assistance

But the fate of Clarendon and the growing ill humour of the Parliament
determined the advisers of Charles to adopt on a sudden a policy which
amazed and delighted the nation.

The English resident at Brussels, Sir William Temple, one of the most
expert diplomatists and most pleasing writers of that age, had already
represented to this court that it was both desirable and practicable
to enter into engagements with the States General for the purpose of
checking the progress of France. For a time his suggestions had been
slighted; but it was now thought expedient to act on them. He was
commissioned to negotiate with the States General. He proceeded to the
Hague, and soon came to an understanding with John De Witt, then the
chief minister of Holland. Sweden, small as her resources were, had,
forty years before, been raised by the genius of Gustavus Adolphus to
a high rank among European powers, and had not yet descended to her
natural position. She was induced to join on this occasion with England
and the States. Thus was formed that coalition known as the Triple
Alliance. Lewis showed signs of vexation and resentment, but did not
think it politic to draw on himself the hostility of such a confederacy
in addition to that of Spain. He consented, therefore, to relinquish
a large part of the territory which his armies had occupied. Peace was
restored to Europe; and the English government, lately an object of
general contempt, was, during a few months, regarded by foreign powers
with respect scarcely less than that which the Protector had inspired.

At home the Triple Alliance was popular in the highest degree. It
gratified alike national animosity and national pride. It put a limit
to the encroachments of a powerful and ambitious neighbour. It bound
the leading Protestant states together in close union. Cavaliers and
Roundheads rejoiced in common: but the joy of the Roundhead was even
greater than that of the Cavalier. For England had now allied herself
strictly with a country republican in government and Presbyterian in
religion, against a country ruled by an arbitrary prince and attached
to the Roman Catholic Church. The House of Commons loudly applauded the
treaty; and some uncourtly grumblers described it as the only good thing
that had been done since the King came in.

The King, however, cared little for the approbation of his Parliament
or of his people. The Triple Alliance he regarded merely as a temporary
expedient for quieting discontents which had seemed likely to become
serious. The independence, the safety, the dignity of the nation
over which he presided were nothing to him. He had begun to find
constitutional restraints galling. Already had been formed in the
Parliament a strong connection known by the name of the Country Party.
That party included all the public men who leaned towards Puritanism
and Republicanism, and many who, though attached to the Church and to
hereditary monarchy, had been driven into opposition by dread of Popery,
by dread of France, and by disgust at the extravagance, dissoluteness,
and faithlessness of the court. The power of this band of politicians
was constantly growing. Every year some of those members who had been
returned to Parliament during the loyal excitement of 1661 had dropped
off; and the vacant seats had generally been filled by persons less
tractable. Charles did not think himself a King while an assembly of
subjects could call for his accounts before paying his debts, and
could insist on knowing which of his mistresses or boon companions had
intercepted the money destined for the equipping and manning of the
fleet. Though not very studious of fame, he was galled by the taunts
which were sometimes uttered in the discussions of the Commons, and on
one occasion attempted to restrain the freedom of speech by disgraceful
means. Sir John Coventry, a country gentleman, had, in debate, sneered
at the profligacy of the court. In any former reign he would probably
have been called before the Privy Council and committed to the Tower. A
different course was now taken. A gang of bullies was secretly sent to
slit the nose of the offender. This ignoble revenge, instead of quelling
the spirit of opposition, raised such a tempest that the King was
compelled to submit to the cruel humiliation of passing an act which
attainted the instruments of his revenge, and which took from him the
power of pardoning them.

But, impatient as he was of constitutional restraints, how was he to
emancipate himself from them? He could make himself despotic only by the
help of a great standing army; and such an army was not in existence.
His revenues did indeed enable him to keep up some regular troops:
but those troops, though numerous enough to excite great jealousy and
apprehension in the House of Commons and in the country, were scarcely
numerous enough to protect Whitehall and the Tower against a rising of
the mob of London. Such risings were, indeed to be dreaded; for it
was calculated that in the capital and its suburbs dwelt not less than
twenty thousand of Oliver's old soldiers.

Since the King was bent on emancipating himself from the control of
Parliament, and since, in such an enterprise, he could not hope for
effectual aid at home, it followed that he must look for aid abroad.
The power and wealth of the King of France might be equal to the arduous
task of establishing absolute monarchy in England. Such an ally would
undoubtedly expect substantial proofs of gratitude for such a service.
Charles must descend to the rank of a great vassal, and must make peace
and war according to the directions of the government which protected
him. His relation to Lewis would closely resemble that in which
the Rajah of Nagpore and the King of Oude now stand to the British
Government. Those princes are bound to aid the East India Company in
all hostilities, defensive and offensive, and to have no diplomatic
relations but such as the East India Company shall sanction. The
Company in return guarantees them against insurrection. As long as they
faithfully discharge their obligations to the paramount power, they
are permitted to dispose of large revenues, to fill their palaces with
beautiful women, to besot themselves in the company of their favourite
revellers, and to oppress with impunity any subject who may incur their
displeasure. 
%[18]
\footnote{ I am happy to say, that, since this passage was written,
the territories both of the Rajah of Nagpore and of the King of Oude
have been added to the British dominions. (1857.)}
 Such a life would be insupportable to a man of high
spirit and of powerful understanding. But to Charles, sensual, indolent,
unequal to any strong intellectual exertion, and destitute alike of
all patriotism and of all sense of personal dignity, the prospect had
nothing unpleasing.

That the Duke of York should have concurred in the design of degrading
that crown which it was probable that he would himself one day wear may
seem more extraordinary. For his nature was haughty and imperious; and,
indeed, he continued to the very last to show, by occasional starts and
struggles, his impatience of the French yoke. But he was almost as much
debased by superstition as his brother by indolence and vice. James
was now a Roman Catholic. Religious bigotry had become the dominant
sentiment of his narrow and stubborn mind, and had so mingled
itself with his love of rule, that the two passions could hardly be
distinguished from each other. It seemed highly improbable that, without
foreign aid, he would be able to obtain ascendency, or even toleration,
for his own faith: and he was in a temper to see nothing humiliating in
any step which might promote the interests of the true Church.

A negotiation was opened which lasted during several months. The chief
agent between the English and French courts was the beautiful, graceful,
and intelligent Henrietta, Duchess of Orleans, sister of Charles, sister
in law of Lewis, and a favourite with both. The King of England offered
to declare himself a Roman Catholic, to dissolve the Triple Alliance,
and to join with France against Holland, if France would engage to lend
him such military and pecuniary aid as might make him independent of
his parliament. Lewis at first affected to receive these propositions
coolly, and at length agreed to them with the air of a man who is
conferring a great favour: but in truth, the course which he had
resolved to take was one by which he might gain and could not lose.

It seems certain that he never seriously thought of establishing
despotism and Popery in England by force of arms. He must have been
aware that such an enterprise would be in the highest degree arduous and
hazardous, that it would task to the utmost all the energies of France
during many years, and that it would be altogether incompatible with
more promising schemes of aggrandisement, which were dear to his heart.
He would indeed willingly have acquired the merit and the glory of doing
a great service on reasonable terms to the Church of which he was a
member. But he was little disposed to imitate his ancestors who, in the
twelfth and thirteenth centuries, had led the flower of French chivalry
to die in Syria and Egypt: and he well knew that a crusade against
Protestantism in Great Britain would not be less perilous than the
expeditions in which the armies of Lewis the Seventh and of Lewis the
Ninth had perished. He had no motive for wishing the Stuarts to be
absolute. He did not regard the English constitution with feelings at
all resembling those which have in later times induced princes to make
war on the free institutions of neighbouring nations. At present a
great party zealous for popular government has ramifications in every
civilised country. And important advantage gained anywhere by that party
is almost certain to be the signal for general commotion. It is not
wonderful that governments threatened by a common danger should combine
for the purpose of mutual insurance. But in the seventeenth century no
such danger existed. Between the public mind of England and the public
mind of France, there was a great gulph. Our institutions and our
factions were as little understood at Paris as at Constantinople. It may
be doubted whether any one of the forty members of the French Academy
had an English volume in his library, or knew Shakespeare, Jonson, or
Spenser, even by name. A few Huguenots, who had inherited the mutinous
spirit of their ancestors, might perhaps have a fellow feeling with
their brethren in the faith, the English Roundheads: but the Huguenots
had ceased to be formidable. The French, as a people, attached to the
Church of Rome, and proud of the greatness of their King and of their
own loyalty, looked on our struggles against Popery and arbitrary power,
not only without admiration or sympathy, but with strong disapprobation
and disgust. It would therefore be a great error to ascribe the conduct
of Lewis to apprehensions at all resembling those which, in our age,
induced the Holy Alliance to interfere in the internal troubles of
Naples and Spain.

Nevertheless, the propositions made by the court of Whitehall were
most welcome to him. He already meditated gigantic designs, which were
destined to keep Europe in constant fermentation during more than forty
years. He wished to humble the United Provinces, and to annex Belgium,
Franche Comte, and Loraine to his dominions. Nor was this all. The King
of Spain was a sickly child. It was likely that he would die without
issue. His eldest sister was Queen of France. A day would almost
certainly come, and might come very soon, when the House of Bourbon
might lay claim to that vast empire on which the sun never set. The
union of two great monarchies under one head would doubtless be opposed
by a continental coalition. But for any continental coalition France
singlehanded was a match. England could turn the scale. On the course
which, in such a crisis, England might pursue, the destinies of the
world would depend; and it was notorious that the English Parliament
and nation were strongly attached to the policy which had dictated the
Triple Alliance. Nothing, therefore, could be more gratifying to Lewis
than to learn that the princes of the House of Stuart needed his help,
and were willing to purchase that help by unbounded subserviency. He
determined to profit by the opportunity, and laid down for himself a
plan to which, without deviation, he adhered, till the Revolution of
1688 disconcerted all his politics. He professed himself desirous to
promote the designs of the English court. He promised large aid. He from
time to time doled out such aid as might serve to keep hope alive, and
as he could without risk or inconvenience spare. In this way, at an
expense very much less than that which he incurred in building and
decorating Versailles or Marli, he succeeded in making England, during
nearly twenty years, almost as insignificant a member of the political
system of Europe as the republic of San Marino.

His object was not to destroy our constitution, but to keep the various
elements of which it was composed in a perpetual state of conflict,
and to set irreconcilable enmity between those who had the power of the
purse and those who had the power of the sword. With this view he bribed
and stimulated both parties in turn, pensioned at once the ministers
of the crown and the chiefs of the opposition, encouraged the court to
withstand the seditious encroachments of the Parliament, and conveyed to
the Parliament intimations of the arbitrary designs of the court.

One of the devices to which he resorted for the purpose of obtaining an
ascendency in the English counsels deserves especial notice. Charles,
though incapable of love in the highest sense of the word, was the
slave of any woman whose person excited his desires, and whose airs and
prattle amused his leisure. Indeed a husband would be justly derided
who should bear from a wife of exalted rank and spotless virtue half the
insolence which the King of England bore from concubines who, while they
owed everything to his bounty, caressed his courtiers almost before his
face. He had patiently endured the termagant passions of Barbara Palmer
and the pert vivacity of Eleanor Gwynn. Lewis thought that the
most useful envoy who could be sent to London, would be a handsome,
licentious, and crafty Frenchwoman. Such a woman was Louisa, a lady of
the House of Querouaille, whom our rude ancestors called Madam Carwell.
She was soon triumphant over all her rivals, was created Duchess of
Portsmouth, was loaded with wealth, and obtained a dominion which ended
only with the life of Charles.

The most important conditions of the alliance between the crowns were
digested into a secret treaty which was signed at Dover in May, 1670,
just ten years after the day on which Charles had landed at that very
port amidst the acclamations and joyful tears of a too confiding people.

By this treaty Charles bound himself to make public profession of the
Roman Catholic religion, to join his arms to those of Lewis for the
purpose of destroying the power of the United Provinces, and to employ
the whole strength of England, by land and sea, in support of the rights
of the House of Bourbon to the vast monarchy of Spain. Lewis, on the
other hand, engaged to pay a large subsidy, and promised that, if any
insurrection should break out in England, he would send an army at his
own charge to support his ally.

This compact was made with gloomy auspices. Six weeks after it had
been signed and sealed, the charming princess, whose influence over her
brother and brother in law had been so pernicious to her country, was
no more. Her death gave rise to horrible suspicions which, for a moment,
seemed likely to interrupt the newly formed friendship between the
Houses of Stuart and Bourbon: but in a short time fresh assurances of
undiminished good will were exchanged between the confederates.

The Duke of York, too dull to apprehend danger, or too fanatical to care
about it, was impatient to see the article touching the Roman Catholic
religion carried into immediate execution: but Lewis had the wisdom
to perceive that, if this course were taken, there would be such an
explosion in England as would probably frustrate those parts of the plan
which he had most at heart. It was therefore determined that Charles
should still call himself a Protestant, and should still, at high
festivals, receive the sacrament according to the ritual of the Church
of England. His more scrupulous brother ceased to appear in the royal
chapel.

About this time died the Duchess of York, daughter of the banished
Earl of Clarendon. She had been, during some years, a concealed Roman
Catholic. She left two daughters, Mary and Anne, afterwards successively
Queens of Great Britain. They were bred Protestants by the positive
command of the King, who knew that it would be vain for him to profess
himself a member of the Church of England, if children who seemed likely
to inherit his throne were, by his permission, brought up as members of
the Church of Rome.

The principal servants of the crown at this time were men whose names
have justly acquired an unenviable notoriety. We must take heed,
however, that we do not load their memory with infamy which of right
belongs to their master. For the treaty of Dover the King himself is
chiefly answerable. He held conferences on it with the French agents:
he wrote many letters concerning it with his own hand: he was the person
who first suggested the most disgraceful articles which it contained;
and he carefully concealed some of those articles from the majority of
his Cabinet.

Few things in our history are more curious than the origin and growth of
the power now possessed by the Cabinet. From an early period the
Kings of England had been assisted by a Privy Council to which the law
assigned many important functions and duties. During several centuries
this body deliberated on the gravest and most delicate affairs. But
by degrees its character changed. It became too large for despatch and
secrecy. The rank of Privy Councillor was often bestowed as an honorary
distinction on persons to whom nothing was confided, and whose opinion
was never asked. The sovereign, on the most important occasions,
resorted for advice to a small knot of leading ministers. The advantages
and disadvantages of this course were early pointed out by Bacon,
with his usual judgment and sagacity: but it was not till after the
Restoration that the interior council began to attract general notice.
During many years old fashioned politicians continued to regard the
Cabinet as an unconstitutional and dangerous board. Nevertheless, it
constantly became more and more important. It at length drew to itself
the chief executive power, and has now been regarded, during several
generations as an essential part of our polity. Yet, strange to say, it
still continues to be altogether unknown to the law: the names of the
noblemen and gentlemen who compose it are never officially announced to
the public: no record is kept of its meetings and resolutions; nor has
its existence ever been recognised by any Act of Parliament.

During some years the word Cabal was popularly used as synonymous with
Cabinet. But it happened by a whimsical coincidence that, in 1671, the
Cabinet consisted of five persons the initial letters of whose names
made up the word Cabal; Clifford, Arlington, Buckingham, Ashley, and
Lauderdale. These ministers were therefore emphatically called the
Cabal; and they soon made that appellation so infamous that it has never
since their time been used except as a term of reproach.

Sir Thomas Clifford was a Commissioner of the Treasury, and had greatly
distinguished himself in the House of Commons. Of the members of the
Cabal he was the most respectable. For, with a fiery and imperious
temper, he had a strong though a lamentably perverted sense of duty and
honour.

Henry Bennet, Lord Arlington, then Secretary of State, had since he came
to manhood, resided principally on the Continent, and had learned that
cosmopolitan indifference to constitutions and religions which is often
observable in persons whose life has been passed in vagrant diplomacy.
If there was any form of government which he liked it was that of
France. If there was any Church for which he felt a preference, it was
that of Rome. He had some talent for conversation, and some talent also
for transacting the ordinary business of office. He had learned, during
a life passed in travelling and negotiating, the art of accommodating
his language and deportment to the society in which he found himself.
His vivacity in the closet amused the King: his gravity in debates and
conferences imposed on the public; and he had succeeded in attaching to
himself, partly by services and partly by hopes, a considerable number
of personal retainers.

Buckingham, Ashley, and Lauderdale were men in whom the immorality which
was epidemic among the politicians of that age appeared in its most
malignant type, but variously modified by greet diversities of temper
and understanding. Buckingham was a sated man of pleasure, who had
turned to ambition as to a pastime. As he had tried to amuse himself
with architecture and music, with writing farces and with seeking for
the philosopher's stone, so he now tried to amuse himself with a secret
negotiation and a Dutch war. He had already, rather from fickleness
and love of novelty than from any deep design, been faithless to every
party. At one time he had ranked among the Cavaliers. At another
time warrants had been out against him for maintaining a treasonable
correspondence with the remains of the Republican party in the city. He
was now again a courtier, and was eager to win the favour of the King
by services from which the most illustrious of those who had fought and
suffered for the royal house would have recoiled with horror.

Ashley, with a far stronger head, and with a far fiercer and more
earnest ambition, had been equally versatile. But Ashley's versatility
was the effect, not of levity, but of deliberate selfishness. He had
served and betrayed a succession of governments. But he had timed all
his treacheries so well that through all revolutions, his fortunes
had constantly been rising. The multitude, struck with admiration by
a prosperity which, while everything else was constantly changing,
remained unchangeable, attributed to him a prescience almost miraculous,
and likened him to the Hebrew statesman of whom it is written that his
counsel was as if a man had inquired of the oracle of God.

Lauderdale, loud and coarse both in mirth and anger, was, perhaps, under
the outward show of boisterous frankness, the most dishonest man in the
whole Cabal. He had made himself conspicuous among the Scotch insurgents
of 1638 by his zeal for the Covenant. He was accused of having been
deeply concerned in the sale of Charles the First to the English
Parliament, and was therefore, in the estimation of good Cavaliers, a
traitor, if possible, of a worse description than those who had sate in
the High Court of Justice. He often talked with a noisy jocularity
of the days when he was a canter and a rebel. He was now the chief
instrument employed by the court in the work of forcing episcopacy
on his reluctant countrymen; nor did he in that cause shrink from the
unsparing use of the sword, the halter, and the boot. Yet those who knew
him knew that thirty years had made no change in his real sentiments,
that he still hated the memory of Charles the First, and that he still
preferred the Presbyterian form of church government to every other.

Unscrupulous as Buckingham, Ashley, and Lauderdale were, it was not
thought safe to intrust to them the King's intention of declaring
himself a Roman Catholic. A false treaty, in which the article
concerning religion was omitted, was shown to them. The names and seals
of Clifford and Arlington are affixed to the genuine treaty. Both these
statesmen had a partiality for the old Church, a partiality which the
brave and vehement Clifford in no long time manfully avowed, but which
the colder and meaner Arlington concealed, till the near approach of
death scared him into sincerity. The three other cabinet ministers,
however, were not men to be kept easily in the dark, and probably
suspected more than was distinctly avowed to them. They were certainly
privy to all the political engagements contracted with France, and were
not ashamed to receive large gratifications from Lewis.

The first object of Charles was to obtain from the Commons supplies
which might be employed in executing the secret treaty. The Cabal,
holding power at a time when our government was in a state of
transition, united in itself two different kinds of vices belonging
to two different ages and to two different systems. As those five evil
counsellors were among the last English statesmen who seriously thought
of destroying the Parliament, so they were the first English statesmen
who attempted extensively to corrupt it. We find in their policy at once
the latest trace of the Thorough of Strafford, and the earliest trace of
that methodical bribery which was afterwards practiced by Walpole. They
soon perceived, however, that, though the House of Commons was chiefly
composed of Cavaliers, and though places and French gold had been
lavished on the members, there was no chance that even the least odious
parts of the scheme arranged at Dover would be supported by a majority.
It was necessary to have recourse to fraud. The King professed great
zeal for the principles of the Triple Alliance, and pretended that, in
order to hold the ambition of France in check, it would be necessary to
augment the fleet. The Commons fell into the snare, and voted a grant of
eight hundred thousand pounds. The Parliament was instantly prorogued;
and the court, thus emancipated from control, proceeded to the execution
of the great design.

The financial difficulties however were serious. A war with Holland
could be carried on only at enormous cost. The ordinary revenue was not
more than sufficient to support the government in time of peace. The
eight hundred thousand pounds out of which the Commons had just been
tricked would not defray the naval and military charge of a single year
of hostilities. After the terrible lesson given by the Long Parliament,
even the Cabal did not venture to recommend benevolences or shipmoney.
In this perplexity Ashley and Clifford proposed a flagitious breach of
public faith. The goldsmiths of London were then not only dealers in the
precious metals, but also bankers, and were in the habit of advancing
large sums of money to the government. In return for these advances they
received assignments on the revenue, and were repaid with interest as
the taxes came in. About thirteen hundred thousand pounds had been
in this way intrusted to the honour of the state. On a sudden it was
announced that it was not convenient to pay the principal, and that the
lenders must content themselves with interest. They were consequently
unable to meet their own engagements. The Exchange was in an uproar:
several great mercantile houses broke; and dismay and distress
spread through all society. Meanwhile rapid strides were made towards
despotism. Proclamations, dispensing with Acts of Parliament, or
enjoining what only Parliament could lawfully enjoin, appeared in rapid
succession. Of these edicts the most important was the Declaration of
Indulgence. By this instrument the penal laws against Roman Catholics
were set aside; and, that the real object of the measure might not
be perceived, the laws against Protestant Nonconformists were also
suspended.

A few days after the appearance of the Declaration of Indulgence, war
was proclaimed against the United Provinces. By sea the Dutch maintained
the struggle with honour; but on land they were at first borne down by
irresistible force. A great French army passed the Rhine. Fortress
after fortress opened its gates. Three of the seven provinces of the
federation were occupied by the invaders. The fires of the hostile camp
were seen from the top of the Stadthouse of Amsterdam. The Republic,
thus fiercely assailed from without, was torn at the same time by
internal dissensions. The government was in the hands of a close
oligarchy of powerful burghers. There were numerous selfelected Town
Councils, each of which exercised within its own sphere, many of the
rights of sovereignty. These councils sent delegates to the Provincial
States, and the Provincial States again sent delegates to the States
General. A hereditary first magistrate was no essential part of this
polity. Nevertheless one family, singularly fertile of great men, had
gradually obtained a large and somewhat indefinite authority. William,
first of the name, Prince of Orange Nassau, and Stadtholder of Holland,
had headed the memorable insurrection against Spain. His son Maurice had
been Captain General and first minister of the States, had, by eminent
abilities and public services, and by some treacherous and cruel
actions, raised himself to almost kingly power, and had bequeathed
a great part of that power to his family. The influence of the
Stadtholders was an object of extreme jealousy to the municipal
oligarchy. But the army, and that great body of citizens which was
excluded from all share in the government, looked on the Burgomasters
and Deputies with a dislike resembling the dislike with which the
legions and the common people of Rome regarded the Senate, and were as
zealous for the House of Orange as the legions and the common people of
Rome for the House of Caesar. The Stadtholder commanded the forces of
the commonwealth, disposed of all military commands, had a large share
of the civil patronage, and was surrounded by pomp almost regal.

Prince William the Second had been strongly opposed by the oligarchical
party. His life had terminated in the year 1650, amidst great civil
troubles. He died childless: the adherents of his house were left for
a short time without a head; and the powers which he had exercised were
divided among the Town Councils, the Provincial States, and the States
General.

But, a few days after William's death, his widow, Mary, daughter of
Charles the first, King of Great Britain, gave birth to a son, destined
to raise the glory and authority of the House of Nassau to the highest
point, to save the United Provinces from slavery, to curb the power
of France, and to establish the English constitution on a lasting
foundation.

This Prince, named William Henry, was from his birth an object of
serious apprehension to the party now supreme in Holland, and of loyal
attachment to the old friends of his line. He enjoyed high consideration
as the possessor of a splendid fortune, as the chief of one of the most
illustrious houses in Europe, as a Magnate of the German empire, as a
prince of the blood royal of England, and, above all, as the descendant
of the founders of Batavian liberty. But the high office which had once
been considered as hereditary in his family remained in abeyance; and
the intention of the aristocratical party was that there should never
be another Stadtholder. The want of a first magistrate was, to a great
extent, supplied by the Grand Pensionary of the Province of Holland,
John De Witt, whose abilities, firmness, and integrity had raised him to
unrivalled authority in the councils of the municipal oligarchy.

The French invasion produced a complete change. The suffering and
terrified people raged fiercely against the government. In their madness
they attacked the bravest captains and the ablest statesmen of the
distressed commonwealth. De Ruyter was insulted by the rabble. De Witt
was torn in pieces before the gate of the palace of the States General
at the Hague. The Prince of Orange, who had no share in the guilt of
the murder, but who, on this occasion, as on another lamentable occasion
twenty years later, extended to crimes perpetrated in his cause an
indulgence which has left a stain on his glory, became chief of
the government without a rival. Young as he was, his ardent and
unconquerable spirit, though disguised by a cold and sullen manner, soon
roused the courage of his dismayed countrymen. It was in vain that both
his uncle and the French King attempted by splendid offers to seduce him
from the cause of the Republic. To the States General he spoke a high
and inspiriting language. He even ventured to suggest a scheme which has
an aspect of antique heroism, and which, if it had been accomplished,
would have been the noblest subject for epic song that is to be found in
the whole compass of modern history. He told the deputies that, even if
their natal soil and the marvels with which human industry had covered
it were buried under the ocean, all was not lost. The Hollanders might
survive Holland. Liberty and pure religion, driven by tyrants and
bigots from Europe, might take refuge in the farthest isles of Asia. The
shipping in the ports of the republic would suffice to carry two
hundred thousand emigrants to the Indian Archipelago. There the Dutch
commonwealth might commence a new and more glorious existence, and might
rear, under the Southern Cross, amidst the sugar canes and nutmeg trees,
the Exchange of a wealthier Amsterdam, and the schools of a more learned
Leyden. The national spirit swelled and rose high. The terms offered
by the allies were firmly rejected. The dykes were opened. The whole
country was turned into one great lake from which the cities, with their
ramparts and steeples, rose like islands. The invaders were forced to
save themselves from destruction by a precipitate retreat. Lewis, who,
though he sometimes thought it necessary to appear at the head of his
troops, greatly preferred a palace to a camp, had already returned
to enjoy the adulation of poets and the smiles of ladies in the newly
planted alleys of Versailles.

And now the tide turned fast. The event of the maritime war had been
doubtful; by land the United Provinces had obtained a respite; and a
respite, though short, was of infinite importance. Alarmed by the vast
designs of Lewis, both the branches of the great House of Austria sprang
to arms. Spain and Holland, divided by the memory of ancient wrongs and
humiliations, were reconciled by the nearness of the common danger.
From every part of Germany troops poured towards the Rhine. The English
government had already expended all the funds which had been obtained by
pillaging the public creditor. No loan could be expected from the City.
An attempt to raise taxes by the royal authority would have at once
produced a rebellion; and Lewis, who had now to maintain a contest
against half Europe, was in no condition to furnish the means of
coercing the people of England. It was necessary to convoke the
Parliament.

In the spring of 1673, therefore, the Houses reassembled after a recess
of near two years. Clifford, now a peer and Lord Treasurer, and Ashley,
now Earl of Shaftesbury and Lord Chancellor, were the persons on whom
the King principally relied as Parliamentary managers. The Country Party
instantly began to attack the policy of the Cabal. The attack was made,
not in the way of storm, but by slow and scientific approaches. The
Commons at first held out hopes that they would give support to the
king's foreign policy, but insisted that he should purchase that support
by abandoning his whole system of domestic policy. Their chief object
was to obtain the revocation of the Declaration of Indulgence. Of all
the many unpopular steps taken by the government the most unpopular was
the publishing of this Declaration. The most opposite sentiments had
been shocked by an act so liberal, done in a manner so despotic. All
the enemies of religious freedom, and all the friends of civil freedom,
found themselves on the same side; and these two classes made up
nineteen twentieths of the nation. The zealous churchman exclaimed
against the favour which had been shown both to the Papist and to the
Puritan. The Puritan, though he might rejoice in the suspension of the
persecution by which he had been harassed, felt little gratitude for a
toleration which he was to share with Antichrist. And all Englishmen who
valued liberty and law, saw with uneasiness the deep inroad which the
prerogative had made into the province of the legislature.

It must in candour be admitted that the constitutional question was then
not quite free from obscurity. Our ancient Kings had undoubtedly claimed
and exercised the right of suspending the operation of penal laws. The
tribunals had recognised that right. Parliaments had suffered it to pass
unchallenged. That some such right was inherent in the crown, few even
of the Country Party ventured, in the face of precedent and authority,
to deny. Yet it was clear that, if this prerogative were without limit,
the English government could scarcely be distinguished from a pure
despotism. That there was a limit was fully admitted by the King and his
ministers. Whether the Declaration of Indulgence lay within or without
the limit was the question; and neither party could succeed in tracing
any line which would bear examination. Some opponents of the government
complained that the Declaration suspended not less than forty statutes.
But why not forty as well as one? There was an orator who gave it as his
opinion that the King might constitutionally dispense with bad laws, but
not with good laws. The absurdity of such a distinction it is needless
to expose. The doctrine which seems to have been generally received
in the House of Commons was, that the dispensing power was confined to
secular matters, and did not extend to laws enacted for the security
of the established religion. Yet, as the King was supreme head of the
Church, it should seem that, if he possessed the dispensing power at
all, he might well possess that power where the Church was concerned.
When the courtiers on the other side attempted to point out the bounds
of this prerogative, they were not more successful than the opposition
had been.

The truth is that the dispensing power was a great anomaly in politics.
It was utterly inconsistent in theory with the principles of mixed
government: but it had grown up in times when people troubled themselves
little about theories. 
%[19]
\footnote{ The most sensible thing said in the House of Commons, on
this subject, came from Sir William Coventry: "Our ancestors never did
draw a line to circumscribe prerogative and liberty."}
 It had not been very grossly abused in
practice. It had therefore been tolerated, and had gradually acquired a
kind of prescription. At length it was employed, after a long interval,
in an enlightened age, and at an important conjuncture, to an extent
never before known, and for a purpose generally abhorred. It was
instantly subjected to a severe scrutiny. Men did not, indeed, at first,
venture to pronounce it altogether unconstitutional. But they began
to perceive that it was at direct variance with the spirit of the
constitution, and would, if left unchecked, turn the English government
from a limited into an absolute monarchy.

Under the influence of such apprehensions, the Commons denied the King's
right to dispense, not indeed with all penal statutes, but with penal
statutes in matters ecclesiastical, and gave him plainly to understand
that, unless he renounced that right, they would grant no supply for the
Dutch war. He, for a moment, showed some inclination to put everything
to hazard; but he was strongly advised by Lewis to submit to necessity,
and to wait for better times, when the French armies, now employed in an
arduous struggle on the Continent, might be available for the purpose
of suppressing discontent in England. In the Cabal itself the signs of
disunion and treachery began to appear. Shaftesbury, with his proverbial
sagacity, saw that a violent reaction was at hand, and that all things
were tending towards a crisis resembling that of 1640. He was determined
that such a crisis should not find him in the situation of Strafford.
He therefore turned suddenly round, and acknowledged, in the House of
Lords, that the Declaration was illegal. The King, thus deserted by
his ally and by his Chancellor, yielded, cancelled the Declaration, and
solemnly promised that it should never be drawn into precedent.

Even this concession was insufficient. The Commons, not content with
having forced their sovereign to annul the Indulgence, next extorted his
unwilling assent to a celebrated law, which continued in force down
to the reign of George the Fourth. This law, known as the Test Act,
provided that all persons holding any office, civil or military, should
take the oath of supremacy, should subscribe a declaration against
transubstantiation, and should publicly receive the sacrament according
to the rites of the Church of England. The preamble expressed hostility
only to the Papists: but the enacting clauses were scarcely more
unfavourable to the Papists than to the rigid Puritans. The Puritans,
however, terrified at the evident leaning of the court towards Popery,
and encouraged by some churchmen to hope that, as soon as the Roman
Catholics should have been effectually disarmed, relief would be
extended to Protestant Nonconformists, made little opposition; nor could
the King, who was in extreme want of money, venture to withhold his
sanction. The act was passed; and the Duke of York was consequently
under the necessity of resigning the great place of Lord High Admiral.

Hitherto the Commons had not declared against the Dutch war. But, when
the King had, in return for money cautiously doled out, relinquished
his whole plan of domestic policy, they fell impetuously on his foreign
policy. They requested him to dismiss Buckingham and Lauderdale from his
councils forever, and appointed a committee to consider the propriety of
impeaching Arlington. In a short time the Cabal was no more. Clifford,
who, alone of the five, had any claim to be regarded as an honest man,
refused to take the new test, laid down his white staff, and retired to
his country seat. Arlington quitted the post of Secretary of State for
a quiet and dignified employment in the Royal household. Shaftesbury
and Buckingham made their peace with the opposition, and appeared at
the head of the stormy democracy of the city. Lauderdale, however, still
continued to be minister for Scotch affairs, with which the English
Parliament could not interfere.

And now the Commons urged the King to make peace with Holland, and
expressly declared that no more supplies should be granted for the war,
unless it should appear that the enemy obstinately refused to consent
to reasonable terms. Charles found it necessary to postpone to a more
convenient season all thought of executing the treaty of Dover, and to
cajole the nation by pretending to return to the policy of the Triple
Alliance. Temple, who, during the ascendency of the Cabal, had lived
in seclusion among his books and flower beds, was called forth from his
hermitage. By his instrumentality a separate peace was concluded with
the United Provinces; and he again became ambassador at the Hague, where
his presence was regarded as a sure pledge for the sincerity of his
court.

The chief direction of affairs was now intrusted to Sir Thomas Osborne,
a Yorkshire baronet, who had, in the House of Commons, shown eminent
talents for business and debate. Osborne became Lord Treasurer, and was
soon created Earl of Danby. He was not a man whose character, if tried
by any high standard of morality, would appear to merit approbation. He
was greedy of wealth and honours, corrupt himself, and a corrupter of
others. The Cabal had bequeathed to him the art of bribing Parliaments,
an art still rude, and giving little promise of the rare perfection to
which it was brought in the following century. He improved greatly on
the plan of the first inventors. They had merely purchased orators:
but every man who had a vote, might sell himself to Danby. Yet the new
minister must not be confounded with the negotiators of Dover. He was
not without the feelings of an Englishman and a Protestant; nor did
he, in his solicitude for his own interests, ever wholly forget the
interests of his country and of his religion. He was desirous, indeed,
to exalt the prerogative: but the means by which he proposed to exalt
it were widely different from those which had been contemplated by
Arlington and Clifford. The thought of establishing arbitrary power, by
calling in the aid of foreign arms, and by reducing the kingdom to the
rank of a dependent principality, never entered into his mind. His plan
was to rally round the monarchy those classes which had been the firm
allies of the monarchy during the troubles of the preceding generation,
and which had been disgusted by the recent crimes and errors of the
court. With the help of the old Cavalier interest, of the nobles, of the
country gentlemen, of the clergy, and of the Universities, it might,
he conceived, be possible to make Charles, not indeed an absolute
sovereign, but a sovereign scarcely less powerful than Elizabeth had
been.

Prompted by these feelings, Danby formed the design of securing to the
Cavalier party the exclusive possession of all political power both
executive and legislative. In the year 1675, accordingly, a bill was
offered to the Lords which provided that no person should hold any
office, or should sit in either House of Parliament, without first
declaring on oath that he considered resistance to the kingly power as
in all cases criminal, and that he would never endeavour to alter the
government either in Church or State. During several weeks the debates,
divisions, and protests caused by this proposition kept the country in a
state of excitement. The opposition in the House of Lords, headed by
two members of the Cabal who were desirous to make their peace with the
nation, Buckingham and Shaftesbury, was beyond all precedent vehement
and pertinacious, and at length proved successful. The bill was not
indeed rejected, but was retarded, mutilated, and at length suffered to
drop.

So arbitrary and so exclusive was Danby's scheme of domestic policy. His
opinions touching foreign policy did him more honour. They were in truth
directly opposed to those of the Cabal and differed little from those of
the Country Party. He bitterly lamented the degraded situation to which
England was reduced, and declared, with more energy than politeness,
that his dearest wish was to cudgel the French into a proper respect
for her. So little did he disguise his feelings that, at a great banquet
where the most illustrious dignitaries of the State and of the Church
were assembled, he not very decorously filled his glass to the confusion
of all who were against a war with France. He would indeed most gladly
have seen his country united with the powers which were then combined
against Lewis, and was for that end bent on placing Temple, the author
of the Triple Alliance, at the head of the department which directed
foreign affairs. But the power of the prime minister was limited. In
his most confidential letters he complained that the infatuation of his
master prevented England from taking her proper place among European
nations. Charles was insatiably greedy of French gold: he had by no
means relinquished the hope that he might, at some future day, be able
to establish absolute monarchy by the help of the French arms; and for
both reasons he wished to maintain a good understanding with the court
of Versailles.

Thus the sovereign leaned towards one system of foreign politics,
and the minister towards a system diametrically opposite. Neither the
sovereign nor the minister, indeed, was of a temper to pursue any object
with undeviating constancy. Each occasionally yielded to the importunity
of the other; and their jarring inclinations and mutual concessions gave
to the whole administration a strangely capricious character. Charles
sometimes, from levity and indolence, suffered Danby to take steps which
Lewis resented as mortal injuries. Danby, on the other hand, rather
than relinquish his great place, sometimes stooped to compliances which
caused him bitter pain and shame. The King was brought to consent to a
marriage between the Lady Mary, eldest daughter and presumptive heiress
of the Duke of York and William of Orange, the deadly enemy of France
and the hereditary champion of the Reformation. Nay, the brave Earl of
Ossory, son of Ormond, was sent to assist the Dutch with some British
troops, who, on the most bloody day of the whole war, signally
vindicated the national reputation for stubborn courage. The Treasurer,
on the other hand, was induced not only to connive at some scandalous
pecuniary transactions which took place between his master and the court
of Versailles, but to become, unwillingly indeed and ungraciously, an
agent in those transactions.

Meanwhile the Country Party was driven by two strong feelings in two
opposite directions. The popular leaders were afraid of the greatness
of Lewis, who was not only making head against the whole strength of the
continental alliance, but was even gaining ground. Yet they were afraid
to entrust their own King with the means of curbing France, lest those
means should be used to destroy the liberties of England. The conflict
between these apprehensions, both of which were perfectly legitimate,
made the policy of the Opposition seem as eccentric and fickle as that
of the Court. The Commons called for a war with France, till the King,
pressed by Danby to comply with their wish, seemed disposed to yield,
and began to raise an army. But, as soon as they saw that the recruiting
had commenced, their dread of Lewis gave place to a nearer dread. They
began to fear that the new levies might be employed on a service in
which Charles took much more interest than in the defence of Flanders.
They therefore refused supplies, and clamoured for disbanding as loudly
as they had just before clamoured for arming. Those historians who
have severely reprehended this inconsistency do not appear to have made
sufficient allowance for the embarrassing situation of subjects who have
reason to believe that their prince is conspiring with a foreign and
hostile power against their liberties. To refuse him military resources
is to leave the state defenceless. Yet to give him military resources
may be only to arm him against the state. In such circumstances
vacillation cannot be considered as a proof of dishonesty or even of
weakness.

These jealousies were studiously fomented by the French King. He had
long kept England passive by promising to support the throne against the
Parliament. He now, alarmed at finding that the patriotic counsels
of Danby seemed likely to prevail in the closet, began to inflame the
Parliament against the throne. Between Lewis and the Country Party there
was one thing, and one only in common, profound distrust of Charles.
Could the Country Party have been certain that their sovereign meant
only to make war on France, they would have been eager to support him.
Could Lewis have been certain that the new levies were intended only to
make war on the constitution of England, he would have made no attempt
to stop them. But the unsteadiness and faithlessness of Charles were
such that the French Government and the English opposition, agreeing in
nothing else, agreed in disbelieving his protestations, and were equally
desirous to keep him poor and without an army. Communications were
opened between Barillon, the Ambassador of Lewis, and those English
politicians who had always professed, and who indeed sincerely felt, the
greatest dread and dislike of the French ascendency. The most upright of
the Country Party, William Lord Russell, son of the Earl of Bedford, did
not scruple to concert with a foreign mission schemes for embarrassing
his own sovereign. This was the whole extent of Russell's offence. His
principles and his fortune alike raised him above all temptations of a
sordid kind: but there is too much reason to believe that some of his
associates were less scrupulous. It would be unjust to impute to them
the extreme wickedness of taking bribes to injure their country. On the
contrary, they meant to serve her: but it is impossible to deny that
they were mean and indelicate enough to let a foreign prince pay them
for serving her. Among those who cannot be acquitted of this degrading
charge was one man who is popularly considered as the personification
of public spirit, and who, in spite of some great moral and intellectual
faults, has a just claim to be called a hero, a philosopher, and a
patriot. It is impossible to see without pain such a name in the list of
the pensioners of France. Yet it is some consolation to reflect that, in
our time, a public man would be thought lost to all sense of duty and
of shame, who should not spurn from him a temptation which conquered the
virtue and the pride of Algernon Sydney.

The effect of these intrigues was that England, though she occasionally
took a menacing attitude, remained inactive till the continental
war, having lasted near seven years, was terminated by the treaty of
Nimeguen. The United Provinces, which in 1672 had seemed to be on the
verge of utter ruin, obtained honourable and advantageous terms. This
narrow escape was generally ascribed to the ability and courage of the
young Stadtholder. His fame was great throughout Europe, and especially
among the English, who regarded him as one of their own princes, and
rejoiced to see him the husband of their future Queen. France retained
many important towns in the Low Countries and the great province of
Franche Comte. Almost the whole loss was borne by the decaying monarchy
of Spain.

A few months after the termination of hostilities on the Continent came
a great crisis in English politics. Towards such a crisis things had
been tending during eighteen years. The whole stock of popularity, great
as it was, with which the King had commenced his administration,
had long been expended. To loyal enthusiasm had succeeded profound
disaffection. The public mind had now measured back again the space
over which it had passed between 1640 and 1660, and was once more in the
state in which it had been when the Long Parliament met.

The prevailing discontent was compounded of many feelings. One of these
was wounded national pride. That generation had seen England, during a
few years, allied on equal terms with France, victorious over Holland
and Spain, the mistress of the sea, the terror of Rome, the head of the
Protestant interest. Her resources had not diminished; and it might have
been expected that she would have been at least as highly considered
in Europe under a legitimate King, strong in the affection and willing
obedience of his subjects, as she had been under an usurper whose utmost
vigilance and energy were required to keep down a mutinous people. Yet
she had, in consequence of the imbecility and meanness of her rulers,
sunk so low that any German or Italian principality which brought
five thousand men into the field was a more important member of the
commonwealth of nations.

With the sense of national humiliation was mingled anxiety for civil
liberty. Rumours, indistinct indeed, but perhaps the more alarming by
reason of their indistinctness, imputed to the court a deliberate design
against all the constitutional rights of Englishmen. It had even
been whispered that this design was to be carried into effect by the
intervention of foreign arms. The thought of Such intervention made the
blood, even of the Cavaliers, boil in their veins. Some who had always
professed the doctrine of non-resistance in its full extent were now
heard to mutter that there was one limitation to that doctrine. If a
foreign force were brought over to coerce the nation, they would not
answer for their own patience.

But neither national pride nor anxiety for public liberty had so
great an influence on the popular mind as hatred of the Roman Catholic
religion. That hatred had become one of the ruling passions of the
community, and was as strong in the ignorant and profane as in those
who were Protestants from conviction. The cruelties of Mary's reign,
cruelties which even in the most accurate and sober narrative excite
just detestation, and which were neither accurately nor soberly related
in the popular martyrologies, the conspiracies against Elizabeth, and
above all the Gunpowder Plot, had left in the minds of the vulgar a deep
and bitter feeling which was kept up by annual commemorations, prayers,
bonfires, and processions. It should be added that those classes which
were peculiarly distinguished by attachment to the throne, the clergy
and the landed gentry, had peculiar reasons for regarding the Church of
Rome with aversion. The clergy trembled for their benefices; the landed
gentry for their abbeys and great tithes. While the memory of the reign
of the Saints was still recent, hatred of Popery had in some degree
given place to hatred of Puritanism; but, during the eighteen years
which had elapsed since the Restoration, the hatred of Puritanism had
abated, and the hatred of Popery had increased. The stipulations of the
treaty of Dover were accurately known to very few; but some hints had
got abroad. The general impression was that a great blow was about to
be aimed at the Protestant religion. The King was suspected by many of a
leaning towards Rome. His brother and heir presumptive was known to be
a bigoted Roman Catholic. The first Duchess of York had died a Roman
Catholic. James had then, in defiance of the remonstrances of the House
of Commons, taken to wife the Princess Mary of Modena, another Roman
Catholic. If there should be sons by this marriage, there was reason to
fear that they might be bred Roman Catholics, and that a long succession
of princes, hostile to the established faith, might sit on the English
throne. The constitution had recently been violated for the purpose of
protecting the Roman Catholics from the penal laws. The ally by whom the
policy of England had, during many years, been chiefly governed, was not
only a Roman Catholic, but a persecutor of the reformed Churches. Under
such circumstances it is not strange that the common people should have
been inclined to apprehend a return of the times of her whom they called
Bloody Mary.

Thus the nation was in such a temper that the smallest spark might raise
a flame. At this conjuncture fire was set in two places at once to the
vast mass of combustible matter; and in a moment the whole was in a
blaze.

The French court, which knew Danby to be its mortal enemy, artfully
contrived to ruin him by making him pass for its friend. Lewis, by the
instrumentality of Ralph Montague, a faithless and shameless man who
had resided in France as minister from England, laid before the House of
Commons proofs that the Treasurer had been concerned in an application
made by the Court of Whitehall to the Court of Versailles for a sum of
money. This discovery produced its natural effect. The Treasurer was,
in truth, exposed to the vengeance of Parliament, not on account of his
delinquencies, but on account of his merits; not because he had been
an accomplice in a criminal transaction, but because he had been a most
unwilling and unserviceable accomplice. But of the circumstances, which
have, in the judgment of posterity, greatly extenuated his fault, his
contemporaries were ignorant. In their view he was the broker who had
sold England to France. It seemed clear that his greatness was at an
end, and doubtful whether his head could be saved.

Yet was the ferment excited by this discovery slight, when compared with
the commotion which arose when it was noised abroad that a great Popish
plot had been detected. One Titus Oates, a clergyman of the Church of
England, had, by his disorderly life and heterodox doctrine, drawn on
himself the censure of his spiritual superiors, had been compelled to
quit his benefice, and had ever since led an infamous and vagrant life.
He had once professed himself a Roman Catholic, and had passed some time
on the Continent in English colleges of the order of Jesus. In those
seminaries he had heard much wild talk about the best means of
bringing England back to the true Church. From hints thus furnished he
constructed a hideous romance, resembling rather the dream of a sick man
than any transaction which ever took place in the real world. The Pope,
he said, had entrusted the government of England to the Jesuits. The
Jesuits had, by commissions under the seal of their society, appointed
Roman Catholic clergymen, noblemen, and gentlemen, to all the highest
offices in Church and State. The Papists had burned down London once.
They had tried to burn it down again. They were at that moment planning
a scheme for setting fire to all the shipping in the Thames. They were
to rise at a signal and massacre all their Protestant neighbours. A
French army was at the same time to land in Ireland. All the leading
statesmen and divines of England were to be murdered. Three or four
schemes had been formed for assassinating the King. He was to be
stabbed. He was to be poisoned in his medicine He was to be shot with
silver bullets. The public mind was so sore and excitable that these
lies readily found credit with the vulgar; and two events which speedily
took place led even some reflecting men to suspect that the tale, though
evidently distorted and exaggerated, might have some foundation.

Edward Coleman, a very busy, and not very honest, Roman Catholic
intriguer, had been among the persons accused. Search was made for his
papers. It was found that he had just destroyed the greater part of
them. But a few which had escaped contained some passages such as,
to minds strongly prepossessed, might seem to confirm the evidence of
Oates. Those passages indeed, when candidly construed, appear to
express little more than the hopes which the posture of affairs, the
predilections of Charles, the still stronger predilections of James,
and the relations existing between the French and English courts, might
naturally excite in the mind of a Roman Catholic strongly attached to
the interests of his Church. But the country was not then inclined to
construe the letters of Papists candidly; and it was urged, with
some show of reason, that, if papers which had been passed over as
unimportant were filled with matter so suspicious, some great mystery
of iniquity must have been contained in those documents which had been
carefully committed to the flames.

A few days later it was known that Sir Edmondsbury Godfrey, an eminent
justice of the peace who had taken the depositions of Oates against
Coleman, had disappeared. Search was made; and Godfrey's corpse was
found in a field near London. It was clear that he had died by violence.
It was equally clear that he had not been set upon by robbers. His fate
is to this day a secret. Some think that he perished by his own
hand; some, that he was slain by a private enemy. The most improbable
supposition is that he was murdered by the party hostile to the court,
in order to give colour to the story of the plot. The most probable
supposition seems, on the whole, to be that some hotheaded Roman
Catholic, driven to frenzy by the lies of Oates and by the insults
of the multitude, and not nicely distinguishing between the perjured
accuser and the innocent magistrate, had taken a revenge of which the
history of persecuted sects furnishes but too many examples. If this
were so, the assassin must have afterwards bitterly execrated his own
wickedness and folly. The capital and the whole nation went mad with
hatred and fear. The penal laws, which had begun to lose something of
their edge, were sharpened anew. Everywhere justices were busied in
searching houses and seizing papers. All the gaols were filled with
Papists. London had the aspect of a city in a state of siege. The
trainbands were under arms all night. Preparations were made for
barricading the great thoroughfares. Patrols marched up and down the
streets. Cannon were planted round Whitehall. No citizen thought himself
safe unless he carried under his coat a small flail loaded with lead to
brain the Popish assassins. The corpse of the murdered magistrate was
exhibited during several days to the gaze of great multitudes, and was
then committed to the grave with strange and terrible ceremonies,
which indicated rather fear and the thirst of vengeance shall sorrow or
religious hope. The Houses insisted that a guard should be placed in the
vaults over which they sate, in order to secure them against a second
Gunpowder Plot. All their proceedings were of a piece with this demand.
Ever since the reign of Elizabeth the oath of supremacy had been exacted
from members of the House of Commons. Some Roman Catholics, however,
had contrived so to interpret this oath that they could take it without
scruple. A more stringent test was now added: every member of Parliament
was required to make the Declaration against Transubstantiation; and
thus the Roman Catholic Lords were for the first time excluded from
their seats. Strong resolutions were adopted against the Queen. The
Commons threw one of the Secretaries of State into prison for having
countersigned commissions directed to gentlemen who were not good
Protestants. They impeached the Lord Treasurer of high treason. Nay,
they so far forgot the doctrine which, while the memory of the civil war
was still recent, they had loudly professed, that they even attempted
to wrest the command of the militia out of the King's hands. To such
a temper had eighteen years of misgovernment brought the most loyal
Parliament that had ever met in England.

Yet it may seem strange that, even in that extremity, the King should
have ventured to appeal to the people; for the people were more excited
than their representatives. The Lower House, discontented as it was,
contained a larger number of Cavaliers than were likely to find seats
again. But it was thought that a dissolution would put a stop to the
prosecution of the Lord Treasurer, a prosecution which might probably
bring to light all the guilty mysteries of the French alliance, and
might thus cause extreme personal annoyance and embarrassment to
Charles. Accordingly, in January, 1679, the Parliament, which had been
in existence ever since the beginning of the year 1661, was dissolved;
and writs were issued for a general election.

During some weeks the contention over the whole country was fierce and
obstinate beyond example. Unprecedented sums were expended. New tactics
were employed. It was remarked by the pamphleteers of that time as
something extraordinary that horses were hired at a great charge for
the conveyance of electors. The practice of splitting freeholds for
the purpose of multiplying votes dates from this memorable struggle.
Dissenting preachers, who had long hidden themselves in quiet nooks from
persecution, now emerged from their retreats, and rode from village to
village, for the purpose of rekindling the zeal of the scattered people
of God. The tide ran strong against the government. Most of the new
members came up to Westminster in a mood little differing from that of
their predecessors who had sent Strafford and Laud to the Tower.

Meanwhile the courts of justice, which ought to be, in the midst of
political commotions, sure places of refuge for the innocent of every
party, were disgraced by wilder passions and fouler corruptions than
were to be found even on the hustings. The tale of Oates, though it had
sufficed to convulse the whole realm, would not, unless confirmed by
other evidence, suffice to destroy the humblest of those whom he had
accused. For, by the old law of England, two witnesses are necessary
to establish a charge of treason. But the success of the first impostor
produced its natural consequences. In a few weeks he had been raised
from penury and obscurity to opulence, to power which made him the dread
of princes and nobles, and to notoriety such as has for low and bad
minds all the attractions of glory. He was not long without coadjutors
and rivals. A wretch named Carstairs, who had earned a livelihood in
Scotland by going disguised to conventicles and then informing against
the preachers, led the way. Bedloe, a noted swindler, followed; and soon
from all the brothels, gambling houses, and spunging houses of London,
false witnesses poured forth to swear away the lives of Roman Catholics.
One came with a story about an army of thirty thousand men who were to
muster in the disguise of pilgrims at Corunna, and to sail thence to
Wales. Another had been promised canonisation and five hundred pounds
to murder the King. A third had stepped into an eating house in Covent
Garden, and had there heard a great Roman Catholic banker vow, in the
hearing of all the guests and drawers, to kill the heretical tyrant.
Oates, that he might not be eclipsed by his imitators, soon added
a large supplement to his original narrative. He had the portentous
impudence to affirm, among other things, that he had once stood behind a
door which was ajar, and had there overheard the Queen declare that she
had resolved to give her consent to the assassination of her husband.
The vulgar believed, and the highest magistrates pretended to believe,
even such fictions as these. The chief judges of the realm were corrupt,
cruel, and timid. The leaders of the Country Party encouraged the
prevailing delusion. The most respectable among them, indeed, were
themselves so far deluded as to believe the greater part of the evidence
of the plot to be true. Such men as Shaftesbury and Buckingham doubtless
perceived that the whole was a romance. But it was a romance which
served their turn; and to their seared consciences the death of an
innocent man gave no more uneasiness than the death of a partridge. The
juries partook of the feelings then common throughout the nation,
and were encouraged by the bench to indulge those feelings without
restraint. The multitude applauded Oates and his confederates, hooted
and pelted the witnesses who appeared on behalf of the accused, and
shouted with joy when the verdict of Guilty was pronounced. It was in
vain that the sufferers appealed to the respectability of their past
lives: for the public mind was possessed with a belief that the more
conscientious a Papist was, the more likely he must be to plot against a
Protestant government. It was in vain that, just before the cart passed
from under their feet, they resolutely affirmed their innocence: for the
general opinion was that a good Papist considered all lies which were
serviceable to his Church as not only excusable but meritorious.

While innocent blood was shedding under the forms of justice, the new
Parliament met; and such was the violence of the predominant party
that even men whose youth had been passed amidst revolutions men who
remembered the attainder of Strafford, the attempt on the five members,
the abolition of the House of Lords, the execution of the King, stood
aghast at the aspect of public affairs. The impeachment of Danby was
resumed. He pleaded the royal pardon. But the Commons treated the
plea with contempt, and insisted that the trial should proceed. Danby,
however, was not their chief object. They were convinced that the only
effectual way of securing the liberties and religion of the nation was
to exclude the Duke of York from the throne.

The King was in great perplexity. He had insisted that his brother, the
sight of whom inflamed the populace to madness, should retire for a
time to Brussels: but this concession did not seem to have produced any
favourable effect. The Roundhead party was now decidedly preponderant.
Towards that party leaned millions who had, at the time of the
Restoration, leaned towards the side of prerogative. Of the old
Cavaliers many participated in the prevailing fear of Popery, and many,
bitterly resenting the ingratitude of the prince for whom they had
sacrificed so much, looked on his distress as carelessly as he had
looked on theirs. Even the Anglican clergy, mortified and alarmed by the
apostasy of the Duke of York, so far countenanced the opposition as to
join cordially in the outcry against the Roman Catholics.

The King in this extremity had recourse to Sir William Temple. Of all
the official men of that age Temple had preserved the fairest character.
The Triple Alliance had been his work. He had refused to take any
part in the politics of the Cabal, and had, while that administration
directed affairs, lived in strict privacy. He had quitted his retreat at
the call of Danby, had made peace between England and Holland, and had
borne a chief part in bringing about the marriage of the Lady Mary to
her cousin the Prince of Orange. Thus he had the credit of every one
of the few good things which had been done by the government since the
Restoration. Of the numerous crimes and blunders of the last eighteen
years none could be imputed to him. His private life, though not
austere, was decorous: his manners were popular; and he was not to be
corrupted either by titles or by money. Something, however, was wanting
to the character of this respectable statesman. The temperature of his
patriotism was lukewarm. He prized his ease and his personal dignity
too much, and shrank from responsibility with a pusillanimous fear. Nor
indeed had his habits fitted him to bear a part in the conflicts of our
domestic factions. He had reached his fiftieth year without having sate
in the English Parliament; and his official experience had been almost
entirely acquired at foreign courts. He was justly esteemed one of the
first diplomatists in Europe: but the talents and accomplishments of a
diplomatist are widely different from those which qualify a politician
to lead the House of Commons in agitated times.

The scheme which he proposed showed considerable ingenuity. Though not
a profound philosopher, he had thought more than most busy men of the
world on the general principles of government; and his mind had been
enlarged by historical studies and foreign travel. He seems to have
discerned more clearly than most of his contemporaries one cause of the
difficulties by which the government was beset. The character of the
English polity was gradually changing. The Parliament was slowly, but
constantly, gaining ground on the prerogative. The line between the
legislative and executive powers was in theory as strongly marked as
ever, but in practice was daily becoming fainter and fainter. The theory
of the constitution was that the King might name his own ministers.
But the House of Commons had driven Clarendon, the Cabal, and
Danby successively from the direction of affairs. The theory of the
constitution was that the King alone had the power of making peace and
war. But the House of Commons had forced him to make peace with Holland,
and had all but forced him to make war with France. The theory of the
constitution was that the King was the sole judge of the cases in which
it might be proper to pardon offenders. Yet he was so much in dread
of the House of Commons that, at that moment, he could not venture to
rescue from the gallows men whom he well knew to be the innocent victims
of perjury.

Temple, it should seem, was desirous to secure to the legislature its
undoubted constitutional powers, and yet to prevent it, if
possible, from encroaching further on the province of the executive
administration. With this view he determined to interpose between the
sovereign and the Parliament a body which might break the shock of their
collision. There was a body ancient, highly honourable, and recognised
by the law, which, he thought, might be so remodelled as to serve this
purpose. He determined to give to the Privy Council a new character and
office in the government. The number of Councillors he fixed at thirty.
Fifteen of them were to be the chief ministers of state, of law, and of
religion. The other fifteen were to be unplaced noblemen and gentlemen
of ample fortune and high character. There was to be no interior
cabinet. All the thirty were to be entrusted with every political
secret, and summoned to every meeting; and the King was to declare that
he would, on every occasion, be guided by their advice.

Temple seems to have thought that, by this contrivance, he could at
once secure the nation against the tyranny of the Crown, and the Crown
against the encroachments of the Parliament. It was, on one hand, highly
improbable that schemes such as had been formed by the Cabal would
be even propounded for discussion in an assembly consisting of thirty
eminent men, fifteen of whom were bound by no tie of interest to the
court. On the other hand, it might be hoped that the Commons, content
with the guarantee against misgovernment which such a Privy Council
furnished, would confine themselves more than they had of late done
to their strictly legislative functions, and would no longer think it
necessary to pry into every part of the executive administration.

This plan, though in some respects not unworthy of the abilities of its
author, was in principle vicious. The new board was half a cabinet and
half a Parliament, and, like almost every other contrivance, whether
mechanical or political, which is meant to serve two purposes altogether
different, failed of accomplishing either. It was too large and too
divided to be a good administrative body. It was too closely connected
with the Crown to be a good checking body. It contained just enough of
popular ingredients to make it a bad council of state, unfit for the
keeping of secrets, for the conducting of delicate negotiations, and
for the administration of war. Yet were these popular ingredients by no
means sufficient to secure the nation against misgovernment. The
plan, therefore, even if it had been fairly tried, could scarcely
have succeeded; and it was not fairly tried. The King was fickle
and perfidious: the Parliament was excited and unreasonable; and the
materials out of which the new Council was made, though perhaps the best
which that age afforded, were still bad.

The commencement of the new system was, however, hailed with general
delight; for the people were in a temper to think any change an
improvement. They were also pleased by some of the new nominations.
Shaftesbury, now their favourite, was appointed Lord President. Russell
and some other distinguished members of the Country Party were sworn
of the Council. But a few days later all was again in confusion. The
inconveniences of having so numerous a cabinet were such that Temple
himself consented to infringe one of the fundamental rules which he
had laid down, and to become one of a small knot which really directed
everything. With him were joined three other ministers, Arthur Capel,
Earl of Essex, George Savile, Viscount Halifax, and Robert Spencer, Earl
of Sunderland.

Of the Earl of Essex, then First Commissioner of the Treasury, it is
sufficient to say that he was a man of solid, though not brilliant
parts, and of grave and melancholy character, that he had been connected
with the Country Party, and that he was at this time honestly desirous
to effect, on terms beneficial to the state, a reconciliation between
that party and the throne.

Among the statesmen of those times Halifax was, in genius, the first.
His intellect was fertile, subtle, and capacious. His polished,
luminous, and animated eloquence, set off by the silver tones of
his voice, was the delight of the House of Lords. His conversation
overflowed with thought, fancy, and wit. His political tracts well
deserve to be studied for their literary merit, and fully entitle him
to a place among English classics. To the weight derived from talents so
great and various he united all the influence which belongs to rank and
ample possessions. Yet he was less successful in politics than many who
enjoyed smaller advantages. Indeed, those intellectual peculiarities
which make his writings valuable frequently impeded him in the contests
of active life. For he always saw passing events, not in the point of
view in which they commonly appear to one who bears a part in them,
but in the point of view in which, after the lapse of many years, they
appear to the philosophic historian. With such a turn of mind he
could not long continue to act cordially with any body of men. All the
prejudices, all the exaggerations, of both the great parties in the
state moved his scorn. He despised the mean arts and unreasonable
clamours of demagogues. He despised still more the doctrines of divine
right and passive obedience. He sneered impartially at the bigotry of
the Churchman and at the bigotry of the Puritan. He was equally unable
to comprehend how any man should object to Saints' days and surplices,
and how any man should persecute any other man for objecting to them. In
temper he was what, in our time, is called a Conservative: in theory
he was a Republican. Even when his dread of anarchy and his disdain
for vulgar delusions led him to side for a time with the defenders of
arbitrary power, his intellect was always with Locke and Milton. Indeed,
his jests upon hereditary monarchy were sometimes such as would have
better become a member of the Calf's Head Club than a Privy Councillor
of the Stuarts. In religion he was so far from being a zealot that
he was called by the uncharitable an atheist: but this imputation he
vehemently repelled; and in truth, though he sometimes gave scandal by
the way in which he exerted his rare powers both of reasoning and
of ridicule on serious subjects, he seems to have been by no means
unsusceptible of religious impressions.

He was the chief of those politicians whom the two great parties
contemptuously called Trimmers. Instead of quarrelling with this
nickname, he assumed it as a title of honour, and vindicated, with great
vivacity, the dignity of the appellation. Everything good, he said,
trims between extremes. The temperate zone trims between the climate
in which men are roasted and the climate in which they are frozen.
The English Church trims between the Anabaptist madness and the Papist
lethargy. The English constitution trims between Turkish despotism and
Polish anarchy. Virtue is nothing but a just temper between propensities
any one of which, if indulged to excess, becomes vice. Nay, the
perfection of the Supreme Being himself consists in the exact
equilibrium of attributes, none of which could preponderate without
disturbing the whole moral and physical order of the world. 
%[20]
\footnote{ Halifax was undoubtedly the real author of the Character
of a Trimmer, which, for a time, went under the name of his kinsman, Sir
William Coventry.}

Thus Halifax was a Trimmer on principle. He was also a Trimmer by the
constitution both of his head and of his heart. His understanding was
keen, sceptical, inexhaustibly fertile in distinctions and objections;
his taste refined; his sense of the ludicrous exquisite; his temper
placid and forgiving, but fastidious, and by no means prone either to
malevolence or to enthusiastic admiration. Such a man could not long
be constant to any band of political allies. He must not, however, be
confounded with the vulgar crowd of renegades. For though, like them,
he passed from side to side, his transition was always in the direction
opposite to theirs. He had nothing in common with those who fly from
extreme to extreme, and who regard the party which they have deserted
with all animosity far exceeding that of consistent enemies. His
place was on the debatable ground between the hostile divisions of the
community, and he never wandered far beyond the frontier of either. The
party to which he at any moment belonged was the party which, at that
moment, he liked least, because it was the party of which at that moment
he had the nearest view. He was therefore always severe upon his violent
associates, and was always in friendly relations with his moderate
opponents. Every faction in the day of its insolent and vindictive
triumph incurred his censure; and every faction, when vanquished and
persecuted, found in him a protector. To his lasting honour it must be
mentioned that he attempted to save those victims whose fate has left
the deepest stain both on the Whig and on the Tory name.

He had greatly distinguished himself in opposition, and had thus drawn
on himself the royal displeasure, which was indeed so strong that he was
not admitted into the Council of Thirty without much difficulty and long
altercation. As soon, however, as he had obtained a footing at court,
the charms of his manner and of his conversation made him a favourite.
He was seriously alarmed by the violence of the public discontent.
He thought that liberty was for the present safe, and that order and
legitimate authority were in danger. He therefore, as was his fashion,
joined himself to the weaker side. Perhaps his conversion was not wholly
disinterested. For study and reflection, though they had emancipated
him from many vulgar prejudices, had left him a slave to vulgar desires.
Money he did not want; and there is no evidence that he ever obtained
it by any means which, in that age, even severe censors considered as
dishonourable; but rank and power had strong attractions for him. He
pretended, indeed, that he considered titles and great offices as baits
which could allure none but fools, that he hated business, pomp, and
pageantry, and that his dearest wish was to escape from the bustle and
glitter of Whitehall to the quiet woods which surrounded his ancient
mansion in Nottinghamshire; but his conduct was not a little at variance
with his professions. In truth he wished to command the respect at
once of courtiers and of philosophers, to be admired for attaining high
dignities, and to be at the same time admired for despising them.

Sunderland was Secretary of State. In this man the political immorality
of his age was personified in the most lively manner. Nature had given
him a keen understanding, a restless and mischievous temper, a cold
heart, and an abject spirit. His mind had undergone a training by
which all his vices had been nursed up to the rankest maturity. At his
entrance into public life, he had passed several years in diplomatic
posts abroad, and had been, during some time, minister in France. Every
calling has its peculiar temptations. There is no injustice in saying
that diplomatists, as a class, have always been more distinguished by
their address, by the art with which they win the confidence of those
with whom they have to deal, and by the ease with which they catch the
tone of every society into which they are admitted, than by generous
enthusiasm or austere rectitude; and the relations between Charles and
Lewis were such that no English nobleman could long reside in France as
envoy, and retain any patriotic or honourable sentiment. Sunderland
came forth from the bad school in which he had been brought up, cunning,
supple, shameless, free from all prejudices, and destitute of all
principles. He was, by hereditary connection, a Cavalier: but with the
Cavaliers he had nothing in common. They were zealous for monarchy, and
condemned in theory all resistance. Yet they had sturdy English hearts
which would never have endured real despotism. He, on the contrary,
had a languid speculative liking for republican institutions which was
compatible with perfect readiness to be in practice the most servile
instrument of arbitrary power. Like many other accomplished flatterers
and negotiators, he was far more skilful in the art of reading the
characters and practising on the weaknesses of individuals, than in the
art of discerning the feelings of great masses, and of foreseeing the
approach of great revolutions. He was adroit in intrigue; and it
was difficult even for shrewd and experienced men who had been amply
forewarned of his perfidy to withstand the fascination of his manner,
and to refuse credit to his professions of attachment. But he was so
intent on observing and courting particular persons, that he often
forgot to study the temper of the nation. He therefore miscalculated
grossly with respect to some of the most momentous events of his time.
More than one important movement and rebound of the public mind took
him by surprise; and the world, unable to understand how so clever a man
could be blind to what was clearly discerned by the politicians of the
coffee houses, sometimes attributed to deep design what were in truth
mere blunders.

It was only in private conference that his eminent abilities displayed
themselves. In the royal closet, or in a very small circle, he exercised
great influence. But at the Council board he was taciturn; and in the
House of Lords he never opened his lips.

The four confidential advisers of the crown soon found that their
position was embarrassing and invidious. The other members of the
Council murmured at a distinction inconsistent with the King's
promises; and some of them, with Shaftesbury at their head, again betook
themselves to strenuous opposition in Parliament. The agitation, which
had been suspended by the late changes, speedily became more violent
than ever. It was in vain that Charles offered to grant to the Commons
any security for the Protestant religion which they could devise,
provided only that they would not touch the order of succession. They
would hear of no compromise. They would have the Exclusion Bill, and
nothing but the Exclusion Bill. The King, therefore, a few weeks after
he had publicly promised to take no step without the advice of his
new Council, went down to the House of Lords without mentioning his
intention in Council, and prorogued the Parliament.

The day of that prorogation, the twenty-sixth of May, 1679, is a great
era in our history. For on that day the Habeas Corpus Act received the
royal assent. From the time of the Great Charter the substantive law
respecting the personal liberty of Englishmen had been nearly the same
as at present: but it had been inefficacious for want of a stringent
system of procedure. What was needed was not a new light, but a prompt
and searching remedy; and such a remedy the Habeas Corpus Act supplied.
The King would gladly have refused his consent to that measure: but he
was about to appeal from his Parliament to his people on the question
of the succession, and he could not venture, at so critical a moment, to
reject a bill which was in the highest degree popular.

On the same day the press of England became for a short time free. In
old times printers had been strictly controlled by the Court of Star
Chamber. The Long Parliament had abolished the Star Chamber, but had,
in spite of the philosophical and eloquent expostulation of Milton,
established and maintained a censorship. Soon after the Restoration, an
Act had been passed which prohibited the printing of unlicensed books;
and it had been provided that this Act should continue in force till
the end of the first session of the next Parliament. That moment had
now arrived; and the King, in the very act of dismissing the House,
emancipated the Press.

Shortly after the prorogation came a dissolution and another general
election. The zeal and strength of the opposition were at the height.
The cry for the Exclusion Bill was louder than ever, and with this cry
was mingled another cry, which fired the blood of the multitude, but
which was heard with regret and alarm by all judicious friends of
freedom. Not only the rights of the Duke of York, an avowed Papist,
but those of his two daughters, sincere and zealous Protestants, were
assailed. It was confidently affirmed that the eldest natural son of the
King had been born in wedlock, and was lawful heir to the crown.

Charles, while a wanderer on the Continent, had fallen in at the
Hague with Lucy Walters, a Welsh girl of great beauty, but of weak
understanding and dissolute manners. She became his mistress, and
presented him with a son. A suspicious lover might have had his doubts;
for the lady had several admirers, and was not supposed to be cruel to
any. Charles, however, readily took her word, and poured forth on little
James Crofts, as the boy was then called, an overflowing fondness, such
as seemed hardly to belong to that cool and careless nature. Soon after
the restoration, the young favourite, who had learned in France the
exercises then considered necessary to a fine gentleman, made his
appearance at Whitehall. He was lodged in the palace, attended by pages,
and permitted to enjoy several distinctions which had till then been
confined to princes of the blood royal. He was married, while still in
tender youth, to Anne Scott, heiress of the noble house of Buccleuch.
He took her name, and received with her hand possession of her ample
domains. The estate which he had acquired by this match was popularly
estimated at not less than ten thousand pounds a year. Titles, and
favours more substantial than titles, were lavished on him. He was made
Duke of Monmouth in England, Duke of Buccleuch in Scotland, a Knight of
the Garter, Master of the Horse, Commander of the first troop of Life
Guards, Chief Justice of Eyre south of Trent, and Chancellor of the
University of Cambridge. Nor did he appear to the public unworthy of his
high fortunes. His countenance was eminently handsome and engaging, his
temper sweet, his manners polite and affable. Though a libertine, he won
the hearts of the Puritans. Though he was known to have been privy
to the shameful attack on Sir John Coventry, he easily obtained the
forgiveness of the Country Party. Even austere moralists owned that, in
such a court, strict conjugal fidelity was scarcely to be expected from
one who, while a child, had been married to another child. Even patriots
were willing to excuse a headstrong boy for visiting with immoderate
vengeance an insult offered to his father. And soon the stain left by
loose amours and midnight brawls was effaced by honourable exploits.
When Charles and Lewis united their forces against Holland, Monmouth
commanded the English auxiliaries who were sent to the Continent, and
approved himself a gallant soldier and a not unintelligent officer. On
his return he found himself the most popular man in the kingdom. Nothing
was withheld from him but the crown; nor did even the crown seem to
be absolutely beyond his reach. The distinction which had most
injudiciously been made between him and the highest nobles had produced
evil consequences. When a boy he had been invited to put on his hat in
the presence chamber, while Howards and Seymours stood uncovered round
him. When foreign princes died, he had mourned for them in the long
purple cloak, which no other subject, except the Duke of York and Prince
Rupert, was permitted to wear. It was natural that these things should
lead him to regard himself as a legitimate prince of the House of
Stuart. Charles, even at a ripe age, was devoted to his pleasures and
regardless of his dignity. It could hardly be thought incredible that he
should at twenty have secretly gone through the form of espousing a lady
whose beauty had fascinated him. While Monmouth was still a child, and
while the Duke of York still passed for a Protestant, it was rumoured
throughout the country, and even in circles which ought to have been
well informed, that the King had made Lucy Walters his wife, and that,
if every one had his right, her son would be Prince of Wales. Much
was said of a certain black box which, according to the vulgar belief,
contained the contract of marriage. When Monmouth had returned from the
Low Countries with a high character for valour and conduct, and when the
Duke of York was known to be a member of a church detested by the great
majority of the nation, this idle story became important. For it
there was not the slightest evidence. Against it there was the solemn
asseveration of the King, made before his Council, and by his order
communicated to his people. But the multitude, always fond of romantic
adventures, drank in eagerly the tale of the secret espousals and the
black box. Some chiefs of the opposition acted on this occasion as they
acted with respect to the more odious fables of Oates, and countenanced
a story which they must have despised. The interest which the populace
took in him whom they regarded as the champion of the true religion, and
the rightful heir of the British throne, was kept up by every artifice.
When Monmouth arrived in London at midnight, the watchmen were ordered
by the magistrates to proclaim the joyful event through the streets of
the City: the people left their beds: bonfires were lighted: the windows
were illuminated: the churches were opened; and a merry peal rose from
all the steeples. When he travelled, he was everywhere received with not
less pomp, and with far more enthusiasm, than had been displayed when
Kings had made progresses through the realm. He was escorted from
mansion to mansion by long cavalcades of armed gentlemen and yeomen.
Cities poured forth their whole population to receive him. Electors
thronged round him, to assure him that their votes were at his disposal.
To such a height were his pretensions carried, that he not only
exhibited on his escutcheon the lions of England and the lilies of
France without the baton sinister under which, according to the law of
heraldry, they should have been debruised in token of his illegitimate
birth, but ventured to touch for the king's evil. At the same time he
neglected no art of condescension by which the love of the multitude
could be conciliated. He stood godfather to the children of the
peasantry, mingled in every rustic sport, wrestled, played at
quarterstaff, and won footraces in his boots against fleet runners in
shoes.

It is a curious circumstance that, at two of the greatest conjunctures
in our history, the chiefs of the Protestant party should have committed
the same error, and should by that error have greatly endangered their
country and their religion. At the death of Edward the Sixth they set up
the Lady Jane, without any show of birthright, in opposition, not only
to their enemy Mary, but also to Elizabeth, the true hope of England
and of the Reformation. Thus the most respectable Protestants, with
Elizabeth at their head, were forced to make common cause with the
Papists. In the same manner, a hundred and thirty years later, a part
of the opposition, by setting up Monmouth as a claimant of the crown,
attacked the rights, not only of James, whom they justly regarded as
an implacable foe of their faith and their liberties, but also of the
Prince and Princess of Orange, who were eminently marked out, both
by situation and by personal qualities, as the defenders of all free
governments and of all reformed churches.

The folly of this course speedily became manifest. At present the
popularity of Monmouth constituted a great part of the strength of the
opposition. The elections went against the court: the day fixed for
the meeting of the Houses drew near; and it was necessary that the
King should determine on some line of conduct. Those who advised him
discerned the first faint signs of a change of public feeling, and hoped
that, by merely postponing the conflict, he would be able to secure the
victory. He therefore, without even asking the opinion of the Council of
the Thirty, resolved to prorogue the new Parliament before it entered
on business. At the same time the Duke of York, who had returned from
Brussels, was ordered to retire to Scotland, and was placed at the head
of the administration of that kingdom.

Temple's plan of government was now avowedly abandoned and very soon
forgotten. The Privy Council again became what it had been. Shaftesbury,
and those who were connected with him in politics resigned their seats.
Temple himself, as was his wont in unquiet times, retired to his garden
and his library. Essex quitted the board of Treasury, and cast in his
lot with the opposition. But Halifax, disgusted and alarmed by the
violence of his old associates, and Sunderland, who never quitted place
while he could hold it, remained in the King's service.

In consequence of the resignations which took place at this conjuncture,
the way to greatness was left clear to a new set of aspirants. Two
statesmen, who subsequently rose to the highest eminence which a British
subject can reach, soon began to attract a large share of the public
attention. These were Lawrence Hyde and Sidney Godolphin.

Lawrence Hyde was the second son of the Chancellor Clarendon, and was
brother of the first Duchess of York. He had excellent parts, which
had been improved by parliamentary and diplomatic experience; but the
infirmities of his temper detracted much from the effective strength of
his abilities. Negotiator and courtier as he was, he never learned the
art of governing or of concealing his emotions. When prosperous, he
was insolent and boastful: when he sustained a check, his undisguised
mortification doubled the triumph of his enemies: very slight
provocations sufficed to kindle his anger; and when he was angry he
said bitter things which he forgot as soon as he was pacified, but which
others remembered many years. His quickness and penetration would have
made him a consummate man of business but for his selfsufficiency and
impatience. His writings proved that he had many of the qualities of an
orator: but his irritability prevented him from doing himself justice
in debate; for nothing was easier than to goad him into a passion; and,
from the moment when he went into a passion, he was at the mercy of
opponents far inferior to him in capacity.

Unlike most of the leading politicians of that generation he was a
consistent, dogged, and rancorous party man, a Cavalier of the old
school, a zealous champion of the Crown and of the Church, and a hater
of Republicans and Nonconformists. He had consequently a great body of
personal adherents. The clergy especially looked on him as their own
man, and extended to his foibles an indulgence of which, to say the
truth, he stood in some need: for he drank deep; and when he was in a
rage,--and he very often was in a rage,--he swore like a porter.

He now succeeded Essex at the treasury. It is to be observed that the
place of First Lord of the Treasury had not then the importance and
dignity which now belong to it. When there was a Lord Treasurer, that
great officer was generally prime minister: but, when the white staff
was in commission, the chief commissioner hardly ranked so high as a
Secretary of State. It was not till the time of Walpole that the First
Lord of the Treasury became, under a humbler name, all that the Lord
High Treasurer had been.

Godolphin had been bred a page at Whitehall, and had early acquired all
the flexibility and the selfpossession of a veteran courtier. He was
laborious, clearheaded, and profoundly versed in the details of finance.
Every government, therefore, found him an useful servant; and there was
nothing in his opinions or in his character which could prevent him from
serving any government. "Sidney Godolphin," said Charles, "is never
in the way, and never out of the way." This pointed remark goes far to
explain Godolphin's extraordinary success in life.

He acted at different times with both the great political parties: but
he never shared in the passions of either. Like most men of cautious
tempers and prosperous fortunes, he had a strong disposition to support
whatever existed. He disliked revolutions; and, for the same reason
for which he disliked revolutions, he disliked counter-revolutions. His
deportment was remarkably grave and reserved: but his personal tastes
were low and frivolous; and most of the time which he could save from
public business was spent in racing, cardplaying, and cockfighting. He
now sate below Rochester at the Board of Treasury, and distinguished
himself there by assiduity and intelligence.

Before the new Parliament was suffered to meet for the despatch of
business a whole year elapsed, an eventful year, which has left
lasting traces in our manners and language. Never before had political
controversy been carried on with so much freedom. Never before had
political clubs existed with so elaborate an organisation or so
formidable an influence. The one question of the Exclusion occupied the
public mind. All the presses and pulpits of the realm took part in
the conflict. On one side it was maintained that the constitution and
religion of the state could never be secure under a Popish King; on the
other, that the right of James to wear the crown in his turn was derived
from God, and could not be annulled, even by the consent of all the
branches of the legislature. Every county, every town, every family,
was in agitation. The civilities and hospitalities of neighbourhood were
interrupted. The dearest ties of friendship and of blood were sundered.
Even schoolboys were divided into angry parties; and the Duke of York
and the Earl of Shaftesbury had zealous adherents on all the forms of
Westminster and Eton. The theatres shook with the roar of the contending
factions. Pope Joan was brought on the stage by the zealous Protestants.
Pensioned poets filled their prologues and epilogues with eulogies
on the King and the Duke. The malecontents besieged the throne with
petitions, demanding that Parliament might be forthwith convened. The
royalists sent up addresses, expressing the utmost abhorrence of all who
presumed to dictate to the sovereign. The citizens of London assembled
by tens of thousands to burn the Pope in effigy. The government posted
cavalry at Temple Bar, and placed ordnance round Whitehall. In that
year our tongue was enriched with two words, Mob and Sham, remarkable
memorials of a season of tumult and imposture. 
%[21]
\footnote{ North's Examen, 231, 574.}
 Opponents of the
court were called Birminghams, Petitioners, and Exclusionists. Those
who took the King's side were Antibirminghams, Abhorrers, and Tantivies.
These appellations soon become obsolete: but at this time were first
heard two nicknames which, though originally given in insult, were soon
assumed with pride, which are still in daily use, which have spread as
widely as the English race, and which will last as long as the English
literature. It is a curious circumstance that one of these nicknames
was of Scotch, and the other of Irish, origin. Both in Scotland and in
Ireland, misgovernment had called into existence bands of desperate men
whose ferocity was heightened by religions enthusiasm. In Scotland some
of the persecuted Covenanters, driven mad by oppression, had lately
murdered the Primate, had taken arms against the government, had
obtained some advantages against the King's forces, and had not been put
down till Monmouth, at the head of some troops from England, had routed
them at Bothwell Bridge. These zealots were most numerous among the
rustics of the western lowlands, who were vulgarly called Whigs. Thus
the appellation of Whig was fastened on the Presbyterian zealots of
Scotland, and was transferred to those English politicians who showed a
disposition to oppose the court, and to treat Protestant Nonconformists
with indulgence. The bogs of Ireland, at the same time, afforded a
refuge to Popish outlaws, much resembling those who were afterwards
known as Whiteboys. These men were then called Tories. The name of Tory
was therefore given to Englishmen who refused to concur in excluding a
Roman Catholic prince from the throne.

The rage of the hostile factions would have been sufficiently violent,
if it had been left to itself. But it was studiously exasperated by the
common enemy of both. Lewis still continued to bribe and flatter
both the court and the opposition. He exhorted Charles to be firm: he
exhorted James to raise a civil war in Scotland: he exhorted the Whigs
not to flinch, and to rely with confidence on the protection of France.

Through all this agitation a discerning eye might have perceived that
the public opinion was gradually changing. The persecution of the Roman
Catholics went on; but convictions were no longer matters of course. A
new brood of false witnesses, among whom a villain named Dangerfield was
the most conspicuous, infested the courts: but the stories of these men,
though better constructed than that of Oates, found less credit. Juries
were no longer so easy of belief as during the panic which had followed
the murder of Godfrey; and Judges, who, while the popular frenzy was at
the height, had been its most obsequious instruments, now ventured to
express some part of what they had from the first thought.

At length, in October 1680, the Parliament met. The Whigs had so great
a majority in the Commons that the Exclusion Bill went through all its
stages there without difficulty. The King scarcely knew on what members
of his own cabinet he could reckon. Hyde had been true to his Tory
opinions, and had steadily supported the cause of hereditary monarchy.
But Godolphin, anxious for quiet, and believing that quiet could be
restored only by concession, wished the bill to pass. Sunderland, ever
false, and ever shortsighted, unable to discern the signs of approaching
reaction, and anxious to conciliate the party which he believed to
be irresistible, determined to vote against the court. The Duchess of
Portsmouth implored her royal lover not to rush headlong to destruction.
If there were any point on which he had a scruple of conscience or of
honour, it was the question of the succession; but during some days
it seemed that he would submit. He wavered, asked what sum the Commons
would give him if he yielded, and suffered a negotiation to be opened
with the leading Whigs. But a deep mutual distrust which had been
many years growing, and which had been carefully nursed by the arts of
France, made a treaty impossible. Neither side would place confidence
in the other. The whole nation now looked with breathless anxiety to the
House of Lords. The assemblage of peers was large. The King himself was
present. The debate was long, earnest, and occasionally furious. Some
hands were laid on the pommels of swords in a manner which revived the
recollection of the stormy Parliaments of Edward the Third and Richard
the Second. Shaftesbury and Essex were joined by the treacherous
Sunderland. But the genius of Halifax bore down all opposition. Deserted
by his most important colleagues, and opposed to a crowd of able
antagonists, he defended the cause of the Duke of York, in a succession
of speeches which, many years later, were remembered as masterpieces of
reasoning, of wit, and of eloquence. It is seldom that oratory changes
votes. Yet the attestation of contemporaries leaves no doubt that,
on this occasion, votes were changed by the oratory of Halifax. The
Bishops, true to their doctrines, supported the principle of hereditary
right, and the bill was rejected by a great majority. 
%[22]
\footnote{ A peer who was present has described the effect of
Halifax's oratory in words which I will quote, because, though they
have been long in print, they are probably known to few even of the
most curious and diligent readers of history. "Of powerful eloquence and
great parts were the Duke's enemies who did assert the Bill; but a noble
Lord appeared against it who, that day, in all the force of speech, in
reason, in arguments of what could concern the public or the private
interests of men, in honour, in conscience, in estate, did outdo himself
and every other man; and in fine his conduct and his parts were
both victorious, and by him all the wit and malice of that party was
overthrown." This passage is taken from a memoir of Henry Earl of
Peterborough, in a volume entitled "Succinct Genealogies, by Robert
Halstead," fol. 1685. The name of Halstead is fictitious. The real
authors were the Earl of Peterborough himself and his chaplain. The book
is extremely rare. Only twenty-four copies were printed, two of which
are now in the British Museum. Of these two one belonged to George the
Fourth, and the other to Mr. Grenville.}


The party which preponderated in the House of Commons, bitterly
mortified by this defeat, found some consolation in shedding the blood
of Roman Catholics. William Howard, Viscount Stafford, one of the
unhappy men who had been accused of a share in the plot, was impeached;
and on the testimony of Oates and of two other false witnesses, Dugdale
and Turberville, was found guilty of high treason, and suffered death.
But the circumstances of his trial and execution ought to have given an
useful warning to the Whig leaders. A large and respectable minority of
the House of Lords pronounced the prisoner not guilty. The multitude,
which a few months before had received the dying declarations of Oates's
victims with mockery and execrations, now loudly expressed a belief that
Stafford was a murdered man. When he with his last breath protested
his innocence, the cry was, "God bless you, my Lord! We believe you, my
Lord." A judicious observer might easily have predicted that the blood
then shed would shortly have blood.

The King determined to try once more the experiment of a dissolution. A
new Parliament was summoned to meet at Oxford, in March, 1681. Since the
days of the Plantagenets the Houses had constantly sat at Westminster,
except when the plague was raging in the capital: but so extraordinary
a conjuncture seemed to require extraordinary precautions. If the
Parliament were held in its usual place of assembling, the House of
Commons might declare itself permanent, and might call for aid on the
magistrates and citizens of London. The trainbands might rise to defend
Shaftesbury as they had risen forty years before to defend Pym and
Hampden. The Guards might be overpowered, the palace forced, the King a
prisoner in the hands of his mutinous subjects. At Oxford there was no
such danger. The University was devoted to the crown; and the gentry of
the neighbourhood were generally Tories. Here, therefore, the opposition
had more reason than the King to apprehend violence.

The elections were sharply contested. The Whigs still composed a
majority of the House of Commons: but it was plain that the Tory
spirit was fast rising throughout the country. It should seem that the
sagacious and versatile Shaftesbury ought to have foreseen the coming
change, and to have consented to the compromise which the court offered:
but he appears to have forgotten his old tactics. Instead of making
dispositions which, in the worst event, would have secured his retreat,
he took up a position in which it was necessary that he should either
conquer or perish. Perhaps his head, strong as it was, had been turned
by popularity, by success, and by the excitement of conflict. Perhaps
he had spurred his party till he could no longer curb it, and was really
hurried on headlong by those whom he seemed to guide.

The eventful day arrived. The meeting at Oxford resembled rather that of
a Polish Diet than that of an English Parliament. The Whig members were
escorted by great numbers of their armed and mounted tenants and
serving men, who exchanged looks of defiance with the royal Guards. The
slightest provocation might, under such circumstances, have produced
a civil war; but neither side dared to strike the first blow. The King
again offered to consent to anything but the Exclusion Bill. The Commons
were determined to accept nothing but the Exclusion Bill. In a few days
the Parliament was again dissolved.

The King had triumphed. The reaction, which had begun some months before
the meeting of the House at Oxford, now went rapidly on. The nation,
indeed, was still hostile to Popery: but, when men reviewed the whole
history of the plot, they felt that their Protestant zeal had hurried
them into folly and crime, and could scarcely believe that they had been
induced by nursery tales to clamour for the blood of fellow subjects
and fellow Christians. The most loyal, indeed, could not deny that the
administration of Charles had often been highly blamable. But men who
had not the full information which we possess touching his dealings with
France, and who were disgusted by the violence of the Whigs, enumerated
the large concessions which, during the last few years he had made to
his Parliaments, and the still larger concessions which he had declared
himself willing to make. He had consented to the laws which excluded
Roman Catholics from the House of Lords, from the Privy Council, and
from all civil and military offices. He had passed the Habeas Corpus
Act. If securities yet stronger had not been provided against the
dangers to which the constitution and the Church might be exposed under
a Roman Catholic sovereign, the fault lay, not with Charles who had
invited the Parliament to propose such securities, but with those Whigs
who had refused to hear of any substitute for the Exclusion Bill. One
thing only had the King denied to his people. He had refused to take
away his brother's birthright. And was there not good reason to believe
that this refusal was prompted by laudable feelings? What selfish motive
could faction itself impute to the royal mind? The Exclusion Bill did
not curtail the reigning King's prerogatives, or diminish his income.
Indeed, by passing it, he might easily have obtained an ample addition
to his own revenue. And what was it to him who ruled after him? Nay, if
he had personal predilections, they were known to be rather in favour
of the Duke of Monmouth than of the Duke of York. The most natural
explanation of the King's conduct seemed to be that, careless as was
his temper and loose as were his morals, he had, on this occasion, acted
from a sense of duty and honour. And, if so, would the nation compel
him to do what he thought criminal and disgraceful? To apply, even by
strictly constitutional means, a violent pressure to his conscience,
seemed to zealous royalists ungenerous and undutiful. But strictly
constitutional means were not the only means which the Whigs were
disposed to employ. Signs were already discernible which portended the
approach of great troubles. Men, who, in the time of the civil war and
of the Commonwealth, had acquired an odious notoriety, had emerged
from the obscurity in which, after the Restoration, they had hidden
themselves from the general hatred, showed their confident and busy
faces everywhere, and appeared to anticipate a second reign of the
Saints. Another Naseby, another High Court of Justice, another usurper
on the throne, the Lords again ejected from their hall by violence, the
Universities again purged, the Church again robbed and persecuted, the
Puritans again dominant, to such results did the desperate policy of the
opposition seem to tend.

Strongly moved by these apprehensions, the majority of the upper and
middle classes hastened to rally round the throne. The situation of the
King bore, at this time, a great resemblance to that in which his father
stood just after the Remonstrance had been voted. But the reaction of
1641 had not been suffered to run its course. Charles the First, at the
very moment when his people, long estranged, were returning to him with
hearts disposed to reconciliation, had, by a perfidious violation of the
fundamental laws of the realm, forfeited their confidence for ever.
Had Charles the Second taken a similar course, had he arrested the Whig
leaders in an irregular manner, had he impeached them of high treason
before a tribunal which had no legal jurisdiction over them, it is
highly probable that they would speedily have regained the ascendancy
which they had lost. Fortunately for himself, he was induced, at this
crisis, to adopt a policy singularly judicious. He determined to conform
to the law, but at the same time to make vigorous and unsparing use
of the law against his adversaries. He was not bound to convoke a
Parliament till three years should have elapsed. He was not much
distressed for money. The produce of the taxes which had been settled on
him for life exceeded the estimate. He was at peace with all the world.
He could retrench his expenses by giving up the costly and useless
settlement of Tangier; and he might hope for pecuniary aid from France.
He had, therefore, ample time and means for a systematic attack on
the opposition under the forms of the constitution. The Judges were
removable at his pleasure: the juries were nominated by the Sheriffs;
and, in almost all the counties of England, the Sheriffs were nominated
by himself. Witnesses, of the same class with those who had recently
sworn away the lives of Papists, were ready to swear away the lives of
Whigs.

The first victim was College, a noisy and violent demagogue of mean
birth and education. He was by trade a joiner, and was celebrated as the
inventor of the Protestant flail. 
%[23]
\footnote{ This is mentioned in the curious work entitled "Ragguaglio
della solenne Comparsa fatta in Roma gli otto di Gennaio, 1687, dall'
illustrissimo et eccellentissimo signor Conte di Castlemaine."}
 He had been at Oxford when the
Parliament sate there, and was accused of having planned a rising and an
attack on the King's guards. Evidence was given against him by Dugdale
and Turberville, the same infamous men who had, a few months earlier,
borne false witness against Stafford. In the sight of a jury of
country squires no Exclusionist was likely to find favour. College was
convicted. The crowd which filled the court house of Oxford received the
verdict with a roar of exultation, as barbarous as that which he and
his friends had been in the habit of raising when innocent Papists were
doomed to the gallows. His execution was the beginning of a new judicial
massacre not less atrocious than that in which he had himself borne a
share.

The government, emboldened by this first victory, now aimed a blow at an
enemy of a very different class. It was resolved that Shaftesbury should
be brought to trial for his life. Evidence was collected which, it was
thought, would support a charge of treason. But the facts which it was
necessary to prove were alleged to have been committed in London. The
Sheriffs of London, chosen by the citizens, were zealous Whigs. They
named a Whig grand jury, which threw out the bill. This defeat, far from
discouraging those who advised the King, suggested to them a new and
daring scheme. Since the charter of the capital was in their way, that
charter must be annulled. It was pretended, therefore, that the City
had by some irregularities forfeited its municipal privileges; and
proceedings were instituted against the corporation in the Court of
King's Bench. At the same time those laws which had, soon after the
Restoration, been enacted against Nonconformists, and which had remained
dormant during the ascendency of the Whigs, were enforced all over the
kingdom with extreme rigour.

Yet the spirit of the Whigs was not subdued. Though in evil plight, they
were still a numerous and powerful party; and as they mustered strong in
the large towns, and especially in the capital, they made a noise and
a show more than proportioned to their real force. Animated by the
recollection of past triumphs, and by the sense of present oppression,
they overrated both their strength and their wrongs. It was not in
their power to make out that clear and overwhelming case which can alone
justify so violent a remedy as resistance to an established government.
Whatever they might suspect, they could not prove that their sovereign
had entered into a treaty with France against the religion and liberties
of England. What was apparent was not sufficient to warrant an appeal
to the sword. If the Lords had thrown out the Exclusion Bill, they had
thrown it out in the exercise of a right coeval with the constitution.
If the King had dissolved the Oxford Parliament, he had done so by
virtue of a prerogative which had never been questioned. If he had,
since the dissolution, done some harsh things, still those things were
in strict conformity with the letter of the law, and with the recent
practice of the malecontents themselves. If he had prosecuted his
opponents, he had prosecuted them according to the proper forms, and
before the proper tribunals. The evidence now produced for the crown was
at least as worthy of credit as the evidence on which the noblest blood
of England had lately been shed by the opposition. The treatment which
an accused Whig had now to expect from judges, advocates, sheriffs,
juries and spectators, was no worse than the treatment which had lately
been thought by the Whigs good enough for an accused Papist. If the
privileges of the City of London were attacked, they were attacked, not
by military violence or by any disputable exercise of prerogative,
but according to the regular practice of Westminster Hall. No tax was
imposed by royal authority. No law was suspended. The Habeas Corpus
Act was respected. Even the Test Act was enforced. The opposition,
therefore, could not bring home to the King that species of
misgovernment which alone could justify insurrection. And, even had his
misgovernment been more flagrant than it was, insurrection would still
have been criminal, because it was almost certain to be unsuccessful.
The situation of the Whigs in 1682 differed widely from that of the
Roundheads forty years before. Those who took up arms against Charles
the First acted under the authority of a Parliament which had been
legally assembled, and which could not, without its own consent, be
legally dissolved. The opponents of Charles the Second were private men.
Almost all the military and naval resources of the kingdom had been at
the disposal of those who resisted Charles the First. All the military
and naval resources of the kingdom were at the disposal of Charles the
Second. The House of Commons had been supported by at least half the
nation against Charles the First. But those who were disposed to levy
war against Charles the Second were certainly a minority. It could
hardly be doubted, therefore, that, if they attempted a rising, they
would fail. Still less could it be doubted that their failure would
aggravate every evil of which they complained. The true policy of the
Whigs was to submit with patience to adversity which was the natural
consequence and the just punishment of their errors, to wait patiently
for that turn of public feeling which must inevitably come, to observe
the law, and to avail themselves of the protection, imperfect indeed,
but by no means nugatory, which the law afforded to innocence. Unhappily
they took a very different course. Unscrupulous and hot-headed chiefs of
the party formed and discussed schemes of resistance, and were heard, if
not with approbation, yet with the show of acquiescence, by much better
men than themselves. It was proposed that there should be simultaneous
insurrections in London, in Cheshire, at Bristol, and at Newcastle.
Communications were opened with the discontented Presbyterians of
Scotland, who were suffering under a tyranny such as England, in the
worst times, had never known. While the leaders of the opposition thus
revolved plans of open rebellion, but were still restrained by fears
or scruples from taking any decisive step, a design of a very different
kind was meditated by some of their accomplices. To fierce spirits,
unrestrained by principle, or maddened by fanaticism, it seemed that to
waylay and murder the King and his brother was the shortest and surest
way of vindicating the Protestant religion and the liberties of England.
A place and a time were named; and the details of the butchery were
frequently discussed, if not definitely arranged. This scheme was known
but to few, and was concealed with especial care from the upright and
humane Russell, and from Monmouth, who, though not a man of delicate
conscience, would have recoiled with horror from the guilt of parricide.
Thus there were two plots, one within the other. The object of the great
Whig plot was to raise the nation in arms against the government. The
lesser plot, commonly called the Rye House Plot, in which only a few
desperate men were concerned, had for its object the assassination of
the King and of the heir presumptive.

Both plots were soon discovered. Cowardly traitors hastened to save
themselves, by divulging all, and more than all, that had passed in
the deliberations of the party. That only a small minority of those
who meditated resistance had admitted into their minds the thought of
assassination is fully established: but, as the two conspiracies ran
into each other, it was not difficult for the government to confound
them together. The just indignation excited by the Rye House Plot was
extended for a time to the whole Whig body. The King was now at
liberty to exact full vengeance for years of restraint and humiliation.
Shaftesbury, indeed, had escaped the fate which his manifold perfidy had
well deserved. He had seen that the ruin of his party was at hand, had
in vain endeavoured to make his peace with the royal brothers, had
fled to Holland, and had died there, under the generous protection of a
government which he had cruelly wronged. Monmouth threw himself at his
father's feet and found mercy, but soon gave new offence, and thought
it prudent to go into voluntary exile. Essex perished by his own hand
in the Tower. Russell, who appears to have been guilty of no offence
falling within the definition of high treason, and Sidney, of whose
guilt no legal evidence could be produced, were beheaded in defiance of
law and justice. Russell died with the fortitude of a Christian, Sidney
with the fortitude of a Stoic. Some active politicians of meaner
rank were sent to the gallows. Many quitted the country. Numerous
prosecutions for misprision of treason, for libel, and for conspiracy
were instituted. Convictions were obtained without difficulty from Tory
juries, and rigorous punishments were inflicted by courtly judges. With
these criminal proceedings were joined civil proceedings scarcely less
formidable. Actions were brought against persons who had defamed
the Duke of York and damages tantamount to a sentence of perpetual
imprisonment were demanded by the plaintiff, and without difficulty
obtained. The Court of King's Bench pronounced that the franchises of
the City of London were forfeited to the Crown. Flushed with this great
victory, the government proceeded to attack the constitutions of other
corporations which were governed by Whig officers, and which had been in
the habit of returning Whig members to Parliament. Borough after borough
was compelled to surrender its privileges; and new charters were granted
which gave the ascendency everywhere to the Tories.

These proceedings, however reprehensible, had yet the semblance of
legality. They were also accompanied by an act intended to quiet the
uneasiness with which many loyal men looked forward to the accession of
a Popish sovereign. The Lady Anne, younger daughter of the Duke of York
by his first wife, was married to George, a prince of the orthodox House
of Denmark. The Tory gentry and clergy might now flatter themselves that
the Church of England had been effectually secured without any violation
of the order of succession. The King and the heir presumptive were
nearly of the same age. Both were approaching the decline of life. The
King's health was good. It was therefore probable that James, if he came
to the throne, would have but a short reign. Beyond his reign there was
the gratifying prospect of a long series of Protestant sovereigns.

The liberty of unlicensed printing was of little or no use to the
vanquished party; for the temper of judges and juries was such that
no writer whom the government prosecuted for a libel had any chance of
escaping. The dread of punishment therefore did all that a censorship
could have done. Meanwhile, the pulpits resounded with harangues against
the sin of rebellion. The treatises in which Filmer maintained that
hereditary despotism was the form of government ordained by God, and
that limited monarchy was a pernicious absurdity, had recently appeared,
and had been favourably received by a large section of the Tory party.
The university of Oxford, on the very day on which Russell was put
to death, adopted by a solemn public act these strange doctrines,
and ordered the political works of Buchanan, Milton, and Baxter to be
publicly burned in the court of the Schools.

Thus emboldened, the King at length ventured to overstep the bounds
which he had during some years observed, and to violate the plain letter
of the law. The law was that not more than three years should pass
between the dissolving of one Parliament and the convoking of another.
But, when three years had elapsed after the dissolution of the
Parliament which sate at Oxford, no writs were issued for an election.
This infraction of the constitution was the more reprehensible, because
the King had little reason to fear a meeting with a new House of
Commons. The counties were generally on his side; and many boroughs in
which the Whigs had lately held sway had been so remodelled that they
were certain to return none but courtiers.

In a short time the law was again violated in order to gratify the Duke
of York. That prince was, partly on account of his religion, and partly
on account of the sternness and harshness of his nature, so unpopular
that it had been thought necessary to keep him out of sight while the
Exclusion Bill was before Parliament, lest his appearance should give
an advantage to the party which was struggling to deprive him of his
birthright. He had therefore been sent to govern Scotland, where the
savage old tyrant Lauderdale was sinking into the grave. Even Lauderdale
was now outdone. The administration of James was marked by odious laws,
by barbarous punishments, and by judgments to the iniquity of which even
that age furnished no parallel. The Scottish Privy Council had power to
put state prisoners to the question. But the sight was so dreadful that,
as soon as the boots appeared, even the most servile and hardhearted
courtiers hastened out of the chamber. The board was sometimes quite
deserted: and it was at length found necessary to make an order that the
members should keep their seats on such occasions. The Duke of York, it
was remarked, seemed to take pleasure in the spectacle which some of
the worst men then living were unable to contemplate without pity
and horror. He not only came to Council when the torture was to be
inflicted, but watched the agonies of the sufferers with that sort of
interest and complacency with which men observe a curious experiment in
science. Thus he employed himself at Edinburgh, till the event of the
conflict between the court and the Whigs was no longer doubtful. He then
returned to England: but he was still excluded by the Test Act from all
public employment; nor did the King at first think it safe to violate a
statute which the great majority of his most loyal subjects regarded as
one of the chief securities of their religion and of their civil rights.
When, however, it appeared, from a succession of trials, that the nation
had patience to endure almost anything that the government had courage
to do, Charles ventured to dispense with the law in his brother's
favour. The Duke again took his seat in the Council, and resumed the
direction of naval affairs.

These breaches of the constitution excited, it is true, some murmurs
among the moderate Tories, and were not unanimously approved even by the
King's ministers. Halifax in particular, now a Marquess and Lord Privy
Seal, had, from the very day on which the Tories had by his help gained
the ascendant, begun to turn Whig. As soon as the Exclusion Bill had
been thrown out, he had pressed the House of Lords to make provision
against the danger to which, in the next reign, the liberties and
religion of the nation might be exposed. He now saw with alarm the
violence of that reaction which was, in no small measure, his own
work. He did not try to conceal the scorn which he felt for the servile
doctrines of the University of Oxford. He detested the French alliance.
He disapproved of the long intermission of Parliaments. He regretted the
severity with which the vanquished party was treated. He who, when the
Whigs were predominant, had ventured to pronounce Stafford not guilty,
ventured, when they were vanquished and helpless, to intercede for
Russell. At one of the last Councils which Charles held a remarkable
scene took place. The charter of Massachusetts had been forfeited. A
question arose how, for the future, the colony should be governed. The
general opinion of the board was that the whole power, legislative as
well as executive, should abide in the crown. Halifax took the opposite
side, and argued with great energy against absolute monarchy, and in
favour of representative government. It was vain, he said, to think that
a population, sprung from the English stock, and animated by English
feelings, would long bear to be deprived of English institutions. Life,
he exclaimed, would not be worth having in a country where liberty and
property were at the mercy of one despotic master. The Duke of York was
greatly incensed by this language, and represented to his brother the
danger of retaining in office a man who appeared to be infected with all
the worst notions of Marvell and Sidney.

Some modern writers have blamed Halifax for continuing in the ministry
while he disapproved of the manner in which both domestic and foreign
affairs were conducted. But this censure is unjust. Indeed it is to be
remarked that the word ministry, in the sense in which we use it, was
then unknown. 
%[24]
\footnote{ North's Examen, 69.}
 The thing itself did not exist; for it belongs to an
age in which parliamentary government is fully established. At present
the chief servants of the crown form one body. They are understood to be
on terms of friendly confidence with each other, and to agree as to
the main principles on which the executive administration ought to be
conducted. If a slight difference of opinion arises among them, it is
easily compromised: but, if one of them differs from the rest on a vital
point, it is his duty to resign. While he retains his office, he is held
responsible even for steps which he has tried to dissuade his colleagues
from taking. In the seventeenth century, the heads of the various
branches of the administration were bound together in no such
partnership. Each of them was accountable for his own acts, for the
use which he made of his own official seal, for the documents which he
signed, for the counsel which he gave to the King. No statesman was held
answerable for what he had not himself done, or induced others to do.
If he took care not to be the agent in what was wrong, and if, when
consulted, he recommended what was right, he was blameless. It would
have been thought strange scrupulosity in him to quit his post, because
his advice as to matters not strictly within his own department was
not taken by his master; to leave the Board of Admiralty, for example,
because the finances were in disorder, or the Board of Treasury because
the foreign relations of the kingdom were in an unsatisfactory state. It
was, therefore, by no means unusual to see in high office, at the same
time, men who avowedly differed from one another as widely as ever
Pulteney differed from Walpole, or Fox from Pitt.

The moderate and constitutional counsels of Halifax were timidly and
feebly seconded by Francis North, Lord Guildford who had lately been
made Keeper of the Great Seal. The character of Guildford has been drawn
at full length by his brother Roger North, a most intolerant Tory, a
most affected and pedantic writer, but a vigilant observer of all those
minute circumstances which throw light on the dispositions of men. It is
remarkable that the biographer, though he was under the influence of the
strongest fraternal partiality, and though he was evidently anxious to
produce a flattering likeness, was unable to portray the Lord Keeper
otherwise than as the most ignoble of mankind. Yet the intellect of
Guildford was clear, his industry great, his proficiency in letters and
science respectable, and his legal learning more than respectable. His
faults were selfishness, cowardice, and meanness. He was not insensible
to the power of female beauty, nor averse from excess in wine. Yet
neither wine nor beauty could ever seduce the cautious and frugal
libertine, even in his earliest youth, into one fit of indiscreet
generosity. Though of noble descent, he rose in his profession by paying
ignominious homage to all who possessed influence in the courts. He
became Chief Justice of the Common Pleas, and as such was party to some
of the foulest judicial murders recorded in our history. He had sense
enough to perceive from the first that Oates and Bedloe were impostors:
but the Parliament and the country were greatly excited: the government
had yielded to the pressure; and North was not a man to risk a good
place for the sake of justice and humanity. Accordingly, while he was in
secret drawing up a refutation of the whole romance of the Popish plot,
he declared in public that the truth of the story was as plain as
the sun in heaven, and was not ashamed to browbeat, from the seat of
judgment, the unfortunate Roman Catholics who were arraigned before him
for their lives. He had at length reached the highest post in the law.
But a lawyer, who, after many years devoted to professional labour,
engages in politics for the first time at an advanced period of life,
seldom distinguishes himself as a statesman; and Guildford was no
exception to the general rule. He was indeed so sensible of his
deficiencies that he never attended the meetings of his colleagues on
foreign affairs. Even on questions relating to his own profession his
opinion had less weight at the Council board than that of any man who
has ever held the Great Seal. Such as his influence was, however, he
used it, as far as he dared, on the side of the laws.

The chief opponent of Halifax was Lawrence Hyde, who had recently
been created Earl of Rochester. Of all Tories, Rochester was the
most intolerant and uncompromising. The moderate members of his party
complained that the whole patronage of the Treasury, while he was First
Commissioner there, went to noisy zealots, whose only claim to promotion
was that they were always drinking confusion to Whiggery, and lighting
bonfires to burn the Exclusion Bill. The Duke of York, pleased with
a spirit which so much resembled his own supported his brother in law
passionately and obstinately.

The attempts of the rival ministers to surmount and supplant each other
kept the court in incessant agitation. Halifax pressed the King to
summon a Parliament, to grant a general amnesty, to deprive the Duke of
York of all share in the government, to recall Monmouth from banishment,
to break with Lewis, and to form a close union with Holland on the
principles of the Triple Alliance. The Duke of York, on the other hand,
dreaded the meeting of a Parliament, regarded the vanquished Whigs with
undiminished hatred, still flattered himself that the design formed
fourteen years before at Dover might be accomplished, daily represented
to his brother the impropriety of suffering one who was at heart a
Republican to hold the Privy Seal, and strongly recommended Rochester
for the great place of Lord Treasurer.

While the two factions were struggling, Godolphin, cautious, silent,
and laborious, observed a neutrality between them. Sunderland, with his
usual restless perfidy, intrigued against them both. He had been turned
out of office in disgrace for having voted in favour of the Exclusion
Bill, but had made his peace by employing the good offices of the
Duchess of Portsmouth and by cringing to the Duke of York, and was once
more Secretary of State.

Nor was Lewis negligent or inactive. Everything at that moment favoured
his designs. He had nothing to apprehend from the German empire, which
was then contending against the Turks on the Danube. Holland could
not, unsupported venture to oppose him. He was therefore at liberty
to indulge his ambition and insolence without restraint. He seized
Strasburg, Courtray, Luxemburg. He exacted from the republic of Genoa
the most humiliating submissions. The power of France at that time
reached a higher point than it ever before or ever after attained,
during the ten centuries which separated the reign of Charlemagne from
the reign of Napoleon. It was not easy to say where her acquisitions
would stop, if only England could be kept in a state of vassalage. The
first object of the court of Versailles was therefore to prevent the
calling of a Parliament and the reconciliation of English parties.
For this end bribes, promises, and menaces were unsparingly employed.
Charles was sometimes allured by the hope of a subsidy, and sometimes
frightened by being told that, if he convoked the Houses, the secret
articles of the treaty of Dover should be published. Several Privy
Councillors were bought; and attempts were made to buy Halifax, but in
vain. When he had been found incorruptible, all the art and influence
of the French embassy were employed to drive him from office: but his
polished wit and his various accomplishments had made him so agreeable
to his master, that the design failed. 
%[25]
\footnote{ Lord Preston, who was envoy at Paris, wrote thence to
Halifax as follows: "I find that your Lordship lies still under the same
misfortune of being no favourite to this court; and Monsieur Barillon
dare not do you the honor to shine upon you, since his master frowneth.
They know very well your lordship's qualifications which make them
fear and consequently hate you; and be assured, my lord, if all their
strength can send you to Rufford, it shall be employed for that end. Two
things, I hear, they particularly object against you, your secrecy, and
your being incapable of being corrupted. Against these two things I know
they have declared." The date of the letter is October 5, N. S. 1683}


Halifax was not content with standing on the defensive. He openly
accused Rochester of malversation. An inquiry took place. It appeared
that forty thousand pounds had been lost to the public by the
mismanagement of the First Lord of the Treasury. In consequence of this
discovery he was not only forced to relinquish his hopes of the white
staff, but was removed from the direction of the finances to the more
dignified but less lucrative and important post of Lord President.
"I have seen people kicked down stairs," said Halifax; "but my Lord
Rochester is the first person that I ever saw kicked up stairs."
Godolphin, now a peer, became First Commissioner of the Treasury.

Still, however, the contest continued. The event depended wholly on
the will of Charles; and Charles could not come to a decision. In
his perplexity he promised everything to everybody. He would stand
by France: he would break with France: he would never meet another
Parliament: he would order writs for a Parliament to be issued without
delay. He assured the Duke of York that Halifax should be dismissed from
office, and Halifax that the Duke should be sent to Scotland. In public
he affected implacable resentment against Monmouth, and in private
conveyed to Monmouth assurances of unalterable affection. How long, if
the King's life had been protracted, his hesitation would have lasted,
and what would have been his resolve, can only be conjectured. Early
in the year 1685, while hostile parties were anxiously awaiting his
determination, he died, and a new scene opened. In a few mouths the
excesses of the government obliterated the impression which had been
made on the public mind by the excesses of the opposition. The violent
reaction which had laid the Whig party prostrate was followed by a still
more violent reaction in the opposite direction; and signs not to be
mistaken indicated that the great conflict between the prerogatives of
the Crown and the privileges of the Parliament, was about to be brought
to a final issue.




\chapter{CHAPTER III.}

I INTEND, in this chapter, to give a description of the state in which
England was at the time when the crown passed from Charles the Second
to his brother. Such a description, composed from scanty and dispersed
materials, must necessarily be very imperfect. Yet it may perhaps
correct some false notions which would make the subsequent narrative
unintelligible or uninstructive.

If we would study with profit the history of our ancestors, we must be
constantly on our guard against that delusion which the well known
names of families, places, and offices naturally produce, and must never
forget that the country of which we read was a very different country
from that in which we live. In every experimental science there is a
tendency towards perfection. In every human being there is a wish to
ameliorate his own condition. These two principles have often
sufficed, even when counteracted by great public calamities and by
bad institutions, to carry civilisation rapidly forward. No ordinary
misfortune, no ordinary misgovernment, will do so much to make a
nation wretched, as the constant progress of physical knowledge and the
constant effort of every man to better himself will do to make a nation
prosperous. It has often been found that profuse expenditure, heavy
taxation, absurd commercial restrictions, corrupt tribunals, disastrous
wars, seditions, persecutions, conflagrations, inundations, have
not been able to destroy capital so fast as the exertions of private
citizens have been able to create it. It can easily be proved that, in
our own land, the national wealth has, during at least six centuries,
been almost uninterruptedly increasing; that it was greater under
the Tudors than under the Plantagenets; that it was greater under the
Stuarts than under the Tudors; that, in spite of battles, sieges, and
confiscations, it was greater on the day of the Restoration than on the
day when the Long Parliament met; that, in spite of maladministration,
of extravagance, of public bankruptcy, of two costly and unsuccessful
wars, of the pestilence and of the fire, it was greater on the day of
the death of Charles the Second than on the day of his Restoration. This
progress, having continued during many ages, became at length, about the
middle of the eighteenth century, portentously rapid, and has proceeded,
during the nineteenth, with accelerated velocity. In consequence partly
of our geographical and partly of our moral position, we have, during
several generations, been exempt from evils which have elsewhere impeded
the efforts and destroyed the fruits of industry. While every part of
the Continent, from Moscow to Lisbon, has been the theatre of bloody
and devastating wars, no hostile standard has been seen here but as a
trophy. While revolutions have taken place all around us, our government
has never once been subverted by violence. During more than a hundred
years there has been in our island no tumult of sufficient importance to
be called an insurrection; nor has the law been once borne down either
by popular fury or by regal tyranny: public credit has been held sacred:
the administration of justice has been pure: even in times which might
by Englishmen be justly called evil times, we have enjoyed what almost
every other nation in the world would have considered as an ample
measure of civil and religious freedom. Every man has felt entire
confidence that the state would protect him in the possession of what
had been earned by his diligence and hoarded by his selfdenial. Under
the benignant influence of peace and liberty, science has flourished,
and has been applied to practical purposes on a scale never before
known. The consequence is that a change to which the history of the old
world furnishes no parallel has taken place in our country. Could the
England of 1685 be, by some magical process, set before our eyes,
we should not know one landscape in a hundred or one building in ten
thousand. The country gentleman would not recognise his own fields. The
inhabitant of the town would not recognise his own street. Everything
has been changed, but the great features of nature, and a few
massive and durable works of human art. We might find out Snowdon and
Windermere, the Cheddar Cliffs and Beachy Head. We might find out here
and there a Norman minster, or a castle which witnessed the wars of the
Roses. But, with such rare exceptions, everything would be strange to
us. Many thousands of square miles which are now rich corn land and
meadow, intersected by green hedgerows and dotted with villages and
pleasant country seats, would appear as moors overgrown with furze, or
fens abandoned to wild ducks. We should see straggling huts built of
wood and covered with thatch, where we now see manufacturing towns and
seaports renowned to the farthest ends of the world. The capital itself
would shrink to dimensions not much exceeding those of its present
suburb on the south of the Thames. Not less strange to us would be the
garb and manners of the people, the furniture and the equipages, the
interior of the shops and dwellings. Such a change in the state of
a nation seems to be at least as well entitled to the notice of a
historian as any change of the dynasty or of the ministry. 
%[26]
\footnote{ During the interval which has elapsed since this chapter
was written, England has continued to advance rapidly in material
prosperity, I have left my text nearly as it originally stood; but I
have added a few notes which may enable the reader to form some notion
of the progress which has been made during the last nine years; and,
in general, I would desire him to remember that there is scarcely a
district which is not more populous, or a source of wealth which is not
more productive, at present than in 1848. (1857.)}


One of the first objects of an inquirer, who wishes to form a correct
notion of the state of a community at a given time, must be to ascertain
of how many persons that community then consisted. Unfortunately the
population of England in 1685, cannot be ascertained with perfect
accuracy. For no great state had then adopted the wise course of
periodically numbering the people. All men were left to conjecture for
themselves; and, as they generally conjectured without examining facts,
and under the influence of strong passions and prejudices, their guesses
were often ludicrously absurd. Even intelligent Londoners ordinarily
talked of London as containing several millions of souls. It was
confidently asserted by many that, during the thirty-five years
which had elapsed between the accession of Charles the First and the
Restoration the population of the City had increased by two millions.

%[27]
\footnote{ Observations on the Bills of Mortality, by Captain John
Graunt (Sir William Petty), chap. xi.}
 Even while the ravages of the plague and fire were recent, it was
the fashion to say that the capital still had a million and a half of
inhabitants. 
%[28]
\footnote{

    "She doth comprehend
     Full fifteen hundred thousand which do spend
     Their days within."

     --Great Britain's Beauty, 1671.}
 Some persons, disgusted by these exaggerations,
ran violently into the opposite extreme. Thus Isaac Vossius, a man of
undoubted parts and learning, strenuously maintained that there were
only two millions of human beings in England, Scotland, and Ireland
taken together. 
%[29]
\footnote{ Isaac Vossius, De Magnitudine Urbium Sinarum, 1685.
Vossius, as we learn from Saint Evremond, talked on this subject oftener
and longer than fashionable circles cared to listen.}


We are not, however, left without the means of correcting the wild
blunders into which some minds were hurried by national vanity and
others by a morbid love of paradox. There are extant three computations
which seem to be entitled to peculiar attention. They are entirely
independent of each other: they proceed on different principles; and yet
there is little difference in the results.

One of these computations was made in the year 1696 by Gregory King,
Lancaster herald, a political arithmetician of great acuteness and
judgment. The basis of his calculations was the number of houses
returned in 1690 by the officers who made the last collection of the
hearth money. The conclusion at which he arrived was that the population
of England was nearly five millions and a half. 
%[30]
\footnote{ King's Natural and Political Observations, 1696 This
valuable treatise, which ought to be read as the author wrote it, and
not as garbled by Davenant, will be found in some editions of Chalmers's
Estimate.}


About the same time King William the Third was desirous to ascertain the
comparative strength of the religious sects into which the community
was divided. An inquiry was instituted; and reports were laid before
him from all the dioceses of the realm. According to these reports the
number of his English subjects must have been about five million two
hundred thousand. 
%[31]
\footnote{ Dalrymple's Appendix to Part II. Book I, The practice of
reckoning the population by sects was long fashionable. Gulliver says
of the King of Brobdignag; "He laughed at my odd arithmetic, as he
was pleased to call it, in reckoning the numbers of our people by
a computation drawn from the several sects among us in religion and
politics."}


Lastly, in our own days, Mr. Finlaison, an actuary of eminent skill,
subjected the ancient parochial registers of baptisms, marriages, and
burials, to all the tests which the modern improvements in statistical
science enabled him to apply. His opinion was, that, at the close of the
seventeenth century, the population of England was a little under five
million two hundred thousand souls. 
%[32]
\footnote{ Preface to the Population Returns of 1831.}


Of these three estimates, framed without concert by different persons
from different sets of materials, the highest, which is that of King,
does not exceed the lowest, which is that of Finlaison, by one twelfth.
We may, therefore, with confidence pronounce that, when James the Second
reigned, England contained between five million and five million five
hundred thousand inhabitants. On the very highest supposition she then
had less than one third of her present population, and less than three
times the population which is now collected in her gigantic capital.

The increase of the people has been great in every part of the kingdom,
but generally much greater in the northern than in the southern shires.
In truth a large part of the country beyond Trent was, down to the
eighteenth century, in a state of barbarism. Physical and moral causes
had concurred to prevent civilisation from spreading to that region. The
air was inclement; the soil was generally such as required skilful and
industrious cultivation; and there could be little skill or industry in
a tract which was often the theatre of war, and which, even when
there was nominal peace, was constantly desolated by bands of Scottish
marauders. Before the union of the two British crowns, and long after
that union, there was as great a difference between Middlesex and
Northumberland as there now is between Massachusetts and the settlements
of those squatters who, far to the west of the Mississippi, administer a
rude justice with the rifle and the dagger. In the reign of Charles the
Second, the traces left by ages of slaughter and pillage were distinctly
perceptible, many miles south of the Tweed, in the face of the country
and in the lawless manners of the people. There was still a large class
of mosstroopers, whose calling was to plunder dwellings and to drive
away whole herds of cattle. It was found necessary, soon after the
Restoration, to enact laws of great severity for the prevention of
these outrages. The magistrates of Northumberland and Cumberland were
authorised to raise bands of armed men for the defence of property and
order; and provision was made for meeting the expense of these levies by
local taxation. 
%[33]
\footnote{ Statutes 14 Car. II. c. 22.; 18 \& 19 Car. II. c. 3., 29 \&
30 Car. II. c. 2.}
 The parishes were required to keep bloodhounds for
the purpose of hunting the freebooters. Many old men who were living in
the middle of the eighteenth century could well remember the time when
those ferocious dogs were common. 
%[34]
\footnote{ Nicholson and Bourne, Discourse on the Ancient State of
the Border, 1777.}
 Yet, even with such auxiliaries,
it was often found impossible to track the robbers to their retreats
among the hills and morasses. For the geography of that wild country was
very imperfectly known. Even after the accession of George the Third,
the path over the fells from Borrowdale to Ravenglas was still a secret
carefully kept by the dalesmen, some of whom had probably in their youth
escaped from the pursuit of justice by that road. 
%[35]
\footnote{ Gray's Journal of a Tour in the Lakes, Oct. 3, 1769.}
 The seats of the
gentry and the larger farmhouses were fortified. Oxen were penned at
night beneath the overhanging battlements of the residence, which was
known by the name of the Peel. The inmates slept with arms at their
sides. Huge stones and boiling water were in readiness to crush and
scald the plunderer who might venture to assail the little garrison. No
traveller ventured into that country without making his will. The Judges
on circuit, with the whole body of barristers, attorneys, clerks, and
serving men, rode on horseback from Newcastle to Carlisle, armed and
escorted by a strong guard under the command of the Sheriffs. It was
necessary to carry provisions; for the country was a wilderness which
afforded no supplies. The spot where the cavalcade halted to dine, under
an immense oak, is not yet forgotten. The irregular vigour with which
criminal justice was administered shocked observers whose lives had been
passed in more tranquil districts. Juries, animated by hatred and by a
sense of common danger, convicted housebreakers and cattle stealers with
the promptitude of a court martial in a mutiny; and the convicts were
hurried by scores to the gallows. 
%[36]
\footnote{ North's Life of Guildford; Hutchinson's History of
Cumberland, Parish of Brampton.}
 Within the memory of some whom
this generation has seen, the sportsman who wandered in pursuit of game
to the sources of the Tyne found the heaths round Keeldar Castle peopled
by a race scarcely less savage than the Indians of California, and heard
with surprise the half naked women chaunting a wild measure, while the
men with brandished dirks danced a war dance. 
%[37]
\footnote{ See Sir Walter Scott's Journal, Oct. 7, 1827, in his Life
by Mr. Lockhart.}


Slowly and with difficulty peace was established on the border. In the
train of peace came industry and all the arts of life. Meanwhile it was
discovered that the regions north of the Trent possessed in their coal
beds a source of wealth far more precious than the gold mines of Peru.
It was found that, in the neighbourhood of these beds, almost every
manufacture might be most profitably carried on. A constant stream of
emigrants began to roll northward. It appeared by the returns of 1841
that the ancient archiepiscopal province of York contained two-sevenths
of the population of England. At the time of the Revolution that
province was believed to contain only one seventh of the population.

%[38]
\footnote{ Dalrymple, Appendix to Part II. Book I. The returns of
the hearth money lead to nearly the same conclusion. The hearths in the
province of York were not a sixth of the hearths of England.}
 In Lancashire the number of inhabitants appear to have increased
ninefold, while in Norfolk, Suffolk, and Northamptonshire it has hardly
doubled. 
%[39]
\footnote{ I do not, of course, pretend to strict accuracy here; but
I believe that whoever will take the trouble to compare the last returns
of hearth money in the reign of William the Third with the census of
1841, will come to a conclusion not very different from mine.}


Of the taxation we can speak with more confidence and precision than of
the population. The revenue of England, when Charles the Second
died, was small, when compared with the resources which she even then
possessed, or with the sums which were raised by the governments of the
neighbouring countries. It had, from the time of the Restoration, been
almost constantly increasing, yet it was little more than three fourths
of the revenue of the United Provinces, and was hardly one fifth of the
revenue of France.

The most important head of receipt was the excise, which, in the last
year of the reign of Charles, produced five hundred and eighty-five
thousand pounds, clear of all deductions. The net proceeds of the
customs amounted in the same year to five hundred and thirty thousand
pounds. These burdens did not lie very heavy on the nation. The tax on
chimneys, though less productive, call forth far louder murmurs. The
discontent excited by direct imposts is, indeed, almost always out of
proportion to the quantity of money which they bring into the Exchequer;
and the tax on chimneys was, even among direct imposts, peculiarly
odious: for it could be levied only by means of domiciliary visits; and
of such visits the English have always been impatient to a degree which
the people of other countries can but faintly conceive. The poorer
householders were frequently unable to pay their hearth money to the
day. When this happened, their furniture was distrained without mercy:
for the tax was farmed; and a farmer of taxes is, of all creditors,
proverbially the most rapacious. The collectors were loudly accused of
performing their unpopular duty with harshness and insolence. It was
said that, as soon as they appeared at the threshold of a cottage, the
children began to wail, and the old women ran to hide their earthenware.
Nay, the single bed of a poor family had sometimes been carried away
and sold. The net annual receipt from this tax was two hundred thousand
pounds. 
%[40]
\footnote{ There are in the Pepysian Library some ballads of that age
on the chimney money. I will give a specimen or two:

   "The good old dames whenever they the chimney man espied,
   Unto their nooks they haste away, their pots and pipkins hide.
   There is not one old dame in ten, and search the nation through,
   But, if you talk of chimney men, will spare a curse or two."

   Again:

   "Like plundering soldiers they'd enter the door,
   And make a distress on the goods of the poor.
   While frighted poor children distractedly cried;
   This nothing abated their insolent pride."

   In the British Museum there are doggrel verses composed on the
   same subject and in the same spirit:

   "Or, if through poverty it be not paid
   For cruelty to tear away the single bed,
   On which the poor man rests his weary head,
   At once deprives him of his rest and bread."

I take this opportunity the first which occurs, of acknowledging most
grateful the kind and liberal manner in which the Master and Vicemaster
of Magdalei College, Cambridge, gave me access to the valuable
collections of Pepys.}


When to the three great sources of income which have been mentioned
we add the royal domains, then far more extensive than at present,
the first fruits and tenths, which had not yet been surrendered to the
Church, the Duchies of Cornwall and Lancaster, the forfeitures, and the
fines, we shall find that the whole annual revenue of the crown may
be fairly estimated at about fourteen hundred thousand pounds. Of this
revenue part was hereditary; the rest had been granted to Charles for
life; and he was at liberty to lay out the whole exactly as he thought
fit. Whatever he could save by retrenching from the expenditure of
the public departments was an addition to his privy purse. Of the Post
Office more will hereafter be said. The profits of that establishment
had been appropriated by Parliament to the Duke of York.

The King's revenue was, or rather ought to have been, charged with the
payment of about eighty thousand pounds a year, the interest of the sum
fraudulently destined in the Exchequer by the Cabal. While Danby was at
the head of the finances, the creditors had received dividends, though
not with the strict punctuality of modern times: but those who had
succeeded him at the treasury had been less expert, or less solicitous
to maintain public faith. Since the victory won by the court over the
Whigs, not a farthing had been paid; and no redress was granted to the
sufferers, till a new dynasty had been many years on the throne. There
can be no greater error than to imagine that the device of meeting the
exigencies of the state by loans was imported into our island by
William the Third. What really dates from his reign is not the system
of borrowing, but the system of funding. From a period of immemorable
antiquity it had been the practice of every English government to
contract debts. What the Revolution introduced was the practice of
honestly paying them. 
%[41]
\footnote{ My chief authorities for this financial statement will be
found in the Commons' Journal, March 1, and March 20, 1688-9.}


By plundering the public creditor, it was possible to make an income of
about fourteen hundred thousand pounds, with some occasional help from
Versailles, support the necessary charges of the government and the
wasteful expenditure of the court. For that load which pressed most
heavily on the finances of the great continental states was here
scarcely felt. In France, Germany, and the Netherlands, armies, such
as Henry the Fourth and Philip the Second had never employed in time
of war, were kept up in the midst of peace. Bastions and raveling
were everywhere rising, constructed on principles unknown to Parma and
Spinola. Stores of artillery and ammunition were accumulated, such as
even Richelieu, whom the preceding generation had regarded as a worker
of prodigies, would have pronounced fabulous. No man could journey many
leagues in those countries without hearing the drums of a regiment
on march, or being challenged by the sentinels on the drawbridge of a
fortress. In our island, on the contrary, it was possible to live long
and to travel far without being once reminded, by any martial sight or
sound, that the defence of nations had become a science and a calling.
The majority of Englishmen who were under twenty-five years of age had
probably never seen a company of regular soldiers. Of the cities which,
in the civil war, had valiantly repelled hostile armies, scarcely one
was now capable of sustaining a siege The gates stood open night and
day. The ditches were dry. The ramparts had been suffered to fall into
decay, or were repaired only that the townsfolk might have a pleasant
walk on summer evenings. Of the old baronial keeps many had been
shattered by the cannon of Fairfax and Cromwell, and lay in heaps of
ruin, overgrown with ivy. Those which remained had lost their martial
character, and were now rural palaces of the aristocracy. The moats were
turned into preserves of carp and pike. The mounds were planted with
fragrant shrubs, through which spiral walks ran up to summer houses
adorned with mirrors and paintings. 
%[42]
\footnote{ See, for example, the picture of the mound at Marlborough,
in Stukeley's Dinerarium Curiosum.}
 On the capes of the sea coast,
and on many inland hills, were still seen tall posts, surmounted by
barrels. Once those barrels had been filled with pitch. Watchmen had
been set round them in seasons of danger; and, within a few hours after
a Spanish sail had been discovered in the Channel, or after a thousand
Scottish mosstroopers had crossed the Tweed, the signal fires were
blazing fifty miles off, and whole counties were rising in arms. But
many years had now elapsed since the beacons had been lighted; and they
were regarded rather as curious relics of ancient manners than as parts
of a machinery necessary to the safety of the state. 
%[43]
\footnote{ Chamberlayne's State of England, 1684.}


The only army which the law recognised was the militia. That force had
been remodelled by two Acts of Parliament, passed shortly after the
Restoration. Every man who possessed five hundred pounds a year derived
from land, or six thousand pounds of personal estate, was bound to
provide, equip, and pay, at his own charge, one horseman. Every man
who had fifty pounds a year derived from land, or six hundred pounds
of personal estate, was charged in like manner with one pikemen or
musketeer. Smaller proprietors were joined together in a kind of
society, for which our language does not afford a special name, but
which an Athenian would have called a Synteleia; and each society was
required to furnish, according to its means, a horse soldier or a foot
soldier. The whole number of cavalry and infantry thus maintained was
popularly estimated at a hundred and thirty thousand men. 
%[44]
\footnote{ 13 and 14 Car. II. c. 3; 15 Car. II. c. 4. Chamberlayne's
State of England, 1684.}


The King was, by the ancient constitution of the realm, and by the
recent and solemn acknowledgment of both Houses of Parliament, the sole
Captain General of this large force. The Lords Lieutenants and their
Deputies held the command under him, and appointed meetings for drilling
and inspection. The time occupied by such meetings, however, was not
to exceed fourteen days in one year. The Justices of the Peace were
authorised to inflict severe penalties for breaches of discipline. Of
the ordinary cost no part was paid by the crown: but when the trainbands
were called out against an enemy, their subsistence became a charge on
the general revenue of the state, and they were subject to the utmost
rigour of martial law.

There were those who looked on the militia with no friendly eye. Men
who had travelled much on the Continent, who had marvelled at the stern
precision with which every sentinel moved and spoke in the citadels
built by Vauban, who had seen the mighty armies which poured along all
the roads of Germany to chase the Ottoman from the Gates of Vienna, and
who had been dazzled by the well ordered pomp of the household troops of
Lewis, sneered much at the way in which the peasants of Devonshire and
Yorkshire marched and wheeled, shouldered muskets and ported pikes. The
enemies of the liberties and religion of England looked with aversion on
a force which could not, without extreme risk, be employed against
those liberties and that religion, and missed no opportunity of throwing
ridicule on the rustic soldiery. 
%[45]
\footnote{ Dryden, in his Cymon and Iphigenia, expressed, with his
usual keenness and energy, the sentiments which had been fashionable
among the sycophants of James the Second:--

     "The country rings around with loud alarms,
     And raw in fields the rude militia swarms;
     Mouths without hands, maintained at vast expense,
     Stout once a month they march, a blustering band,
     And ever, but in time of need at hand.
     This was the morn when, issuing on the guard,
     Drawn up in rank and file, they stood prepared
     Of seeming arms to make a short essay.
     Then hasten to be drunk, the business of the day."}
 Enlightened patriots, when they
contrasted these rude levies with the battalions which, in time of war,
a few hours might bring to the coast of Kent or Sussex, were forced
to acknowledge that, dangerous as it might be to keep up a permanent
military establishment, it might be more dangerous still to stake
the honour and independence of the country on the result of a contest
between plowmen officered by Justices of the Peace, and veteran warriors
led by Marshals of France. In Parliament, however, it was necessary
to express such opinions with some reserve; for the militia was an
institution eminently popular. Every reflection thrown on it excited the
indignation of both the great parties in the state, and especially of
that party which was distinguished by peculiar zeal for monarchy and
for the Anglican Church. The array of the counties was commanded almost
exclusively by Tory noblemen and gentlemen. They were proud of their
military rank, and considered an insult offered to the service to which
they belonged as offered to themselves. They were also perfectly
aware that whatever was said against a militia was said in favour of a
standing army; and the name of standing army was hateful to them. One
such army had held dominion in England; and under that dominion the King
had been murdered, the nobility degraded, the landed gentry plundered,
the Church persecuted. There was scarcely a rural grandee who could
not tell a story of wrongs and insults suffered by himself, or by his
father, at the hands of the parliamentary soldiers. One old Cavalier had
seen half his manor house blown up. The hereditary elms of another had
been hewn down. A third could never go into his parish church without
being reminded by the defaced scutcheons and headless statues of his
ancestry, that Oliver's redcoats had once stabled their horses there.
The consequence was that those very Royalists, who were most ready
to fight for the King themselves, were the last persons whom he could
venture to ask for the means of hiring regular troops.

Charles, however, had, a few months after his restoration, begun to form
a small standing army. He felt that, without some better protection
than that of the trainbands and beefeaters, his palace and person would
hardly be secure, in the vicinity of a great city swarming with warlike
Fifth Monarchy men who had just been disbanded. He therefore, careless
and profuse as he was, contrived to spare from his pleasures a sum
sufficient to keep up a body of guards. With the increase of trade and
of public wealth his revenues increased; and he was thus enabled,
in spite of the occasional murmurs of the Commons, to make gradual
additions to his regular forces. One considerable addition was made
a few months before the close of his reign. The costly, useless, and
pestilential settlement of Tangier was abandoned to the barbarians who
dwelt around it; and the garrison, consisting of one regiment of horse
and two regiments of foot, was brought to England.

The little army formed by Charles the Second was the germ of that great
and renowned army which has, in the present century, marched triumphant
into Madrid and Paris, into Canton and Candahar. The Life Guards, who
now form two regiments, were then distributed into three troops, each of
which consisted of two hundred carabineers, exclusive of officers. This
corps, to which the safety of the King and royal family was confided,
had a very peculiar character. Even the privates were designated as
gentlemen of the Guard. Many of them were of good families, and had held
commissions in the civil war. Their pay was far higher than that of
the most favoured regiment of our time, and would in that age have been
thought a respectable provision for the younger son of a country squire.
Their fine horses, their rich housings, their cuirasses, and their
buff coats adorned with ribands, velvet, and gold lace, made a splendid
appearance in Saint James's Park. A small body of grenadier dragoons,
who came from a lower class and received lower pay, was attached to each
troop. Another body of household cavalry distinguished by blue coats
and cloaks, and still called the Blues, was generally quartered in the
neighbourhood of the capital. Near the capital lay also the corps which
is now designated as the first regiment of dragoons, but which was
then the only regiment of dragoons on the English establishment. It had
recently been formed out of the cavalry which had returned from Tangier.
A single troop of dragoons, which did not form part of any regiment, was
stationed near Berwick, for the purpose of keeping, the peace among the
mosstroopers of the border. For this species of service the dragoon
was then thought to be peculiarly qualified. He has since become a
mere horse soldier. But in the seventeenth century he was accurately
described by Montecuculi as a foot soldier who used a horse only in
order to arrive with more speed at the place where military service was
to be performed.

The household infantry consisted of two regiments, which were then,
as now, called the first regiment of Foot Guards, and the Coldstream
Guards. They generally did duty near Whitehall and Saint James's Palace.
As there were then no barracks, and as, by the Petition of Right, it
had been declared unlawful to quarter soldiers on private families, the
redcoats filled all the alehouses of Westminster and the Strand.

There were five other regiments of foot. One of these, called the
Admiral's Regiment, was especially destined to service on board of the
fleet. The remaining four still rank as the first four regiments of the
line. Two of these represented two brigades which had long sustained on
the Continent the fame of British valour. The first, or Royal regiment,
had, under the great Gustavus, borne a conspicuous part in the
deliverance of Germany. The third regiment, distinguished by
fleshcoloured facings, from which it had derived the well known name of
the Buffs, had, under Maurice of Nassau, fought not less bravely for the
deliverance of the Netherlands. Both these gallant bands had at length,
after many vicissitudes, been recalled from foreign service by Charles
the Second, and had been placed on the English establishment.

The regiments which now rank as the second and fourth of the line
had, in 1685, just returned from Tangier, bringing with them cruel and
licentious habits contracted in a long course of warfare with the
Moors. A few companies of infantry which had not been regimented lay in
garrison at Tilbury Fort, at Portsmouth, at Plymouth, and at some other
important stations on or near the coast.

Since the beginning of the seventeenth century a great change had taken
place in the arms of the infantry. The pike had been gradually giving
place to the musket; and, at the close of the reign of Charles the
Second, most of his foot were musketeers. Still, however, there was a
large intermixture of pikemen. Each class of troops was occasionally
instructed in the use of the weapon which peculiarly belonged to the
other class. Every foot soldier had at his side a sword for close fight.
The musketeer was generally provided with a weapon which had, during
many years, been gradually coming into use, and which the English then
called a dagger, but which, from the time of William the Third, has been
known among us by the French name of bayonet. The bayonet seems not
to have been then so formidable an instrument of destruction as it
has since become; for it was inserted in the muzzle of the gun; and in
action much time was lost while the soldier unfixed his bayonet in
order to fire, and fixed it again in order to charge. The dragoon, when
dismounted, fought as a musketeer.

The regular army which was kept up in England at the beginning of the
year 1685 consisted, all ranks included, of about seven thousand foot,
and about seventeen hundred cavalry and dragoons. The whole charge
amounted to about two hundred and ninety thousand pounds a year, less
then a tenth part of what the military establishment of France then cost
in time of peace. The daily pay of a private in the Life Guards was
four shillings, in the Blues two shillings and sixpence, in the Dragoons
eighteen pence, in the Foot Guards tenpence, and in the line eightpence.
The discipline was lax, and indeed could not be otherwise. The common
law of England knew nothing of courts martial, and made no distinction,
in time of peace, between a soldier and any other subject; nor could
the government then venture to ask even the most loyal Parliament for
a Mutiny Bill. A soldier, therefore, by knocking down his colonel,
incurred only the ordinary penalties of assault and battery, and by
refusing to obey orders, by sleeping on guard, or by deserting his
colours, incurred no legal penalty at all. Military punishments were
doubtless inflicted during the reign of Charles the Second; but they
were inflicted very sparingly, and in such a manner as not to attract
public notice, or to produce an appeal to the courts of Westminster
Hall.

Such an army as has been described was not very likely to enslave five
millions of Englishmen. It would indeed have been unable to suppress
an insurrection in London, if the trainbands of the City had joined the
insurgents. Nor could the King expect that, if a rising took place in
England, he would obtain effectual help from his other dominions.
For, though both Scotland and Ireland supported separate military
establishments, those establishments were not more than sufficient to
keep down the Puritan malecontents of the former kingdom and the Popish
malecontents of the latter. The government had, however, an important
military resource which must not be left unnoticed. There were in the
pay of the United Provinces six fine regiments, of which three had
been raised in England and three in Scotland. Their native prince had
reserved to himself the power of recalling them, if he needed their
help against a foreign or domestic enemy. In the meantime they were
maintained without any charge to him, and were kept under an excellent
discipline to which he could not have ventured to subject them. 
%[46]
\footnote{ Most of the materials which I have used for this
account of the regular army will be found in the Historical Records of
Regiments, published by command of King William the Fourth, and under
the direction of the Adjutant General. See also Chamberlayne's State of
England, 1684; Abridgment of the English Military Discipline, printed by
especial command, 1688; Exercise of Foot, by their Majesties' command,
1690.}


If the jealousy of the Parliament and of the nation made it impossible
for the King to maintain a formidable standing army, no similar
impediment prevented him from making England the first of maritime
powers. Both Whigs and Tories were ready to applaud every step tending
to increase the efficiency of that force which, while it was the best
protection of the island against foreign enemies, was powerless against
civil liberty. All the greatest exploits achieved within the memory of
that generation by English soldiers had been achieved in war against
English princes. The victories of our sailors had been won over foreign
foes, and had averted havoc and rapine from our own soil. By at least
half the nation the battle of Naseby was remembered with horror, and the
battle of Dunbar with pride chequered by many painful feelings: but the
defeat of the Armada, and the encounters of Blake with the Hollanders
and Spaniards were recollected with unmixed exultation by all parties.
Ever since the Restoration, the Commons, even when most discontented
and most parsimonious, had always been bountiful to profusion where the
interest of the navy was concerned. It had been represented to them,
while Danby was minister, that many of the vessels in the royal fleet
were old and unfit for sea; and, although the House was, at that time,
in no giving mood, an aid of near six hundred thousand pounds had been
granted for the building of thirty new men of war.

But the liberality of the nation had been made fruitless by the vices of
the government. The list of the King's ships, it is true, looked well.
There were nine first rates, fourteen second rates, thirty-nine third
rates, and many smaller vessels. The first rates, indeed, were less than
the third rates of our time; and the third rates would not now rank
as very large frigates. This force, however, if it had been efficient,
would in those days have been regarded by the greatest potentate as
formidable. But it existed only on paper. When the reign of Charles
terminated, his navy had sunk into degradation and decay, such as would
be almost incredible if it were not certified to us by the independent
and concurring evidence of witnesses whose authority is beyond
exception. Pepys, the ablest man in the English Admiralty, drew up,
in the year 1684, a memorial on the state of his department, for the
information of Charles. A few months later Bonrepaux, the ablest man in
the French Admiralty, having visited England for the especial purpose
of ascertaining her maritime strength, laid the result of his inquiries
before Lewis. The two reports are to the same effect. Bonrepaux declared
that he found everything in disorder and in miserable condition, that
the superiority of the French marine was acknowledged with shame and
envy at Whitehall, and that the state of our shipping and dockyards
was of itself a sufficient guarantee that we should not meddle in
the disputes of Europe. 
%[47]
\footnote{ I refer to a despatch of Bonrepaux to Seignelay, dated
Feb. 8/18. 1686. It was transcribed for Mr. Fox from the French
archives, during the peace of Amiens, and, with the other materials
brought together by that great man, was entrusted to me by the kindness
of the late Lady Holland, and of the present Lord Holland. I ought to
add that, even in the midst of the troubles which have lately agitated
Paris, I found no difficulty in obtaining, from the liberality of
the functionaries there, extracts supplying some chasms in Mr. Fox's
collection. (1848.)}
 Pepys informed his master that the naval
administration was a prodigy of wastefulness, corruption, ignorance,
and indolence, that no estimate could be trusted, that no contract was
performed, that no check was enforced. The vessels which the recent
liberality of Parliament had enabled the government to build, and which
had never been out of harbour, had been made of such wretched timber
that they were more unfit to go to sea than the old hulls which had been
battered thirty years before by Dutch and Spanish broadsides. Some
of the new men of war, indeed, were so rotten that, unless speedily
repaired, they would go down at their moorings. The sailors were paid
with so little punctuality that they were glad to find some usurer who
would purchase their tickets at forty per cent. discount. The commanders
who had not powerful friends at court were even worse treated. Some
officers, to whom large arrears were due, after vainly importuning the
government during many years, had died for want of a morsel of bread.

Most of the ships which were afloat were commanded by men who had not
been bred to the sea. This, it is true, was not an abuse introduced by
the government of Charles. No state, ancient or modern, had, before that
time, made a complete separation between the naval and military service.
In the great civilised nations of antiquity, Cimon and Lysander, Pompey
and Agrippa, had fought battles by sea as well as by land. Nor had the
impulse which nautical science received at the close of the fifteenth
century produced any new division of labour. At Flodden the right wing
of the victorious army was led by the Admiral of England. At Jarnac and
Moncontour the Huguenot ranks were marshalled by the Admiral of France.
Neither John of Austria, the conqueror of Lepanto, nor Lord Howard of
Effingham, to whose direction the marine of England was confided when
the Spanish invaders were approaching our shores, had received the
education of a sailor. Raleigh, highly celebrated as a naval commander,
had served during many years as a soldier in France, the Netherlands,
and Ireland. Blake had distinguished himself by his skilful and valiant
defence of an inland town before he humbled the pride of Holland and
of Castile on the ocean. Since the Restoration the same system had been
followed. Great fleets had been entrusted to the direction of Rupert
and Monk; Rupert, who was renowned chiefly as a hot and daring cavalry
officer, and Monk, who, when he wished his ship to change her course,
moved the mirth of his crew by calling out, "Wheel to the left!"

But about this time wise men began to perceive that the rapid
improvement, both of the art of war and of the art of navigation, made
it necessary to draw a line between two professions which had hitherto
been confounded. Either the command of a regiment or the command of
a ship was now a matter quite sufficient to occupy the attention of
a single mind. In the year 1672 the French government determined to
educate young men of good family from a very early age especially for
the sea service. But the English government, instead of following this
excellent example, not only continued to distribute high naval commands
among landsmen, but selected for such commands landsmen who, even on
land, could not safely have been put in any important trust. Any lad
of noble birth, any dissolute courtier for whom one of the King's
mistresses would speak a word, might hope that a ship of the line, and
with it the honour of the country and the lives of hundreds of brave
men, would be committed to his care. It mattered not that he had never
in his life taken a voyage except on the Thames, that he could not
keep his feet in a breeze, that he did not know the difference between
latitude and longitude. No previous training was thought necessary; or,
at most, he was sent to make a short trip in a man of war, where he was
subjected to no discipline, where he was treated with marked respect,
and where he lived in a round of revels and amusements. If, in the
intervals of feasting, drinking, and gambling, he succeeded in learning
the meaning of a few technical phrases and the names of the points
of the compass, he was thought fully qualified to take charge of a
three-decker. This is no imaginary description. In 1666, John Sheffield,
Earl of Mulgrave, at seventeen years of age, volunteered to serve at sea
against the Dutch. He passed six weeks on board, diverting himself, as
well as he could, in the society of some young libertines of rank, and
then returned home to take the command of a troop of horse. After this
he was never on the water till the year 1672, when he again joined
the fleet, and was almost immediately appointed Captain of a ship
of eighty-four guns, reputed the finest in the navy. He was then
twenty-three years old, and had not, in the whole course of his life,
been three months afloat. As soon as he came back from sea he was made
Colonel of a regiment of foot. This is a specimen of the manner in which
naval commands of the highest importance were then given; and a very
favourable specimen; for Mulgrave, though he wanted experience, wanted
neither parts nor courage. Others were promoted in the same way who not
only were not good officers, but who were intellectually and morally
incapable of ever becoming good officers, and whose only recommendation
was that they had been ruined by folly and vice. The chief bait which
allured these men into the service was the profit of conveying bullion
and other valuable commodities from port to port; for both the Atlantic
and the Mediterranean were then so much infested by pirates from Barbary
that merchants were not willing to trust precious cargoes to any custody
but that of a man of war. A Captain might thus clear several thousands
of pounds by a short voyage; and for this lucrative business he too
often neglected the interests of his country and the honour of his
flag, made mean submissions to foreign powers, disobeyed the most direct
injunctions of his superiors, lay in port when he was ordered to chase
a Sallee rover, or ran with dollars to Leghorn when his instructions
directed him to repair to Lisbon. And all this he did with impunity.
The same interest which had placed him in a post for which he was unfit
maintained him there. No Admiral, bearded by these corrupt and dissolute
minions of the palace, dared to do more than mutter something about a
court martial. If any officer showed a higher sense of duty than his
fellows, he soon found out he lost money without acquiring honor. One
Captain, who, by strictly obeying the orders of the Admiralty, missed a
cargo which would have been worth four thousand pounds to him, was told
by Charles, with ignoble levity, that he was a great fool for his pains.

The discipline of the navy was of a piece throughout. As the courtly
Captain despised the Admiralty, he was in turn despised by his crew.
It could not be concealed that he was inferior in Seamanship to every
foremast man on board. It was idle to expect that old sailors, familiar
with the hurricanes of the tropics and with the icebergs of the Arctic
Circle, would pay prompt and respectful obedience to a chief who knew no
more of winds and waves than could be learned in a gilded barge between
Whitehall Stairs and Hampton Court. To trust such a novice with the
working of a ship was evidently impossible. The direction of the
navigation was therefore taken from the Captain and given to the Master;
but this partition of authority produced innumerable inconveniences.
The line of demarcation was not, and perhaps could not be, drawn
with precision. There was therefore constant wrangling. The Captain,
confident in proportion to his ignorance, treated the Master with
lordly contempt. The Master, well aware of the danger of disobliging
the powerful, too often, after a struggle, yielded against his better
judgment; and it was well if the loss of ship and crew was not the
consequence. In general the least mischievous of the aristocratical
Captains were those who completely abandoned to others the direction of
the vessels, and thought only of making money and spending it. The way
in which these men lived was so ostentatious and voluptuous that, greedy
as they were of gain, they seldom became rich. They dressed as if for
a gala at Versailles, ate off plate, drank the richest wines, and kept
harems on board, while hunger and scurvy raged among the crews, and
while corpses were daily flung out of the portholes.

Such was the ordinary character of those who were then called gentlemen
Captains. Mingled with them were to be found, happily for our country,
naval commanders of a very different description, men whose whole life
had been passed on the deep, and who had worked and fought their way
from the lowest offices of the forecastle to rank and distinction. One
of the most eminent of these officers was Sir Christopher Mings, who
entered the service as a cabin boy, who fell fighting bravely against
the Dutch, and whom his crew, weeping and vowing vengeance, carried to
the grave. From him sprang, by a singular kind of descent, a line of
valiant and expert sailors. His cabin boy was Sir John Narborough; and
the cabin boy of Sir John Narborough was Sir Cloudesley Shovel. To the
strong natural sense and dauntless courage of this class of men England
owes a debt never to be forgotten. It was by such resolute hearts that,
in spite of much maladministration, and in spite of the blunders and
treasons of more courtly admirals, our coasts were protected and the
reputation of our flag upheld during many gloomy and perilous years. But
to a landsman these tarpaulins, as they were called, seemed a strange
and half savage race. All their knowledge was professional; and their
professional knowledge was practical rather than scientific. Off their
own element they were as simple as children. Their deportment was
uncouth. There was roughness in their very good nature; and their talk,
where it was not made up of nautical phrases, was too commonly made
up of oaths and curses. Such were the chiefs in whose rude school were
formed those sturdy warriors from whom Smollett, in the next age, drew
Lieutenant Bowling and Commodore Trunnion. But it does not appear that
there was in the service of any of the Stuarts a single naval officer
such as, according to the notions of our times, a naval officer ought
to be, that is to say, a man versed in the theory and practice of his
calling, and steeled against all the dangers of battle and tempest, yet
of cultivated mind and polished manners. There were gentlemen and there
were seamen in the navy of Charles the Second. But the seamen were not
gentlemen; and the gentlemen were not seamen.

The English navy at that time might, according to the most exact
estimates which have come down to us, have been kept in an efficient
state for three hundred and eighty thousand pounds a year. Four hundred
thousand pounds a year was the sum actually expended, but expended, as
we have seen, to very little purpose. The cost of the French marine was
nearly the same the cost of the Dutch marine considerably more. 
%[48]
\footnote{ My information respecting the condition of the navy,
at this time, is chiefly derived from Pepys. His report, presented to
Charles the Second in May, 1684, has never, I believe, been printed. The
manuscript is at Magdalene College Cambridge. At Magdalene College is
also a valuable manuscript containing a detailed account of the maritime
establishments of the country in December 1684. Pepys's "Memoirs
relating to the State of the Royal Navy for Ten Years determined
December, 1688," and his diary and correspondence during his mission
to Tangier, are in print. I have made large use of them. See also
Sheffield's Memoirs, Teonge's Diary, Aubrey's Life of Monk, the Life of
Sir Cloudesley Shovel, 1708, Commons' Journals, March 1 and March 20.
1688-9.}


The charge of the English ordnance in the seventeenth century was, as
compared with other military and naval charges, much smaller than at
present. At most of the garrisons there were gunners: and here and
there, at an important post, an engineer was to be found. But there was
no regiment of artillery, no brigade of sappers and miners, no college
in which young soldiers could learn the scientific part of the art of
war. The difficulty of moving field pieces was extreme. When, a few
years later, William marched from Devonshire to London, the apparatus
which he brought with him, though such as had long been in constant use
on the Continent, and such as would now be regarded at Woolwich as rude
and cumbrous, excited in our ancestors an admiration resembling that
which the Indians of America felt for the Castilian harquebusses. The
stock of gunpowder kept in the English forts and arsenals was boastfully
mentioned by patriotic writers as something which might well impress
neighbouring nations with awe. It amounted to fourteen or fifteen
thousand barrels, about a twelfth of the quantity which it is now
thought necessary to have in store. The expenditure under the head of
ordnance was on an average a little above sixty thousand pounds a year.

%[49]
\footnote{ Chamberlayne's State of England, 1684; Commons' Journals,
March 1, and March 20, 1688-9. In 1833, it was determined, after full
enquiry, that a hundred and seventy thousand barrels of gunpowder should
constantly be kept in store.}


The whole effective charge of the army, navy, and ordnance, was about
seven hundred and fifty thousand pounds. The noneffective charge, which
is now a heavy part of our public burdens, can hardly be said to have
existed. A very small number of naval officers, who were not employed
in the public service, drew half pay. No Lieutenant was on the list, nor
any Captain who had not commanded a ship of the first or second rate. As
the country then possessed only seventeen ships of the first and second
rate that had ever been at sea, and as a large proportion of the persons
who had commanded such ships had good posts on shore, the expenditure
under this head must have been small indeed. 
%[50]
\footnote{ It appears from the records of the Admiralty, that Flag
officers were allowed half pay in 1668, Captains of first and second
rates not till 1674.}
 In the army, half pay
was given merely as a special and temporary allowance to a small number
of officers belonging to two regiments, which were peculiarly situated.

%[51]
\footnote{ Warrant in the War Office Records; dated March 26, 1678.}
 Greenwich Hospital had not been founded. Chelsea Hospital was
building: but the cost of that institution was defrayed partly by
a deduction from the pay of the troops, and partly by private
subscription. The King promised to contribute only twenty thousand
pounds for architectural expenses, and five thousand a year for the
maintenance of the invalids. 
%[52]
\footnote{ Evelyn's Diary. Jan. 27, 1682. I have seen a privy seal,
dated May 17. 1683, which confirms Evelyn's testimony.}
 It was no part of the plan that there
should be outpensioners. The whole noneffective charge, military and
naval, can scarcely have exceeded ten thousand pounds a year. It now
exceeds ten thousand pounds a day.

Of the expense of civil government only a small portion was defrayed by
the crown. The great majority of the functionaries whose business was to
administer justice and preserve order either gave their services to the
public gratuitously, or were remunerated in a manner which caused no
drain on the revenue of the state. The Sheriffs, mayors, and aldermen
of the towns, the country gentlemen who were in the commission of the
peace, the headboroughs, bailiffs, and petty constables, cost the King
nothing. The superior courts of law were chiefly supported by fees.

Our relations with foreign courts had been put on the most economical
footing. The only diplomatic agent who had the title of Ambassador
resided at Constantinople, and was partly supported by the Turkish
Company. Even at the court of Versailles England had only an Envoy; and
she had not even an Envoy at the Spanish, Swedish, and Danish courts.
The whole expense under this head cannot, in the last year of the reign
of Charles the Second, have much exceeded twenty thousand pounds. 
%[53]
\footnote{ James the Second sent Envoys to Spain, Sweden, and
Denmark; yet in his reign the diplomatic expenditure was little more
than 30,000£. a year. See the Commons' Journals, March 20, 1688-9.
Chamberlayne's State of England, 1684.}


In this frugality there was nothing laudable. Charles was, as usual,
niggardly in the wrong place, and munificent in the wrong place. The
public service was starved that courtiers might be pampered. The expense
of the navy, of the ordnance, of pensions to needy old officers, of
missions to foreign courts, must seem small indeed to the present
generation. But the personal favourites of the sovereign, his ministers,
and the creatures of those ministers, were gorged with public money.
Their salaries and pensions, when compared with the incomes of the
nobility, the gentry, the commercial and professional men of that age,
will appear enormous. The greatest estates in the kingdom then
very little exceeded twenty thousand a year. The Duke of Ormond had
twenty-two thousand a year. 
%[54]
\footnote{ Carte's Life of Ormond.}
 The Duke of Buckingham, before his
extravagance had impaired his great property, had nineteen thousand
six hundred a year. 
%[55]
\footnote{ Pepys's Diary, Feb. 14, 1668-9.}
 George Monk, Duke of Albemarle, who had been
rewarded for his eminent services with immense grants of crown land,
and who had been notorious both for covetousness and for parsimony, left
fifteen thousand a year of real estate, and sixty thousand pounds in
money which probably yielded seven per cent. 
%[56]
\footnote{ See the Report of the Bath and Montague case, which was
decided by Lord Keeper Somers, in December, 1693.}
 These three Dukes
were supposed to be three of the very richest subjects in England. The
Archbishop of Canterbury can hardly have had five thousand a year.

%[57]
\footnote{ During three quarters of a year, beginning from Christmas,
1689, the revenues of the see of Canterbury were received by an officer
appointed by the crown. That officer's accounts are now in the British
Museum. (Lansdowne MSS. 885.) The gross revenue for the three quarters
was not quite four thousand pounds; and the difference between the gross
and the net revenue was evidently something considerable.}
 The average income of a temporal peer was estimated, by the best
informed persons, at about three thousand a year, the average income of
a baronet at nine hundred a year, the average income of a member of the
House of Commons at less than eight hundred a year. 
%[58]
\footnote{ King's Natural and Political Conclusions. Davenant on
the Balance of Trade. Sir W. Temple says, "The revenues of a House of
Commons have seldom exceeded four hundred thousand pounds." Memoirs,
Third Part.}
 A thousand a
year was thought a large revenue for a barrister. Two thousand a year
was hardly to be made in the Court of King's Bench, except by the crown
lawyers. 
%[59]
\footnote{ Langton's Conversations with Chief Justice Hale, 1672.}
 It is evident, therefore, that an official man would have
been well paid if he had received a fourth or fifth part of what would
now be an adequate stipend. In fact, however, the stipends of the
higher class of official men were as large as at present, and not seldom
larger. The Lord Treasurer, for example, had eight thousand a year,
and, when the Treasury was in commission, the junior Lords had sixteen
hundred a year each. The Paymaster of the Forces had a poundage,
amounting, in time of peace, to about five thousand a year, on all the
money which passed through his hands. The Groom of the Stole had five
thousand a year, the Commissioners of the Customs twelve hundred a
year each, the Lords of the Bedchamber a thousand a year each. 
%[60]
\footnote{ Commons' Journals, April 27,1689; Chamberlayne's State of
England, 1684.}

The regular salary, however, was the smallest part of the gains of an
official man at that age. From the noblemen who held the white staff and
the great seal, down to the humblest tidewaiter and gauger, what would
now be called gross corruption was practiced without disguise and
without reproach. Titles, places, commissions, pardons, were daily sold
in market overt by the great dignitaries of the realm; and every
clerk in every department imitated, to the best of his power, the evil
example.

During the last century no prime minister, however powerful, has become
rich in office; and several prime ministers have impaired their private
fortune in sustaining their public character. In the seventeenth
century, a statesman who was at the head of affairs might easily, and
without giving scandal, accumulate in no long time an estate amply
sufficient to support a dukedom. It is probable that the income of the
prime minister, during his tenure of power, far exceeded that of any
other subject. The place of Lord Lieutenant of Ireland was popularly
reported to be worth forty thousand pounds a year. 
%[61]
\footnote{ See the Travels of the Grand Duke Cosmo.}
 The gains of the
Chancellor Clarendon, of Arlington, of Lauderdale, and of Danby, were
certainly enormous. The sumptuous palace to which the populace of London
gave the name of Dunkirk Mouse, the stately pavilions, the fishponds,
the deer park and the orangery of Euston, the more than Italian luxury
of Ham, with its busts, fountains, and aviaries, were among the many
signs which indicated what was the shortest road to boundless wealth.
This is the true explanation of the unscrupulous violence with which the
statesmen of that day struggled for office, of the tenacity with which,
in spite of vexations, humiliations and dangers, they clung to it, and
of the scandalous compliances to which they stooped in order to retain
it. Even in our own age, formidable as is the power of opinion, and
high as is the standard of integrity, there would be great risk of a
lamentable change in the character of our public men, if the place of
First Lord of the Treasury or Secretary of State were worth a hundred
thousand pounds a year. Happy for our country the emoluments of the
highest class of functionaries have not only not grown in proportion to
the general growth of our opulence, but have positively diminished.

The fact that the sum raised in England by taxation has, in a time not
exceeding two long lives, been multiplied forty-fold, is strange, and
may at first sight seem appalling. But those who are alarmed by the
increase of the public burdens may perhaps be reassured when they have
considered the increase of the public resources. In the year 1685, the
value of the produce of the soil far exceeded the value of all the
other fruits of human industry. Yet agriculture was in what would now
be considered as a very rude and imperfect state. The arable land and
pasture land were not supposed by the best political arithmeticians of
that age to amount to much more than half the area of the kingdom. 
%[62]
\footnote{ King's Natural and Political Conclusions. Davenant on the
Balance of Trade.}

The remainder was believed to consist of moor, forest, and fen. These
computations are strongly confirmed by the road books and maps of the
seventeenth century. From those books and maps it is clear that many
routes which now pass through an endless succession of orchards,
cornfields, hayfields, and beanfields, then ran through nothing but
heath, swamp, and warren. 
%[63]
\footnote{ See the Itinerarium Angliae, 1675, by John Ogilby,
Cosmographer Royal. He describes great part of the land as wood, fen,
heath on both sides, marsh on both sides. In some of his maps the roads
through enclosed country are marked by lines, and the roads through
unenclosed country by dots. The proportion of unenclosed country, which,
if cultivated, must have been wretchedly cultivated, seems to have been
very great. From Abingdon to Gloucester, for example, a distance of
forty or fifty miles, there was not a single enclosure, and scarcely one
enclosure between Biggleswade and Lincoln.}
 In the drawings of English landscapes
made in that age for the Grand Duke Cosmo, scarce a hedgerow is to be
seen, and numerous tracts; now rich with cultivation, appear as bare as
Salisbury Plain. 
%[64]
\footnote{ Large copies of these highly interesting drawings are in
the noble collection bequeathed by Mr. Grenville to the British Museum.
See particularly the drawings of Exeter and Northampton.}
 At Enfield, hardly out of sight of the smoke of
the capital, was a region of five and twenty miles in circumference,
which contained only three houses and scarcely any enclosed fields.
Deer, as free as in an American forest, wandered there by thousands.

%[65]
\footnote{ Evelyn's Diary, June 2, 1675.}
 It is to be remarked, that wild animals of large size were then far
more numerous than at present. The last wild boars, indeed, which had
been preserved for the royal diversion, and had been allowed to ravage
the cultivated land with their tusks, had been slaughtered by the
exasperated rustics during the license of the civil war. The last wolf
that has roamed our island had been slain in Scotland a short time
before the close of the reign of Charles the Second. But many breeds,
now extinct, or rare, both of quadrupeds and birds, were still common.
The fox, whose life is now, in many counties, held almost as sacred as
that of a human being, was then considered as a mere nuisance. Oliver
Saint John told the Long Parliament that Strafford was to be regarded,
not as a stag or a hare, to whom some law was to be given, but as a fox,
who was to be snared by any means, and knocked on the head without pity.
This illustration would be by no means a happy one, if addressed to
country gentlemen of our time: but in Saint John's days there were not
seldom great massacres of foxes to which the peasantry thronged with all
the dogs that could be mustered. Traps were set: nets were spread: no
quarter was given; and to shoot a female with cub was considered as a
feat which merited the warmest gratitude of the neighbourhood. The red
deer were then as common in Gloucestershire and Hampshire, as they now
are among the Grampian Hills. On one occasion Queen Anne, travelling to
Portsmouth, saw a herd of no less than five hundred. The wild bull with
his white mane was still to be found wandering in a few of the southern
forests. The badger made his dark and tortuous hole on the side of every
hill where the copsewood grew thick. The wild cats were frequently heard
by night wailing round the lodges of the rangers of whittlebury and
Needwood. The yellow-breasted martin was still pursued in Cranbourne
Chase for his fur, reputed inferior only to that of the sable. Fen
eagles, measuring more than nine feet between the extremities of the
wings, preyed on fish along the coast of Norfolk. On all the downs,
from the British Channel to Yorkshire huge bustards strayed in troops
of fifty or sixty, and were often hunted with greyhounds. The marshes of
Cambridgeshire and Lincolnshire were covered during some months of every
year by immense clouds of cranes. Some of these races the progress of
cultivation has extirpated. Of others the numbers are so much diminished
that men crowd to gaze at a specimen as at a Bengal tiger, or a Polar
bear. 
%[66]
\footnote{ See White's Selborne; Bell's History of British
Quadrupeds, Gentleman's Recreation, 1686; Aubrey's Natural History
of Wiltshire, 1685; Morton's History of Northamptonshire, 1712;
Willoughby's Ornithology, by Ray, 1678; Latham's General Synopsis of
Birds; and Sir Thomas Browne's Account of Birds found in Norfolk.}


The progress of this great change can nowhere be more clearly traced
than in the Statute Book. The number of enclosure acts passed since King
George the Second came to the throne exceeds four thousand. The area
enclosed under the authority of those acts exceeds, on a moderate
calculation, ten thousand square miles. How many square miles, which
were formerly uncultivated or ill cultivated, have, during the same
period, been fenced and carefully tilled by the proprietors without any
application to the legislature, can only be conjectured. But it seems
highly probable that a fourth part of England has been, in the course of
little more than a century, turned from a wild into a garden.

Even in those parts of the kingdom which at the close of the reign of
Charles the Second were the best cultivated, the farming, though greatly
improved since the civil war, was not such as would now be thought
skilful. To this day no effectual steps have been taken by public
authority for the purpose of obtaining accurate accounts of the produce
of the English soil. The historian must therefore follow, with some
misgivings, the guidance of those writers on statistics whose reputation
for diligence and fidelity stands highest. At present an average crop of
wheat, rye, barley, oats, and beans, is supposed considerably to exceed
thirty millions of quarters. The crop of wheat would be thought wretched
if it did not exceed twelve millions of quarters. According to the
computation made in the year 1696 by Gregory King, the whole quantity of
wheat, rye, barley, oats, and beans, then annually grown in the kingdom,
was somewhat less than ten millions of quarters. The wheat, which was
then cultivated only on the strongest clay, and consumed only by those
who were in easy circumstances, he estimated at less than two millions
of quarters. Charles Davenant, an acute and well informed though most
unprincipled and rancorous politician, differed from King as to some
of the items of the account, but came to nearly the same general
conclusions. 
%[67]
\footnote{ King's Natural and Political Conclusions. Davenant on the
Balance of Trade.}


The rotation of crops was very imperfectly understood. It was known,
indeed, that some vegetables lately introduced into our island,
particularly the turnip, afforded excellent nutriment in winter to sheep
and oxen: but it was not yet the practice to feed cattle in this manner.
It was therefore by no means easy to keep them alive during the season
when the grass is scanty. They were killed and salted in great numbers
at the beginning of the cold weather; and, during several months, even
the gentry tasted scarcely any fresh animal food, except game and
river fish, which were consequently much more important articles
in housekeeping than at present. It appears from the Northumberland
Household Book that, in the reign of Henry the Seventh, fresh meat was
never eaten even by the gentlemen attendant on a great Earl, except
during the short interval between Midsummer and Michaelmas. But in
the course of two centuries an improvement had taken place; and under
Charles the Second it was not till the beginning of November that
families laid in their stock of salt provisions, then called Martinmas
beef. 
%[68]
\footnote{ See the Almanacks of 1684 and 1685.}


The sheep and the ox of that time were diminutive when compared with
the sheep and oxen which are now driven to our markets. 
%[69]
\footnote{ See Mr. M'Culloch's Statistical Account of the British
Empire, Part III. chap. i. sec. 6.}
 Our native
horses, though serviceable, were held in small esteem, and fetched low
prices. They were valued, one with another, by the ablest of those who
computed the national wealth, at not more than fifty shillings each.
Foreign breeds were greatly preferred. Spanish jennets were regarded
as the finest chargers, and were imported for purposes of pageantry and
war. The coaches of the aristocracy were drawn by grey Flemish mares,
which trotted, as it was thought, with a peculiar grace, and endured
better than any cattle reared in our island the work of dragging a
ponderous equipage over the rugged pavement of London. Neither the
modern dray horse nor the modern race horse was then known. At a
much later period the ancestors of the gigantic quadrupeds, which all
foreigners now class among the chief wonders of London, were brought
from the marshes of Walcheren; the ancestors of Childers and Eclipse
from the sands of Arabia. Already, however, there was among our nobility
and gentry a passion for the amusements of the turf. The importance of
improving our studs by an infusion of new blood was strongly felt; and
with this view a considerable number of barbs had lately been brought
into the country. Two men whose authority on such subjects was held in
great esteem, the Duke of Newcastle and Sir John Fenwick, pronounced
that the meanest hack ever imported from Tangier would produce a diner
progeny than could be expected from the best sire of our native breed.
They would not readily have believed that a time would come when the
princes and nobles of neighbouring lands would be as eager to obtain
horses from England as ever the English had been to obtain horses from
Barbary. 
%[70]
\footnote{ King and Davenant as before The Duke of Newcastle on
Horsemanship; Gentleman's Recreation, 1686. The "dappled Flanders mares"
were marks of greatness in the time of Pope, and even later. The vulgar
proverb, that the grey mare is the better horse, originated, I suspect,
in the preference generally given to the grey mares of Flanders over the
finest coach horses of England.}


The increase of vegetable and animal produce, though great, seems small
when compared with the increase of our mineral wealth. In 1685 the tin
of Cornwall, which had, more than two thousand years before, attracted
the Tyrian sails beyond the pillars of Hercules, was still one of the
most valuable subterranean productions of the island. The quantity
annually extracted from the earth was found to be, some years later,
sixteen hundred tons, probably about a third of what it now is. 
%[71]
\footnote{ See a curious note by Tonkin, in Lord De Dunstanville's
edition of Carew's Survey of Cornwall.}
 But
the veins of copper which lie in the same region were, in the time of
Charles the Second, altogether neglected, nor did any landowner take
them into the account in estimating the value of his property. Cornwall
and Wales at present yield annually near fifteen thousand tons of
copper, worth near a million and a half sterling; that is to say, worth
about twice as much as the annual produce of all English mines of all
descriptions in the seventeenth century. 
%[72]
\footnote{ Borlase's Natural History of Cornwall, 1758. The quantity
of copper now produced, I have taken from parliamentary returns.
Davenant, in 1700, estimated the annual produce of all the mines of
England at between seven and eight hundred thousand pounds}
 The first bed of rock salt
had been discovered in Cheshire not long after the Restoration, but
does not appear to have been worked till much later. The salt which
was obtained by a rude process from brine pits was held in no high
estimation. The pans in which the manufacture was carried on exhaled a
sulphurous stench; and, when the evaporation was complete, the substance
which was left was scarcely fit to be used with food. Physicians
attributed the scorbutic and pulmonary complaints which were common
among the English to this unwholesome condiment. It was therefore
seldom used by the upper and middle classes; and there was a regular and
considerable importation from France. At present our springs and mines
not only supply our own immense demand, but send annually more than
seven hundred millions of pounds of excellent salt to foreign countries.

%[73]
\footnote{ Philosophical Transactions, No. 53. Nov. 1669, No. 66.
Dec. 1670, No. 103. May 1674, No 156. Feb. 1683-4}


Far more important has been the improvement of our iron works. Such
works had long existed in our island, but had not prospered, and had
been regarded with no favourable eye by the government and by the
public. It was not then the practice to employ coal for smelting the
ore; and the rapid consumption of wood excited the alarm of politicians.
As early as the reign of Elizabeth, there had been loud complaints that
whole forests were cut down for the purpose of feeding the furnaces; and
the Parliament had interfered to prohibit the manufacturers from burning
timber. The manufacture consequently languished. At the close of the
reign of Charles the Second, great part of the iron which was used in
this country was imported from abroad; and the whole quantity cast here
annually seems not to have exceeded ten thousand tons. At present the
trade is thought to be in a depressed state if less than a million of
tons are produced in a year. 
%[74]
\footnote{ Yarranton, England's Improvement by Sea and Land, 1677;
Porter's Progress of the Nation. See also a remarkably perspicnous
history, in small compass of the English iron works, in Mr. M'Culloch's
Statistical Account of the British Empire.}


One mineral, perhaps more important than iron itself, remains to be
mentioned. Coal, though very little used in any species of manufacture,
was already the ordinary fuel in some districts which were fortunate
enough to possess large beds, and in the capital, which could easily be
supplied by water carriage, It seems reasonable to believe that at least
one half of the quantity then extracted from the pits was consumed in
London. The consumption of London seemed to the writers of that age
enormous, and was often mentioned by them as a proof of the greatness of
the imperial city. They scarcely hoped to be believed when they affirmed
that two hundred and eighty thousand chaldrons that is to say, about
three hundred and fifty thousand tons, were, in the last year of the
reign of Charles the Second, brought to the Thames. At present three
millions and a half of tons are required yearly by the metropolis; and
the whole annual produce cannot, on the most moderate computation, be
estimated at less than thirty millions of tons. 
%[75]
\footnote{ See Chamberlayne's State of England, 1684, 1687, Angliae,
Metropolis, 1691; M'Culloch's Statistical Account of the British Empire
Part III. chap. ii. (edition of 1847). In 1845 the quantity of coal
brought into London appeared, by the Parliamentary returns, to be
3,460,000 tons. (1848.) In 1854 the quantity of coal brought into London
amounted to 4,378,000 tons. (1857.)}


While these great changes have been in progress, the rent of land has,
as might be expected, been almost constantly rising. In some districts
it has multiplied more than tenfold. In some it has not more than
doubled. It has probably, on the average, quadrupled.

Of the rent, a large proportion was divided among the country gentlemen,
a class of persons whose position and character it is most important
that we should clearly understand; for by their influence and by their
passions the fate of the nation was, at several important conjunctures,
determined.

We should be much mistaken if we pictured to ourselves the squires of
the seventeenth century as men bearing a close resemblance to their
descendants, the county members and chairmen of quarter sessions with
whom we are familiar. The modern country gentleman generally receives a
liberal education, passes from a distinguished school to a distinguished
college, and has ample opportunity to become an excellent scholar. He
has generally seen something of foreign countries. A considerable
part of his life has generally been passed in the capital; and the
refinements of the capital follow him into the country. There is perhaps
no class of dwellings so pleasing as the rural seats of the English
gentry. In the parks and pleasure grounds, nature, dressed yet not
disguised by art, wears her most alluring form. In the buildings, good
sense and good taste combine to produce a happy union of the comfortable
and the graceful. The pictures, the musical instruments, the library,
would in any other country be considered as proving the owner to be
an eminently polished and accomplished man. A country gentleman who
witnessed the Revolution was probably in receipt of about a fourth
part of the rent which his acres now yield to his posterity. He was,
therefore, as compared with his posterity, a poor man, and was generally
under the necessity of residing, with little interruption, on his
estate. To travel on the Continent, to maintain an establishment in
London, or even to visit London frequently, were pleasures in which only
the great proprietors could indulge. It may be confidently affirmed that
of the squires whose names were then in the Commissions of Peace and
Lieutenancy not one in twenty went to town once in five years, or had
ever in his life wandered so far as Paris. Many lords of manors had
received an education differing little from that of their menial
servants. The heir of an estate often passed his boyhood and youth
at the seat of his family with no better tutors than grooms and
gamekeepers, and scarce attained learning enough to sign his name to
a Mittimus. If he went to school and to college, he generally returned
before he was twenty to the seclusion of the old hall, and there,
unless his mind were very happily constituted by nature, soon forgot his
academical pursuits in rural business and pleasures. His chief serious
employment was the care of his property. He examined samples of grain,
handled pigs, and, on market days, made bargains over a tankard with
drovers and hop merchants. His chief pleasures were commonly derived
from field sports and from an unrefined sensuality. His language and
pronunciation were such as we should now expect to hear only from the
most ignorant clowns. His oaths, coarse jests, and scurrilous terms of
abuse, were uttered with the broadest accent of his province. It was
easy to discern, from the first words which he spoke, whether he came
from Somersetshire or Yorkshire. He troubled himself little about
decorating his abode, and, if he attempted decoration, seldom produced
anything but deformity. The litter of a farmyard gathered under the
windows of his bedchamber, and the cabbages and gooseberry bushes grew
close to his hall door. His table was loaded with coarse plenty; and
guests were cordially welcomed to it. But, as the habit of drinking to
excess was general in the class to which he belonged, and as his fortune
did not enable him to intoxicate large assemblies daily with claret
or canary, strong beer was the ordinary beverage. The quantity of beer
consumed in those days was indeed enormous. For beer then was to the
middle and lower classes, not only all that beer is, but all that wine,
tea, and ardent spirits now are. It was only at great houses, or on
great occasions, that foreign drink was placed on the board. The ladies
of the house, whose business it had commonly been to cook the repast,
retired as soon as the dishes had been devoured, and left the gentlemen
to their ale and tobacco. The coarse jollity of the afternoon was often
prolonged till the revellers were laid under the table.

It was very seldom that the country gentleman caught glimpses of the
great world; and what he saw of it tended rather to confuse than
to enlighten his understanding. His opinions respecting religion,
government, foreign countries and former times, having been derived,
not from study, from observation, or from conversation with enlightened
companions, but from such traditions as were current in his own small
circle, were the opinions of a child. He adhered to them, however, with
the obstinacy which is generally found in ignorant men accustomed to be
fed with flattery. His animosities were numerous and bitter. He
hated Frenchmen and Italians, Scotchmen and Irishmen, Papists and
Presbyterians, Independents and Baptists, Quakers and Jews. Towards
London and Londoners he felt an aversion which more than once produced
important political effects. His wife and daughter were in tastes and
acquirements below a housekeeper or a stillroom maid of the present day.
They stitched and spun, brewed gooseberry wine, cured marigolds, and
made the crust for the venison pasty.

From this description it might be supposed that the English esquire of
the seventeenth century did not materially differ from a rustic miller
or alehouse keeper of our time. There are, however, some important
parts of his character still to be noted, which will greatly modify this
estimate. Unlettered as he was and unpolished, he was still in some most
important points a gentleman. He was a member of a proud and powerful
aristocracy, and was distinguished by many both of the good and of the
bad qualities which belong to aristocrats. His family pride was beyond
that of a Talbot or a Howard. He knew the genealogies and coats of
arms of all his neighbours, and could tell which of them had assumed
supporters without any right, and which of them were so unfortunate as
to be greatgrandsons of aldermen. He was a magistrate, and, as
such, administered gratuitously to those who dwelt around him a rude
patriarchal justice, which, in spite of innumerable blunders and of
occasional acts of tyranny, was yet better than no justice at all. He
was an officer of the trainbands; and his military dignity, though it
might move the mirth of gallants who had served a campaign in Flanders,
raised his character in his own eyes and in the eyes of his neighbours.
Nor indeed was his soldiership justly a subject of derision. In every
county there were elderly gentlemen who had seen service which was no
child's play. One had been knighted by Charles the First, after the
battle of Edgehill. Another still wore a patch over the scar which he
had received at Naseby. A third had defended his old house till
Fairfax had blown in the door with a petard. The presence of these
old Cavaliers, with their old swords and holsters, and with their old
stories about Goring and Lunsford, gave to the musters of militia an
earnest and warlike aspect which would otherwise have been wanting. Even
those country gentlemen who were too young to have themselves exchanged
blows with the cuirassiers of the Parliament had, from childhood, been
surrounded by the traces of recent war, and fed with stories of the
martial exploits of their fathers and uncles. Thus the character of
the English esquire of the seventeenth century was compounded of
two elements which we seldom or never find united. His ignorance and
uncouthness, his low tastes and gross phrases, would, in our time, be
considered as indicating a nature and a breeding thoroughly plebeian.
Yet he was essentially a patrician, and had, in large measure both the
virtues and the vices which flourish among men set from their birth
in high place, and used to respect themselves and to be respected by
others. It is not easy for a generation accustomed to find chivalrous
sentiments only in company with liberal Studies and polished manners
to image to itself a man with the deportment, the vocabulary, and
the accent of a carter, yet punctilious on matters of genealogy and
precedence, and ready to risk his life rather than see a stain cast on
the honour of his house. It is however only by thus joining together
things seldom or never found together in our own experience, that we can
form a just idea of that rustic aristocracy which constituted the main
strength of the armies of Charles the First, and which long supported,
with strange fidelity, the interest of his descendants.

The gross, uneducated; untravelled country gentleman was commonly a
Tory; but, though devotedly attached to hereditary monarchy, he had no
partiality for courtiers and ministers. He thought, not without reason,
that Whitehall was filled with the most corrupt of mankind, and that of
the great sums which the House of Commons had voted to the crown since
the Restoration part had been embezzled by cunning politicians, and part
squandered on buffoons and foreign courtesans. His stout English heart
swelled with indignation at the thought that the government of his
country should be subject to French dictation. Being himself generally
an old Cavalier, or the son of an old Cavalier, he reflected with bitter
resentment on the ingratitude with which the Stuarts had requited their
best friends. Those who heard him grumble at the neglect with which he
was treated, and at the profusion with which wealth was lavished on the
bastards of Nell Gwynn and Madam Carwell, would have supposed him ripe
for rebellion. But all this ill humour lasted only till the throne was
really in danger. It was precisely when those whom the sovereign had
loaded with wealth and honours shrank from his side that the country
gentlemen, so surly and mutinous in the season of his prosperity,
rallied round him in a body. Thus, after murmuring twenty years at the
misgovernment of Charles the Second, they came to his rescue in his
extremity, when his own Secretaries of State and the Lords of his own
Treasury had deserted him, and enabled him to gain a complete victory
over the opposition; nor can there be any doubt that they would have
shown equal loyalty to his brother James, if James would, even at the
last moment, have refrained from outraging their strongest feeling. For
there was one institution, and one only, which they prized even more
than hereditary monarchy; and that institution was the Church of
England. Their love of the Church was not, indeed, the effect of study
or meditation. Few among them could have given any reason, drawn from
Scripture or ecclesiastical history, for adhering to her doctrines, her
ritual, and her polity; nor were they, as a class, by any means strict
observers of that code of morality which is common to all Christian
sects. But the experience of many ages proves that men may be ready to
fight to the death, and to persecute without pity, for a religion
whose creed they do not understand, and whose precepts they habitually
disobey. 
%[76]
\footnote{ My notion of the country gentleman of the seventeenth
century has been derived from sources too numerous to be recapitulated.
I must leave my description to the judgment of those who have studied
the history and the lighter literature of that age.}


The rural clergy were even more vehement in Toryism than the rural
gentry, end were a class scarcely less important. It is to be observed,
however, that the individual clergyman, as compared with the individual
gentleman, then ranked much lower than in our days. The main support of
the Church was derived from the tithe; and the tithe bore to the rent a
much smaller ratio than at present. King estimated the whole income
of the parochial and collegiate clergy at only four hundred and eighty
thousand pounds a year; Davenant at only five hundred and forty-four
thousand a year. It is certainly now more than seven times as great
as the larger of these two sums. The average rent of the land has not,
according to any estimate, increased proportionally. It follows that
the rectors and vicars must have been, as compared with the neighbouring
knights and squires, much poorer in the seventeenth than in the
nineteenth century.

The place of the clergyman in society had been completely changed by the
Reformation. Before that event, ecclesiastics had formed the majority
of the House of Lords, had, in wealth and splendour, equalled, and
sometimes outshone, the greatest of the temporal barons, and had
generally held the highest civil offices. Many of the Treasurers, and
almost all the Chancellors of the Plantagenets were Bishops. The Lord
Keeper of the Privy Seal and the Master of the Rolls were ordinarily
churchmen. Churchmen transacted the most important diplomatic business.
Indeed all that large portion of the administration which rude and
warlike nobles were incompetent to conduct was considered as especially
belonging to divines. Men, therefore, who were averse to the life of
camps, and who were, at the same time, desirous to rise in the state,
commonly received the tonsure. Among them were sons of all the most
illustrious families, and near kinsmen of the throne, Scroops and
Nevilles, Bourchiers, Staffords and Poles. To the religious houses
belonged the rents of immense domains, and all that large portion of
the tithe which is now in the hands of laymen. Down to the middle of the
reign of Henry the Eighth, therefore, no line of life was so attractive
to ambitious and covetous natures as the priesthood. Then came a violent
revolution. The abolition of the monasteries deprived the Church at once
of the greater part of her wealth, and of her predominance in the Upper
House of Parliament. There was no longer an Abbot of Glastonbury or
an Abbot of Reading, seated among the peers, and possessed of revenues
equal to those of a powerful Earl. The princely splendour of William of
Wykeham and of William of Waynflete had disappeared. The scarlet hat of
the Cardinal, the silver cross of the Legate, were no more. The clergy
had also lost the ascendency which is the natural reward of superior
mental cultivation. Once the circumstance that a man could read had
raised a presumption that he was in orders. But, in an age which
produced such laymen as William Cecil and Nicholas Bacon, Roger Ascham
and Thomas Smith, Walter Mildmay and Francis Walsingham, there was
no reason for calling away prelates from their dioceses to negotiate
treaties, to superintend the finances, or to administer justice. The
spiritual character not only ceased to be a qualification for high civil
office, but began to be regarded as a disqualification. Those worldly
motives, therefore, which had formerly induced so many able, aspiring,
and high born youths to assume the ecclesiastical habit, ceased to
operate. Not one parish in two hundred then afforded what a man of
family considered as a maintenance. There were still indeed prizes in
the Church: but they were few; and even the highest were mean, when
compared with the glory which had once surrounded the princes of the
hierarchy. The state kept by Parker and Grindal seemed beggarly to
those who remembered the imperial pomp of Wolsey, his palaces, which had
become the favorite abodes of royalty, Whitehall and Hampton Court, the
three sumptuous tables daily spread in his refectory, the forty-four
gorgeous copes in his chapel, his running footmen in rich liveries, and
his body guards with gilded poleaxes. Thus the sacerdotal office lost
its attraction for the higher classes. During the century which followed
the accession of Elizabeth, scarce a single person of noble descent took
orders. At the close of the reign of Charles the Second, two sons of
peers were Bishops; four or five sons of peers were priests, and held
valuable preferment: but these rare exceptions did not take away the
reproach which lay on the body. The clergy were regarded as, on the
whole, a plebeian class. 
%[77]
\footnote{ In the eighteenth century the great increase in the value
of benefices produced a change. The younger sons of the nobility were
allured back to the clerical profession. Warburton in a letter to
Hurd, dated the 6th of July, 1762, mentions this change, which was then
recent. "Our grandees have at last found their way back into the Church.
I only wonder they have been so long about it. But be assured that
nothing but a new religious revolution, to sweep away the fragments that
Henry the Eighth left after banqueting his courtiers, will drive them
out again."}
 And, indeed, for one who made the figure
of a gentleman, ten were mere menial servants. A large proportion of
those divines who had no benefices, or whose benefices were too small to
afford a comfortable revenue, lived in the houses of laymen. It had
long been evident that this practice tended to degrade the priestly
character. Laud had exerted himself to effect a change; and Charles the
First had repeatedly issued positive orders that none but men of
high rank should presume to keep domestic chaplains. 
%[78]
\footnote{ See Heylin's Cyprianus Anglicus.}
 But these
injunctions had become obsolete. Indeed during the domination of the
Puritan, many of the ejected ministers of the Church of England could
obtain bread and shelter only by attaching themselves to the households
of royalist gentlemen; and the habits which had been formed in those
times of trouble continued long after the reestablishment of monarchy
and episcopacy. In the mansions of men of liberal sentiments and
cultivated understandings, the chaplain was doubtless treated with
urbanity and kindness. His conversation, his literary assistance, his
spiritual advice, were considered as an ample return for his food, his
lodging, and his stipend. But this was not the general feeling of the
country gentlemen. The coarse and ignorant squire, who thought that it
belonged to his dignity to have grace said every day at his table by an
ecclesiastic in full canonicals, found means to reconcile dignity with
economy. A young Levite--such was the phrase then in use--might be had
for his board, a small garret, and ten pounds a year, and might not
only perform his own professional functions, might not only be the most
patient of butts and of listeners, might not only be always ready in
fine weather for bowls, and in rainy weather for shovelboard, but
might also save the expense of a gardener, or of a groom. Sometimes the
reverend man nailed up the apricots; and sometimes he curried the coach
horses. He cast up the farrier's bills. He walked ten miles with a
message or a parcel. He was permitted to dine with the family; but he
was expected to content himself with the plainest fare. He might fill
himself with the corned beef and the carrots: but, as soon as the tarts
and cheesecakes made their appearance, he quitted his seat, and stood
aloof till he was summoned to return thanks for the repast, from a great
part of which he had been excluded. 
%[79]
\footnote{ Eachard, Causes of the Contempt of the Clergy; Oldham,
Satire addressed to a Friend about to leave the University; Tatler, 255,
258. That the English clergy were a lowborn class, is remarked in the
Travels of the Grand Duke Cosmo, Appendix A.}


Perhaps, after some years of service, he was presented to a living
sufficient to support him; but he often found it necessary to purchase
his preferment by a species of Simony, which furnished an inexhaustible
subject of pleasantry to three or four generations of scoffers. With his
cure he was expected to take a wife. The wife had ordinarily been in the
patron's service; and it was well if she was not suspected of standing
too high in the patron's favor. Indeed the nature of the matrimonial
connections which the clergymen of that age were in the habit of forming
is the most certain indication of the place which the order held in
the social system. An Oxonian, writing a few months after the death
of Charles the Second, complained bitterly, not only that the country
attorney and the country apothecary looked down with disdain on the
country clergyman but that one of the lessons most earnestly inculcated
on every girl of honourable family was to give no encouragement to a
lover in orders, and that, if any young lady forgot this precept, she
was almost as much disgraced as by an illicit amour. 
%[80]
\footnote{ "A causidico, medicastro, ipsaque artificum farragine,
ecclesiae rector aut vicarius contemnitur et fit ludibrio. Gentis et
familiae nitor sacris ordinibus pollutus censetur: foeminisque natalitio
insignibus unicum inculcatur saepius praeceptum, ne modestiae naufragium
faciant, aut, (quod idem auribus tam delicatulis sonat,) ne clerico se
nuptas dari patiantur."--Angliae Notitia, by T. Wood, of New College
Oxford 1686.}
 Clarendon, who
assuredly bore no ill will to the priesthood, mentions it as a sign of
the confusion of ranks which the great rebellion had produced, that some
damsels of noble families had bestowed themselves on divines. 
%[81]
\footnote{ Clarendon's Life, ii. 21.}
 A
waiting woman was generally considered as the most suitable helpmate for
a parson. Queen Elizabeth, as head of the Church, had given what seemed
to be a formal sanction to this prejudice, by issuing special orders
that no clergyman should presume to espouse a servant girl, without
the consent of the master or mistress. 
%[82]
\footnote{ See the injunctions of 1559, In Bishop Sparrow's
Collection. Jeremy Collier, in his Essay on Pride, speaks of this
injunction with a bitterness which proves that his own pride had not
been effectually tamed.}
 During several generations
accordingly the relation between divines and handmaidens was a theme
for endless jest; nor would it be easy to find, in the comedy of the
seventeenth century, a single instance of a clergyman who wins a spouse
above the rank of cook. 
%[83]
\footnote{ Roger and Abigail in Fletcher's Scornful Lady, Bull
and the Nurse in Vanbrugh's Relapse, Smirk and Susan in Shadwell's
Lancashire Witches, are instances.}
 Even so late as the time of George the
Second, the keenest of all observers of life and manners, himself
a priest, remarked that, in a great household, the chaplain was the
resource of a lady's maid whose character had been blown upon, and who
was therefore forced to give up hopes of catching the steward. 
%[84]
\footnote{ Swift's Directions to Servants. In Swift's Remarks on the
Clerical Residence Bill, he describes the family of an English vicar
thus: "His wife is little better than a Goody, in her birth, education,
or dress..... His daughters shall go to service, or be sent apprentice
to the sempstress of the next town."}


In general the divine who quitted his chaplainship for a benefice and
a wife found that he had only exchanged one class of vexations for
another. Hardly one living in fifty enabled the incumbent to bring up
a family comfortably. As children multiplied end grew, the household of
the priest became more and more beggarly. Holes appeared more and more
plainly in the thatch of his parsonage and in his single cassock. Often
it was only by toiling on his glebe, by feeding swine, and by loading
dungcarts, that he could obtain daily bread; nor did his utmost
exertions always prevent the bailiffs from taking his concordance and
his inkstand in execution. It was a white day on which he was admitted
into the kitchen of a great house, and regaled by the servants with
cold meat and ale. His children were brought up like the children of the
neighbouring peasantry. His boys followed the plough; and his girls went
out to service. 
%[85]
\footnote{ Even in Tom Jones, published two generations later. Mrs.
Seagrim, the wife of a gamekeeper, and Mrs. Honour, a waitingwoman,
boast of their descent from clergymen, "It is to be hoped," says
Fielding, "such instances will in future ages, when some provision is
made for the families of the inferior clergy, appear stranger than they
can be thought at present."}
 Study he found impossible: for the advowson of his
living would hardly have sold for a sum sufficient to purchase a good
theological library; and he might be considered as unusually lucky if
he had ten or twelve dogeared volumes among the pots and pans on his
shelves. Even a keen and strong intellect might be expected to rust in
so unfavourable a situation.

Assuredly there was at that time no lack in the English Church of
ministers distinguished by abilities and learning But it is to be
observed that these ministers were not scattered among the rural
population. They were brought together at a few places where the means
of acquiring knowledge were abundant, and where the opportunities of
vigorous intellectual exercise were frequent. 
%[86]
\footnote{ This distinction between country clergy and town clergy is
strongly marked by Eachard, and cannot but be observed by every person
who has studied the ecclesiastical history of that age.}
 At such places were
to be found divines qualified by parts, by eloquence, by wide knowledge
of literature, of science, and of life, to defend their Church
victoriously against heretics and sceptics, to command the attention
of frivolous and worldly congregations, to guide the deliberations of
senates, and to make religion respectable, even in the most dissolute
of courts. Some laboured to fathom the abysses of metaphysical theology:
some were deeply versed in biblical criticism; and some threw light
on the darkest parts of ecclesiastical history. Some proved themselves
consummate masters of logic. Some cultivated rhetoric with such
assiduity and success that their discourses are still justly valued as
models of style. These eminent men were to be found, with scarcely a
single exception, at the Universities, at the great Cathedrals, or in
the capital. Barrow had lately died at Cambridge; and Pearson had gone
thence to the episcopal bench. Cudworth and Henry More were still living
there. South and Pococke, Jane and Aldrich, were at Oxford, Prideaux was
in the close of Norwich, and Whitby in the close of Salisbury. But it
was chiefly by the London clergy, who were always spoken of as a class
apart, that the fame of their profession for learning and eloquence was
upheld. The principal pulpits of the metropolis were occupied about this
time by a crowd of distinguished men, from among whom was selected a
large proportion of the rulers of the Church. Sherlock preached at the
Temple, Tillotson at Lincoln's Inn, Wake and Jeremy Collier at Gray's
Inn, Burnet at the Rolls, Stillingfleet at Saint Paul's Cathedral,
Patrick at Saint Paul's in Covent Garden, Fowler at Saint Giles's,
Cripplegate, Sharp at Saint Giles's in the Fields, Tenison at Saint
Martin's, Sprat at Saint Margaret's, Beveridge at Saint Peter's in
Cornhill. Of these twelve men, all of high note in ecclesiastical
history, ten became Bishops, and four Archbishops. Meanwhile almost the
only important theological works which came forth from a rural parsonage
were those of George Bull, afterwards Bishop of Saint David's; and Bull
never would have produced those works, had he not inherited an estate,
by the sale of which he was enabled to collect a library, such as
probably no other country clergyman in England possessed. 
%[87]
\footnote{ Nelson's Life of Bull. As to the extreme difficulty which
the country clergy found in procuring books, see the Life of Thomas
Bray, the founder of the Society for the Propagation of the Gospel.}


Thus the Anglican priesthood was divided into two sections, which, in
acquirements, in manners, and in social position, differed widely from
each other. One section, trained for cities and courts, comprised men
familiar with all ancient and modern learning; men able to encounter
Hobbes or Bossuet at all the weapons of controversy; men who could, in
their sermons, set forth the majesty and beauty of Christianity with
such justness of thought, and such energy of language, that the indolent
Charles roused himself to listen and the fastidious Buckingham forgot
to sneer; men whose address, politeness, and knowledge of the world
qualified them to manage the consciences of the wealthy and noble; men
with whom Halifax loved to discuss the interests of empires, and from
whom Dryden was not ashamed to own that he had learned to write. 
%[88]
\footnote{ "I have frequently heard him (Dryden) own with pleasure,
that if he had any talent for English prose it was owing to his having
often read the writings of the great Archbishop Tillotson."--Congreve's
Dedication of Dryden's Plays.}

The other section was destined to ruder and humbler service. It was
dispersed over the country, and consisted chiefly of persons not at
all wealthier, and not much more refined, than small farmers or upper
servants. Yet it was in these rustic priests, who derived but a scanty
subsistence from their tithe sheaves and tithe pigs, and who had not the
smallest chance of ever attaining high professional honours, that the
professional spirit was strongest. Among those divines who were the
boast of the Universities and the delight of the capital, and who had
attained, or might reasonably expect to attain, opulence and lordly
rank, a party, respectable in numbers, and more respectable in
character, leaned towards constitutional principles of government, lived
on friendly terms with Presbyterians, Independents, and Baptists, would
gladly have seen a full toleration granted to all Protestant sects, and
would even have consented to make alterations in the Liturgy, for the
purpose of conciliating honest and candid Nonconformists. But such
latitudinarianism was held in horror by the country parson. He took,
indeed, more pride in his ragged gown than his superiors in their lawn
and their scarlet hoods. The very consciousness that there was little in
his worldly circumstances to distinguish him from the villagers to
whom he preached led him to hold immoderately high the dignity of that
sacerdotal office which was his single title to reverence. Having
lived in seclusion, and having had little opportunity of correcting his
opinions by reading or conversation, he held and taught the doctrines
of indefeasible hereditary right, of passive obedience, and of
nonresistance, in all their crude absurdity. Having been long engaged in
a petty war against the neighbouring dissenters, he too often hated them
for the wrong which he had done them, and found no fault with the Five
Mile Act and the Conventicle Act, except that those odious laws had not
a sharper edge. Whatever influence his office gave him was exerted with
passionate zeal on the Tory side; and that influence was immense. It
would be a great error to imagine, because the country rector was in
general not regarded as a gentleman, because he could not dare to aspire
to the hand of one of the young ladies at the manor house, because he
was not asked into the parlours of the great, but was left to drink and
smoke with grooms and butlers, that the power of the clerical body
was smaller than at present. The influence of a class is by no means
proportioned to the consideration which the members of that class
enjoy in their individual capacity. A Cardinal is a much more exalted
personage than a begging friar: but it would be a grievous mistake to
suppose that the College of Cardinals has exercised greater dominion
over the public mind of Europe than the Order of Saint Francis. In
Ireland, at present, a peer holds a far higher station in society than
a Roman Catholic priest: yet there are in Munster and Connaught few
counties where a combination of priests would not carry an election
against a combination of peers. In the seventeenth century the pulpit
was to a large portion of the population what the periodical press now
is. Scarce any of the clowns who came to the parish church ever saw a
Gazette or a political pamphlet. Ill informed as their spiritual pastor
might be, he was yet better informed than themselves: he had every
week an opportunity of haranguing them; and his harangues were never
answered. At every important conjuncture, invectives against the Whigs
and exhortations to obey the Lord's anointed resounded at once from many
thousands of pulpits; and the effect was formidable indeed. Of all the
causes which, after the dissolution of the Oxford Parliament, produced
the violent reaction against the Exclusionists, the most potent seems to
have been the oratory of the country clergy.

The power which the country gentleman and the country clergyman
exercised in the rural districts was in some measure counterbalanced by
the power of the yeomanry, an eminently manly and truehearted race. The
petty proprietors who cultivated their own fields with their own hands,
and enjoyed a modest competence, without affecting to have scutcheons
and crests, or aspiring to sit on the bench of justice, then formed a
much more important part of the nation than at present. If we may trust
the best statistical writers of that age, not less than a hundred and
sixty thousand proprietors, who with their families must have made up
more than a seventh of the whole population, derived their subsistence
from little freehold estates. The average income of these small
landholders, an income mace up of rent, profit, and wages, was estimated
at between sixty and seventy pounds a year. It was computed that the
number of persons who tilled their own land was greater than the number
of those who farmed the land of others. 
%[89]
\footnote{ I have taken Davenant's estimate, which is a little lower
than King's.}
 A large portion of
the yeomanry had, from the time of the Reformation, leaned towards
Puritanism, had, in the civil war, taken the side of the Parliament,
had, after the Restoration, persisted in hearing Presbyterian and
Independent preachers, had, at elections, strenuously supported the
Exclusionists and had continued even after the discovery of the Rye
House plot and the proscription of the Whig leaders, to regard Popery
and arbitrary power with unmitigated hostility.

Great as has been the change in the rural life of England since the
Revolution, the change which has come to pass in the cities is still
more amazing. At present above a sixth part of the nation is crowded
into provincial towns of more than thirty thousand inhabitants. In the
reign of Charles the second no provincial town in the kingdom contained
thirty thousand inhabitants; and only four provincial towns contained so
many as ten thousand inhabitants.

Next to the capital, but next at an immense distance, stood Bristol,
then the first English seaport, and Norwich, then the first English
manufacturing town. Both have since that time been far outstripped
by younger rivals; yet both have made great positive advances. The
population of Bristol has quadrupled. The population of Norwich has more
than doubled.

Pepys, who visited Bristol eight years after the Restoration, was struck
by the splendour of the city. But his standard was not high; for he
noted down as a wonder the circumstance that, in Bristol, a man might
look round him and see nothing but houses. It seems that, in no other
place with which he was acquainted, except London, did the buildings
completely shut out the woods and fields. Large as Bristol might then
appear, it occupied but a very small portion of the area on which it
now stands. A few churches of eminent beauty rose out of a labyrinth
of narrow lanes built upon vaults of no great solidity. If a coach or
a cart entered those alleys, there was danger that it would be wedged
between the houses, and danger also that it would break in the cellars.
Goods were therefore conveyed about the town almost exclusively in
trucks drawn by dogs; and the richest inhabitants exhibited their
wealth, not by riding in gilded carriages, but by walking the streets
with trains of servants in rich liveries, and by keeping tables loaded
with good cheer. The pomp of the christenings and burials far exceeded
what was seen at any other place in England. The hospitality of the city
was widely renowned, and especially the collations with which the sugar
refiners regaled their visitors. The repast was dressed in the furnace,
and was accompanied by a rich beverage made of the best Spanish wine,
and celebrated over the whole kingdom as Bristol milk. This luxury was
supported by a thriving trade with the North American plantations and
with the West Indies. The passion for colonial traffic was so strong
that there was scarcely a small shopkeeper in Bristol who had not a
venture on board of some ship bound for Virginia or the Antilles. Some
of these ventures indeed were not of the most honourable kind. There
was, in the Transatlantic possessions of the crown, a great demand for
labour; and this demand was partly supplied by a system of crimping and
kidnapping at the principal English seaports. Nowhere was this system
in such active and extensive operation as at Bristol. Even the first
magistrates of that city were not ashamed to enrich themselves by so
odious a commerce. The number of houses appears, from the returns of the
hearth money, to have been in the year 1685, just five thousand three
hundred. We can hardly suppose the number of persons in a house to have
been greater than in the city of London; and in the city of London we
learn from the best authority that there were then fifty-five persons
to ten houses. The population of Bristol must therefore have been about
twenty-nine thousand souls. 
%[90]
\footnote{ Evelvn's Diary, June 27. 1654; Pepys's Diary, June 13.
1668; Roger North's Lives of Lord Keeper Guildford, and of Sir Dudley
North; Petty's Political Arithmetic. I have taken Petty's facts, but, in
drawing inferences from them, I have been guided by King and Davenant,
who, though not abler men than he, had the advantage of coming after
him. As to the kidnapping for which Bristol was infamous, see North's
Life of Guildford, 121, 216, and the harangue of Jeffreys on the
subject, in the Impartial History of his Life and Death, printed with
the Bloody Assizes. His style was, as usual, coarse, but I cannot reckon
the reprimand which he gave to the magistrates of Bristol among his
crimes.}


Norwich was the capital of a large and fruitful province. It was the
residence of a Bishop and of a Chapter. It was the chief seat of the
chief manufacture of the realm. Some men distinguished by learning and
science had recently dwelt there and no place in the kingdom, except the
capital and the Universities, had more attractions for the curious. The
library, the museum, the aviary, and the botanical garden of Sir Thomas
Browne, were thought by Fellows of the Royal Society well worthy of a
long pilgrimage. Norwich had also a court in miniature. In the heart
of the city stood an old palace of the Dukes of Norfolk, said to be the
largest town house in the kingdom out of London. In this mansion, to
which were annexed a tennis court, a bowling green, and a wilderness
stretching along the banks of the Wansum, the noble family of
Howard frequently resided, and kept a state resembling that of petty
sovereigns. Drink was served to guests in goblets of pure gold. The very
tongs and shovels were of silver. Pictures by Italian masters adorned
the walls. The cabinets were filled with a fine collection of gems
purchased by that Earl of Arundel whose marbles are now among the
ornaments of Oxford. Here, in the year 1671, Charles and his court were
sumptuously entertained. Here, too, all comers were annually welcomed,
from Christmas to Twelfth Night. Ale flowed in oceans for the populace.
Three coaches, one of which had been built at a cost of five hundred
pounds to contain fourteen persons, were sent every afternoon round
the city to bring ladies to the festivities; and the dances were always
followed by a luxurious banquet. When the Duke of Norfolk came to
Norwich, he was greeted like a King returning to his capital. The bells
of the Cathedral and of St. Peter Mancroft were rung: the guns of
the castle were fired; and the Mayor and Aldermen waited on their
illustrious fellow citizen with complimentary addresses. In the year
1693 the population of Norwich was found by actual enumeration, to be
between twenty-eight and twenty-nine thousand souls. 
%[91]
\footnote{ Fuller's Worthies; Evelyn's Diary, Oct. 17,1671; Journal
of T. Browne, son of Sir Thomas Browne, Jan. 1663-4; Blomefield's
History of Norfolk; History of the City and County of Norwich, 2 vols.
1768.}


Far below Norwich, but still high in dignity and importance, were some
other ancient capitals of shires. In that age it was seldom that a
country gentleman went up with his family to London. The county town was
his metropolis. He sometimes made it his residence during part of the
year. At all events, he was often attracted thither by business and
pleasure, by assizes, quarter sessions, elections, musters of militia,
festivals, and races. There were the halls where the judges, robed
in scarlet and escorted by javelins and trumpets, opened the King's
commission twice a year. There were the markets at which the corn, the
cattle, the wool, and the hops of the surrounding country were exposed
to sale. There were the great fairs to which merchants came clown from
London, and where the rural dealer laid in his annual stores of sugar,
stationery, cutlery, and muslin. There were the shops at which the best
families of the neighbourhood bought grocery and millinery. Some of
these places derived dignity from interesting historical recollections,
from cathedrals decorated by all the art and magnificence of the middle
ages, from palaces where a long succession of prelates had dwelt, from
closes surrounded by the venerable abodes of deans and canons, and from
castles which had in the old time repelled the Nevilles or de Veres, and
which bore more recent traces of the vengeance of Rupert or of Cromwell.

Conspicuous amongst these interesting cities were York, the capital
of the north, and Exeter, the capital of the west. Neither can have
contained much more than ten thousand inhabitants. Worcester, the queen
of the cider land had but eight thousand; Nottingham probably as many.
Gloucester, renowned for that resolute defence which had been fatal to
Charles the First, had certainly between four and five thousand; Derby
not quite four thousand. Shrewsbury was the chief place of an extensive
and fertile district. The Court of the Marches of Wales was held there.
In the language of the gentry many miles round the Wrekin, to go to
Shrewsbury was to go to town. The provincial wits and beauties imitated,
as well as they could, the fashions of Saint James's Park, in the walks
along the side of the Severn. The inhabitants were about seven thousand.

%[92]
\footnote{ The population of York appears, from the return of
baptisms and burials in Drake's History, to have been about 13,000 in
1730. Exeter had only 17,000 inhabitants in 1801. The population of
Worcester was numbered just before the siege in 1646. See Nash's History
of Worcestershire. I have made allowance for the increase which must be
supposed to have taken place in forty years. In 1740, the population of
Nottingham was found, by enumeration, to be just 10,000. See Dering's
History. The population of Gloucester may readily be inferred from the
number of houses which King found in the returns of hearth money,
and from the number of births and burials which is given in Atkyns's
History. The population of Derby was 4,000 in 1712. See Wolley's MS.
History, quoted in Lyson's Magna Britannia. The population of Shrewsbury
was ascertained, in 1695, by actual enumeration. As to the gaieties of
Shrewsbury, see Farquhar's Recruiting Officer. Farquhar's description
is borne out by a ballad in the Pepysian Library, of which the burden is
"Shrewsbury for me."}


The population of every one of these places has, since the Revolution,
much more than doubled. The population of some has multiplied sevenfold.
The streets have been almost entirely rebuilt. Slate has succeeded to
thatch, and brick to timber. The pavements and the lamps, the display
of wealth in the principal shops, and the luxurious neatness of the
dwellings occupied by the gentry would, in the seventeenth century, have
seemed miraculous. Yet is the relative importance of the old capitals of
counties by no means what it was. Younger towns, towns which are
rarely or never mentioned in our early history and which sent no
representatives to our early Parliaments, have, within the memory
of persons still living, grown to a greatness which this generation
contemplates with wonder and pride, not unaccompanied by awe and
anxiety.

The most eminent of these towns were indeed known in the seventeenth
century as respectable seats of industry. Nay, their rapid progress
and their vast opulence were then sometimes described in language which
seems ludicrous to a man who has seen their present grandeur. One of the
most populous and prosperous among them was Manchester. Manchester
had been required by the Protector to send one representative to his
Parliament, and was mentioned by writers of the time of Charles the
Second as a busy and opulent place. Cotton had, during half a century,
been brought thither from Cyprus and Smyrna; but the manufacture was in
its infancy. Whitney had not yet taught how the raw material might be
furnished in quantities almost fabulous. Arkwright had not yet taught
how it might be worked up with a speed and precision which seem magical.
The whole annual import did not, at the end of the seventeenth century,
amount to two millions of pounds, a quantity which would now hardly
supply the demand of forty-eight hours. That wonderful emporium, which
in population and wealth far surpassed capitals so much renowned as
Berlin, Madrid, and Lisbon, was then a mean and ill built market town
containing under six thousand people. It then had not a single press. It
now supports a hundred printing establishments. It then had not a single
coach. It now Supports twenty coach makers. 
%[93]
\footnote{ Blome's Britannia, 1673; Aikin's Country round Manchester;
Manchester Directory, 1845: Baines, History of the Cotton Manufacture.
The best information which I have been able to find, touching the
population of Manchester in the seventeenth century is contained in
a paper drawn up by the Reverend R. Parkinson, and published in the
Journal of the Statistical Society for October 1842.}


Leeds was already the chief seat of the woollen manufactures of
Yorkshire; but the elderly inhabitants could still remember the time
when the first brick house, then and long after called the Red House,
was built. They boasted loudly of their increasing wealth, and of the
immense sales of cloth which took place in the open air on the bridge.
Hundreds, nay thousands of pounds, had been paid down in the course of
one busy market day. The rising importance of Leeds had attracted
the notice of successive governments. Charles the First had granted
municipal privileges to the town. Oliver had invited it to send one
member to the House of Commons. But from the returns of the hearth money
it seems certain that the whole population of the borough, an extensive
district which contains many hamlets, did not, in the reign of Charles
the Second, exceed seven thousand souls. In 1841 there were more than a
hundred and fifty thousand. 
%[94]
\footnote{ Thoresby's Ducatus Leodensis; Whitaker's Loidis and
Elmete; Wardell's Municipal History of the Borough of Leeds. (1848.) In
1851 Leeds had 172,000 Inhabitants. (1857.)}


About a day's journey south of Leeds, on the verge of a wild moorland
tract, lay an ancient manor, now rich with cultivation, then barren and
unenclosed, which was known by the name of Hallamshire. Iron abounded
there; and, from a very early period, the rude whittles fabricated there
had been sold all over the kingdom. They had indeed been mentioned by
Geoffrey Chaucer in one of his Canterbury Tales. But the manufacture
appears to have made little progress during the three centuries which
followed his time. This languor may perhaps be explained by the fact
that the trade was, during almost the whole of this long period, subject
to such regulations as the lord and his court feet thought fit to
impose. The more delicate kinds of cutlery were either made in the
capital or brought from the Continent. Indeed it was not till the reign
of George the First that the English surgeons ceased to import from
France those exquisitely fine blades which are required for operations
on the human frame. Most of the Hallamshire forges were collected in a
market town which had sprung up near the castle of the proprietor, and
which, in the reign of James the First, had been a singularly miserable
place, containing about two thousand inhabitants, of whom a third were
half starved and half naked beggars. It seems certain from the parochial
registers that the population did not amount to four thousand at the
end of the reign of Charles the Second. The effects of a species of toil
singularly unfavourable to the health and vigour of the human frame were
at once discerned by every traveller. A large proportion of the
people had distorted limbs. This is that Sheffield which now, with its
dependencies, contains a hundred and twenty thousand souls, and which
sends forth its admirable knives, razors, and lancets to the farthest
ends of the world. 
%[95]
\footnote{ Hunter's History of Hallamshire. (1848.) In 1851 the
population of Sheffield had increased to 135,000. (1857.)}


Birmingham had not been thought of sufficient importance to return a
member to Oliver's Parliament. Yet the manufacturers of Birmingham were
already a busy and thriving race. They boasted that their hardware was
highly esteemed, not indeed as now, at Pekin and Lima, at Bokhara and
Timbuctoo, but in London, and even as far off as Ireland. They had
acquired a less honourable renown as coiners of bad money. In allusion
to their spurious groats, some Tory wit had fixed on demagogues,
who hypocritically affected zeal against Popery, the nickname of
Birminghams. Yet in 1685 the population, which is now little less
than two hundred thousand, did not amount to four thousand. Birmingham
buttons were just beginning to be known: of Birmingham guns nobody had
yet heard; and the place whence, two generations later, the magnificent
editions of Baskerville went forth to astonish all the librarians
of Europe, did not contain a single regular shop where a Bible or an
almanack could be bought. On Market days a bookseller named Michael
Johnson, the father of the great Samuel Johnson, came over from
Lichfield, and opened stall during a few hours. This supply of
literature was long found equal to the demand. 
%[96]
\footnote{ Blome's Britannia, 1673; Dugdale's Warwickshire, North's
Examen, 321; Preface to Absalom and Achitophel; Hutton's History of
Birmingham; Boswell's Life of Johnson. In 1690 the burials at Birmingham
were 150, the baptisms 125. I think it probable that the annual
mortality was little less than one in twenty-five. In London it was
considerably greater. A historian of Nottingham, half a century later,
boasted of the extraordinary salubrity of his town, where the annual
mortality was one in thirty. See Doring's History of Nottingham. (1848.)
In 1851 the population of Birmingham had increased to 222,000. (1857.)}


These four chief seats of our great manufactures deserve especial
mention. It would be tedious to enumerate all the populous and opulent
hives of industry which, a hundred and fifty years ago, were hamlets
without parish churches, or desolate moors, inhabited only by grouse and
wild deer. Nor has the change been less signal in those outlets by which
the products of the English looms and forges are poured forth over
the whole world. At present Liverpool contains more than three hundred
thousand inhabitants. The shipping registered at her port amounts to
between four and five hundred thousand tons. Into her custom house has
been repeatedly paid in one year a sum more than thrice as great as
the whole income of the English crown in 1685. The receipts of her post
office, even since the great reduction of the duty, exceed the sum
which the postage of the whole kingdom yielded to the Duke of York. Her
endless docks, quays, and warehouses are among the wonders of the world.
Yet even those docks and quays and warehouses seem hardly to suffice for
the gigantic trade of the Mersey; and already a rival city is growing
fast on the opposite shore. In the days of Charles the Second Liverpool
was described as a rising town which had recently made great advances,
and which maintained a profitable intercourse with Ireland and with the
sugar colonies. The customs had multiplied eight-fold within sixteen
years, and amounted to what was then considered as the immense sum of
fifteen thousand pounds annually. But the population can hardly have
exceeded four thousand: the shipping was about fourteen hundred tons,
less than the tonnage of a single modern Indiaman of the first class,
and the whole number of seamen belonging to the port cannot be estimated
at more than two hundred. 
%[97]
\footnote{ Blome's Britannia; Gregson's Antiquities of the County
Palatine and Duchy of Lancaster, Part II.; Petition from Liverpool in
the Privy Council Book, May 10, 1686. In 1690 the burials at Liverpool
were 151, the baptisms 120. In 1844 the net receipt of the customs at
Liverpool was 4,366,526£. 1s. 8d. (1848.) In 1851 Liverpool contained
375,000 inhabitants, (1857.)}


Such has been the progress of those towns where wealth is created and
accumulated. Not less rapid has been the progress of towns of a
very different kind, towns in which wealth, created and accumulated
elsewhere, is expended for purposes of health and recreation. Some of
the most remarkable of these gay places have sprung into existence since
the time of the Stuarts. Cheltenham is now a greater city than any which
the kingdom contained in the seventeenth century, London alone excepted.
But in the seventeenth century, and at the beginning of the eighteenth,
Cheltenham was mentioned by local historians merely as a rural parish
lying under the Cotswold Hills, and affording good ground both for
tillage and pasture. Corn grew and cattle browsed over the space now
covered by that long succession of streets and villas. 
%[98]
\footnote{ Atkyne's Gloucestershire.}
 Brighton was
described as a place which had once been thriving, which had possessed
many small fishing barks, and which had, when at the height of
prosperity, contained above two thousand inhabitants, but which was
sinking fast into decay. The sea was gradually gaining on the buildings,
which at length almost entirely disappeared. Ninety years ago the ruins
of an old fort were to be seen lying among the pebbles and seaweed
on the beach; and ancient men could still point out the traces of
foundations on a spot where a street of more than a hundred huts had
been swallowed up by the waves. So desolate was the place after this
calamity, that the vicarage was thought scarcely worth having. A few
poor fishermen, however, still continued to dry their nets on those
cliffs, on which now a town, more than twice as large and populous
as the Bristol of the Stuarts, presents, mile after mile, its gay and
fantastic front to the sea. 
%[99]
\footnote{ Magna Britannia; Grose's Antiquities; New Brighthelmstone
Directory.}


England, however, was not, in the seventeenth century, destitute of
watering places. The gentry of Derbyshire and of the neighbouring
counties repaired to Buxton, where they were lodged in low rooms under
bare rafters, and regaled with oatcake, and with a viand which the hosts
called mutton, but which the guests suspected to be dog. A single good
house stood near the spring. 
%[100]
\footnote{ Tour in Derbyshire, by Thomas Browne, son of Sir Thomas.}
 Tunbridge Wells, lying within a
day's journey of the capital, and in one of the richest and most highly
civilised parts of the kingdom, had much greater attractions. At present
we see there a town which would, a hundred and sixty years ago, have
ranked, in population, fourth or fifth among the towns of England. The
brilliancy of the shops and the luxury of the private dwellings far
surpasses anything that England could then show. When the court, soon
after the Restoration, visited Tunbridge Wells, there was no town:
but, within a mile of the spring, rustic cottages, somewhat cleaner and
neater than the ordinary cottages of that time, were scattered over the
heath. Some of these cabins were movable and were carried on sledges
from one part of the common to another. To these huts men of fashion,
wearied with the din and smoke of London, sometimes came in the summer
to breathe fresh air, and to catch a glimpse of rural life. During the
season a kind of fair was daily held near the fountain. The wives and
daughters of the Kentish farmers came from the neighbouring villages
with cream, cherries, wheatears, and quails. To chaffer with them,
to flirt with them, to praise their straw hats and tight heels, was a
refreshing pastime to voluptuaries sick of the airs of actresses and
maids of honour. Milliners, toymen, and jewellers came down from London,
and opened a bazaar under the trees. In one booth the politician might
find his coffee and the London Gazette; in another were gamblers playing
deep at basset; and, on fine evenings, the fiddles were in attendance
and there were morris dances on the elastic turf of the bowling green.
In 1685 a subscription had just been raised among those who frequented
the wells for building a church, which the Tories, who then domineered
everywhere, insisted on dedicating to Saint Charles the Martyr. 
%[101]
\footnote{ Memoires de Grammont; Hasted's History of Kent; Tunbridge
Wells, a Comedy, 1678; Causton's Tunbridgialia, 1688; Metellus, a poem
on Tunbridge Wells, 1693.}


But at the head of the English watering places, without a rival. was
Bath. The springs of that city had been renowned from the days of the
Romans. It had been, during many centuries, the seat of a Bishop. The
sick repaired thither from every part of the realm. The King sometimes
held his court there. Nevertheless, Bath was then a maze of only four or
five hundred houses, crowded within an old wall in the vicinity of the
Avon. Pictures of what were considered as the finest of those houses are
still extant, and greatly resemble the lowest rag shops and pothouses of
Ratcliffe Highway. Travellers indeed complained loudly of the narrowness
and meanness of the streets. That beautiful city which charms even eyes
familiar with the masterpieces of Bramante and Palladio, and which the
genius of Anstey and of Smollett, of Frances Burney and of Jane Austen,
has made classic ground, had not begun to exist. Milsom Street itself
was an open field lying far beyond the walls; and hedgerows intersected
the space which is now covered by the Crescent and the Circus. The poor
patients to whom the waters had been recommended lay on straw in a place
which, to use the language of a contemporary physician, was a covert
rather than a lodging. As to the comforts and luxuries which were to be
found in the interior of the houses of Bath by the fashionable visitors
who resorted thither in search of health or amusement, we possess
information more complete and minute than can generally be obtained
on such subjects. A writer who published an account of that city about
sixty years after the Revolution has accurately described the changes
which had taken place within his own recollection. He assures us that,
in his younger days, the gentlemen who visited the springs slept in
rooms hardly as good as the garrets which he lived to see occupied
by footmen. The floors of the dining rooms were uncarpeted, and were
coloured brown with a wash made of soot and small beer, in order to hide
the dirt. Not a wainscot was painted. Not a hearth or a chimneypiece
was of marble. A slab of common free-stone and fire irons which had cost
from three to four shillings were thought sufficient for any fireplace.
The best-apartments were hung with coarse woollen stuff, and were
furnished with rushbottomed chairs. Readers who take an interest in the
progress of civilisation and of the useful arts will be grateful to the
humble topographer who has recorded these facts, and will perhaps wish
that historians of far higher pretensions had sometimes spared a few
pages from military evolutions and political intrigues, for the purpose
of letting us know how the parlours and bedchambers of our ancestors
looked. 
%[102]
\footnote{ See Wood's History of Bath, 1719; Evelyn's Diary, June
27,1654; Pepys's Diary, June 12, 1668; Stukeley's Itinerarium Curiosum;
Collinson's Somersetshire; Dr. Peirce's History and Memoirs of the Bath,
1713, Book I. chap. viii. obs. 2, 1684. I have consulted several
old maps and pictures of Bath, particularly one curious map which is
surrounded by views of the principal buildings. It Dears the date of
1717.}


The position of London, relatively to the other towns of the empire,
was, in the time of Charles the Second, far higher than at present. For
at present the population of London is little more than six times the
population of Manchester or of Liverpool. In the days of Charles the
Second the population of London was more than seventeen times the
population of Bristol or of Norwich. It may be doubted whether any other
instance can be mentioned of a great kingdom in which the first city
was more than seventeen times as large as the second. There is reason to
believe that, in 1685, London had been, during about half a century, the
most populous capital in Europe. The inhabitants, who are now at least
nineteen hundred thousand, were then probably little more shall half a
million. 
%[103]
\footnote{ According to King 530,000. (1848.) In 1851 the population
of London exceeded, 2,300,000. (1857.)}
 London had in the world only one commercial rival, now
long ago outstripped, the mighty and opulent Amsterdam. English writers
boasted of the forest of masts and yardarms which covered the river from
the Bridge to the Tower, and of the stupendous sums which were collected
at the Custom House in Thames Street. There is, indeed, no doubt that
the trade of the metropolis then bore a far greater proportion than at
present to the whole trade of the country; yet to our generation the
honest vaunting of our ancestors must appear almost ludicrous. The
shipping which they thought incredibly great appears not to have
exceeded seventy thousand tons. This was, indeed, then more than a third
of the whole tonnage of the kingdom, but is now less than a fourth of
the tonnage of Newcastle, and is nearly equalled by the tonnage of the
steam vessels of the Thames.

The customs of London amounted, in 1685, to about three hundred and
thirty thousand pounds a year. In our time the net duty paid annually,
at the same place, exceeds ten millions. 
%[104]
\footnote{ Macpherson's History of Commerce; Chalmers's Estimate;
Chamberlayne's State of England, 1684. The tonnage of the steamers
belonging to the port of London was, at the end of 1847, about 60,000
tons. The customs of the port, from 1842 to 1845, very nearly averaged
11,000,000£. (1848.) In 1854 the tonnage of the steamers of the port of
London amounted to 138,000 tons, without reckoning vessels of less than
fifty tons. (1857.)}


Whoever examines the maps of London which were published towards the
close of the reign of Charles the Second will see that only the nucleus
of the present capital then existed. The town did not, as now, fade
by imperceptible degrees into the country. No long avenues of villas,
embowered in lilacs and laburnums, extended from the great centre of
wealth and civilisation almost to the boundaries of Middlesex and far
into the heart of Kent and Surrey. In the east, no part of the immense
line of warehouses and artificial lakes which now stretches from the
Tower to Blackwall had even been projected. On the west, scarcely one
of those stately piles of building which are inhabited by the noble and
wealthy was in existence; and Chelsea, which is now peopled by more than
forty thousand human beings, was a quiet country village with about
a thousand inhabitants. 
%[105]
\footnote{ Lyson's Environs of London. The baptisms at Chelsea,
between 1680 and 1690, were only 42 a year.}
 On the north, cattle fed, and sportsmen
wandered with dogs and guns, over the site of the borough of Marylebone,
and over far the greater part of the space now covered by the boroughs
of Finsbury and of the Tower Hamlets. Islington was almost a solitude;
and poets loved to contrast its silence and repose with the din and
turmoil of the monster London. 
%[106]
\footnote{ Cowley, Discourse of Solitude.}
 On the south the capital is
now connected with its suburb by several bridges, not inferior in
magnificence and solidity to the noblest works of the Caesars. In 1685,
a single line of irregular arches, overhung by piles of mean and crazy
houses, and garnished, after a fashion worthy of the naked barbarians of
Dahomy, with scores of mouldering heads, impeded the navigation of the
river.

Of the metropolis, the City, properly so called, was the most important
division. At the time of the Restoration it had been built, for the most
part, of wood and plaster; the few bricks that were used were ill baked;
the booths where goods were exposed to sale projected far into the
streets, and were overhung by the upper stories. A few specimens of this
architecture may still be seen in those districts which were not reached
by the great fire. That fire had, in a few days, covered a space of
little less shall a square mile with the ruins of eighty-nine churches
and of thirteen thousand houses. But the City had risen again with a
celerity which had excited the admiration of neighbouring countries.
Unfortunately, the old lines of the streets had been to a great extent
preserved; and those lines, originally traced in an age when even
princesses performed their journeys on horseback, were often too narrow
to allow wheeled carriages to pass each other with ease, and were
therefore ill adapted for the residence of wealthy persons in an age
when a coach and six was a fashionable luxury. The style of building
was, however, far superior to that of the City which had perished. The
ordinary material was brick, of much better quality than had formerly
been used. On the sites of the ancient parish churches had arisen a
multitude of new domes, towers, and spires which bore the mark of the
fertile genius of Wren. In every place save one the traces of the great
devastation had been completely effaced. But the crowds of workmen, the
scaffolds, and the masses of hewn stone were still to be seen where the
noblest of Protestant temples was slowly rising on the ruins of the Old
Cathedral of Saint Paul. 
%[107]
\footnote{ The fullest and most trustworthy information about the
state of the buildings of London at this time is to be derived from the
maps and drawings in the British Museum and in the Pepysian Library.
The badness of the bricks in the old buildings of London is particularly
mentioned in the Travels of the Grand Duke Cosmo. There is an account of
the works at Saint Paul's in Ward's London Spy. I am almost ashamed to
quote such nauseous balderdash; but I have been forced to descend even
lower, if possible, in search of materials.}


The whole character of the City has, since that time, undergone a
complete change. At present the bankers, the merchants, and the chief
shopkeepers repair thither on six mornings of every week for the
transaction of business; but they reside in other quarters of the
metropolis, or at suburban country seats surrounded by shrubberies
and flower gardens. This revolution in private habits has produced
a political revolution of no small importance. The City is no longer
regarded by the wealthiest traders with that attachment which every man
naturally feels for his home. It is no longer associated in their minds
with domestic affections and endearments. The fireside, the nursery,
the social table, the quiet bed are not there. Lombard Street and
Threadneedle Street are merely places where men toil and accumulate.
They go elsewhere to enjoy and to expend. On a Sunday, or in an evening
after the hours of business, some courts and alleys, which a few hours
before had been alive with hurrying feet and anxious faces, are as
silent as the glades of a forest. The chiefs of the mercantile interest
are no longer citizens. They avoid, they almost contemn, municipal
honours and duties. Those honours and duties are abandoned to men who,
though useful and highly respectable, seldom belong to the princely
commercial houses of which the names are renowned throughout the world.

In the seventeenth century the City was the merchant's residence. Those
mansions of the great old burghers which still exist have been turned
into counting houses and warehouses: but it is evident that they were
originally not inferior in magnificence to the dwellings which were then
inhabited by the nobility. They sometimes stand in retired and gloomy
courts, and are accessible only by inconvenient passages: but their
dimensions are ample, and their aspect stately. The entrances are
decorated with richly carved pillars and canopies. The staircases and
landing places are not wanting in grandeur. The floors are sometimes of
wood tessellated after the fashion of France. The palace of Sir Robert
Clayton, in the Old Jewry, contained a superb banqueting room wainscoted
with cedar, and adorned with battles of gods and giants in fresco. 
%[108]
\footnote{ Evelyn's Diary, Sept. 20. 1672.}

Sir Dudley North expended four thousand pounds, a sum which would then
have been important to a Duke, on the rich furniture of his reception
rooms in Basinghall Street. 
%[109]
\footnote{ Roger North's Life of Sir Dudley North.}
 In such abodes, under the last
Stuarts, the heads of the great firms lived splendidly and hospitably.
To their dwelling place they were bound by the strongest ties of
interest and affection. There they had passed their youth, had made
their friendships, had courted their wives had seen their children grow
up, had laid the remains of their parents in the earth, and expected
that their own remains would be laid. That intense patriotism which is
peculiar to the members of societies congregated within a narrow space
was, in such circumstances, strongly developed. London was, to the
Londoner, what Athens was to the Athenian of the age of Pericles, what
Florence was to the Florentine of the fifteenth century. The citizen
was proud of the grandeur of his city, punctilious about her claims to
respect, ambitious of her offices, and zealous for her franchises.

At the close of the reign of Charles the Second the pride of the
Londoners was smarting from a cruel mortification. The old charter had
been taken away; and the magistracy had been remodelled. All the civic
functionaries were Tories: and the Whigs, though in numbers and in
wealth superior to their opponents, found themselves excluded from every
local dignity. Nevertheless, the external splendour of the municipal
government was not diminished, nay, was rather increased by this change.
For, under the administration of some Puritans who had lately borne
rule, the ancient fame of the City for good cheer had declined: but
under the new magistrates, who belonged to a more festive party, and
at whose boards guests of rank and fashion from beyond Temple Bar were
often seen, the Guildhall and the halls of the great companies were
enlivened by many sumptuous banquets. During these repasts, odes
composed by the poet laureate of the corporation, in praise of the King,
the Duke, and the Mayor, were sung to music. The drinking was deep and
the shouting loud. An observant Tory, who had often shared in these
revels, has remarked that the practice of huzzaing after drinking
healths dates from this joyous period. 
%[110]
\footnote{ North's Examen. This amusing writer has preserved
a specimen of the sublime raptures in which the Pindar of the City
indulged:--

     "The worshipful sir John Moor!
     After age that name adore!"}


The magnificence displayed by the first civic magistrate was almost
regal. The gilded coach, indeed, which is now annually admired by the
crowd, was not yet a part of his state. On great occasions he appeared
on horseback, attended by a long cavalcade inferior in magnificence
only to that which, before a coronation, escorted the sovereign from the
Tower to Westminster. The Lord Mayor was never seen in public without
his rich robe, his hood of black velvet, his gold chain, his jewel, and
a great attendance of harbingers and guards. 
%[111]
\footnote{ Chamberlayne's State of England, 1684; Anglie Metropolis,
1690; Seymour's London, 1734.}
 Nor did the world
find anything ludicrous in the pomp which constantly surrounded him. For
it was not more than became the place which, as wielding the strength
and representing the dignity of the City of London, he was entitled to
occupy in the State. That City, being then not only without equal in the
country, but without second, had, during five and forty years, exercised
almost as great an influence on the politics of England as Paris has,
in our own time, exercised on the politics of France. In intelligence
London was greatly in advance of every other part of the kingdom. A
government, supported and trusted by London, could in a day obtain such
pecuniary means as it would have taken months to collect from the rest
of the island. Nor were the military resources of the capital to be
despised. The power which the Lord Lieutenants exercised in other
parts of the kingdom was in London entrusted to a Commission of eminent
citizens. Under the order of this Commission were twelve regiments of
foot and two regiments of horse. An army of drapers' apprentices and
journeymen tailors, with common councilmen for captains and aldermen for
colonels, might not indeed have been able to stand its ground against
regular troops; but there were then very few regular troops in the
kingdom. A town, therefore, which could send forth, at an hour's notice,
thousands of men, abounding in natural courage, provided with tolerable
weapons, and not altogether untinctured with martial discipline, could
not but be a valuable ally and a formidable enemy. It was not forgotten
that Hampden and Pym had been protected from lawless tyranny by the
London trainbands; that, in the great crisis of the civil war, the
London trainbands had marched to raise the siege of Gloucester; or that,
in the movement against the military tyrants which followed the downfall
of Richard Cromwell, the London trainbands had borne a signal part. In
truth, it is no exaggeration to say that, but for the hostility of the
City, Charles the First would never have been vanquished, and that,
without the help of the City, Charles the Second could scarcely have
been restored.

These considerations may serve to explain why, in spite of that
attraction which had, during a long course of years, gradually drawn the
aristocracy westward, a few men of high rank had continued, till a
very recent period, to dwell in the vicinity of the Exchange and of
the Guildhall. Shaftesbury and Buckingham, while engaged in bitter and
unscrupulous opposition to the government, had thought that they could
nowhere carry on their intrigues so conveniently or so securely as under
the protection of the City magistrates and the City militia. Shaftesbury
had therefore lived in Aldersgate Street, at a house which may still
be easily known by pilasters and wreaths, the graceful work of Inigo.
Buckingham had ordered his mansion near Charing Cross, once the abode
of the Archbishops of York, to be pulled down; and, while streets and
alleys which are still named after him were rising on that site, chose
to reside in Dowgate. 
%[112]
\footnote{ North's Examen, 116; Wood, Ath. Ox. Shaftesbury; The Duke
of B.'s Litany.}


These, however, were rare exceptions. Almost all the noble families of
England had long migrated beyond the walls. The district where most of
their town houses stood lies between the city and the regions which
are now considered as fashionable. A few great men still retained their
hereditary hotels in the Strand. The stately dwellings on the south and
west of Lincoln's Inn Fields, the Piazza of Covent Garden, Southampton
Square, which is now called Bloomsbury Square, and King's Square in Soho
Fields, which is now called Soho Square, were among the favourite spots.
Foreign princes were carried to see Bloomsbury Square, as one of the
wonders of England. 
%[113]
\footnote{ Travels of the Grand Duke Cosmo.}
 Soho Square, which had just been built, was to
our ancestors a subject of pride with which their posterity will hardly
sympathise. Monmouth Square had been the name while the fortunes of
the Duke of Monmouth flourished; and on the southern side towered his
mansion. The front, though ungraceful, was lofty and richly adorned.
The walls of the principal apartments were finely sculptured with fruit,
foliage, and armorial bearings, and were hung with embroidered satin.

%[114]
\footnote{ Chamberlayne's State of England, 1684; Pennant's London;
Smith's Life of Nollekens.}
 Every trace of this magnificence has long disappeared; and no
aristocratical mansion is to be found in that once aristocratical
quarter. A little way north from Holborn, and on the verge of the
pastures and corn-fields, rose two celebrated palaces, each with
an ample garden. One of them, then called Southampton House, and
subsequently Bedford House, was removed about fifty years ago to make
room for a new city, which now covers with its squares, streets, and
churches, a vast area, renowned in the seventeenth century for peaches
and snipes. The other, Montague House, celebrated for its frescoes and
furniture, was, a few months after the death of Charles the Second,
burned to the ground, and was speedily succeeded by a more magnificent
Montague House, which, having been long the repository of such various
and precious treasures of art, science, and learning as were scarcely
ever before assembled under a single roof, has now given place to an
edifice more magnificent still. 
%[115]
\footnote{ Evelyn's Diary, Oct. 10, 1683, Jan. 19, 1685-6.}


Nearer to the Court, on a space called St. James's Fields, had just
been built St. James's Square and Jermyn Street. St. James's Church had
recently been opened for the accommodation of the inhabitants of this
new quarter. 
%[116]
\footnote{ Stat. 1 Jac. II. c. 22; Evelyn's Diary, Dec, 7, 1684.}
 Golden Square, which was in the next generation
inhabited by lords and ministers of state, had not yet been begun.
Indeed the only dwellings to be seen on the north of Piccadilly were
three or four isolated and almost rural mansions, of which the most
celebrated was the costly pile erected by CIarendon, and nicknamed
Dunkirk House. It had been purchased after its founder's downfall by
the Duke of Albemarle. The Clarendon Hotel and Albemarle Street still
preserve the memory of the site.

He who then rambled to what is now the gayest and most crowded part
of Regent Street found himself in a solitude, and, was sometimes so
fortunate as to have a shot at a woodcock. 
%[117]
\footnote{ Old General Oglethorpe, who died in 1785, used to boast
that he had shot birds here in Anne's reign. See Pennant's London, and
the Gentleman's Magazine for July, 1785.}
 On the north the Oxford
road ran between hedges. Three or four hundred yards to the south were
the garden walls of a few great houses which were considered as quite
out of town. On the west was a meadow renowned for a spring from which,
long afterwards, Conduit Street was named. On the east was a field not
to be passed without a shudder by any Londoner of that age. There, as in
a place far from the haunts of men, had been dug, twenty years before,
when the great plague was raging, a pit into which the dead carts had
nightly shot corpses by scores. It was popularly believed that the earth
was deeply tainted with infection, and could not be disturbed without
imminent risk to human life. No foundations were laid there till two
generations had passed without any return of the pestilence, and till
the ghastly spot had long been surrounded by buildings. 
%[118]
\footnote{ The pest field will be seen in maps of London as late as
the end of George the First's reign.}


We should greatly err if we were to suppose that any of the streets and
squares then bore the same aspect as at present. The great majority of
the houses, indeed have, since that time, been wholly, or in great part,
rebuilt. If the most fashionable parts of the capital could be placed
before us such as they then were, we should be disgusted by their
squalid appearance, and poisoned by their noisome atmosphere.

In Covent Garden a filthy and noisy market was held close to the
dwellings of the great. Fruit women screamed, carters fought, cabbage
stalks and rotten apples accumulated in heaps at the thresholds of the
Countess of Berkshire and of the Bishop of Durham. 
%[119]
\footnote{ See a very curious plan of Covent Garden made about 1690,
and engraved for Smith's History of Westminster. See also Hogarth's
Morning, painted while some of the houses in the Piazza were still
occupied by people of fashion.}


The centre of Lincoln's Inn Fields was an open space where the rabble
congregated every evening, within a few yards of Cardigan House and
Winchester House, to hear mountebanks harangue, to see bears dance, and
to set dogs at oxen. Rubbish was shot in every part of the area. Horses
were exercised there. The beggars were as noisy and importunate as in
the worst governed cities of the Continent. A Lincoln's Inn mumper was
a proverb. The whole fraternity knew the arms and liveries of every
charitably disposed grandee in the neighbourhood, and as soon as his
lordship's coach and six appeared, came hopping and crawling in crowds
to persecute him. These disorders lasted, in spite of many accidents,
and of some legal proceedings, till, in the reign of George the Second,
Sir Joseph Jekyll, Master of the Rolls, was knocked down and nearly
killed in the middle of the Square. Then at length palisades were set
up, and a pleasant garden laid out. 
%[120]
\footnote{ London Spy, Tom Brown's comical View of London and
Westminster; Turner's Propositions for the employing of the Poor, 1678;
Daily Courant and Daily Journal of June 7, 1733; Case of Michael v.
Allestree, in 1676, 2 Levinz, p. 172. Michael had been run over by
two horses which Allestree was breaking in Lincoln's Inn Fields.
The declaration set forth that the defendant "porta deux chivals
ungovernable en un coach, et improvide, incante, et absque debita
consideratione ineptitudinis loci la eux drive pur eux faire tractable
et apt pur an coach, quels chivals, pur ceo que, per leur ferocite, ne
poientestre rule, curre sur le plaintiff et le noie."}


Saint James's Square was a receptacle for all the offal and cinders,
for all the dead cats and dead dogs of Westminster. At one time a cudgel
player kept the ring there. At another time an impudent squatter settled
himself there, and built a shed for rubbish under the windows of the
gilded saloons in which the first magnates of the realm, Norfolk,
Ormond, Kent, and Pembroke, gave banquets and balls. It was not till
these nuisances had lasted through a whole generation, and till much had
been written about them, that the inhabitants applied to Parliament for
permission to put up rails, and to plant trees. 
%[121]
\footnote{ Stat. 12 Geo. I. c. 25; Commons' Journals, Feb. 25, March
2, 1725-6; London Gardener, 1712; Evening Post, March, 23, 1731. I have
not been able to find this number of the Evening Post; I therefore
quote it on the faith of Mr. Malcolm, who mentions it in his History of
London.}


When such was the state of the region inhabited by the most luxurious
portion of society, we may easily believe that the great body of the
population suffered what would now be considered as insupportable
grievances. The pavement was detestable: all foreigners cried shame
upon it. The drainage was so bad that in rainy weather the gutters soon
became torrents. Several facetious poets have commemorated the fury
with which these black rivulets roared down Snow Hill and Ludgate Hill,
bearing to Fleet Ditch a vast tribute of animal and vegetable filth from
the stalls of butchers and greengrocers. This flood was profusely thrown
to right and left by coaches and carts. To keep as far from the carriage
road as possible was therefore the wish of every pedestrian. The
mild and timid gave the wall. The bold and athletic took it. If two
roisterers met they cocked their hats in each other's faces, and pushed
each other about till the weaker was shoved towards the kennel. If he
was a mere bully he sneaked off, mattering that he should find a time.
If he was pugnacious, the encounter probably ended in a duel behind
Montague House. 
%[122]
\footnote{ Lettres sur les Anglois, written early in the reign of
William the Third; Swift's City Shower; Gay's Trivia. Johnson used to
relate a curious conversation which he had with his mother about giving
and taking the wall.}


The houses were not numbered. There would indeed have been little
advantage in numbering them; for of the coachmen, chairmen, porters,
and errand boys of London, a very small proportion could read. It was
necessary to use marks which the most ignorant could understand. The
shops were therefore distinguished by painted or sculptured signs, which
gave a gay and grotesque aspect to the streets. The walk from Charing
Cross to Whitechapel lay through an endless succession of Saracens'
Heads, Royal Oaks, Blue Bears, and Golden Lambs, which disappeared when
they were no longer required for the direction of the common people.

When the evening closed in, the difficulty and danger of walking about
London became serious indeed. The garret windows were opened, and pails
were emptied, with little regard to those who were passing below. Falls,
bruises and broken bones were of constant occurrence. For, till the last
year of the reign of Charles the Second, most of the streets were
left in profound darkness. Thieves and robbers plied their trade with
impunity: yet they were hardly so terrible to peaceable citizens as
another class of ruffians. It was a favourite amusement of dissolute
young gentlemen to swagger by night about the town, breaking windows,
upsetting sedans, beating quiet men, and offering rude caresses
to pretty women. Several dynasties of these tyrants had, since the
Restoration, domineered over the streets. The Muns and Tityre Tus had
given place to the Hectors, and the Hectors had been recently succeeded
by the Scourers. At a later period arose the Nicker, the Hawcubite, and
the yet more dreaded name of Mohawk. 
%[123]
\footnote{ Oldham's Imitation of the 3d Satire of Juvenal, 1682;
Shadwell's Scourers, 1690. Many other authorities will readily occur
to all who are acquainted with the popular literature of that and the
succeeding generation. It may be suspected that some of the Tityre
Tus, like good Cavaliers, broke Milton's windows shortly after the
Restoration. I am confident that he was thinking of those pests of
London when he dictated the noble lines:

     "And in luxurious cities, when the noise
     Of riot ascends above their loftiest towers,
     And injury and outrage, and when night
     Darkens the streets, then wander forth the sons
     Of Belial, flown With innocence and wine."}
 The machinery for keeping the
peace was utterly contemptible. There was an Act of Common Council which
provided that more than a thousand watchmen should be constantly on the
alert in the city, from sunset to sunrise, and that every inhabitant
should take his turn of duty. But this Act was negligently executed.
Few of those who were summoned left their homes; and those few generally
found it more agreeable to tipple in alehouses than to pace the streets.

%[124]
\footnote{ Seymour's London.}


It ought to be noticed that, in the last year of the reign of Charles
the Second, began a great change in the police of London, a change which
has perhaps added as much to the happiness of the body of the people as
revolutions of much greater fame. An ingenious projector, named Edward
Heming, obtained letters patent conveying to him, for a term of years,
the exclusive right of lighting up London. He undertook, for a moderate
consideration, to place a light before every tenth door, on moonless
nights, from Michaelmas to Lady Day, and from six to twelve of the
clock. Those who now see the capital all the year round, from dusk to
dawn, blazing with a splendour beside which the illuminations for La
Hogue and Blenheim would have looked pale, may perhaps smile to think of
Heming's lanterns, which glimmered feebly before one house in ten during
a small part of one night in three. But such was not the feeling of his
contemporaries. His scheme was enthusiastically applauded, and furiously
attacked. The friends of improvement extolled him as the greatest of
all the benefactors of his city. What, they asked, were the boasted
inventions of Archimedes, when compared with the achievement of the man
who had turned the nocturnal shades into noon-day? In spite of these
eloquent eulogies the cause of darkness was not left undefended. There
were fools in that age who opposed the introduction of what was called
the new light as strenuously as fools in our age have opposed the
introduction of vaccination and railroads, as strenuously as the
fools of an age anterior to the dawn of history doubtless opposed the
introduction of the plough and of alphabetical writing. Many years after
the date of Heming's patent there were extensive districts in which no
lamp was seen. 
%[125]
\footnote{ Angliae Metropolis, 1690, Sect. 17, entitled, "Of the new
lights"; Seymour's London.}


We may easily imagine what, in such times, must have been the state of
the quarters of London which were peopled by the outcasts of society.
Among those quarters one had attained a scandalous preeminence. On the
confines of the City and the Temple had been founded, in the thirteenth
century, a House of Carmelite Friars, distinguished by their white
hoods. The precinct of this house had, before the Reformation, been a
sanctuary for criminals, and still retained the privilege of protecting
debtors from arrest. Insolvents consequently were to be found in every
dwelling, from cellar to garret. Of these a large proportion were
knaves and libertines, and were followed to their asylum by women more
abandoned than themselves. The civil power was unable to keep order in a
district swarming with such inhabitants; and thus Whitefriars became the
favourite resort of all who wished to be emancipated from the restraints
of the law. Though the immunities legally belonging to the place
extended only to cases of debt, cheats, false witnesses, forgers, and
highwaymen found refuge there. For amidst a rabble so desperate no
peace officer's life was in safety. At the cry of "Rescue," bullies
with swords and cudgels, and termagant hags with spits and broomsticks,
poured forth by hundreds; and the intruder was fortunate if he escaped
back into Fleet Street, hustled, stripped, and pumped upon. Even the
warrant of the Chief Justice of England could not be executed without
the help of a company of musketeers. Such relics of the barbarism of the
darkest ages were to be found within a short walk of the chambers where
Somers was studying history and law, of the chapel where Tillotson was
preaching, of the coffee house where Dryden was passing judgment on
poems and plays, and of the hall where the Royal Society was examining
the astronomical system of Isaac Newton. 
%[126]
\footnote{ Stowe's Survey of London; Shadwell's Squire of Alsatia;
Ward's London Spy; Stat. 8 \& 9 Gul. III. cap. 27.}


Each of the two cities which made up the capital of England had its
own centre of attraction. In the metropolis of commerce the point of
convergence was the Exchange; in the metropolis of fashion the Palace.
But the Palace did not retain influence so long as the Exchange. The
Revolution completely altered the relations between the Court and the
higher classes of society. It was by degrees discovered that the King,
in his individual capacity, had very little to give; that coronets
and garters, bishoprics and embassies, lordships of the Treasury and
tellerships of the Exchequer, nay, even charges in the royal stud and
bedchamber, were really bestowed, not by him, but by his advisers.
Every ambitious and covetous man perceived that he would consult his own
interest far better by acquiring the dominion of a Cornish borough, and
by rendering good service to the ministry during a critical session,
than by becoming the companion, or even the minion, of his prince. It
was therefore in the antechambers, not of George the First and of
George the Second, but of Walpole and of Pelham, that the daily crowd
of courtiers was to be found. It is also to be remarked that the same
Revolution, which made it impossible that our Kings should use the
patronage of the state merely for the purpose of gratifying their
personal predilections, gave us several Kings unfitted by their
education and habits to be gracious and affable hosts. They had been
born and bred on the Continent. They never felt themselves at home in
our island. If they spoke our language, they spoke it inelegantly and
with effort. Our national character they never fully understood. Our
national manners they hardly attempted to acquire. The most important
part of their duty they performed better than any ruler who preceded
them: for they governed strictly according to law: but they could not be
the first gentlemen of the realm, the heads of polite society. If ever
they unbent, it was in a very small circle where hardly an English face
was to be seen; and they were never so happy as when they could escape
for a summer to their native land. They had indeed their days of
reception for our nobility and gentry; but the reception was a mere
matter of form, and became at last as solemn a ceremony as a funeral.

Not such was the court of Charles the Second. Whitehall, when he dwelt
there, was the focus of political intrigue and of fashionable gaiety.
Half the jobbing and half the flirting of the metropolis went on under
his roof. Whoever could make himself agreeable to the prince, or could
secure the good offices of the mistress, might hope to rise in the world
without rendering any service to the government, without being even
known by sight to any minister of state. This courtier got a frigate,
and that a company; a third, the pardon of a rich offender; a fourth,
a lease of crown land on easy terms. If the King notified his pleasure
that a briefless lawyer should be made a judge, or that a libertine
baronet should be made a peer, the gravest counsellors, after a little
murmuring, submitted. 
%[127]
\footnote{ See Sir Roger North's account of the way in which Wright
was made a judge, and Clarendon's account of the way in which Sir George
Savile was made a peer.}
 Interest, therefore, drew a constant press
of suitors to the gates of the palace; and those gates always stood
wide. The King kept open house every day, and all day long, for the good
society of London, the extreme Whigs only excepted. Hardly any gentleman
had any difficulty in making his way to the royal presence. The levee
was exactly what the word imports. Some men of quality came every
morning to stand round their master, to chat with him while his wig
was combed and his cravat tied, and to accompany him in his early walk
through the Park. All persons who had been properly introduced might,
without any special invitation, go to see him dine, sup, dance, and
play at hazard, and might have the pleasure of hearing him tell stories,
which indeed he told remarkably well, about his flight from Worcester,
and about the misery which he had endured when he was a state prisoner
in the hands of the canting meddling preachers of Scotland. Bystanders
whom His Majesty recognised often came in for a courteous word. This
proved a far more successful kingcraft than any that his father
or grandfather had practiced. It was not easy for the most austere
republican of the school of Marvel to resist the fascination of so much
good humour and affability; and many a veteran Cavalier, in whose heart
the remembrance of unrequited sacrifices and services had been festering
during twenty years, was compensated in one moment for wounds and
sequestrations by his sovereign's kind nod, and "God bless you, my old
friend!"

Whitehall naturally became the chief staple of news. Whenever there was
a rumour that anything important had happened or was about to happen,
people hastened thither to obtain intelligence from the fountain head.
The galleries presented the appearance of a modern club room at an
anxious time. They were full of people enquiring whether the Dutch mail
was in, what tidings the express from France had brought, whether John
Sobiesky had beaten the Turks, whether the Doge of Genoa was really
at Paris These were matters about which it was safe to talk aloud. But
there were subjects concerning which information was asked and given
in whispers. Had Halifax got the better of Rochester? Was there to be a
Parliament? Was the Duke of York really going to Scotland? Had Monmouth
really been summoned from the Hague? Men tried to read the countenance
of every minister as he went through the throng to and from the royal
closet. All sorts of auguries were drawn from the tone in which His
Majesty spoke to the Lord President, or from the laugh with which His
Majesty honoured a jest of the Lord Privy Seal; and in a few hours the
hopes and fears inspired by such slight indications had spread to all
the coffee houses from Saint James's to the Tower. 
%[128]
\footnote{ The sources from which I have drawn my information about
the state of the Court are too numerous to recapitulate. Among them
are the Despatches of Barillon, Van Citters, Ronquillo, and Adda, the
Travels of the Grand Duke Cosmo, the works of Roger North, the Diares of
Pepys, Evelyn, and Teonge, and the Memoirs of Grammont and Reresby.}


The coffee house must not be dismissed with a cursory mention. It might
indeed at that time have been not improperly called a most important
political institution. No Parliament had sat for years The municipal
council of the City had ceased to speak the sense of the citizens.
Public meetings, harangues, resolutions, and the rest of the modern
machinery of agitation had not yet come into fashion. Nothing resembling
the modern newspaper existed. In such circumstances the coffee houses
were the chief organs through which the public opinion of the metropolis
vented itself.

The first of these establishments had been set up by a Turkey merchant,
who had acquired among the Mahometans a taste for their favourite
beverage. The convenience of being able to make appointments in any part
of the town, and of being able to pass evenings socially at a very small
charge, was so great that the fashion spread fast. Every man of the
upper or middle class went daily to his coffee house to learn the news
and to discuss it. Every coffee house had one or more orators to whose
eloquence the crowd listened with admiration, and who soon became, what
the journalists of our time have been called, a fourth Estate of the
realm. The Court had long seen with uneasiness the growth of this
new power in the state. An attempt had been made, during Danby's
administration, to close the coffee houses. But men of all parties
missed their usual places of resort so much that there was an universal
outcry. The government did not venture, in opposition to a feeling so
strong and general, to enforce a regulation of which the legality might
well be questioned. Since that time ten years had elapsed, and during
those years the number and influence of the coffee houses had been
constantly increasing. Foreigners remarked that the coffee house was
that which especially distinguished London from all other cities; that
the coffee house was the Londoner's home, and that those who wished to
find a gentleman commonly asked, not whether he lived in Fleet Street
or Chancery Lane, but whether he frequented the Grecian or the Rainbow.
Nobody was excluded from these places who laid down his penny at the
bar. Yet every rank and profession, and every shade of religious and
political opinion, had its own headquarters. There were houses near
Saint James's Park where fops congregated, their heads and shoulders
covered with black or flaxen wigs, not less ample than those which are
now worn by the Chancellor and by the Speaker of the House of Commons.
The wig came from Paris and so did the rest of the fine gentleman's
ornaments, his embroidered coat, his fringed gloves, and the tassel
which upheld his pantaloons. The conversation was in that dialect which,
long after it had ceased to be spoken in fashionable circles, continued,
in the mouth of Lord Foppington, to excite the mirth of theatres. 
%[129]
\footnote{ The chief peculiarity of this dialect was that, in
a large class of words, the O was pronounced like A. Thus Lord was
pronounced Lard. See Vanbrugh's Relapse. Lord Sunderland was a great
master of this court tune, as Roger North calls it; and Titus Oates
affected it in the hope of passing for a fine gentleman. Examen, 77,
254.}

The atmosphere was like that of a perfumer's shop. Tobacco in any other
form than that of richly scented snuff was held in abomination. If
any clown, ignorant of the usages of the house, called for a pipe, the
sneers of the whole assembly and the short answers of the waiters soon
convinced him that he had better go somewhere else. Nor, indeed, would
he have had far to go. For, in general the coffee rooms reeked with
tobacco like a guardroom: and strangers sometimes expressed their
surprise that so many people should leave their own firesides to sit
in the midst of eternal fog and stench. Nowhere was the smoking more
constant than at Will's. That celebrated house, situated between Covent
Garden and Bow Street, was sacred to polite letters. There the talk was
about poetical justice and the unities of place and time. There was
a faction for Perrault and the moderns, a faction for Boileau and the
ancients. One group debated whether Paradise Lost ought not to have
been in rhyme. To another an envious poetaster demonstrated that Venice
Preserved ought to have been hooted from the stage. Under no roof was
a greater variety of figures to be seen. There were Earls in stars and
garters, clergymen in cassocks and bands, pert Templars, sheepish lads
from the Universities, translators and index makers in ragged coats
of frieze. The great press was to get near the chair where John Dryden
sate. In winter that chair was always in the warmest nook by the fire;
in summer it stood in the balcony. To bow to the Laureate, and to hear
his opinion of Racine's last tragedy or of Bossu's treatise on epic
poetry, was thought a privilege. A pinch from his snuff box was an
honour sufficient to turn the head of a young enthusaist. There were
coffee houses where the first medical men might be consulted. Doctor
John Radcliffe, who, in the year 1685, rose to the largest practice in
London, came daily, at the hour when the Exchange was full, from
his house in Bow Street, then a fashionable part of the capital,
to Garraway's, and was to be found, surrounded by surgeons and
apothecaries, at a particular table. There were Puritan coffee houses
where no oath was heard, and where lankhaired men discussed election and
reprobation through their noses; Jew coffee houses where darkeyed money
changers from Venice and Amsterdam greeted each other; and Popish coffee
houses where, as good Protestants believed, Jesuits planned, over their
cups, another great fire, and cast silver bullets to shoot the King.

%[130]
\footnote{ Lettres sur les Anglois; Tom Brown's Tour; Ward's London
Spy; The Character of a Coffee House, 1673; Rules and Orders of the
Coffee House, 1674; Coffee Houses vindicated, 1675; A Satyr against
Coffee; North's Examen, 138; Life of Guildford, 152; Life of Sir Dudley
North, 149; Life of Dr. Radcliffe, published by Curll in 1715. The
liveliest description of Will's is in the City and Country Mouse. There
is a remarkable passage about the influence of the coffee house orators
in Halstead's Succinct Genealogies, printed in 1685.}


These gregarious habits had no small share in forming the character of
the Londoner of that age. He was, indeed, a different being from the
rustic Englishman. There was not then the intercourse which now exists
between the two classes. Only very great men were in the habit of
dividing the year between town and country. Few esquires came to the
capital thrice in their lives. Nor was it yet the practice of all
citizens in easy circumstances to breathe the fresh air of the fields
and woods during some weeks of every summer. A cockney, in a rural
village, was stared at as much as if he had intruded into a Kraal
of Hottentots. On the other hand, when the lord of a Lincolnshire
or Shropshire manor appeared in Fleet Street, he was as easily
distinguished from the resident population as a Turk or a Lascar. His
dress, his gait, his accent, the manner in which he gazed at the shops,
stumbled into the gutters, ran against the porters, and stood under the
waterspouts, marked him out as an excellent subject for the operations
of swindlers and barterers. Bullies jostled him into the kennel. Hackney
coachmen splashed him from head to foot. Thieves explored with perfect
security the huge pockets of his horseman's coat, while he stood
entranced by the splendour of the Lord Mayor's show. Moneydroppers, sore
from the cart's tail, introduced themselves to him, and appeared to him
the most honest friendly gentlemen that he had ever seen. Painted women,
the refuse of Lewkner Lane and Whetstone Park, passed themselves on
him for countesses and maids of honour. If he asked his way to Saint
James's, his informants sent him to Mile End. If he went into a shop, he
was instantly discerned to be a fit purchaser of everything that nobody
else would buy, of second-hand embroidery, copper rings, and watches
that would not go. If he rambled into any fashionable coffee house, he
became a mark for the insolent derision of fops and the grave waggery
of Templars. Enraged and mortified, he soon returned to his mansion,
and there, in the homage of his tenants and the conversation of his boon
companions, found consolation for the vexatious and humiliations which
he had undergone. There he was once more a great man, and saw nothing
above himself except when at the assizes he took his seat on the bench
near the Judge, or when at the muster of the militia he saluted the Lord
Lieutenant.

The chief cause which made the fusion of the different elements of
society so imperfect was the extreme difficulty which our ancestors
found in passing from place to place. Of all inventions, the alphabet
and the printing press alone excepted, those inventions which abridge
distance have done most for the civilisation of our species. Every
improvement of the means of locomotion benefits mankind morally and
intellectually as well as materially, and not only facilitates the
interchange of the various productions of nature and art, but tends to
remove national and provincial antipathies, and to bind together all
the branches of the great human family. In the seventeenth century the
inhabitants of London were, for almost every practical purpose,
farther from Reading than they now are from Edinburgh, and farther from
Edinburgh than they now are from Vienna.

The subjects of Charles the Second were not, it is true, quite
unacquainted with that principle which has, in our own time, produced an
unprecedented revolution in human affairs, which has enabled navies to
advance in face of wind and tide, and brigades of troops, attended by
all their baggage and artillery, to traverse kingdoms at a pace equal to
that of the fleetest race horse. The Marquess of Worcester had recently
observed the expansive power of moisture rarefied by heat. After many
experiments he had succeeded in constructing a rude steam engine, which
he called a fire water work, and which he pronounced to be an admirable
and most forcible instrument of propulsion. 
%[131]
\footnote{ Century of inventions, 1663, No. 68.}
 But the Marquess was
suspected to be a madman, and known to be a Papist. His inventions,
therefore found no favourable reception. His fire water work might,
perhaps, furnish matter for conversation at a meeting of the Royal
Society, but was not applied to any practical purpose. There were no
railways, except a few made of timber, on which coals were carried from
the mouths of the Northumbrian pits to the banks of the Tyne. 
%[132]
\footnote{ North's Life of Guildford, 136.}

There was very little internal communication by water. A few attempts
had been made to deepen and embank the natural streams, but with slender
success. Hardly a single navigable canal had been even projected. The
English of that day were in the habit of talking with mingled admiration
and despair of the immense trench by which Lewis the Fourteenth had
made a junction between the Atlantic and the Mediterranean. They little
thought that their country would, in the course of a few generations,
be intersected, at the cost of private adventurers, by artificial rivers
making up more than four times the length of the Thames, the Severn, and
the Trent together.

It was by the highways that both travellers and goods generally passed
from place to place; and those highways appear to have been far worse
than might have been expected from the degree of wealth and civilisation
which the nation had even then attained. On the best lines of
communication the ruts were deep, the descents precipitous, and the way
often such as it was hardly possible to distinguish, in the dusk, from
the unenclosed heath and fen which lay on both sides. Ralph Thorseby,
the antiquary, was in danger of losing his way on the great North road,
between Barnby Moor and Tuxford, and actually lost his way between
Doncaster and York. 
%[133]
\footnote{ Thoresby's Diary Oct. 21,1680, Aug. 3, 1712.}
 Pepys and his wife, travelling in their own
coach, lost their way between Newbury and Reading. In the course of
the same tour they lost their way near Salisbury, and were in danger of
having to pass the night on the plain. 
%[134]
\footnote{ Pepys's Diary, June 12 and 16,1668.}
 It was only in fine weather
that the whole breadth of the road was available for wheeled vehicles.
Often the mud lay deep on the right and the left; and only a narrow
track of firm ground rose above the quagmire. 
%[135]
\footnote{ Ibid. Feb. 28, 1660.}
 At such times
obstructions and quarrels were frequent, and the path was sometimes
blocked up during a long time by carriers, neither of whom would break
the way. It happened, almost every day, that coaches stuck fast, until
a team of cattle could be procured from some neighbouring farm, to
tug them out of the slough. But in bad seasons the traveller had to
encounter inconveniences still more serious. Thoresby, who was in the
habit of travelling between Leeds and the capital, has recorded, in
his Diary, such a series of perils and disasters as might suffice for a
journey to the Frozen Ocean or to the Desert of Sahara. On one occasion
he learned that the floods were out between Ware and London, that
passengers had to swim for their lives, and that a higgler had perished
in the attempt to cross. In consequence of these tidings he turned out
of the high road, and was conducted across some meadows, where it was
necessary for him to ride to the saddle skirts in water. 
%[136]
\footnote{ Thoresby's Diary, May 17,1695.}
 In the
course of another journey he narrowly escaped being swept away by an
inundation of the Trent. He was afterwards detained at Stamford four
days, on account of the state of the roads, and then ventured to proceed
only because fourteen members of the House of Commons, who were going
up in a body to Parliament with guides and numerous attendants, took him
into their company. 
%[137]
\footnote{ Ibid. Dec. 27,1708.}
 On the roads of Derbyshire, travellers were in
constant fear for their necks, and were frequently compelled to alight
and lead their beasts. 
%[138]
\footnote{ Tour in Derbyshire, by J. Browne, son of Sir Thomas
Browne, 1662; Cotton's Angler, 1676.}
 The great route through Wales to Holyhead
was in such a state that, in 1685, a viceroy, going to Ireland, was five
hours in travelling fourteen miles, from Saint Asaph to Conway. Between
Conway and Beaumaris he was forced to walk a great part of the way; and
his lady was carried in a litter. His coach was, with much difficulty,
and by the help of many hands, brought after him entire. In general,
carriages were taken to pieces at Conway, and borne, on the shoulders of
stout Welsh peasants, to the Menai Straits. 
%[139]
\footnote{ Correspondence of Henry Earl of Clarendon, Dec. 30, 1685,
Jan. 1, 1686.}
 In some parts of Kent
and Sussex, none but the strongest horses could, in winter, get through
the bog, in which, at every step, they sank deep. The markets were often
inaccessible during several months. It is said that the fruits of the
earth were sometimes suffered to rot in one place, while in another
place, distant only a few miles, the supply fell far short of the
demand. The wheeled carriages were, in this district, generally pulled
by oxen. 
%[140]
\footnote{ Postlethwaite's Dictionary, Roads; History of Hawkhurst,
in the Bibliotheca Topographica Britannica.}
 When Prince George of Denmark visited the stately mansion
of Petworth in wet weather, he was six hours in going nine miles; and it
was necessary that a body of sturdy hinds should be on each side of his
coach, in order to prop it. Of the carriages which conveyed his retinue
several were upset and injured. A letter from one of the party has been
preserved, in which the unfortunate courtier complains that, during
fourteen hours, he never once alighted, except when his coach was
overturned or stuck fast in the mud. 
%[141]
\footnote{ Annals of Queen Anne, 1703, Appendix, No. 3.}


One chief cause of the badness of the roads seems to have been the
defective state of the law. Every parish was bound to repair the
highways which passed through it. The peasantry were forced to
give their gratuitous labour six days in the year. If this was not
sufficient, hired labour was employed, and the expense was met by a
parochial rate. That a route connecting two great towns, which have a
large and thriving trade with each other, should be maintained at the
cost of the rural population scattered between them is obviously unjust;
and this injustice was peculiarly glaring in the case of the great North
road, which traversed very poor and thinly inhabited districts, and
joined very rich and populous districts. Indeed it was not in the
power of the parishes of Huntingdonshire to mend a high-way worn by the
constant traffic between the West Riding of Yorkshire and London. Soon
after the Restoration this grievance attracted the notice of Parliament;
and an act, the first of our many turnpike acts, was passed imposing
a small toll on travellers and goods, for the purpose of keeping some
parts of this important line of communication in good repair. 
%[142]
\footnote{ 15 Car. II. c. 1.}
 This
innovation, however, excited many murmurs; and the other great avenues
to the capital were long left under the old system. A change was at
length effected, but not without much difficulty. For unjust and absurd
taxation to which men are accustomed is often borne far more willingly
than the most reasonable impost which is new. It was not till many
toll bars had been violently pulled down, till the troops had in many
districts been forced to act against the people, and till much blood
had been shed, that a good system was introduced. 
%[143]
\footnote{ The evils of the old system are strikingly set forth
in many petitions which appear in the Commons' Journal of 172 5/6. How
fierce an opposition was offered to the new system may be learned from
the Gentleman's Magazine of 1749.}
 By slow degrees
reason triumphed over prejudice; and our island is now crossed in every
direction by near thirty thousand miles of turnpike road.

On the best highways heavy articles were, in the time of Charles the
Second, generally conveyed from place to place by stage waggons. In the
straw of these vehicles nestled a crowd of passengers, who could not
afford to travel by coach or on horseback, and who were prevented by
infirmity, or by the weight of their luggage, from going on foot. The
expense of transmitting heavy goods in this way was enormous. From
London to Birmingham the charge was seven pounds a ton; from London to
Exeter twelve pounds a ton. 
%[144]
\footnote{ Postlethwaite's Dict., Roads.}
 This was about fifteen pence a ton
for every mile, more by a third than was afterwards charged on turnpike
roads, and fifteen times what is now demanded by railway companies.
The cost of conveyance amounted to a prohibitory tax on many useful
articles. Coal in particular was never seen except in the districts
where it was produced, or in the districts to which it could be carried
by sea, and was indeed always known in the south of England by the name
of sea coal.

On byroads, and generally throughout the country north of York and west
of Exeter, goods were carried by long trains of packhorses. These strong
and patient beasts, the breed of which is now extinct, were attended by
a class of men who seem to have borne much resemblance to the Spanish
muleteers. A traveller of humble condition often found it convenient to
perform a journey mounted on a packsaddle between two baskets, under the
care of these hardy guides. The expense of this mode of conveyance was
small. But the caravan moved at a foot's pace; and in winter the cold
was often insupportable. 
%[145]
\footnote{ Loidis and Elmete; Marshall's Rural Economy of England,
In 1739 Roderic Random came from Scotland to Newcastle on a packhorse.}


The rich commonly travelled in their own carriages, with at least four
horses. Cotton, the facetious poet, attempted to go from London to the
Peak with a single pair, but found at Saint Albans that the journey
would be insupportably tedious, and altered his Plan. 
%[146]
\footnote{ Cotton's Epistle to J. Bradshaw.}
 A coach
and six is in our time never seen, except as part of some pageant. The
frequent mention therefore of such equipages in old books is likely to
mislead us. We attribute to magnificence what was really the effect of a
very disagreeable necessity. People, in the time of Charles the Second,
travelled with six horses, because with a smaller number there was great
danger of sticking fast in the mire. Nor were even six horses always
sufficient. Vanbrugh, in the succeeding generation, described with great
humour the way in which a country gentleman, newly chosen a member of
Parliament, went up to London. On that occasion all the exertions of six
beasts, two of which had been taken from the plough, could not save the
family coach from being embedded in a quagmire.

Public carriages had recently been much improved. During the years which
immediately followed the Restoration, a diligence ran between London and
Oxford in two days. The passengers slept at Beaconsfield. At length, in
the spring of 1669, a great and daring innovation was attempted. It was
announced that a vehicle, described as the Flying Coach, would perform
the whole journey between sunrise and sunset. This spirited undertaking
was solemnly considered and sanctioned by the Heads of the University,
and appears to have excited the same sort of interest which is excited
in our own time by the opening of a new railway. The Vicechancellor, by
a notice affixed in all public places, prescribed the hour and place
of departure. The success of the experiment was complete. At six in the
morning the carriage began to move from before the ancient front of All
Souls College; and at seven in the evening the adventurous gentlemen
who had run the first risk were safely deposited at their inn in London.

%[147]
\footnote{ Anthony a Wood's Life of himself.}
 The emulation of the sister University was moved; and soon a
diligence was set up which in one day carried passengers from Cambridge
to the capital. At the close of the reign of Charles the Second flying
carriages ran thrice a week from London to the chief towns. But no stage
coach, indeed no stage waggon, appears to have proceeded further north
than York, or further west than Exeter. The ordinary day's journey of
a flying coach was about fifty miles in the summer; but in winter, when
the ways were bad and the nights long, little more than thirty. The
Chester coach, the York coach, and the Exeter coach generally reached
London in four days during the fine season, but at Christmas not till
the sixth day. The passengers, six in number, were all seated in the
carriage. For accidents were so frequent that it would have been
most perilous to mount the roof. The ordinary fare was about twopence
halfpenny a mile in summer, and somewhat more in winter. 
%[148]
\footnote{ Chamberlayne's State of England, 1684. See also the list
of stage coaches and waggons at the end of the book, entitled Angliae
Metropolis, 1690.}


This mode of travelling, which by Englishmen of the present day would be
regarded as insufferably slow, seemed to our ancestors wonderfully and
indeed alarmingly rapid. In a work published a few months before the
death of Charles the Second, the flying coaches are extolled as far
superior to any similar vehicles ever known in the world. Their velocity
is the subject of special commendation, and is triumphantly contrasted
with the sluggish pace of the continental posts. But with boasts like
these was mingled the sound of complaint and invective. The interests of
large classes had been unfavourably affected by the establishment of the
new diligences; and, as usual, many persons were, from mere stupidity
and obstinacy, disposed to clamour against the innovation, simply
because it was an innovation. It was vehemently argued that this mode of
conveyance would be fatal to the breed of horses and to the noble art of
horsemanship; that the Thames, which had long been an important nursery
of seamen, would cease to be the chief thoroughfare from London up to
Windsor and down to Gravesend; that saddlers and spurriers would be
ruined by hundreds; that numerous inns, at which mounted travellers had
been in the habit of stopping, would be deserted, and would no longer
pay any rent; that the new carriages were too hot in summer and too cold
in winter; that the passengers were grievously annoyed by invalids and
crying children; that the coach sometimes reached the inn so late that
it was impossible to get supper, and sometimes started so early that
it was impossible to get breakfast. On these grounds it was gravely
recommended that no public coach should be permitted to have more than
four horses, to start oftener than once a week, or to go more than
thirty miles a day. It was hoped that, if this regulation were adopted,
all except the sick and the lame would return to the old mode of
travelling. Petitions embodying such opinions as these were presented to
the King in council from several companies of the City of London, from
several provincial towns, and from the justices of several counties. We
Smile at these things. It is not impossible that our descendants,
when they read the history of the opposition offered by cupidity and
prejudice to the improvements of the nineteenth century, may smile in
their turn. 
%[149]
\footnote{ John Cresset's Reasons for suppressing Stage Coaches,
1672. These reason were afterwards inserted in a tract, entitled "The
Grand Concern of England explained, 1673." Cresset's attack on stage
coaches called forth some answers which I have consulted.}


In spite of the attractions of the flying coaches, it was still usual
for men who enjoyed health and vigour, and who were not encumbered by
much baggage, to perform long journeys on horseback. If the traveller
wished to move expeditiously he rode post. Fresh saddle horses and
guides were to be procured at convenient distances along all the great
lines of road. The charge was threepence a mile for each horse, and
fourpence a stage for the guide. In this manner, when the ways were
good, it was possible to travel, for a considerable time, as rapidly
as by any conveyance known in England, till vehicles were propelled by
steam. There were as yet no post chaises; nor could those who rode
in their own coaches ordinarily procure a change of horses. The King,
however, and the great officers of state were able to command relays.
Thus Charles commonly went in one day from Whitehall to New-market, a
distance of about fifty-five miles through a level country; and this was
thought by his subjects a proof of great activity. Evelyn performed the
same journey in company with the Lord Treasurer Clifford. The coach was
drawn by six horses, which were changed at Bishop Stortford and again at
Chesterford. The travellers reached Newmarket at night. Such a mode of
conveyance seems to have been considered as a rare luxury confined to
princes and ministers. 
%[150]
\footnote{ Chamberlayne's State of England, 1684; North's Examen,
105; Evelyn's Diary, Oct. 9,10, 1671.}


Whatever might be the way in which a journey was performed, the
travellers, unless they were numerous and well armed, ran considerable
risk of being stopped and plundered. The mounted highwayman, a marauder
known to our generation only from books, was to be found on every main
road. The waste tracts which lay on the great routes near London were
especially haunted by plunderers of this class. Hounslow Heath, on the
Great Western Road, and Finchley Common, on the Great Northern Road,
were perhaps the most celebrated of these spots. The Cambridge scholars
trembled when they approached Epping Forest, even in broad daylight.
Seamen who had just been paid off at Chatham were often compelled
to deliver their purses on Gadshill, celebrated near a hundred years
earlier by the greatest of poets as the scene of the depredations of
Falstaff. The public authorities seem to have been often at a loss
how to deal with the plunderers. At one time it was announced in the
Gazette, that several persons, who were strongly suspected of being
highwaymen, but against whom there was not sufficient evidence, would be
paraded at Newgate in riding dresses: their horses would also be shown;
and all gentlemen who had been robbed were invited to inspect this
singular exhibition. On another occasion a pardon was publicly offered
to a robber if he would give up some rough diamonds, of immense value,
which he had taken when he stopped the Harwich mail. A short time after
appeared another proclamation, warning the innkeepers that the eye
of the government was upon them. Their criminal connivance, it was
affirmed, enabled banditti to infest the roads with impunity. That these
suspicions were not without foundation, is proved by the dying speeches
of some penitent robbers of that age, who appear to have received from
the innkeepers services much resembling those which Farquhar's Boniface
rendered to Gibbet. 
%[151]
\footnote{ See the London Gazette, May 14, 1677, August 4, 1687,
Dec. 5, 1687. The last confession of Augustin King, who was the son of
an eminent divine, and had been educated at Cambridge but was hanged at
Colchester in March, 1688, is highly curious.}


It was necessary to the success and even to the safety of the highwayman
that he should be a bold and skilful rider, and that his manners and
appearance should be such as suited the master of a fine horse. He
therefore held an aristocratical position in the community of thieves,
appeared at fashionable coffee houses and gaming houses, and betted with
men of quality on the race ground. 
%[152]
\footnote{ Aimwell. Pray sir, han't I seen your face at Will's
coffeehouse? Gibbet. Yes sir, and at White's too.--Beaux' Stratagem.}
 Sometimes, indeed, he was a man
of good family and education. A romantic interest therefore attached,
and perhaps still attaches, to the names of freebooters of this class.
The vulgar eagerly drank in tales of their ferocity and audacity, of
their occasional acts of generosity and good nature, of their amours,
of their miraculous escapes, of their desperate struggles, and of their
manly bearing at the bar and in the cart. Thus it was related of William
Nevison, the great robber of Yorkshire, that he levied a quarterly
tribute on all the northern drovers, and, in return, not only spared
them himself, but protected them against all other thieves; that he
demanded purses in the most courteous manner; that he gave largely to
the poor what he had taken from the rich; that his life was once spared
by the royal clemency, but that he again tempted his fate, and at length
died, in 1685, on the gallows of York. 
%[153]
\footnote{ Gent's History of York. Another marauder of the same
description, named Biss, was hanged at Salisbury in 1695. In a ballad
which is in the Pepysian Library, he is represented as defending himself
thus before the Judge:

     "What say you now, my honoured Lord
     What harm was there in this?
     Rich, wealthy misers were abhorred
     By brave, freehearted Biss."}
 It was related how Claude
Duval, the French page of the Duke of Richmond, took to the road, became
captain of a formidable gang, and had the honour to be named first in a
royal proclamation against notorious offenders; how at the head of his
troop he stopped a lady's coach, in which there was a booty of four
hundred pounds; how he took only one hundred, and suffered the fair
owner to ransom the rest by dancing a coranto with him on the heath;
how his vivacious gallantry stole away the hearts of all women; how
his dexterity at sword and pistol made him a terror to all men; how, at
length, in the year 1670, he was seized when overcome by wine; how dames
of high rank visited him in prison, and with tears interceded for his
life; how the King would have granted a pardon, but for the interference
of Judge Morton, the terror of highwaymen, who threatened to resign his
office unless the law were carried into full effect; and how, after the
execution, the corpse lay in state with all the pomp of scutcheons, wax
lights, black hangings and mutes, till the same cruel Judge, who
had intercepted the mercy of the crown, sent officers to disturb the
obsequies. 
%[154]
\footnote{ Pope's Memoirs of Duval, published immediately after the
execution. Oates's Eikwg basilikh, Part I.}
 In these anecdotes there is doubtless a large mixture
of fable; but they are not on that account unworthy of being recorded;
for it is both an authentic and an important fact that such tales,
whether false or true, were heard by our ancestors with eagerness and
faith.

All the various dangers by which the traveller was beset were greatly
increased by darkness. He was therefore commonly desirous of having
the shelter of a roof during the night; and such shelter it was not
difficult to obtain. From a very early period the inns of England
had been renowned. Our first great poet had described the excellent
accommodation which they afforded to the pilgrims of the fourteenth
century. Nine and twenty persons, with their horses, found room in the
wide chambers and stables of the Tabard in Southwark. The food was of
the best, and the wines such as drew the company on to drink largely.
Two hundred years later, under the reign of Elizabeth, William Harrison
gave a lively description of the plenty and comfort of the great
hostelries. The Continent of Europe, he said, could show nothing like
them. There were some in which two or three hundred people, with their
horses, could without difficulty be lodged and fed. The bedding, the
tapestry, above all, the abundance of clean and fine linen was matter
of wonder. Valuable plate was often set on the tables. Nay, there were
signs which had cost thirty or forty pounds. In the seventeenth century
England abounded with excellent inns of every rank. The traveller
sometimes, in a small village, lighted on a public house such as Walton
has described, where the brick floor was swept clean, where the walls
were stuck round with ballads, where the sheets smelt of lavender, and
where a blazing fire, a cup of good ale, and a dish of trouts fresh
from the neighbouring brook, were to be procured at small charge. At
the larger houses of entertainment were to be found beds hung with silk,
choice cookery, and claret equal to the best which was drunk in London.

%[155]
\footnote{ See the prologue to the Canterbury Tales, Harrison's
Historical Description of the Island of Great Britain, and Pepys's
account of his tour in the summer of 1668. The excellence of the English
inns is noticed in the Travels of the Grand Duke Cosmo.}
 The innkeepers too, it was said, were not like other innkeepers.
On the Continent the landlord was the tyrant of those who crossed the
threshold. In England he was a servant. Never was an Englishman more
at home than when he took his ease in his inn. Even men of fortune, who
might in their own mansions have enjoyed every luxury, were often in
the habit of passing their evenings in the parlour of some neighbouring
house of public entertainment. They seem to have thought that comfort
and freedom could in no other place be enjoyed with equal perfection.
This feeling continued during many generations to be a national
peculiarity. The liberty and jollity of inns long furnished matter to
our novelists and dramatists. Johnson declared that a tavern chair was
the throne of human felicity; and Shenstone gently complained that no
private roof, however friendly, gave the wanderer so warm a welcome as
that which was to be found at an inn.

Many conveniences, which were unknown at Hampton Court and Whitehall in
the seventeenth century, are in all modern hotels. Yet on the whole it
is certain that the improvement of our houses of public entertainment
has by no means kept pace with the improvement of our roads and of our
conveyances. Nor is this strange; for it is evident that, all other
circumstances being supposed equal, the inns will be best where the
means of locomotion are worst. The quicker the rate of travelling, the
less important is it that there should be numerous agreeable resting
places for the traveller. A hundred and sixty years ago a person who
came up to the capital from a remote county generally required, by the
way, twelve or fifteen meals, and lodging for five or six nights. If he
were a great man, he expected the meals and lodging to be comfortable,
and even luxurious. At present we fly from York or Exeter to London by
the light of a single winter's day. At present, therefore, a traveller
seldom interrupts his journey merely for the sake of rest and
refreshment. The consequence is that hundreds of excellent inns
have fallen into utter decay. In a short time no good houses of that
description will be found, except at places where strangers are likely
to be detained by business or pleasure.

The mode in which correspondence was carried on between distant places
may excite the scorn of the present generation; yet it was such as might
have moved the admiration and envy of the polished nations of antiquity,
or of the contemporaries of Raleigh and Cecil. A rude and imperfect
establishment of posts for the conveyance of letters had been set up by
Charles the First, and had been swept away by the civil war. Under the
Commonwealth the design was resumed. At the Restoration the proceeds of
the Post Office, after all expenses had been paid, were settled on the
Duke of York. On most lines of road the mails went out and came in only
on the alternate days. In Cornwall, in the fens of Lincolnshire, and
among the hills and lakes of Cumberland, letters were received only once
a week. During a royal progress a daily post was despatched from the
capital to the place where the court sojourned. There was also daily
communication between London and the Downs; and the same privilege was
sometimes extended to Tunbridge Wells and Bath at the seasons when those
places were crowded by the great. The bags were carried on horseback day
and night at the rate of about five miles an hour. 
%[156]
\footnote{ Stat. 12 Car. II. c. 36; Chamberlayne's State of England,
1684; Angliae Metropolis, 1690; London Gazette, June 22, 1685, August
15, 1687.}


The revenue of this establishment was not derived solely from the charge
for the transmission of letters. The Post Office alone was entitled to
furnish post horses; and, from the care with which this monopoly was
guarded, we may infer that it was found profitable. 
%[157]
\footnote{ Lond. Gaz., Sept. 14, 1685.}
 If, indeed, a
traveller had waited half an hour without being supplied he might hire a
horse wherever he could.

To facilitate correspondence between one part of London and another was
not originally one of the objects of the Post Office. But, in the
reign of Charles the Second, an enterprising citizen of London, William
Dockwray, set up, at great expense, a penny post, which delivered
letters and parcels six or eight times a day in the busy and crowded
streets near the Exchange, and four times a day in the outskirts of
the capital. This improvement was, as usual, strenuously resisted. The
porters complained that their interests were attacked, and tore down the
placards in which the scheme was announced to the public. The excitement
caused by Godfrey's death, and by the discovery of Coleman's papers, was
then at the height. A cry was therefore raised that the penny post was a
Popish contrivance. The great Doctor Oates, it was affirmed, had hinted
a suspicion that the Jesuits were at the bottom of the scheme, and that
the bags, if examined, would be found full of treason. 
%[158]
\footnote{ Smith's Current intelligence, March 30, and April 3,
1680.}
 The utility
of the enterprise was, however, so great and obvious that all opposition
proved fruitless. As soon as it became clear that the speculation would
be lucrative, the Duke of York complained of it as an infraction of his
monopoly; and the courts of law decided in his favour. 
%[159]
\footnote{ Anglias Metropolis, 1690.}


The revenue of the Post Office was from the first constantly increasing.
In the year of the Restoration a committee of the House of Commons,
after strict enquiry, had estimated the net receipt at about twenty
thousand pounds. At the close of the reign of Charles the Second, the
net receipt was little short of fifty thousand pounds; and this was then
thought a stupendous sum. The gross receipt was about seventy thousand
pounds. The charge for conveying a single letter was twopence for eighty
miles, and threepence for a longer distance. The postage increased in
proportion to the weight of the packet. 
%[160]
\footnote{ Commons' Journals, Sept. 4, 1660, March 1, 1688-9;
Chamberlayne, 1684; Davenant on the Public Revenue, Discourse IV.}
 At present a single letter
is carried to the extremity of Scotland or of Ireland for a penny; and
the monopoly of post horses has long ceased to exist. Yet the gross
annual receipts of the department amount to more than eighteen hundred
thousand pounds, and the net receipts to more than seven hundred
thousand pounds. It is, therefore, scarcely possible to doubt that the
number of letters now conveyed by mail is seventy times the number which
was so conveyed at the time of the accession of James the Second. 
%[161]
\footnote{ I have left the text as it stood in 1848. In the year
1856 the gross receipt of the Post Office was more than 2,800,000£.; and
the net receipt was about 1,200,000£. The number of letters conveyed by
post was 478,000,000. (1857).}


No part of the load which the old mails carried out was more important
than the newsletters. In 1685 nothing like the London daily paper of
our time existed, or could exist. Neither the necessary capital nor
the necessary skill was to be found. Freedom too was wanting, a want as
fatal as that of either capital or skill. The press was not indeed at
that moment under a general censorship. The licensing act, which had
been passed soon after the Restoration, had expired in 1679. Any person
might therefore print, at his own risk, a history, a sermon, or a poem,
without the previous approbation of any officer; but the Judges were
unanimously of opinion that this liberty did not extend to Gazettes, and
that, by the common law of England, no man, not authorised by the crown,
had a right to publish political news. 
%[162]
\footnote{ London Gazette, May 5, and 17, 1680.}
 While the Whig party was
still formidable, the government thought it expedient occasionally to
connive at the violation of this rule. During the great battle of the
Exclusion Bill, many newspapers were suffered to appear, the Protestant
Intelligence, the Current Intelligence, the Domestic Intelligence, the
True News, the London Mercury. 
%[163]
\footnote{ There is a very curious, and, I should think, unique
collection of these papers in the British Museum.}
 None of these was published oftener
than twice a week. None exceeded in size a single small leaf. The
quantity of matter which one of them contained in a year was not more
than is often found in two numbers of the Times. After the defeat of the
Whigs it was no longer necessary for the King to be sparing in the
use of that which all his Judges had pronounced to be his undoubted
prerogative. At the close of his reign no newspaper was suffered to
appear without his allowance: and his allowance was given exclusively
to the London Gazette. The London Gazette came out only on Mondays and
Thursdays. The contents generally were a royal proclamation, two or
three Tory addresses, notices of two or three promotions, an account
of a skirmish between the imperial troops and the Janissaries on the
Danube, a description of a highwayman, an announcement of a grand
cockfight between two persons of honour, and an advertisement offering a
reward for a strayed dog. The whole made up two pages of moderate size.
Whatever was communicated respecting matters of the highest moment was
communicated in the most meagre and formal style. Sometimes, indeed,
when the government was disposed to gratify the public curiosity
respecting an important transaction, a broadside was put forth giving
fuller details than could be found in the Gazette: but neither the
Gazette nor any supplementary broadside printed by authority ever
contained any intelligence which it did not suit the purposes of the
Court to publish. The most important parliamentary debates, the most
important state trials recorded in our history, were passed over in
profound silence. 
%[164]
\footnote{ For example, there is not a word in the Gazette about
the important parliamentary proceedings of November, 1685, or about the
trial and acquittal of the Seven Bishops.}
 In the capital the coffee houses supplied in
some measure the place of a journal. Thither the Londoners flocked, as
the Athenians of old flocked to the market place, to hear whether
there was any news. There men might learn how brutally a Whig, had been
treated the day before in Westminster Hall, what horrible accounts the
letters from Edinburgh gave of the torturing of Covenanters, how grossly
the Navy Board had cheated the crown in the Victualling of the fleet,
and what grave charges the Lord Privy Seal had brought against the
Treasury in the matter of the hearth money. But people who lived at a
distance from the great theatre of political contention could be
kept regularly informed of what was passing there only by means of
newsletters. To prepare such letters became a calling in London, as it
now is among the natives of India. The newswriter rambled from coffee
room to coffee room, collecting reports, squeezed himself into the
Sessions House at the Old Bailey if there was an interesting trial, nay
perhaps obtained admission to the gallery of Whitehall, and noticed how
the King and Duke looked. In this way he gathered materials for weekly
epistles destined to enlighten some county town or some bench of rustic
magistrates. Such were the sources from which the inhabitants of the
largest provincial cities, and the great body of the gentry and clergy,
learned almost all that they knew of the history of their own time. We
must suppose that at Cambridge there were as many persons curious
to know what was passing in the world as at almost any place in the
kingdom, out of London. Yet at Cambridge, during a great part of the
reign of Charles the Second, the Doctors of Laws and the Masters of
Arts had no regular supply of news except through the London Gazette.
At length the services of one of the collectors of intelligence in
the capital were employed. That was a memorable day on which the first
newsletter from London was laid on the table of the only coffee room
in Cambridge. 
%[165]
\footnote{ Roger North's Life of Dr. John North. On the subject of
newsletters, see the Examen, 133.}
 At the seat of a man of fortune in the country the
newsletter was impatiently expected. Within a week after it had arrived
it had been thumbed by twenty families. It furnished the neighboring
squires with matter for talk over their October, and the neighboring
rectors with topics for sharp sermons against Whiggery or Popery.
Many of these curious journals might doubtless still be detected by a
diligent search in the archives of old families. Some are to be found
in our public libraries; and one series, which is not the least valuable
part of the literary treasures collected by Sir James Mackintosh, will
be occasionally quoted in the course of this work. 
%[166]
\footnote{ I take this opportunity of expressing my warm gratitude
to the family of my dear and honoured friend sir James Mackintosh
for confiding to me the materials collected by him at a time when he
meditated a work similar to that which I have undertaken. I have never
seen, and I do not believe that there anywhere exists, within the same
compass, so noble a collection of extracts from public and private
archives The judgment with which sir James in great masses of the
rudest ore of history, selected what was valuable, and rejected what was
worthless, can be fully appreciated only by one who has toiled after him
in the same mine.}


It is scarcely necessary to say that there were then no provincial
newspapers. Indeed, except in the capital and at the two Universities,
there was scarcely a printer in the kingdom. The only press in England
north of Trent appears to have been at York. 
%[167]
\footnote{ Life of Thomas Gent. A complete list of all printing
houses in 1724 will be found in Nichols's Literary Anecdotae of the
eighteenth century. There had then been a great increase within a few
years in the number of presses, and yet there were thirty-four counties
in which there was no printer, one of those counties being Lancashire.}


It was not only by means of the London Gazette that the government
undertook to furnish political instruction to the people. That journal
contained a scanty supply of news without comment. Another journal,
published under the patronage of the court, consisted of comment without
news. This paper, called the Observator, was edited by an old Tory
pamphleteer named Roger Lestrange. Lestrange was by no means deficient
in readiness and shrewdness; and his diction, though coarse, and
disfigured by a mean and flippant jargon which then passed for wit in
the green room and the tavern, was not without keenness and vigour. But
his nature, at once ferocious and ignoble, showed itself in every line
that he penned. When the first Observators appeared there was some
excuse for his acrimony. The Whigs were then powerful; and he had to
contend against numerous adversaries, whose unscrupulous violence might
seem to justify unsparing retaliation. But in 1685 all the opposition
had been crushed. A generous spirit would have disdained to insult a
party which could not reply, and to aggravate the misery of prisoners,
of exiles, of bereaved families: but; from the malice of Lestrange the
grave was no hiding place, and the house of mourning no sanctuary. In
the last month of the reign of Charles the Second, William Jenkyn, an
aged dissenting pastor of great note, who had been cruelly persecuted
for no crime but that of worshipping God according to the fashion
generally followed throughout protestant Europe, died of hardships and
privations at Newgate. The outbreak of popular sympathy could not be
repressed. The corpse was followed to the grave by a train of a hundred
and fifty coaches. Even courtiers looked sad. Even the unthinking King
showed some signs of concern. Lestrange alone set up a howl of savage
exultation, laughed at the weak compassion of the Trimmers, proclaimed
that the blasphemous old impostor had met with a most righteous
punishment, and vowed to wage war, not only to the death, but after
death, with all the mock saints and martyrs. 
%[168]
\footnote{ Observator, Jan. 29, and 31, 1685; Calamy's Life of
Baxter; Nonconformist Memorial.}
 Such was the spirit
of the paper which was at this time the oracle of the Tory party, and
especially of the parochial clergy.

Literature which could be carried by the post bag then formed the
greater part of the intellectual nutriment ruminated by the country
divines and country justices. The difficulty and expense of conveying
large packets from place to place was so great, that an extensive work
was longer in making its way from Paternoster Row to Devonshire or
Lancashire than it now is in reaching Kentucky. How scantily a rural
parsonage was then furnished, even with books the most necessary to a
theologian, has already been remarked. The houses of the gentry were
not more plentifully supplied. Few knights of the shire had libraries so
good as may now perpetually be found in a servants' hall or in the back
parlour of a small shopkeeper. An esquire passed among his neighbours
for a great scholar, if Hudibras and Baker's Chronicle, Tarlton's Jests
and the Seven Champions of Christendom, lay in his hall window among
the fishing rods and fowling pieces. No circulating library, no book
society, then existed even in the capital: but in the capital those
students who could not afford to purchase largely had a resource. The
shops of the great booksellers, near Saint Paul's Churchyard, were
crowded every day and all day long with readers; and a known customer
was often permitted to carry a volume home. In the country there was
no such accommodation; and every man was under the necessity of buying
whatever he wished to read. 
%[169]
\footnote{ Cotton seems, from his Angler, to have found room for his
whole library in his hall window; and Cotton was a man of letters. Even
when Franklin first visited London in 1724, circulating libraries were
unknown there. The crowd at the booksellers' shops in Little Britain is
mentioned by Roger North in his life of his brother John.}


As to the lady of the manor and her daughters, their literary stores
generally consisted of a prayer book and receipt book. But in truth
they lost little by living in rural seclusion. For, even in the highest
ranks, and in those situations which afforded the greatest facilities
for mental improvement, the English women of that generation were
decidedly worse educated than they have been at any other time since
the revival of learning. At an early period they had studied the
masterpieces of ancient genius. In the present day they seldom bestow
much attention on the dead languages; but they are familiar with the
tongue of Pascal and Moliere, with the tongue of Dante and Tasso,
with the tongue of Goethe and Schiller; nor is there any purer or more
graceful English than that which accomplished women now speak and write.
But, during the latter part of the seventeenth century, the culture
of the female mind seems to have been almost entirely neglected. If
a damsel had the least smattering of literature she was regarded as a
prodigy. Ladies highly born, highly bred, and naturally quick witted,
were unable to write a line in their mother tongue without solecisms
and faults of spelling such as a charity girl would now be ashamed to
commit. 
%[170]
\footnote{ One instance will suffice. Queen Mary, the daughter of
James, had excellent natural abilities, had been educated by a Bishop,
was fond of history and poetry and was regarded by very eminent men as a
superior woman. There is, in the library at the Hague, a superb English
Bible which was delivered to her when she was crowned in Westminster
Abbey. In the titlepage are these words in her own hand, "This book was
given the King and I, at our crownation. Marie R."}


The explanation may easily be found. Extravagant licentiousness,
the natural effect of extravagant austerity, was now the mode;
and licentiousness had produced its ordinary effect, the moral and
intellectual degradation of women. To their personal beauty, it was the
fashion to pay rude and impudent homage. But the admiration and desire
which they inspired were seldom mingled with respect, with affection,
or with any chivalrous sentiment. The qualities which fit them to
be companions, advisers, confidential friends, rather repelled than
attracted the libertines of Whitehall. In that court a maid of honour,
who dressed in such a manner as to do full justice to a white bosom,
who ogled significantly, who danced voluptuously, who excelled in pert
repartee, who was not ashamed to romp with Lords of the Bedchamber and
Captains of the Guards, to sing sly verses with sly expression, or to
put on a page's dress for a frolic, was more likely to be followed and
admired, more likely to be honoured with royal attentions, more likely
to win a rich and noble husband than Jane Grey or Lucy Hutchinson would
have been. In such circumstances the standard of female attainments was
necessarily low; and it was more dangerous to be above that standard
than to be beneath it. Extreme ignorance and frivolity were thought less
unbecoming in a lady than the slightest tincture of pedantry. Of the
too celebrated women whose faces we still admire on the walls of Hampton
Court, few indeed were in the habit of reading anything more valuable
than acrostics, lampoons, and translations of the Clelia and the Grand
Cyrus.

The literary acquirements, even of the accomplished gentlemen of that
generation, seem to have been somewhat less solid and profound than at
an earlier or a later period. Greek learning, at least, did not flourish
among us in the days of Charles the Second, as it had flourished before
the civil war, or as it again flourished long after the Revolution.
There were undoubtedly scholars to whom the whole Greek literature,
from Homer to Photius, was familiar: but such scholars were to be found
almost exclusively among the clergy resident at the Universities, and
even at the Universities were few, and were not fully appreciated. At
Cambridge it was not thought by any means necessary that a divine should
be able to read the Gospels in the original. 
%[171]
\footnote{ Roger North tells us that his brother John, who was Greek
professor at Cambridge, complained bitterly of the general neglect of
the Greek tongue among the academical clergy.}
 Nor was the standard
at Oxford higher. When, in the reign of William the Third, Christ
Church rose up as one man to defend the genuineness of the Epistles
of Phalaris, that great college, then considered as the first seat
of philology in the kingdom, could not muster such a stock of Attic
learning as is now possessed by several youths at every great public
school. It may easily be supposed that a dead language, neglected at the
Universities, was not much studied by men of the world. In a former age
the poetry and eloquence of Greece had been the delight of Raleigh and
Falkland. In a later age the poetry and eloquence of Greece were the
delight of Pitt and Fox, of Windham and Grenville. But during the
latter part of the seventeenth century there was in England scarcely one
eminent statesman who could read with enjoyment a page of Sophocles or
Plato.

Good Latin scholars were numerous. The language of Rome, indeed, had not
altogether lost its imperial prerogatives, and was still, in many parts
of Europe, almost indispensable to a traveller or a negotiator. To speak
it well was therefore a much more common accomplishment shall in our
time; and neither Oxford nor Cambridge wanted poets who, on a great
occasion, could lay at the foot of the throne happy imitations of
the verses in which Virgil and Ovid had celebrated the greatness of
Augustus.

Yet even the Latin was giving way to a younger rival. France united at
that time almost every species of ascendency. Her military glory was
at the height. She had vanquished mighty coalitions. She had dictated
treaties. She had subjugated great cities and provinces. She had forced
the Castilian pride to yield her the precedence. She had summoned
Italian princes to prostrate themselves at her footstool. Her authority
was supreme in all matters of good breeding, from a duel to a minuet.
She determined how a gentleman's coat must be cut, how long his peruke
must be, whether his heels must be high or low, and whether the lace
on his hat must be broad or narrow. In literature she gave law to the
world. The fame of her great writers filled Europe. No other country
could produce a tragic poet equal to Racine, a comic poet equal to
Moliere, a trifler so agreeable as La Fontaine, a rhetorician so skilful
as Bossuet. The literary glory of Italy and of Spain had set; that of
Germany had not yet dawned. The genius, therefore, of the eminent men
who adorned Paris shone forth with a splendour which was set off to full
advantage by contrast. France, indeed, had at that time an empire over
mankind, such as even the Roman Republic never attained. For, when Rome
was politically dominant, she was in arts and letters the humble pupil
of Greece. France had, over the surrounding countries, at once the
ascendency which Rome had over Greece, and the ascendency which Greece
had over Rome. French was fast becoming the universal language, the
language of fashionable society, the language of diplomacy. At several
courts princes and nobles spoke it more accurately and politely than
their mother tongue. In our island there was less of this servility than
on the Continent. Neither our good nor our bad qualities were those of
imitators. Yet even here homage was paid, awkwardly indeed and sullenly,
to the literary supremacy of our neighbours. The melodious Tuscan, so
familiar to the gallants and ladies of the court of Elizabeth, sank into
contempt. A gentleman who quoted Horace or Terence was considered in
good company as a pompous pedant. But to garnish his conversation with
scraps of French was the best proof which he could give of his parts
and attainments. 
%[172]
\footnote{ Butler, in a satire of great asperity, says,

     "For, though to smelter words of Greek
     And Latin be the rhetorique
     Of pedants counted, and vainglorious,
     To smatter French is meritorious."}
 New canons of criticism, new models of style
came into fashion. The quaint ingenuity which had deformed the verses of
Donne, and had been a blemish on those of Cowley, disappeared from our
poetry. Our prose became less majestic, less artfully involved, less
variously musical than that of an earlier age, but more lucid, more
easy, and better fitted for controversy and narrative. In these changes
it is impossible not to recognise the influence of French precept and of
French example. Great masters of our language, in their most dignified
compositions, affected to use French words, when English words, quite as
expressive and sonorous, were at hand: 
%[173]
\footnote{ The most offensive instance which I remember is in a poem
on the coronation of Charles the Second by Dryden, who certainly could
not plead poverty as an excuse for borrowing words from any foreign
tongue:--

     "Hither in summer evenings you repair
     To taste the fraicheur of the cooler air."}
 and from France was imported
the tragedy in rhyme, an exotic which, in our soil, drooped, and
speedily died.

It would have been well if our writers had also copied the decorum which
their great French contemporaries, with few exceptions, preserved; for
the profligacy of the English plays, satires, songs, and novels of that
age is a deep blot on our national fame. The evil may easily be traced
to its source. The wits and the Puritans had never been on friendly
terms. There was no sympathy between the two classes. They looked on
the whole system of human life from different points and in different
lights. The earnest of each was the jest of the other. The pleasures
of each were the torments of the other. To the stern precisian even the
innocent sport of the fancy seemed a crime. To light and festive natures
the solemnity of the zealous brethren furnished copious matter of
ridicule. From the Reformation to the civil war, almost every writer,
gifted with a fine sense of the ludicrous, had taken some opportunity of
assailing the straighthaired, snuffling, whining saints, who christened
their children out of the Book of Nehemiah, who groaned in spirit at
the sight of Jack in the Green, and who thought it impious to taste plum
porridge on Christmas day. At length a time came when the laughers began
to look grave in their turn. The rigid, ungainly zealots, after having
furnished much good sport during two generations, rose up in arms,
conquered, ruled, and, grimly smiling, trod down under their feet the
whole crowd of mockers. The wounds inflicted by gay and petulant malice
were retaliated with the gloomy and implacable malice peculiar to bigots
who mistake their own rancour for virtue. The theatres were closed.
The players were flogged. The press was put under the guardianship of
austere licensers. The Muses were banished from their own favourite
haunts, Cambridge and Oxford. Cowly, Crashaw, and Cleveland were ejected
from their fellowships. The young candidate for academical honours was
no longer required to write Ovidian epistles or Virgilian pastorals, but
was strictly interrogated by a synod of lowering Supralapsarians as to
the day and hour when he experienced the new birth. Such a system was
of course fruitful of hypocrites. Under sober clothing and under visages
composed to the expression of austerity lay hid during several years
the intense desire of license and of revenge. At length that desire was
gratified. The Restoration emancipated thousands of minds from a yoke
which had become insupportable. The old fight recommenced, but with an
animosity altogether new. It was now not a sportive combat, but a war to
the death. The Roundhead had no better quarter to expect from those whom
he had persecuted than a cruel slavedriver can expect from insurgent
slaves still bearing the marks of his collars and his scourges.

The war between wit and Puritanism soon became a war between wit and
morality. The hostility excited by a grotesque caricature of virtue did
not spare virtue herself. Whatever the canting Roundhead had regarded
with reverence was insulted. Whatever he had proscribed was favoured.
Because he had been scrupulous about trifles, all scruples were treated
with derision. Because he had covered his failings with the mask of
devotion, men were encouraged to obtrude with Cynic impudence all their
most scandalous vices on the public eye. Because he had punished illicit
love with barbarous severity, virgin purity and conjugal fidelity were
made a jest. To that sanctimonious jargon which was his Shibboleth, was
opposed another jargon not less absurd and much more odious. As he never
opened his mouth except in scriptural phrase, the new breed of wits and
fine gentlemen never opened their mouths without uttering ribaldry of
which a porter would now be ashamed, and without calling on their Maker
to curse them, sink them, confound them, blast them, and damn them.

It is not strange, therefore, that our polite literature, when it
revived with the revival of the old civil and ecclesiastical polity,
should have been profoundly immoral. A few eminent men, who belonged to
an earlier and better age, were exempt from the general contagion. The
verse of Waller still breathed the sentiments which had animated a more
chivalrous generation. Cowley, distinguished as a loyalist and as a man
of letters, raised his voice courageously against the immorality which
disgraced both letters and loyalty. A mightier poet, tried at once by
pain, danger, poverty, obloquy, and blindness, meditates, undisturbed by
the obscene tumult which raged all around him, a song so sublime and so
holy that it would not have misbecome the lips of those ethereal
Virtues whom he saw, with that inner eye which no calamity could darken,
flinging down on the jasper pavement their crowns of amaranth and gold.
The vigourous and fertile genius of Butler, if it did not altogether
escape the prevailing infection, took the disease in a mild form. But
these were men whose minds had been trained in a world which had passed
away. They gave place in no long time to a younger generation of
wits; and of that generation, from Dryden down to Durfey, the common
characteristic was hard-hearted, shameless, swaggering licentiousness,
at once inelegant and inhuman. The influence of these writers was
doubtless noxious, yet less noxious than it would have been had they
been less depraved. The poison which they administered was so strong
that it was, in no long time, rejected with nausea. None of them
understood the dangerous art of associating images of unlawful pleasure
with all that is endearing and ennobling. None of them was aware that a
certain decorum is essential even to voluptuousness, that drapery may
be more alluring than exposure, and that the imagination may be far more
powerfully moved by delicate hints which impel it to exert itself, than
by gross descriptions which it takes in passively.

The spirit of the Antipuritan reaction pervades almost the whole polite
literature of the reign of Charles the Second. But the very quintessence
of that spirit will be found in the comic drama. The playhouses, shut
by the meddling fanatic in the day of his power, were again crowded. To
their old attractions new and more powerful attractions had been added.
Scenery, dresses, and decorations, such as would now be thought mean or
absurd, but such as would have been esteemed incredibly magnificent by
those who, early in the seventeenth century, sate on the filthy benches
of the Hope, or under the thatched roof of the Rose, dazzled the eyes
of the multitude. The fascination of sex was called in to aid the
fascination of art: and the young spectator saw, with emotions unknown
to the contemporaries of Shakspeare and Johnson, tender and sprightly
heroines personated by lovely women. From the day on which the theatres
were reopened they became seminaries of vice; and the evil propagated
itself. The profligacy of the representations soon drove away sober
people. The frivolous and dissolute who remained required every year
stronger and stronger stimulants. Thus the artists corrupted the
spectators, and the spectators the artists, till the turpitude of the
drama became such as must astonish all who are not aware that extreme
relaxation is the natural effect of extreme restraint, and that an age
of hypocrisy is, in the regular course of things, followed by all age of
impudence.

Nothing is more characteristic of the times than the care with which
the poets contrived to put all their loosest verses into the mouths of
women. The compositions in which the greatest license was taken were the
epilogues. They were almost always recited by favourite actresses; and
nothing charmed the depraved audience so much as to hear lines grossly
indecent repeated by a beautiful girl, who was supposed to have not yet
lost her innocence 
%[174]
\footnote{ Jeremy Collier has censured this odious practice with his
usual force and keenness.}
.

Our theatre was indebted in that age for many plots and characters
to Spain, to France, and to the old English masters: but whatever our
dramatists touched they tainted. In their imitations the houses of
Calderon's stately and highspirited Castilian gentlemen became sties of
vice, Shakspeare's Viola a procuress, Moliere's Misanthrope a ravisher,
Moliere's Agnes an adulteress. Nothing could be so pure or so heroic but
that it became foul and ignoble by transfusion through those foul and
ignoble minds.

Such was the state of the drama; and the drama was the department of
polite literature in which a poet had the best chance of obtaining a
subsistence by his pen. The sale of books was so small that a man of the
greatest name could hardly expect more than a pittance for the copyright
of the best performance. There cannot be a stronger instance than the
fate of Dryden's last production, the Fables. That volume was published
when he was universally admitted to be the chief of living English
poets. It contains about twelve thousand lines. The versification is
admirable, the narratives and descriptions full of life. To this day
Palamon and Arcite, Cymon and Iphigenia, Theodore and Honoria, are
the delight both of critics and of schoolboys. The collection includes
Alexander's Feast, the noblest ode in our language. For the copyright
Dryden received two hundred and fifty pounds, less than in our days has
sometimes been paid for two articles in a review. 
%[175]
\footnote{ The contrast will be found in Sir Walter Scott's edition
of Dryden.}
 Nor does the
bargain seem to have been a hard one. For the book went off slowly; and
the second edition was not required till the author had been ten years
in his grave. By writing for the theatre it was possible to earn a much
larger sum with much less trouble. Southern made seven hundred pounds by
one play. 
%[176]
\footnote{ See the Life of Southern. by Shiels.}
 Otway was raised from beggary to temporary affluence
by the success of his Don Carlos. 
%[177]
\footnote{ See Rochester's Trial of the Poets.}
 Shadwell cleared a hundred and
thirty pounds by a single representation of the Squire of Alsatia. 
%[178]
\footnote{ Some Account of the English Stage.}

The consequence was that every man who had to live by his wit wrote
plays, whether he had any internal vocation to write plays or not.
It was thus with Dryden. As a satirist he has rivalled Juvenal. As a
didactic poet he perhaps might, with care and meditation, have rivalled
Lucretius. Of lyric poets he is, if not the most sublime, the most
brilliant and spiritstirring. But nature, profuse to him of many rare
gifts, had withheld from him the dramatic faculty. Nevertheless all the
energies of his best years were wasted on dramatic composition. He
had too much judgment not to be aware that in the power of exhibiting
character by means of dialogue he was deficient. That deficiency he
did his best to conceal, sometimes by surprising and amusing incidents,
sometimes by stately declamation, sometimes by harmonious numbers,
sometimes by ribaldry but too well suited to the taste of a profane and
licentious pit. Yet he never obtained any theatrical success equal to
that which rewarded the exertions of some men far inferior to him in
general powers. He thought himself fortunate if he cleared a hundred
guineas by a play; a scanty remuneration, yet apparently larger than he
could have earned in any other way by the same quantity of labour. 
%[179]
\footnote{ Life of Southern, by Shiels.}


The recompense which the wits of that age could obtain from the public
was so small, that they were under the necessity of eking out
their incomes by levying contributions on the great. Every rich
and goodnatured lord was pestered by authors with a mendicancy
so importunate, and a flattery so abject, as may in our time seem
incredible. The patron to whom a work was inscribed was expected to
reward the writer with a purse of gold. The fee paid for the dedication
of a book was often much larger than the sum which any publisher would
give for the copyright. Books were therefore frequently printed merely
that they might be dedicated. This traffic in praise produced the effect
which might have been expected. Adulation pushed to the verge, sometimes
of nonsense, and sometimes of impiety, was not thought to disgrace a
poet. Independence, veracity, selfrespect, were things not required
by the world from him. In truth, he was in morals something between a
pandar and a beggar.

To the other vices which degraded the literary character was added,
towards the close of the reign of Charles the Second, the most savage
intemperance of party spirit. The wits, as a class, had been impelled
by their old hatred of Puritanism to take the side of the court, and had
been found useful allies. Dryden, in particular, had done good service
to the government. His Absalom and Achitophel, the greatest satire of
modern times had amazed the town, had made its way with unprecedented
rapidity even into rural districts, and had, wherever it appeared
bitterly annoyed the Exclusionists and raised the courage of the Tories.
But we must not, in the admiration which we naturally feel for noble
diction and versification, forget the great distinctions of good and
evil. The spirit by which Dryden and several of his compeers were at
this time animated against the Whigs deserves to be called fiendish. The
servile Judges and Sheriffs of those evil days could not shed blood
as fast as the poets cried out for it. Calls for more victims, hideous
jests on hanging, bitter taunts on those who, having stood by the King
in the hour of danger, now advised him to deal mercifully and generously
by his vanquished enemies, were publicly recited on the stage, and, that
nothing might be wanting to the guilt and the shame, were recited by
women, who, having long been taught to discard all modesty, were now
taught to discard all compassion. 
%[180]
\footnote{ If any reader thinks my expressions too severe, I would
advise him to read Dryden's Epilogue to the Duke of Guise, and to
observe that it was spoken by a woman.}


It is a remarkable fact that, while the lighter literature of England
was thus becoming a nuisance and a national disgrace, the English genius
was effecting in science a revolution which will, to the end of time,
be reckoned among the highest achievements of the human intellect. Bacon
had sown the good seed in a sluggish soil and an ungenial season. He
had not expected an early crop, and in his last testament had solemnly
bequeathed his fame to the next age. During a whole generation his
philosophy had, amidst tumults wars, and proscriptions, been slowly
ripening in a few well constituted minds. While factions were struggling
for dominion over each other, a small body of sages had turned away with
benevolent disdain from the conflict, and had devoted themselves to the
nobler work of extending the dominion of man over matter. As soon
as tranquillity was restored, these teachers easily found attentive
audience. For the discipline through which the nation had passed had
brought the public mind to a temper well fitted for the reception of the
Verulamian doctrine. The civil troubles had stimulated the faculties of
the educated classes, and had called forth a restless activity and an
insatiable curiosity, such as had not before been known among us. Yet
the effect of those troubles was that schemes of political and religious
reform were generally regarded with suspicion and contempt. During
twenty years the chief employment of busy and ingenious men had been to
frame constitutions with first magistrates, without first magistrates,
with hereditary senates, with senates appointed by lot, with annual
senates, with perpetual senates. In these plans nothing was omitted. All
the detail, all the nomenclature, all the ceremonial of the imaginary
government was fully set forth, Polemarchs and Phylarchs, Tribes and
Galaxies, the Lord Archon and the Lord Strategus. Which ballot boxes
were to be green and which red, which balls were to be of gold and which
of silver, which magistrates were to wear hats and which black velvet
caps with peaks, how the mace was to be carried and when the heralds
were to uncover, these, and a hundred more such trifles, were gravely
considered and arranged by men of no common capacity and learning.

%[181]
\footnote{ See particularly Harrington's Oceana.}
 But the time for these visions had gone by; and, if any steadfast
republican still continued to amuse himself with them, fear of public
derision and of a criminal information generally induced him to keep
his fancies to himself. It was now unpopular and unsafe to mutter a word
against the fundamental laws of the monarchy: but daring and ingenious
men might indemnify themselves by treating with disdain what had lately
been considered as the fundamental laws of nature. The torrent which
had been dammed up in one channel rushed violently into another. The
revolutionary spirit, ceasing to operate in politics, began to exert
itself with unprecedented vigour and hardihood in every department
of physics. The year 1660, the era of the restoration of the old
constitution, is also the era from which dates the ascendency of the new
philosophy. In that year the Royal Society, destined to be a chief agent
in a long series of glorious and salutary reforms, began to exist.

%[182]
\footnote{ See Sprat's History of the Royal Society.}
 In a few months experimental science became all the mode. The
transfusion of blood, the ponderation of air, the fixation of mercury,
succeeded to that place in the public mind which had been lately
occupied by the controversies of the Rota. Dreams of perfect forms of
government made way for dreams of wings with which men were to fly from
the Tower to the Abbey, and of doublekeeled ships which were never to
founder in the fiercest storm. All classes were hurried along by the
prevailing sentiment. Cavalier and Roundhead, Churchman and Puritan,
were for once allied. Divines, jurists, statesmen, nobles, princes,
swelled the triumph of the Baconian philosophy. Poets sang with emulous
fervour the approach of the golden age. Cowley, in lines weighty
with thought and resplendent with wit, urged the chosen seed to take
possession of the promised land flowing with milk and honey, that land
which their great deliverer and lawgiver had seen, as from the summit
of Pisgah, but had not been permitted to enter. 
%[183]
\footnote{ Cowley's Ode to the Royal Society.}
 Dryden, with more
zeal than knowledge, joined voice to the general acclamation to enter,
and foretold things which neither he nor anybody else understood. The
Royal Society, he predicted, would soon lead us to the extreme verge of
the globe, and there delight us with a better view of the moon. 
%[184]
\footnote{

     "Then we upon the globe's last verge shall go,
     And view the ocean leaning  on the  sky;
     From  thence our rolling neighbours we shall know,
     And on the lunar world secretly pry."}

Two able and aspiring prelates, Ward, Bishop of Salisbury, and Wilkins,
Bishop of Chester, were conspicuous among the leaders of the movement.
Its history was eloquently written by a younger divine, who was rising
to high distinction in his profession, Thomas Sprat, afterwards Bishop
of Rochester. Both Chief Justice Hale and Lord Keeper Guildford stole
some hours from the business of their courts to write on hydrostatics.
Indeed it was under the immediate direction of Guildford that the
first barometers ever exposed to sale in London were constructed. 
%[185]
\footnote{ North's Life of Guildford.}

Chemistry divided, for a time, with wine and love, with the stage and
the gaming table, with the intrigues of a courtier and the intrigues
of a demagogue, the attention of the fickle Buckingham. Rupert has the
credit of having invented mezzotinto; from him is named that curious
bubble of glass which has long amused children and puzzled philosophers.
Charles himself had a laboratory at Whitehall, and was far more active
and attentive there than at the council board. It was almost necessary
to the character of a fine gentleman to have something to say about air
pumps and telescopes; and even fine ladies, now and then, thought it
becoming to affect a taste for science, went in coaches and six to
visit the Gresham curiosities, and broke forth into cries of delight at
finding that a magnet really attracted a needle, and that a microscope
really made a fly loom as large as a sparrow. 
%[186]
\footnote{ Pepys's Diary, May 30, 1667.}


In this, as in every great stir of the human mind, there was doubtless
something which might well move a smile. It is the universal law that
whatever pursuit, whatever doctrine, becomes fashionable, shall lose a
portion of that dignity which it had possessed while it was confined to
a small but earnest minority, and was loved for its own sake alone. It
is true that the follies of some persons who, without any real
aptitude for science, professed a passion for it, furnished matter of
contemptuous mirth to a few malignant satirists who belonged to the
preceding generation, and were not disposed to unlearn the lore of their
youth. 
%[187]
\footnote{ Butler was, I think, the only man of real genius who,
between the Restoration and the Revolution showed a bitter enmity to
the new philosophy, as it was then called. See the Satire on the Royal
Society, and the Elephant in the Moon.}
 But it is not less true that the great work of interpreting
nature was performed by the English of that age as it had never before
been performed in any age by any nation. The spirit of Francis Bacon was
abroad, a spirit admirably compounded of audacity and sobriety. There
was a strong persuasion that the whole world was full of secrets of high
moment to the happiness of man, and that man had, by his Maker, been
entrusted with the key which, rightly used, would give access to
them. There was at the same time a conviction that in physics it was
impossible to arrive at the knowledge of general laws except by the
careful observation of particular facts. Deeply impressed with these
great truths, the professors of the new philosophy applied themselves
to their task, and, before a quarter of a century had expired, they had
given ample earnest of what has since been achieved. Already a reform
of agriculture had been commenced. New vegetables were cultivated. New
implements of husbandry were employed. New manures were applied to the
soil. 
%[188]
\footnote{ The eagerness with which the agriculturists of that
age tried experiments and introduced improvements is well described by
Aubrey. See the Natural history of Wiltshire, 1685.}
 Evelyn had, under the formal sanction of the Royal Society,
given instruction to his countrymen in planting. Temple, in his
intervals of leisure, had tried many experiments in horticulture, and
had proved that many delicate fruits, the natives of more favoured
climates, might, with the help of art, be grown on English ground.
Medicine, which in France was still in abject bondage, and afforded an
inexhaustible subject of just ridicule to Moliere, had in England become
an experimental and progressive science, and every day made some
new advance in defiance of Hippocrates and Galen. The attention of
speculative men had been, for the first time, directed to the important
subject of sanitary police. The great plague of 1665 induced them to
consider with care the defective architecture, draining, and ventilation
of the capital. The great fire of 1666 afforded an opportunity for
effecting extensive improvements. The whole matter was diligently
examined by the Royal Society; and to the suggestions of that body must
be partly attributed the changes which, though far short of what the
public welfare required, yet made a wide difference between the new
and the old London, and probably put a final close to the ravages of
pestilence in our country. 
%[189]
\footnote{ Sprat's History of the Royal Society.}
 At the same time one of the founders
of the Society, Sir William Petty, created the science of political
arithmetic, the humble but indispensable handmaid of political
philosophy. No kingdom of nature was left unexplored. To that period
belong the chemical discoveries of Boyle, and the earliest botanical
researches of Sloane. It was then that Ray made a new classification
of birds and fishes, and that the attention of Woodward was first drawn
towards fossils and shells. One after another phantoms which had haunted
the world through ages of darkness fled before the light. Astrology and
alchymy became jests. Soon there was scarcely a county in which some of
the Quorum did not smile contemptuously when an old woman was brought
before them for riding on broomsticks or giving cattle the murrain. But
it was in those noblest and most arduous departments of knowledge
in which induction and mathematical demonstration cooperate for the
discovery of truth, that the English genius won in that age the most
memorable triumphs. John Wallis placed the whole system of statics on
a new foundation. Edmund Halley investigated the properties of the
atmosphere, the ebb and flow of the sea, the laws of magnetism, and the
course of the comets; nor did he shrink from toil, peril and exile in
the cause of science. While he, on the rock of Saint Helena, mapped the
constellations of the southern hemisphere, our national observatory was
rising at Greenwich: and John Flamsteed, the first Astronomer Royal,
was commencing that long series of observations which is never mentioned
without respect and gratitude in any part of the globe. But the glory
of these men, eminent as they were, is cast into the shade by the
transcendent lustre of one immortal name. In Isaac Newton two kinds of
intellectual power, which have little in common, and which are not often
found together in a very high degree of vigour, but which nevertheless
are equally necessary in the most sublime departments of physics, were
united as they have never been united before or since. There may have
been minds as happily constituted as his for the cultivation of pure
mathematical science: there may have been minds as happily constituted
for the cultivation of science purely experimental; but in no other mind
have the demonstrative faculty and the inductive faculty coexisted in
such supreme excellence and perfect harmony. Perhaps in the days of
Scotists and Thomists even his intellect might have run to waste, as
many intellects ran to waste which were inferior only to his. Happily
the spirit of the age on which his lot was cast, gave the right
direction to his mind; and his mind reacted with tenfold force on the
spirit of the age. In the year 1685 his fame, though splendid, was only
dawning; but his genius was in the meridian. His great work, that work
which effected a revolution in the most important provinces of natural
philosophy, had been completed, but was not yet published, and was just
about to be submitted to the consideration of the Royal Society.

It is not very easy to explain why the nation which was so far before
its neighbours in science should in art have been far behind them. Yet
such was the fact. It is true that in architecture, an art which is half
a science, an art in which none but a geometrician can excel, an art
which has no standard of grace but what is directly or indirectly
dependent on utility, an art of which the creations derive a part, at
least, of their majesty from mere bulk, our country could boast of one
truly great man, Christopher Wren; and the fire which laid London in
ruins had given him an opportunity, unprecedented in modern history, of
displaying his powers. The austere beauty of the Athenian portico,
the gloomy sublimity of the Gothic arcade, he was like almost all
his contemporaries, incapable of emulating, and perhaps incapable of
appreciating; but no man born on our side of the Alps, has imitated with
so much success the magnificence of the palacelike churches of Italy.
Even the superb Lewis has left to posterity no work which can bear a
comparison with Saint Paul's. But at the close of the reign of Charles
the Second there was not a single English painter or statuary whose name
is now remembered. This sterility is somewhat mysterious; for painters
and statuaries were by no means a despised or an ill paid class. Their
social position was at least as high as at present. Their gains, when
compared with the wealth of the nation and with the remuneration of
other descriptions of intellectual labour, were even larger than at
present. Indeed the munificent patronage which was extended to artists
drew them to our shores in multitudes. Lely, who has preserved to us
the rich curls, the full lips, and the languishing eyes of the frail
beauties celebrated by Hamilton, was a Westphalian. He had died in 1680,
having long lived splendidly, having received the honour of knighthood,
and having accumulated a good estate out of the fruits of his skill.
His noble collection of drawings and pictures was, after his decease,
exhibited by the royal permission in the Banqueting House at Whitehall,
and was sold by auction for the almost incredible sum of twenty-six
thousand pounds, a sum which bore a greater proportion to the fortunes
of the rich men of that day than a hundred thousand pounds would bear
to the fortunes of the rich men of our time. 
%[190]
\footnote{ Walpole's Anecdotes of Painting, London Gazette, May 31,
1683; North's Life of Guildford.}
 Lely was succeeded by
his countryman Godfrey Kneller, who was made first a knight and then a
baronet, and who, after keeping up a sumptuous establishment, and after
losing much money by unlucky speculations, was still able to bequeath
a large fortune to his family. The two Vandeveldes, natives of Holland,
had been tempted by English liberality to settle here, and had produced
for the King and his nobles some of the finest sea pieces in the world.
Another Dutchman, Simon Varelst, painted glorious sunflowers and tulips
for prices such as had never before been known. Verrio, a Neapolitan,
covered ceilings and staircases with Gorgons and Muses, Nymphs and
Satyrs, Virtues and Vices, Gods quaffing nectar, and laurelled princes
riding in triumph. The income which he derived from his performances
enabled him to keep one of the most expensive tables in England. For his
pieces at Windsor alone he received seven thousand pounds, a sum then
sufficient to make a gentleman of moderate wishes perfectly easy for
life, a sum greatly exceeding all that Dryden, during a literary life of
forty years, obtained from the booksellers. 
%[191]
\footnote{ The great prices paid to Varelst and Verrio are mentioned
in Walpole's Anecdotes of Painting.}
 Verrio's assistant
and successor, Lewis Laguerre, came from France. The two most celebrated
sculptors of that day were also foreigners. Cibber, whose pathetic
emblems of Fury and Melancholy still adorn Bedlam, was a Dane. Gibbons,
to whose graceful fancy and delicate touch many of our palaces,
colleges, and churches owe their finest decorations, was a Dutchman.
Even the designs for the coin were made by French artists. Indeed, it
was not till the reign of George the Second that our country could glory
in a great painter; and George the Third was on the throne before she
had reason to be proud of any of her sculptors.

It is time that this description of the England which Charles the Second
governed should draw to a close. Yet one subject of the highest moment
still remains untouched. Nothing has yet been said of the great body
of the people, of those who held the ploughs, who tended the oxen, who
toiled at the looms of Norwich, and squared the Portland stone for Saint
Paul's. Nor can very much be said. The most numerous class is precisely
the class respecting which we have the most meagre information. In those
times philanthropists did not yet regard it as a sacred duty, nor had
demagogues yet found it a lucrative trade, to talk and write about the
distress of the labourer. History was too much occupied with courts and
camps to spare a line for the hut of the peasant or the garret of the
mechanic. The press now often sends forth in a day a greater quantity of
discussion and declamation about the condition of the working man than
was published during the twenty-eight years which elapsed between the
Restoration and the Revolution. But it would be a great error to infer
from the increase of complaint that there has been any increase of
misery.

The great criterion of the state of the common people is the amount
of their wages; and as four-fifths of the common people were, in the
seventeenth century, employed in agriculture, it is especially important
to ascertain what were then the wages of agricultural industry. On this
subject we have the means of arriving at conclusions sufficiently exact
for our purpose.

Sir William Petty, whose mere assertion carries great weight, informs us
that a labourer was by no means in the lowest state who received for
a day's work fourpence with food, or eightpence without food. Four
shillings a week therefore were, according to Petty's calculation, fair
agricultural wages. 
%[192]
\footnote{ Petty's Political Arithmetic.}


That this calculation was not remote from the truth we have
abundant proof. About the beginning of the year 1685 the justices of
Warwickshire, in the exercise of a power entrusted to them by an Act of
Elizabeth, fixed, at their quarter sessions, a scale of wages for
the county, and notified that every employer who gave more than the
authorised sum, and every working man who received more, would be liable
to punishment. The wages of the common agricultural labourer, from
March to September, were fixed at the precise amount mentioned by Petty,
namely four shillings a week without food. From September to March the
wages were to be only three and sixpence a week. 
%[193]
\footnote{ Stat 5 Eliz. c. 4; Archaeologia, vol. xi.}


But in that age, as in ours, the earnings of the peasant were very
different in different parts of the kingdom. The wages of Warwickshire
were probably about the average, and those of the counties near the
Scottish border below it: but there were more favoured districts. In
the same year, 1685, a gentleman of Devonshire, named Richard Dunning,
published a small tract, in which he described the condition of the poor
of that county. That he understood his subject well it is impossible to
doubt; for a few months later his work was reprinted, and was, by
the magistrates assembled in quarter sessions at Exeter, strongly
recommended to the attention of all parochial officers. According to
him, the wages of the Devonshire peasant were, without food, about five
shillings a week. 
%[194]
\footnote{ Plain and easy Method showing how the office of Overseer
of the Poor may be managed, by Richard Dunning; 1st edition, 1685; 2d
edition, 1686.}


Still better was the condition of the labourer in the neighbourhood of
Bury Saint Edmund's. The magistrates of Suffolk met there in the spring
of 1682 to fix a rate of wages, and resolved that, where the labourer
was not boarded, he should have five shillings a week in winter, and six
in summer. 
%[195]
\footnote{ Cullum's History of Hawsted.}


In 1661 the justices at Chelmsford had fixed the wages of the Essex
labourer, who was not boarded, at six shillings in winter and seven in
summer. This seems to have been the highest remuneration given in
the kingdom for agricultural labour between the Restoration and the
Revolution; and it is to be observed that, in the year in which this
order was made, the necessaries of life were immoderately dear. Wheat
was at seventy shillings the quarter, which would even now be considered
as almost a famine price. 
%[196]
\footnote{ Ruggles on the Poor.}


These facts are in perfect accordance with another fact which seems to
deserve consideration. It is evident that, in a country where no man can
be compelled to become a soldier, the ranks of an army cannot be filled
if the government offers much less than the wages of common rustic
labour. At present the pay and beer money of a private in a regiment of
the line amount to seven shillings and sevenpence a week. This stipend,
coupled with the hope of a pension, does not attract the English
youth in sufficient numbers; and it is found necessary to supply the
deficiency by enlisting largely from among the poorer population of
Munster and Connaught. The pay of the private foot soldier in 1685 was
only four shillings and eightpence a week; yet it is certain that the
government in that year found no difficulty in obtaining many thousands
of English recruits at very short notice. The pay of the private foot
soldier in the army of the Commonwealth had been seven shillings a
week, that is to say, as much as a corporal received under Charles the
Second; 
%[197]
\footnote{ See, in Thurloe's State Papers, the memorandum of the
Dutch Deputies dated August 2-12, 1653.}
 and seven shillings a week had been found sufficient to fill
the ranks with men decidedly superior to the generality of the people.
On the whole, therefore, it seems reasonable to conclude that, in the
reign of Charles the Second, the ordinary wages of the peasant did not
exceed four shillings a week; but that, in some parts of the kingdom,
five shillings, six shillings, and, during the summer months, even seven
shillings were paid. At present a district where a labouring man earns
only seven shillings a week is thought to be in a state shocking to
humanity. The average is very much higher; and in prosperous counties,
the weekly wages of husbandmen amount to twelve, fourteen, and even
sixteen shillings. The remuneration of workmen employed in manufactures
has always been higher than that of the tillers of the soil. In the year
1680, a member of the House of Commons remarked that the high wages
paid in this country made it impossible for our textures to maintain a
competition with the produce of the Indian looms. An English mechanic,
he said, instead of slaving like a native of Bengal for a piece of
copper, exacted a shilling a day. 
%[198]
\footnote{ The orator was Mr. John Basset, member for Barnstaple.
See Smith's Memoirs of Wool, chapter lxviii.}
 Other evidence is extant,
which proves that a shilling a day was the pay to which the English
manufacturer then thought himself entitled, but that he was often forced
to work for less. The common people of that age were not in the habit
of meeting for public discussion, of haranguing, or of petitioning
Parliament. No newspaper pleaded their cause. It was in rude rhyme
that their love and hatred, their exultation and their distress, found
utterance. A great part of their history is to be learned only from
their ballads. One of the most remarkable of the popular lays chaunted
about the streets of Norwich and Leeds in the time of Charles the Second
may still be read on the original broadside. It is the vehement and
bitter cry of labour against capital. It describes the good old times
when every artisan employed in the woollen manufacture lived as well
as a farmer. But those times were past. Sixpence a day was now all that
could be earned by hard labour at the loom. If the poor complained that
they could not live on such a pittance, they were told that they were
free to take it or leave it. For so miserable a recompense were the
producers of wealth compelled to toil rising early and lying down late,
while the master clothier, eating, sleeping, and idling, became rich by
their exertions. A shilling a day, the poet declares, is what the weaver
would have if justice were done. 
%[199]
\footnote{ This ballad is in the British Museum. The precise year
is not given; but the Imprimatur of Roger Lestrange fixes the date
sufficiently for my purpose. I will quote some of the lines. The master
clothier is introduced speaking as follows:

     "In former ages we used to give,
     So that our workfolks like farmers did live;
     But the times are changed, we will make them know."

     "We will make them to work hard for sixpence a day,
     Though a shilling they deserve if they kind their just pay;
     If at all they murmur and say 'tis too small,
     We bid them choose whether they'll work at all.
     And thus we forgain all our wealth and estate,
     By many poor men that work early and late.
     Then hey for the clothing trade! It goes on brave;
     We scorn for to toyl and moyl, nor yet to slave.
     Our workmen do work hard, but we live at ease,
     We go when we will, and we come when we please."}
 We may therefore conclude that,
in the generation which preceded the Revolution, a workman employed in
the great staple manufacture of England thought himself fairly paid if
he gained six shillings a week.

It may here be noticed that the practice of setting children prematurely
to work, a practice which the state, the legitimate protector of those
who cannot protect themselves, has, in our time, wisely and humanely
interdicted, prevailed in the seventeenth century to an extent which,
when compared with the extent of the manufacturing system, seems almost
incredible. At Norwich, the chief seat of the clothing trade, a little
creature of six years old was thought fit for labour. Several writers
of that time, and among them some who were considered as eminently
benevolent, mention, with exultation, the fact that, in that single
city, boys and girls of very tender age created wealth exceeding what
was necessary for their own subsistence by twelve thousand pounds a
year. 
%[200]
\footnote{ Chamberlayne's State of England; Petty's Political
Arithmetic, chapter viii.; Dunning's Plain and Easy Method; Firmin's
Proposition for the Employing of the Poor. It ought to be observed that
Firmin was an eminent philanthropist.}
 The more carefully we examine the history of the past, the
more reason shall we find to dissent from those who imagine that our age
has been fruitful of new social evils. The truth is that the evils are,
with scarcely an exception, old. That which is new is the intelligence
which discerns and the humanity which remedies them.

When we pass from the weavers of cloth to a different class of artisans,
our enquiries will still lead us to nearly the same conclusions. During
several generations, the Commissioners of Greenwich Hospital have kept a
register of the wages paid to different classes of workmen who have been
employed in the repairs of the building. From this valuable record it
appears that, in the course of a hundred and twenty years, the daily
earnings of the bricklayer have risen from half a crown to four and
tenpence, those of the mason from half a crown to five and threepence,
those of the carpenter from half a crown to five and fivepence, and
those of the plumber from three shillings to five and sixpence.

It seems clear, therefore, that the wages of labour, estimated in money,
were, in 1685, not more than half of what they now are; and there were
few articles important to the working man of which the price was not,
in 1685, more than half of what it now is. Beer was undoubtedly much
cheaper in that age than at present. Meat was also cheaper, but was
still so dear that hundreds of thousands of families scarcely knew
the taste of it. 
%[201]
\footnote{ King in his Natural and Political Conclusions roughly
estimated the common people of England at 880,000 families. Of these
families 440,000, according to him ate animal food twice a week. The
remaining 440,000, ate it not at all, or at most not oftener than once a
week.}
 In the cost of wheat there has been very little
change. The average price of the quarter, during the last twelve years
of Charles the Second, was fifty shillings. Bread, therefore, such as is
now given to the inmates of a workhouse, was then seldom seen, even on
the trencher of a yeoman or of a shopkeeper. The great majority of the
nation lived almost entirely on rye, barley, and oats.

The produce of tropical countries, the produce of the mines, the
produce of machinery, was positively dearer than at present. Among the
commodities for which the labourer would have had to pay higher in
1685 than his posterity now pay were sugar, salt, coals, candles,
soap, shoes, stockings, and generally all articles of clothing and all
articles of bedding. It may be added, that the old coats and blankets
would have been, not only more costly, but less serviceable than the
modern fabrics.

It must be remembered that those labourers who were able to maintain
themselves and their families by means of wages were not the most
necessitous members of the community. Beneath them lay a large class
which could not subsist without some aid from the parish. There can
hardly be a more important test of the condition of the common people
than the ratio which this class bears to the whole society. At present,
the men, women, and children who receive relief appear from the official
returns to be, in bad years, one tenth of the inhabitants of England,
and, in good years, one thirteenth. Gregory King estimated them in his
time at about a fourth; and this estimate, which all our respect for
his authority will scarcely prevent us from calling extravagant, was
pronounced by Davenant eminently judicious.

We are not quite without the means of forming an estimate for ourselves.
The poor rate was undoubtedly the heaviest tax borne by our ancestors in
those days. It was computed, in the reign of Charles the Second, at near
seven hundred thousand pounds a year, much more than the produce either
of the excise or of the customs, and little less than half the entire
revenue of the crown. The poor rate went on increasing rapidly, and
appears to have risen in a short time to between eight and nine hundred
thousand a year, that is to say, to one sixth of what it now is. The
population was then less than a third of what it now is. The minimum
of wages, estimated in money, was half of what it now is; and we can
therefore hardly suppose that the average allowance made to a pauper can
have been more than half of what it now is. It seems to follow that the
proportion of the English people which received parochial relief then
must have been larger than the proportion which receives relief now. It
is good to speak on such questions with diffidence: but it has certainly
never yet been proved that pauperism was a less heavy burden or a less
serious social evil during the last quarter of the seventeenth century
than it is in our own time. 
%[202]
\footnote{ Fourteenth Report of the Poor Law Commissioners, Appendix
B. No. 2, Appendix C. No 1, 1848. Of the two estimates of the poor rate
mentioned in the text one was formed by Arthur Moore, the other, some
years later, by Richard Dunning. Moore's estimate will be found in
Davenant's Essay on Ways and Means; Dunning's in Sir Frederic Eden's
valuable work on the poor. King and Davenant estimate the paupers
and beggars in 1696, at the incredible number of 1,330,000 out of a
population of 5,500,000. In 1846 the number of persons who received
relief appears from the official returns to have been only 1,332,089 out
of a population of about 17,000,000. It ought also to be observed that,
in those returns, a pauper must very often be reckoned more than once.
I would advise the reader to consult De Foe's pamphlet entitled "Giving
Alms no Charity," and the Greenwich tables which will be found in Mr.
M'Culloch's Commercial Dictionary under the head Prices.}


In one respect it must be admitted that the progress of civilization has
diminished the physical comforts of a portion of the poorest class. It
has already been mentioned that, before the Revolution, many thousands
of square miles, now enclosed and cultivated, were marsh, forest, and
heath. Of this wild land much was, by law, common, and much of what was
not common by law was worth so little that the proprietors suffered it
to be common in fact. In such a tract, squatters and trespassers were
tolerated to an extent now unknown. The peasant who dwelt there could,
at little or no charge, procure occasionally some palatable addition to
his hard fare, and provide himself with fuel for the winter. He kept a
flock of geese on what is now an orchard rich with apple blossoms.
He snared wild fowl on the fell which has long since been drained and
divided into corn-fields and turnip fields. He cut turf among the furze
bushes on the moor which is now a meadow bright with clover and renowned
for butter and cheese. The progress of agriculture and the increase of
population necessarily deprived him of these privileges. But against
this disadvantage a long list of advantages is to be set off. Of the
blessings which civilisation and philosophy bring with them a large
proportion is common to all ranks, and would, if withdrawn, be missed
as painfully by the labourer as by the peer. The market-place which the
rustic can now reach with his cart in an hour was, a hundred and sixty
years ago, a day's journey from him. The street which now affords to
the artisan, during the whole night, a secure, a convenient, and a
brilliantly lighted walk was, a hundred and sixty years ago, so dark
after sunset that he would not have been able to see his hand, so ill
paved that he would have run constant risk of breaking his neck, and so
ill watched that he would have been in imminent danger of being knocked
down and plundered of his small earnings. Every bricklayer who falls
from a scaffold, every sweeper of a crossing who is run over by a
carriage, may now have his wounds dressed and his limbs set with a skill
such as, a hundred and sixty years ago, all the wealth of a great
lord like Ormond, or of a merchant prince like Clayton, could not have
purchased. Some frightful diseases have been extirpated by science;
and some have been banished by police. The term of human life has been
lengthened over the whole kingdom, and especially in the towns. The year
1685 was not accounted sickly; yet in the year 1685 more than one in
twenty-three of the inhabitants of the capital died. 
%[203]
\footnote{ The deaths were 23,222. Petty's Political Arithmetic.}
 At present
only one inhabitant of the capital in forty dies annually. The
difference in salubrity between the London of the nineteenth century
and the London of the seventeenth century is very far greater than the
difference between London in an ordinary year and London in a year of
cholera.

Still more important is the benefit which all orders of society, and
especially the lower orders, have derived from the mollifying influence
of civilisation on the national character. The groundwork of that
character has indeed been the same through many generations, in the
sense in which the groundwork of the character of an individual may be
said to be the same when he is a rude and thoughtless schoolboy and when
he is a refined and accomplished man. It is pleasing to reflect that the
public mind of England has softened while it has ripened, and that we
have, in the course of ages, become, not only a wiser, but also a kinder
people. There is scarcely a page of the history or lighter literature
of the seventeenth century which does not contain some proof that our
ancestors were less humane than their posterity. The discipline of
workshops, of schools, of private families, though not more efficient
than at present, was infinitely harsher. Masters, well born and bred,
were in the habit of beating their servants. Pedagogues knew no way of
imparting knowledge but by beating their pupils. Husbands, of decent
station, were not ashamed to beat their wives. The implacability of
hostile factions was such as we can scarcely conceive. Whigs were
disposed to murmur because Stafford was suffered to die without seeing
his bowels burned before his face. Tories reviled and insulted Russell
as his coach passed from the Tower to the scaffold in Lincoln's Inn
Fields. 
%[204]
\footnote{ Burnet, i. 560.}
 As little mercy was shown by the populace to sufferers of
a humbler rank. If an offender was put into the pillory, it was well
if he escaped with life from the shower of brickbats and paving stones.

%[205]
\footnote{ Muggleton's Acts of the Witnesses of the Spirit.}
 If he was tied to the cart's tail, the crowd pressed round him,
imploring the hangman to give it the fellow well, and make him howl.

%[206]
\footnote{ Tom Brown describes such a scene in lines which I do not
venture to quote.}
 Gentlemen arranged parties of pleasure to Bridewell on court
days for the purpose of seeing the wretched women who beat hemp there
whipped. 
%[207]
\footnote{ Ward's London Spy.}
 A man pressed to death for refusing to plead, a woman
burned for coining, excited less sympathy than is now felt for a galled
horse or an overdriven ox. Fights compared with which a boxing match is
a refined and humane spectacle were among the favourite diversions of a
large part of the town. Multitudes assembled to see gladiators hack each
other to pieces with deadly weapons, and shouted with delight when one
of the combatants lost a finger or an eye. The prisons were hells on
earth, seminaries of every crime and of every disease. At the assizes
the lean and yellow culprits brought with them from their cells to the
dock an atmosphere of stench and pestilence which sometimes avenged them
signally on bench, bar, and jury. But on all this misery society looked
with profound indifference. Nowhere could be found that sensitive
and restless compassion which has, in our time, extended a powerful
protection to the factory child, to the Hindoo widow, to the negro
slave, which pries into the stores and watercasks of every emigrant
ship, which winces at every lash laid on the back of a drunken
soldier, which will not suffer the thief in the hulks to be ill fed or
overworked, and which has repeatedly endeavoured to save the life
even of the murderer. It is true that compassion ought, like all other
feelings, to be under the government of reason, and has, for want of
such government, produced some ridiculous and some deplorable effects.
But the more we study the annals of the past, the more shall we rejoice
that we live in a merciful age, in an age in which cruelty is abhorred,
and in which pain, even when deserved, is inflicted reluctantly and from
a sense of duty. Every class doubtless has gained largely by this great
moral change: but the class which has gained most is the poorest, the
most dependent, and the most defenceless.

The general effect of the evidence which has been submitted to the
reader seems hardly to admit of doubt. Yet, in spite of evidence, many
will still image to themselves the England of the Stuarts as a more
pleasant country than the England in which we live. It may at first
sight seem strange that society, while constantly moving forward with
eager speed, should be constantly looking backward with tender regret.
But these two propensities, inconsistent as they may appear, can easily
be resolved into the same principle. Both spring from our impatience of
the state in which we actually are. That impatience, while it stimulates
us to surpass preceding generations, disposes us to overrate their
happiness. It is, in some sense, unreasonable and ungrateful in us to be
constantly discontented with a condition which is constantly improving.
But, in truth, there is constant improvement precisely because there is
constant discontent. If we were perfectly satisfied with the present,
we should cease to contrive, to labour, and to save with a view to the
future. And it is natural that, being dissatisfied with the present, we
should form a too favourable estimate of the past.

In truth we are under a deception similar to that which misleads the
traveller in the Arabian desert. Beneath the caravan all is dry and
bare: but far in advance, and far in the rear, is the semblance of
refreshing waters. The pilgrims hasten forward and find nothing but sand
where an hour before they had seen a lake. They turn their eyes and see
a lake where, an hour before, they were toiling through sand. A similar
illusion seems to haunt nations through every stage of the long progress
from poverty and barbarism to the highest degrees of opulence and
civilisation. But if we resolutely chase the mirage backward, we shall
find it recede before us into the regions of fabulous antiquity. It
is now the fashion to place the golden age of England in times
when noblemen were destitute of comforts the want of which would
be intolerable to a modern footman, when farmers and shopkeepers
breakfasted on loaves the very sight of which would raise a riot in a
modern workhouse, when to have a clean shirt once a week was a privilege
reserved for the higher class of gentry, when men died faster in the
purest country air than they now die in the most pestilential lanes of
our towns, and when men died faster in the lanes of our towns than
they now die on the coast of Guiana. We too shall, in our turn, be
outstripped, and in our turn be envied. It may well be, in the twentieth
century, that the peasant of Dorsetshire may think himself miserably
paid with twenty shillings a week; that the carpenter at Greenwich may
receive ten shillings a day; that labouring men may be as little used to
dine without meat as they now are to eat rye bread; that sanitary police
and medical discoveries may have added several more years to the average
length of human life; that numerous comforts and luxuries which are now
unknown, or confined to a few, may be within the reach of every diligent
and thrifty working man. And yet it may then be the mode to assert that
the increase of wealth and the progress of science have benefited
the few at the expense of the many, and to talk of the reign of Queen
Victoria as the time when England was truly merry England, when all
classes were bound together by brotherly sympathy, when the rich did
not grind the faces of the poor, and when the poor did not envy the
splendour of the rich.




\chapter{CHAPTER IV.}

THE death of King Charles the Second took the nation by surprise. His
frame was naturally strong, and did not appear to have suffered from
excess. He had always been mindful of his health even in his pleasures;
and his habits were such as promise a long life and a robust old age.
Indolent as he was on all occasions which required tension of the mind,
he was active and persevering in bodily exercise. He had, when young,
been renowned as a tennis player, 
%[208]
\footnote{ Pepys's Diary, Dec. 28, 1663, Sept. 2, 1667.}
 and was, even in the decline of
life, an indefatigable walker. His ordinary pace was such that those who
were admitted to the honour of his society found it difficult to keep up
with him. He rose early, and generally passed three or four hours a day
in the open air. He might be seen, before the dew was off the grass in
St. James's Park, striding among the trees, playing with his spaniels,
and flinging corn to his ducks; and these exhibitions endeared him to
the common people, who always love to See the great unbend. 
%[209]
\footnote{ Burnet, i, 606; Spectator, No. 462; Lords' Journals,
October 28, 1678; Cibber's Apology.}


At length, towards the close of the year 1684, he was prevented, by a
slight attack of what was supposed to be gout, from rambling as usual.
He now spent his mornings in his laboratory, where he amused himself
with experiments on the properties of mercury. His temper seemed to have
suffered from confinement. He had no apparent cause for disquiet. His
kingdom was tranquil: he was not in pressing want of money: his power
was greater than it had ever been: the party which had long thwarted
him had been beaten down; but the cheerfulness which had supported him
against adverse fortune had vanished in this season of prosperity. A
trifle now sufficed to depress those elastic spirits which had borne
up against defeat, exile, and penury. His irritation frequently showed
itself by looks and words such as could hardly have been expected from a
man so eminently distinguished by good humour and good breeding. It was
not supposed however that his constitution was seriously impaired. 
%[210]
\footnote{ Burnet, i. 605, 606, Welwood, North's Life of Guildford,
251.}


His palace had seldom presented a gayer or a more scandalous appearance
than on the evening of Sunday the first of February 1685. 
%[211]
\footnote{ I may take this opportunity of mentioning that whenever
I give only one date, I follow the old style, which was, in the
seventeenth century, the style of England; but I reckon the year from
the first of January.}
 Some
grave persons who had gone thither, after the fashion of that age, to
pay their duty to their sovereign, and who had expected that, on such a
day, his court would wear a decent aspect, were struck with astonishment
and horror. The great gallery of Whitehall, an admirable relic of the
magnificence of the Tudors, was crowded with revellers and gamblers. The
king sate there chatting and toying with three women, whose charms were
the boast, and whose vices were the disgrace, of three nations. Barbara
Palmer, Duchess of Cleveland, was there, no longer young, but still
retaining some traces of that superb and voluptuous loveliness which
twenty years before overcame the hearts of all men. There too was the
Duchess of Portsmouth, whose soft and infantine features were lighted up
with the vivacity of France. Hortensia Mancini, Duchess of Mazarin, and
niece of the great Cardinal, completed the group. She had been early
removed from her native Italy to the court where her uncle was supreme.
His power and her own attractions had drawn a crowd of illustrious
suitors round her. Charles himself, during his exile, had sought her
hand in vain. No gift of nature or of fortune seemed to be wanting
to her. Her face was beautiful with the rich beauty of the South,
her understanding quick, her manners graceful, her rank exalted, her
possessions immense; but her ungovernable passions had turned all these
blessings into curses. She had found the misery of an ill assorted
marriage intolerable, had fled from her husband, had abandoned her
vast wealth, and, after having astonished Rome and Piedmont by her
adventures, had fixed her abode in England. Her house was the favourite
resort of men of wit and pleasure, who, for the sake of her smiles
and her table, endured her frequent fits of insolence and ill humour.
Rochester and Godolphin sometimes forgot the cares of state in
her company. Barillon and Saint Evremond found in her drawing room
consolation for their long banishment from Paris. The learning of
Vossius, the wit of Waller, were daily employed to flatter and amuse
her. But her diseased mind required stronger stimulants, and sought them
in gallantry, in basset, and in usquebaugh. 
%[212]
\footnote{ Saint Everemond, passim; Saint Real, Memoires de la
Duchesse de Mazarin; Rochester's Farewell; Evelyn's Diary, Sept. 6,
1676, June 11, 1699.}
 While Charles. flirted
with his three sultanas, Hortensia's French page, a handsome boy, whose
vocal performances were the delight of Whitehall, and were rewarded by
numerous presents of rich clothes, ponies, and guineas, warbled some
amorous verses. 
%[213]
\footnote{ Evelyn's Diary, Jan. 28, 1684-5, Saint Evremond's Letter
to Dery.}
 A party of twenty courtiers was seated at cards
round a large table on which gold was heaped in mountains. 
%[214]
\footnote{ Id., February 4, 1684-5.}
 Even
then the King had complained that he did not feel quite well. He had
no appetite for his supper: his rest that night was broken; but on the
following morning he rose, as usual, early.

To that morning the contending factions in his council had, during some
days, looked forward with anxiety. The struggle between Halifax and
Rochester seemed to be approaching a decisive crisis. Halifax, not
content with having already driven his rival from the Board of Treasury,
had undertaken to prove him guilty of such dishonesty or neglect in the
conduct of the finances as ought to be punished by dismission from the
public service. It was even whispered that the Lord President would
probably be sent to the Tower. The King had promised to enquire into the
matter. The second of February had been fixed for the investigation; and
several officers of the revenue had been ordered to attend with their
books on that day. 
%[215]
\footnote{ Roger North's Life of Sir Dudley North, 170; The true
Patriot vindicated, or a Justification of his Excellency the E-of
R-; Burnet, i. 605. The Treasury Books prove that Burnet had good
intelligence.}
 But a great turn of fortune was at hand.

Scarcely had Charles risen from his bed when his attendants perceived
that his utterance was indistinct, and that his thoughts seemed to be
wandering. Several men of rank had, as usual, assembled to see their
sovereign shaved and dressed. He made an effort to converse with them
in his usual gay style; but his ghastly look surprised and alarmed them.
Soon his face grew black; his eyes turned in his head; he uttered a cry,
staggered, and fell into the arms of one of his lords. A physician who
had charge of the royal retorts and crucibles happened to be present.
He had no lances; but he opened a vein with a penknife. The blood flowed
freely; but the King was still insensible.

He was laid on his bed, where, during a short time, the Duchess of
Portsmouth hung over him with the familiarity of a wife. But the alarm
had been given. The Queen and the Duchess of York were hastening to
the room. The favourite concubine was forced to retire to her own
apartments. Those apartments had been thrice pulled down and thrice
rebuilt by her lover to gratify her caprice. The very furniture of
the chimney was massy silver. Several fine paintings, which properly
belonged to the Queen, had been transferred to the dwelling of the
mistress. The sideboards were piled with richly wrought plate. In
the niches stood cabinets, the masterpieces of Japanese art. On the
hangings, fresh from the looms of Paris, were depicted, in tints which
no English tapestry could rival, birds of gorgeous plumage, landscapes,
hunting matches, the lordly terrace of Saint Germains, the statues and
fountains of Versailles. 
%[216]
\footnote{ Evelyn's Diary, Jan. 24, 1681-2, Oct. 4, 1683.}
 In the midst of this splendour, purchased
by guilt and shame, the unhappy woman gave herself up to an agony of
grief, which, to do her justice, was not wholly selfish.

And now the gates of Whitehall, which ordinarily stood open to all
comers, were closed. But persons whose faces were known were still
permitted to enter. The antechambers and galleries were soon filled
to overflowing; and even the sick room was crowded with peers, privy
councillors, and foreign ministers. All the medical men of note in
London were summoned. So high did political animosities run that the
presence of some Whig physicians was regarded as an extraordinary
circumstance. 
%[217]
\footnote{ Dugdale's Correspondence.}
 One Roman Catholic, whose skill was then widely
renowned, Doctor Thomas Short, was in attendance. Several of the
prescriptions have been preserved. One of them is signed by fourteen
Doctors. The patient was bled largely. Hot iron was applied to his head.
A loathsome volatile salt, extracted from human skulls, was forced into
his mouth. He recovered his senses; but he was evidently in a situation
of extreme danger.

The Queen was for a time assiduous in her attendance. The Duke of York
scarcely left his brother's bedside. The Primate and four other bishops
were then in London. They remained at Whitehall all day, and took it
by turns to sit up at night in the King's room. The news of his illness
filled the capital with sorrow and dismay. For his easy temper and
affable manners had won the affection of a large part of the nation;
and those who most disliked him preferred his unprincipled levity to the
stern and earnest bigotry of his brother.

On the morning of Thursday the fifth of February, the London Gazette
announced that His Majesty was going on well, and was thought by the
physicians to be out of danger. The bells of all the churches rang
merrily; and preparations for bonfires were made in the streets. But in
the evening it was known that a relapse had taken place, and that the
medical attendants had given up all hope. The public mind was greatly
disturbed; but there was no disposition to tumult. The Duke of York, who
had already taken on himself to give orders, ascertained that the City
was perfectly quiet, and that he might without difficulty be proclaimed
as soon as his brother should expire.

The King was in great pain, and complained that he felt as if a fire
was burning within him. Yet he bore up against his sufferings with a
fortitude which did not seem to belong to his soft and luxurious nature.
The sight of his misery affected his wife so much that she fainted, and
was carried senseless to her chamber. The prelates who were in waiting
had from the first exhorted him to prepare for his end. They now thought
it their duty to address him in a still more urgent manner. William
Sancroft, Archbishop of Canterbury, an honest and pious, though
narrowminded, man, used great freedom. "It is time," he said, "to
speak out; for, Sir, you are about to appear before a Judge who is no
respecter of persons." The King answered not a word.

Thomas Ken, Bishop of Bath and Wells, then tried his powers of
persuasion. He was a man of parts and learning, of quick sensibility and
stainless virtue. His elaborate works have long been forgotten; but
his morning and evening hymns are still repeated daily in thousands of
dwellings. Though, like most of his order, zealous for monarchy, he was
no sycophant. Before he became a Bishop, he had maintained the honour of
his gown by refusing, when the court was at Winchester, to let Eleanor
Gwynn lodge in the house which he occupied there as a prebendary. 
%[218]
\footnote{ Hawkins's Life of Ken, 1713.}

The King had sense enough to respect so manly a spirit. Of all the
prelates he liked Ken the best. It was to no purpose, however, that the
good Bishop now put forth all his eloquence. His solemn and pathetic
exhortation awed and melted the bystanders to such a degree that some
among them believed him to be filled with the same spirit which, in the
old time, had, by the mouths of Nathan and Elias, called sinful princes
to repentance. Charles however was unmoved. He made no objection indeed
when the service for the visitation of the sick was read. In reply to
the pressing questions of the divines, he said that he was sorry for
what he had done amiss; and he suffered the absolution to be pronounced
over him according to the forms of the Church of England: but, when he
was urged to declare that he died in the communion of that Church, he
seemed not to hear what was said; and nothing could induce him to take
the Eucharist from the hands of the Bishops. A table with bread and wine
was brought to his bedside, but in vain. Sometimes he said that there
was no hurry, and sometimes that he was too weak.

Many attributed this apathy to contempt for divine things, and many to
the stupor which often precedes death. But there were in the palace a
few persons who knew better. Charles had never been a sincere member of
the Established Church. His mind had long oscillated between Hobbism and
Popery. When his health was good and his spirits high he was a scoffer.
In his few serious moments he was a Roman Catholic. The Duke of York
was aware of this, but was entirely occupied with the care of his own
interests. He had ordered the outports to be closed. He had posted
detachments of the Guards in different parts of the city. He had also
procured the feeble signature of the dying King to an instrument by
which some duties, granted only till the demise of the Crown, were let
to farm for a term of three years. These things occupied the attention
of James to such a degree that, though, on ordinary occasions, he was
indiscreetly and unseasonably eager to bring over proselytes to his
Church, he never reflected that his brother was in danger of dying
without the last sacraments. This neglect was the more extraordinary
because the Duchess of York had, at the request of the Queen, suggested,
on the morning on which the King was taken ill, the propriety of
procuring spiritual assistance. For such assistance Charles was at last
indebted to an agency very different from that of his pious wife and
sister-in-law. A life of frivolty and vice had not extinguished in the
Duchess of Portsmouth all sentiments of religion, or all that kindness
which is the glory of her sex. The French ambassador Barillon, who had
come to the palace to enquire after the King, paid her a visit. He found
her in an agony of sorrow. She took him into a secret room, and poured
out her whole heart to him. "I have," she said, "a thing of great moment
to tell you. If it were known, my head would be in danger. The King is
really and truly a Catholic; but he will die without being reconciled
to the Church. His bedchamber is full of Protestant clergymen. I cannot
enter it without giving scandal. The Duke is thinking only of himself.
Speak to him. Remind him that there is a soul at stake. He is master
now. He can clear the room. Go this instant, or it will be too late."

Barillon hastened to the bedchamber, took the Duke aside, and delivered
the message of the mistress. The conscience of James smote him. He
started as if roused from sleep, and declared that nothing should
prevent him from discharging the sacred duty which had been too long
delayed. Several schemes were discussed and rejected. At last the Duke
commanded the crowd to stand aloof, went to the bed, stooped down, and
whispered something which none of the spectators could hear, but which
they supposed to be some question about affairs of state. Charles
answered in an audible voice, "Yes, yes, with all my heart." None of
the bystanders, except the French Ambassador, guessed that the King was
declaring his wish to be admitted into the bosom of the Church of Rome.

"Shall I bring a priest?" said the Duke. "Do, brother," replied the Sick
man. "For God's sake do, and lose no time. But no; you will get into
trouble." "If it costs me my life," said the Duke, "I will fetch a
priest."

To find a priest, however, for such a purpose, at a moment's notice,
was not easy. For, as the law then stood, the person who admitted a
proselyte into the Roman Catholic Church was guilty of a capital crime.
The Count of Castel Melhor, a Portuguese nobleman, who, driven by
political troubles from his native land, had been hospitably received at
the English court, undertook to procure a confessor. He had recourse to
his countrymen who belonged to the Queen's household; but he found that
none of her chaplains knew English or French enough to shrive the King.
The Duke and Barillon were about to send to the Venetian Minister for
a clergyman when they heard that a Benedictine monk, named John
Huddleston, happened to be at Whitehall. This man had, with great risk
to himself, saved the King's life after the battle of Worcester, and
had, on that account, been, ever since the Restoration, a privileged
person. In the sharpest proclamations which had been put forth against
Popish priests, when false witnesses had inflamed the nation to fury,
Huddleston had been excepted by name. 
%[219]
\footnote{ See the London Gazette of Nov. 21, 1678. Barillon
and Burnet say that Huddleston was excepted out of all the Acts of
Parliament made against priests; but this is a mistake.}
 He readily consented to put
his life a second time in peril for his prince; but there was still a
difficulty. The honest monk was so illiterate that he did not know what
he ought to say on an occasion of such importance. He however obtained
some hints, through the intervention of Castel Melhor, from a Portuguese
ecclesiastic, and, thus instructed, was brought up the back stairs by
Chiffinch, a confidential servant, who, if the satires of that age
are to be credited, had often introduced visitors of a very different
description by the same entrance. The Duke then, in the King's name,
commanded all who were present to quit the room, except Lewis Duras,
Earl of Feversham, and John Granville, Earl of Bath. Both these Lords
professed the Protestant religion; but James conceived that he could
count on their fidelity. Feversham, a Frenchman of noble birth, and
nephew of the great Turenne, held high rank in the English army, and was
Chamberlain to the Queen. Bath was Groom of the Stole.

The Duke's orders were obeyed; and even the physicians withdrew. The
back door was then opened; and Father Huddleston entered. A cloak
had been thrown over his sacred vestments; and his shaven crown was
concealed by a flowing wig. "Sir," said the Duke, "this good man once
saved your life. He now comes to save your soul." Charles faintly
answered, "He is welcome." Huddleston went through his part better than
had been expected. He knelt by the bed, listened to the confession,
pronounced the absolution, and administered extreme unction. He asked
if the King wished to receive the Lord's supper. "Surely," said Charles,
"if I am not unworthy." The host was brought in. Charles feebly strove
to rise and kneel before it. The priest made him lie still, and assured
him that God would accept the humiliation of the soul, and would not
require the humiliation of the body. The King found so much difficulty
in swallowing the bread that it was necessary to open the door and
procure a glass of water. This rite ended, the monk held up a crucifix
before the penitent, charged him to fix his last thoughts on the
sufferings of the Redeemer, and withdrew. The whole ceremony had
occupied about three quarters of an hour; and, during that time, the
courtiers who filled the outer room had communicated their suspicions to
each other by whispers and significant glances. The door was at length
thrown open, and the crowd again filled the chamber of death.

It was now late in the evening. The King seemed much relieved by what
had passed. His natural children were brought to his bedside, the Dukes
of Grafton, Southampton, and Northumberland, sons of the Duchess of
Cleveland, the Duke of Saint Albans, son of Eleanor Gwynn, and the Duke
of Richmond, son of the Duchess of Portsmouth. Charles blessed them all,
but spoke with peculiar tenderness to Richmond. One face which should
have been there was wanting. The eldest and best loved child was an
exile and a wanderer. His name was not once mentioned by his father.

During the night Charles earnestly recommended the Duchess of Portsmouth
and her boy to the care of James; "And do not," he good-naturedly added,
"let poor Nelly starve." The Queen sent excuses for her absence by
Halifax. She said that she was too much disordered to resume her post
by the couch, and implored pardon for any offence which she might
unwittingly have given. "She ask my pardon, poor woman!" cried Charles;
"I ask hers with all my heart."

The morning light began to peep through the windows of Whitehall; and
Charles desired the attendants to pull aside the curtains, that he might
have one more look at the day. He remarked that it was time to wind up
a clock which stood near his bed. These little circumstances were long
remembered because they proved beyond dispute that, when he declared
himself a Roman Catholic, he was in full possession of his faculties.
He apologised to those who had stood round him all night for the trouble
which he had caused. He had been, he said, a most unconscionable time
dying; but he hoped that they would excuse it. This was the last glimpse
of the exquisite urbanity, so often found potent to charm away the
resentment of a justly incensed nation. Soon after dawn the speech of
the dying man failed. Before ten his senses were gone. Great numbers had
repaired to the churches at the hour of morning service. When the prayer
for the King was read, loud groans and sobs showed how deeply his people
felt for him. At noon on Friday, the sixth of February, he passed away
without a struggle. 
%[220]
\footnote{ Clark's Life of James the Second, i, 746. Orig. Mem.;
Barillon's Despatch of Feb. 1-18, 1685; Van Citters's Despatches of Feb.
3-13 and Feb. 1-16. Huddleston's Narrative; Letters of Philip, second
Earl of Chesterfield, 277; Sir H. Ellis's Original Letters, First
Series. iii. 333: Second Series, iv 74; Chaillot MS.; Burnet, i. 606:
Evelyn's Diary, Feb. 4. 1684-5: Welwood's Memoires 140; North's Life of
Guildford. 252; Examen, 648; Hawkins's Life of Ken; Dryden's Threnodia
Augustalis; Sir H. Halford's Essay on Deaths of Eminent Persons. See
also a fragment of a letter written by the Earl of Ailesbury, which is
printed in the European Magazine for April, 1795. Ailesbury calls Burnet
an impostor. Yet his own narrative and Burnet's will not, to any candid
and sensible reader, appear to contradict each other. I have seen in
the British Museum, and also in the Library of the Royal Institution, a
curious broadside containing an account of the death of Charles. It will
be found in the Somers Collections. The author was evidently a zealous
Roman Catholic, and must have had access to good sources of information.
I strongly suspect that he had been in communication, directly or
indirectly, with James himself. No name is given at length; but the
initials are perfectly intelligible, except in one place. It is said
that the D. of Y. was reminded of the duty which he owed to his brother
by P.M.A.C.F. I must own myself quite unable to decipher the last
five letters. It is some consolation that Sir Walter Scott was
equally unsuccessful. (1848.) Since the first edition of this work
was published, several ingenious conjectures touching these mysterious
letters have been communicated to me, but I am convinced that the true
solution has not yet been suggested. (1850.) I still greatly
doubt whether the riddle has been solved. But the most plausible
interpretation is one which, with some variations, occurred, almost at
the same time, to myself and to several other persons; I am inclined to
read "Pere Mansuete A Cordelier Friar." Mansuete, a Cordelier, was
then James's confessor. To Mansuete therefore it peculiarly belonged
to remind James of a sacred duty which had been culpably neglected. The
writer of the broadside must have been unwilling to inform the world
that a soul which many devout Roman Catholics had left to perish had
been snatched from destruction by the courageous charity of a woman of
loose character. It is therefore not unlikely that he would prefer a
fiction, at once probable and edifying, to a truth which could not
fail to give scandal. (1856.)----It should seem that no transactions in
history ought to be more accurately known to us than those which
took place round the deathbed of Charles the Second. We have several
relations written by persons who were actually in his room. We have
several relations written by persons who, though not themselves
eyewitnesses, had the best opportunity of obtaining information from
eyewitnesses. Yet whoever attempts to digest this vast mass of materials
into a consistent narrative will find the task a difficult one. Indeed
James and his wife, when they told the story to the nuns of Chaillot,
could not agree as to some circumstances. The Queen said that, after
Charles had received the last sacraments the Protestant Bishops renewed
their exhortations. The King said that nothing of the kind took place.
"Surely," said the Queen, "you told me so yourself." "It is impossible
that I have told you so," said the King, "for nothing of the sort
happened."----It is much to be regretted that Sir Henry Halford should
have taken so little trouble ascertain the facts on which he pronounced
judgment. He does not seem to have been aware of the existence of the
narrative of James, Barillon, and Huddleston.----As this is the first
occasion on which I cite the correspondence of the Dutch ministers
at the English court, I ought here to mention that a series of their
despatches, from the accession of James the Second to his flight,
forms one of the most valuable parts of the Mackintosh collection.
The subsequent despatches, down to the settlement of the government in
February, 1689, I procured from the Hague. The Dutch archives have been
far too little explored. They abound with information interesting in the
highest degree to every Englishman. They are admirably arranged and they
are in the charge of gentlemen whose courtesy, liberality and zeal for
the interests of literature, cannot be too highly praised. I wish to
acknowledge, in the strongest manner, my own obligations to Mr. De Jonge
and to Mr. Van Zwanne.}


At that time the common people throughout Europe, and nowhere more
than in England, were in the habit of attributing the death of princes,
especially when the prince was popular and the death unexpected, to the
foulest and darkest kind of assassination. Thus James the First had
been accused of poisoning Prince Henry. Thus Charles the First had been
accused of poisoning James the First. Thus when, in the time of the
Commonwealth, the Princess Elizabeth died at Carisbrook, it was loudly
asserted that Cromwell had stooped to the senseless and dastardly
wickedness of mixing noxious drugs with the food of a young girl whom he
had no conceivable motive to injure. 
%[221]
\footnote{ Clarendon mentions this calumny with just scorn.
"According to the charity of the time towards Cromwell, very many would
have it believed to be by poison, of which there was no appearance, nor
any proof ever after made."--Book xiv.}
 A few years later, the rapid
decomposition of Cromwell's own corpse was ascribed by many to a deadly
potion administered in his medicine. The death of Charles the Second
could scarcely fail to occasion similar rumours. The public ear had been
repeatedly abused by stories of Popish plots against his life. There
was, therefore, in many minds, a strong predisposition to suspicion; and
there were some unlucky circumstances which, to minds so predisposed,
might seem to indicate that a crime had been perpetrated. The fourteen
Doctors who deliberated on the King's case contradicted each other and
themselves. Some of them thought that his fit was epileptic, and that
he should be suffered to have his doze out. The majority pronounced
him apoplectic, and tortured him during some hours like an Indian at
a stake. Then it was determined to call his complaint a fever, and to
administer doses of bark. One physician, however, protested against
this course, and assured the Queen that his brethren would kill the
King among them. Nothing better than dissension and vacillation could be
expected from such a multitude of advisers. But many of the vulgar not
unnaturally concluded, from the perplexity of the great masters of the
healing art, that the malady had some extraordinary origin. There is
reason to believe that a horrible suspicion did actually cross the mind
of Short, who, though skilful in his profession, seems to have been a
nervous and fanciful man, and whose perceptions were probably confused
by dread of the odious imputations to which he, as a Roman Catholic,
was peculiarly exposed. We cannot, therefore, wonder that wild stories
without number were repeated and believed by the common people. His
Majesty's tongue had swelled to the size of a neat's tongue. A cake of
deleterious powder had been found in his brain. There were blue spots on
his breast, There were black spots on his shoulder. Something had been,
put in his snuff-box. Something had been put into his broth. Something
had been put into his favourite dish of eggs and ambergrease. The
Duchess of Portsmouth had poisoned him in a cup of chocolate. The
Queen had poisoned him in a jar of dried pears. Such tales ought to be
preserved; for they furnish us with a measure of the intelligence and
virtue of the generation which eagerly devoured them. That no rumour of
the same kind has ever, in the present age, found credit among us, even
when lives on which great interest depended have been terminated
by unforeseen attacks of disease, is to be attributed partly to the
progress of medical and chemical science, but partly also, it may be
hoped, to the progress which the nation has made in good sense, justice,
and humanity. 
%[222]
\footnote{ Welwood, 139 Burnet, i. 609; Sheffield's Character
of Charles the Second; North's Life of Guildford, 252; Examen,
648; Revolution Politics; Higgons on Burnet. What North says of the
embarrassment and vacillation of the physicians is confirmed by the
despatches of Van Citters. I have been much perplexed by the strange
story about Short's suspicions. I was, at one time, inclined to adopt
North's solution. But, though I attach little weight to the authority of
Welwood and Burnet in such a case, I cannot reject the testimony of so
well informed and so unwilling a witness as Sheffield.}


When all was over, James retired from the bedside to his closet, where,
during a quarter of an hour, he remained alone. Meanwhile the Privy
Councillors who were in the palace assembled. The new King came
forth, and took his place at the head of the board. He commenced his
administration, according to usage, by a speech to the Council. He
expressed his regret for the loss which he had just sustained, and he
promised to imitate the singular lenity which had distinguished the late
reign. He was aware, he said, that he had been accused of a fondness for
arbitrary power. But that was not the only falsehood which had been told
of him. He was resolved to maintain the established government both in
Church and State. The Church of England he knew to be eminently loyal.
It should therefore always be his care to support and defend her. The
laws of England, he also knew, were sufficient to make him as great a
King as he could wish to be. He would not relinquish his own rights; but
he would respect the rights of others. He had formerly risked his life
in defense of his country; and he would still go as far as any man in
support of her just liberties.

This speech was not, like modern speeches on similar occasions,
carefully prepared by the advisers of the sovereign. It was the
extemporaneous expression of the new King's feelings at a moment of
great excitement. The members of the Council broke forth into clamours
of delight and gratitude. The Lord President, Rochester, in the name
of his brethren, expressed a hope that His Majesty's most welcome
declaration would be made public. The Solicitor General, Heneage Finch,
offered to act as clerk. He was a zealous churchman, and, as such, was
naturally desirous that there should be some permanent record of the
gracious promises which had just been uttered. "Those promises," he
said, "have made so deep an impression on me that I can repeat them word
for word." He soon produced his report. James read it, approved of it,
and ordered it to be published. At a later period he said that he had
taken this step without due consideration, that his unpremeditated
expressions touching the Church of England were too strong, and that
Finch had, with a dexterity which at the time escaped notice, made them
still stronger. 
%[223]
\footnote{ London Gazette, Feb. 9. 1684-5; Clarke's Life of James
the Second, ii. 3; Barillon, Feb. 9-19: Evelyn's Diary, Feb. 6.}


The King had been exhausted by long watching and by many violent
emotions. He now retired to rest. The Privy Councillors, having
respectfully accompanied him to his bedchamber, returned to their seats,
and issued orders for the ceremony of proclamation. The Guards were
under arms; the heralds appeared in their gorgeous coats; and the
pageant proceeded without any obstruction. Casks of wine were broken up
in the streets, and all who passed were invited to drink to the health
of the new sovereign. But, though an occasional shout was raised, the
people were not in a joyous mood. Tears were seen in many eyes; and it
was remarked that there was scarcely a housemaid in London who had not
contrived to procure some fragment of black crepe in honour of King
Charles. 
%[224]
\footnote{ See the authorities cited in the last note. See also the
Examen, 647; Burnet, i. 620; Higgons on Burnet.}


The funeral called forth much censure. It would, indeed, hardly have
been accounted worthy of a noble and opulent subject. The Tories gently
blamed the new King's parsimony: the Whigs sneered at his want of
natural affection; and the fiery Covenanters of Scotland exultingly
proclaimed that the curse denounced of old against wicked princes had
been signally fulfilled, and that the departed tyrant had been buried
with the burial of an ass. 
%[225]
\footnote{ London Gazette, Feb. 14, 1684-5; Evelyn's Diary of the
same day; Burnet, i. 610: The Hind let loose.}
 Yet James commenced his administration
with a large measure of public good will. His speech to the Council
appeared in print, and the impression which it produced was highly
favourable to him. This, then, was the prince whom a faction had driven
into exile and had tried to rob of his birthright, on the ground that
he was a deadly enemy to the religion and laws of England. He had
triumphed: he was on the throne; and his first act was to declare that
he would defend the Church, and would strictly respect the rights of
his people. The estimate which all parties had formed of his character,
added weight to every word that fell from him. The Whigs called him
haughty, implacable, obstinate, regardless of public opinion. The
Tories, while they extolled his princely virtues, had often lamented his
neglect of the arts which conciliate popularity. Satire itself had never
represented him as a man likely to court public favour by professing
what he did not feel, and by promising what he had no intention of
performing. On the Sunday which followed his accession, his speech was
quoted in many pulpits. "We have now for our Church," cried one loyal
preacher, "the word of a King, and of a King who was never worse than
his word." This pointed sentence was fast circulated through town and
country, and was soon the watchword of the whole Tory party. 
%[226]
\footnote{ Burnet, i. 628; Lestrange, Observator, Feb. 11, 1684.}


The great offices of state had become vacant by the demise of the crown
and it was necessary for James to determine how they should be filled.
Few of the members of the late cabinet had any reason to expect his
favour. Sunderland, who was Secretary of State, and Godolphin, who was
First Lord of the Treasury, had supported the Exclusion Bill. Halifax,
who held the Privy Seal, had opposed that bill with unrivalled powers
of argument and eloquence. But Halifax was the mortal enemy of despotism
and of Popery. He saw with dread the progress of the French arms on the
Continent and the influence of French gold in the counsels of England.
Had his advice been followed, the laws would have been strictly
observed: clemency would have been extended to the vanquished Whigs: the
Parliament would have been convoked in due season: an attempt would have
been made to reconcile our domestic factions; and the principles of
the Triple Alliance would again have guided our foreign policy. He
had therefore incurred the bitter animosity of James. The Lord Keeper
Guildford could hardly be said to belong to either of the parties into
which the court was divided. He could by no means be called a friend of
liberty; and yet he had so great a reverence for the letter of the
law that he was not a serviceable tool of arbitrary power. He was
accordingly designated by the vehement Tories as a Trimmer, and was to
James an object of aversion with which contempt was largely mingled.
Ormond, who was Lord Steward of the Household and Viceroy of Ireland,
then resided at Dublin. His claims on the royal gratitude were superior
to those of any other subject. He had fought bravely for Charles the
First: he had shared the exile of Charles the Second; and, since the
Restoration, he had, in spite of many provocations, kept his loyalty
unstained. Though he had been disgraced during the predominance of the
Cabal, he had never gone into factious opposition, and had, in the
days of the Popish Plot and the Exclusion Bill, been foremost among the
supporters of the throne. He was now old, and had been recently tried by
the most cruel of all calamities. He had followed to the grave a son who
should have been his own chief mourner, the gallant Ossory. The eminent
services, the venerable age, and the domestic misfortunes of Ormond made
him an object of general interest to the nation. The Cavaliers regarded
him as, both by right of seniority and by right of merit, their head;
and the Whigs knew that, faithful as he had always been to the cause of
monarchy, he was no friend either to Popery or to arbitrary power. But,
high as he stood in the public estimation, he had little favor to expect
from his new master. James, indeed, while still a subject, had urged his
brother to make a complete change in the Irish administration. Charles
had assented; and it had been arranged that, in a few months, there
should be a new Lord Lieutenant. 
%[227]
\footnote{ The letters which passed between Rochester and Ormond on
this subject will be found in the Clarendon Correspondence.}


Rochester was the only member of the cabinet who stood high in the
favour of the King. The general expectation was that he would be
immediately placed at the head of affairs, and that all the other great
officers of the state would be changed. This expectation proved to be
well founded in part only. Rochester was declared Lord Treasurer, and
thus became prime minister. Neither a Lord High Admiral nor a Board of
Admiralty was appointed. The new King, who loved the details of naval
business, and would have made a respectable clerk in a dockyard at
Chatham, determined to be his own minister of marine. Under him the
management of that important department was confided to Samuel Pepys,
whose library and diary have kept his name fresh to our time. No servant
of the late sovereign was publicly disgraced. Sunderland exerted so much
art and address, employed so many intercessors, and was in possession of
so many secrets, that he was suffered to retain his seals. Godolphin's
obsequiousness, industry, experience and taciturnity, could ill
be spared. As he was no longer wanted at the Treasury, he was made
Chamberlain to the Queen. With these three Lords the King took counsel
on all important questions. As to Halifax, Ormond, and Guildford, he
determined not yet to dismiss them, but merely to humble and annoy them.

Halifax was told that he must give up the Privy seal and accept the
Presidency of the Council. He submitted with extreme reluctance. For,
though the President of the Council had always taken precedence of
the Lord Privy Seal, the Lord Privy Seal was, in that age a much more
important officer than the Lord President. Rochester had not forgotten
the jest which had been made a few months before on his own removal from
the Treasury, and enjoyed in his turn the pleasure of kicking his rival
up stairs. The Privy Seal was delivered to Rochester's elder brother,
Henry Earl of Clarendon.

To Barillon James expressed the strongest dislike of Halifax. "I
know him well, I never can trust him. He shall have no share in the
management of public business. As to the place which I have given him,
it will just serve to show how little influence he has." But to Halifax
it was thought convenient to hold a very different language. "All the
past is forgotten," said the King, "except the service which you did me
in the debate on the Exclusion Bill." This speech has often been cited
to prove that James was not so vindictive as he had been called by
his enemies. It seems rather to prove that he by no means deserved the
praises which have been bestowed on his sincerity by his friends. 
%[228]
\footnote{ The ministerial changes are announced in the London
Gazette, Feb. 19, 1684-5. See Burnet, i. 621; Barillon, Feb. 9-19,
16-26; and Feb. 19,/Mar. 1.}


Ormond was politely informed that his services were no longer needed
in Ireland, and was invited to repair to Whitehall, and to perform the
functions of Lord Steward. He dutifully submitted, but did not affect to
deny that the new arrangement wounded his feelings deeply. On the eve of
his departure he gave a magnificent banquet at Kilmainham Hospital, then
just completed, to the officers of the garrison of Dublin. After dinner
he rose, filled a goblet to the brim with wine, and, holding it up,
asked whether he had spilt one drop. "No, gentlemen; whatever the
courtiers may say, I am not yet sunk into dotage. My hand does not fail
me yet: and my hand is not steadier than my heart. To the health of King
James!" Such was the last farewell of Ormond to Ireland. He left the
administration in the hands of Lords Justices, and repaired to London,
where he was received with unusual marks of public respect. Many persons
of rank went forth to meet him on the road. A long train of eguipages
followed him into Saint James's Square, where his mansion stood; and
the Square was thronged by a multitude which greeted him with loud
acclamations. 
%[229]
\footnote{ Carte's Life of Ormond; Secret Consults of the Romish
Party in Ireland, 1690; Memoirs of Ireland, 1716.}


The Great Seal was left in Guildford's custody; but a marked indignity
was at the same time offered to him. It was determined that another
lawyer of more vigour and audacity should be called to assist in the
administration. The person selected was Sir George Jeffreys, Chief
Justice of the Court of King's Bench. The depravity of this man has
passed into a proverb. Both the great English parties have attacked his
memory with emulous violence: for the Whigs considered him as their most
barbarous enemy; and the Tories found it convenient to throw on him the
blame of all the crimes which had sullied their triumph. A diligent and
candid enquiry will show that some frightful stories which have been
told concerning him are false or exaggerated. Yet the dispassionate
historian will be able to make very little deduction from the vast mass
of infamy with which the memory of the wicked judge has been loaded.

He was a man of quick and vigorous parts, but constitutionally prone to
insolence and to the angry passions. When just emerging from boyhood
he had risen into practice at the Old Bailey bar, a bar where advocates
have always used a license of tongue unknown in Westminster Hall. Here,
during many years his chief business was to examine and crossexamine
the most hardened miscreants of a great capital. Daily conflicts
with prostitutes and thieves called out and exercised his powers so
effectually that he became the most consummate bully ever known in his
profession. Tenderness for others and respect for himself were feelings
alike unknown to him. He acquired a boundless command of the rhetoric
in which the vulgar express hatred and contempt. The profusion of
maledictions and vituperative epithets which composed his vocabulary
could hardly have been rivalled in the fishmarket or the beargarden.
His countenance and his voice must always have been unamiable. But these
natural advantages,--for such he seems to have thought them,--he had
improved to such a degree that there were few who, in his paroxysms of
rage, could see or hear him without emotion. Impudence and ferocity sate
upon his brow. The glare of his eyes had a fascination for the unhappy
victim on whom they were fixed. Yet his brow and his eye were less
terrible than the savage lines of his mouth. His yell of fury, as was
said by one who had often heard it, sounded like the thunder of the
judgment day. These qualifications he carried, while still a young man,
from the bar to the bench. He early became Common Serjeant, and then
Recorder of London. As a judge at the City sessions he exhibited the
same propensities which afterwards, in a higher post, gained for him an
unenviable immortality. Already might be remarked in him the most odious
vice which is incident to human nature, a delight in misery merely
as misery. There was a fiendish exultation in the way in which he
pronounced sentence on offenders. Their weeping and imploring seemed
to titillate him voluptuously; and he loved to scare them into fits by
dilating with luxuriant amplification on all the details of what they
were to suffer. Thus, when he had an opportunity of ordering an unlucky
adventuress to be whipped at the cart's tail, "Hangman," he would
exclaim, "I charge you to pay particular attention to this lady! Scourge
her soundly man! Scourge her till the blood runs down! It is Christmas,
a cold time for Madam to strip in! See that you warm her shoulders
thoroughly!" 
%[230]
\footnote{ Christmas Sessions Paper of 1678.}
 He was hardly less facetious when he passed judgment
on poor Lodowick Muggleton, the drunken tailor who fancied himself a
prophet. "Impudent rogue!" roared Jeffreys, "thou shalt have an easy,
easy, easy punishment!" One part of this easy punishment was the
pillory, in which the wretched fanatic was almost killed with brickbats.

%[231]
\footnote{ The Acts of the Witnesses of the Spirit, part v chapter
v. In this work Lodowick, after his fashion, revenges himself on the
"bawling devil," as he calls Jeffreys, by a string of curses which
Ernulphus, or Jeffreys himself, might have envied. The trial was in
January, 1677.}


By this time the heart of Jeffreys had been hardened to that temper
which tyrants require in their worst implements. He had hitherto looked
for professional advancement to the corporation of London. He had
therefore professed himself a Roundhead, and had always appeared to be
in a higher state of exhilaration when he explained to Popish priests
that they were to be cut down alive, and were to see their own bowels
burned, than when he passed ordinary sentences of death. But, as soon
as he had got all that the city could give, he made haste to sell his
forehead of brass and his tongue of venom to the Court. Chiffinch, who
was accustomed to act as broker in infamous contracts of more than one
kind, lent his aid. He had conducted many amorous and many political
intrigues; but he assuredly never rendered a more scandalous service to
his masters than when he introduced Jeffreys to Whitehall. The renegade
soon found a patron in the obdurate and revengeful James, but was always
regarded with scorn and disgust by Charles, whose faults, great as they
were, had no affinity with insolence and cruelty. "That man," said the
King, "has no learning, no sense, no manners, and more impudence than
ten carted street-walkers." 
%[232]
\footnote{ This saying is to be found in many contemporary
pamphlets. Titus Oates was never tired of quoting it. See his Eikwg
Basilikh.}
 Work was to be done, however, which
could be trusted to no man who reverenced law or was sensible of
shame; and thus Jeffreys, at an age at which a barrister thinks himself
fortunate if he is employed to conduct an important cause, was made
Chief Justice of the King's Bench.

His enemies could not deny that he possessed some of the qualities of
a great judge. His legal knowledge, indeed, was merely such as he had
picked up in practice of no very high kind. But he had one of those
happily constituted intellects which, across labyrinths of sophistry,
and through masses of immaterial facts, go straight to the true point.
Of his intellect, however, he seldom had the full use. Even in civil
causes his malevolent and despotic temper perpetually disordered his
judgment. To enter his court was to enter the den of a wild beast,
which none could tame, and which was as likely to be roused to rage by
caresses as by attacks. He frequently poured forth on plaintiffs and
defendants, barristers and attorneys, witnesses and jurymen, torrents of
frantic abuse, intermixed with oaths and curses. His looks and tones
had inspired terror when he was merely a young advocate struggling into
practice. Now that he was at the head of the most formidable tribunal
in the realm, there were few indeed who did not tremble before him.
Even when he was sober, his violence was sufficiently frightful. But in
general his reason was overclouded and his evil passions stimulated
by the fumes of intoxication. His evenings were ordinarily given to
revelry. People who saw him only over his bottle would have supposed him
to be a man gross indeed, sottish, and addicted to low company and low
merriment, but social and goodhumoured. He was constantly surrounded on
such occasions by buffoons selected, for the most part, from among the
vilest pettifoggers who practiced before him. These men bantered and
abused each other for his entertainment. He joined in their ribald talk,
sang catches with them, and, when his head grew hot, hugged and kissed
them in an ecstasy of drunken fondness. But though wine at first seemed
to soften his heart, the effect a few hours later was very different. He
often came to the judgment seat, having kept the court waiting long, and
yet having but half slept off his debauch, his cheeks on fire, his eyes
staring like those of a maniac. When he was in this state, his boon
companions of the preceding night, if they were wise, kept out of his
way: for the recollection of the familiarity to which he had admitted
them inflamed his malignity; and he was sure to take every opportunity
of overwhelming them with execration and invective. Not the least odious
of his many odious peculiarities was the pleasure which he took in
publicly browbeating and mortifying those whom, in his fits of maudlin
tenderness, he had encouraged to presume on his favour.

The services which the government had expected from him were performed,
not merely without flinching, but eagerly and triumphantly. His first
exploit was the judicial murder of Algernon Sidney. What followed was
in perfect harmony with this beginning. Respectable Tories lamented the
disgrace which the barbarity and indecency of so great a functionary
brought upon the administration of justice. But the excesses which
filled such men with horror were titles to the esteem of James.
Jeffreys, therefore, very soon after the death of Charles, obtained a
seat in the cabinet and a peerage. This last honour was a signal mark of
royal approbation. For, since the judicial system of the realm had been
remodelled in the thirteenth century, no Chief Justice had been a Lord
of Parliament. 
%[233]
\footnote{ The chief sources of information concerning Jeffreys are
the State Trials and North's Life of Lord Guildford. Some touches of
minor importance I owe to contemporary pamphlets in verse and prose.
Such are the Bloody Assizes the life and Death of George Lord Jeffreys,
the Panegyric on the late Lord Jeffreys, the Letter to the Lord
Chancellor, Jeffreys's Elegy. See also Evelyn's Diary, Dec. 5, 1683,
Oct. 31. 1685. I scarcely need advise every reader to consult Lord
Campbell's excellent Life of Jeffreys.}


Guildford now found himself superseded in all his political functions,
and restricted to his business as a judge in equity. At Council he was
treated by Jeffreys with marked incivility. The whole legal patronage
was in the hands of the Chief Justice; and it was well known by the bar
that the surest way to propitiate the Chief Justice was to treat the
Lord Keeper with disrespect.

James had not been many hours King when a dispute arose between the two
heads of the law. The customs had been settled on Charles for life only,
and could not therefore be legally exacted by the new sovereign. Some
weeks must elapse before a House of Commons could be chosen. If, in
the meantime, the duties were suspended, the revenue would suffer; the
regular course of trade would be interrupted; the consumer would derive
no benefit, and the only gainers would be those fortunate speculators
whose cargoes might happen to arrive during the interval between the
demise of the crown and the meeting of the Parliament. The Treasury was
besieged by merchants whose warehouses were filled with goods on which
duty had been paid, and who were in grievous apprehension of being
undersold and ruined. Impartial men must admit that this was one of
those cases in which a government may be justified in deviating from the
strictly constitutional course. But when it is necessary to deviate from
the strictly constitutional course, the deviation clearly ought to be
no greater than the necessity requires. Guildford felt this, and gave
advice which did him honour. He proposed that the duties should be
levied, but should be kept in the Exchequer apart from other sums till
the Parliament should meet. In this way the King, while violating
the letter of the laws, would show that he wished to conform to their
spirit, Jeffreys gave very different counsel. He advised James to put
forth an edict declaring it to be His Majesty's will and pleasure that
the customs should continue to be paid. This advice was well suited
to the King's temper. The judicious proposition of the Lord Keeper
was rejected as worthy only of a Whig, or of what was still worse,
a Trimmer. A proclamation, such as the Chief Justice had suggested,
appeared. Some people had expected that a violent outbreak of public
indignation would be the consequence; but they were deceived. The spirit
of opposition had not yet revived; and the court might safely venture to
take steps which, five years before, would have produced a rebellion.
In the City of London, lately so turbulent, scarcely a murmur was heard.

%[234]
\footnote{ London Gazette, Feb. 12, 1684-5. North's Life of
Guildford, 254.}


The proclamation, which announced that the customs would still be
levied, announced also that a Parliament would shortly meet. It was not
without many misgivings that James had determined to call the Estates of
his realm together. The moment was, indeed most auspicious for a general
election. Never since the accession of the House of Stuart had the
constituent bodies been so favourably disposed towards the Court.
But the new sovereign's mind was haunted by an apprehension not to be
mentioned even at this distance of time, without shame and indignation.
He was afraid that by summoning his Parliament he might incur the
displeasure of the King of France.

To the King of France it mattered little which of the two English
factions triumphed at the elections: for all the Parliaments which had
met since the Restoration, whatever might have been their temper as to
domestic politics, had been jealous of the growing power of the House of
Bourbon. On this subject there was little difference between the Whigs
and the sturdy country gentlemen who formed the main strength of the
Tory party. Lewis had therefore spared neither bribes nor menaces to
prevent Charles from convoking the Houses; and James, who had from the
first been in the secret of his brother's foreign politics, had, in
becoming King of England, become also a hireling and vassal of France.

Rochester, Godolphin, and Sunderland, who now formed the interior
cabinet, were perfectly aware that their late master had been in
the habit of receiving money from the court of Versailles. They were
consulted by James as to the expediency of convoking the legislature.
They acknowledged the importance of keeping Lewis in good humour: but
it seemed to them that the calling of a Parliament was not a matter of
choice. Patient as the nation appeared to be, there were limits to its
patience. The principle, that the money of the subject could not be
lawfully taken by the King without the assent of the Commons, was firmly
rooted in the public mind; and though, on all extraordinary emergency
even Whigs might be willing to pay, during a few weeks, duties not
imposed by statute, it was certain that even Tories would become
refractory if such irregular taxation should continue longer than the
special circumstances which alone justified it. The Houses then must
meet; and since it was so, the sooner they were summoned the better.
Even the short delay which would be occasioned by a reference to
Versailles might produce irreparable mischief. Discontent and suspicion
would spread fast through society. Halifax would complain that the
fundamental principles of the constitution were violated. The Lord
Keeper, like a cowardly pedantic special pleader as he was, would take
the same side. What might have been done with a good grace would at last
be done with a bad grace. Those very ministers whom His Majesty most
wished to lower in the public estimation would gain popularity at his
expense. The ill temper of the nation might seriously affect the result
of the elections. These arguments were unanswerable. The King therefore
notified to the country his intention of holding a Parliament. But he
was painfully anxious to exculpate himself from the guilt of having
acted undutifully and disrespectfully towards France. He led Barillon
into a private room, and there apologised for having dared to take so
important a step without the previous sanction of Lewis. "Assure your
master," said James, "of my gratitude and attachment. I know that
without his protection I can do nothing. I know what troubles my brother
brought on himself by not adhering steadily to France. I will take good
care not to let the Houses meddle with foreign affairs. If I see in them
any disposition to make mischief, I will send them about their business.
Explain this to my good brother. I hope that he will not take it
amiss that I have acted without consulting him. He has a right to be
consulted; and it is my wish to consult him about everything. But
in this case the delay even of a week might have produced serious
consequences."

These ignominious excuses were, on the following morning, repeated by
Rochester. Barillon received them civilly. Rochester, grown bolder,
proceeded to ask for money. "It will be well laid out," he said: "your
master cannot employ his revenues better. Represent to him strongly how
important it is that the King of England should be dependent, not on his
own people, but on the friendship of France alone." 
%[235]
\footnote{ The chief authority for these transactions is Barillon's
despatch of February 9-19, 1685. It will be found in the Appendix to Mr.
Fox's History. See also Preston's Letter to James, dated April 18-28,
1685, in Dalrymple.}


Barillon hastened to communicate to Lewis the wishes of the English
government; but Lewis had already anticipated them. His first act,
after he was apprised of the death of Charles, was to collect bills of
exchange on England to the amount of five hundred thousand livres, a sum
equivalent to about thirty-seven thousand five hundred pounds sterling
Such bills were not then to be easily procured in Paris at day's notice.
In a few hours, however, the purchase was effected, and a courier
started for London. 
%[236]
\footnote{ Lewis to Barillon, Feb. 16-26, 1685.}
 As soon as Barillon received the remittance,
he flew to Whitehall, and communicated the welcome news. James was not
ashamed to shed, or pretend to shed, tears of delight and gratitude.
"Nobody but your King," he said, "does such kind, such noble things. I
never can be grateful enough. Assure him that my attachment will last
to the end of my days." Rochester, Sunderland, and Godolphin came, one
after another, to embrace the ambassador, and to whisper to him that he
had given new life to their royal master. 
%[237]
\footnote{ Barillon, Feb. 16-26, 1685.}


But though James and his three advisers were pleased with the
promptitude which Lewis had shown, they were by no means satisfied with
the amount of the donation. As they were afraid, however, that they
might give offence by importunate mendicancy, they merely hinted their
wishes. They declared that they had no intention of haggling with so
generous a benefactor as the French King, and that they were willing to
trust entirely to his munificence. They, at the same time, attempted
to propitiate him by a large sacrifice of national honour. It was
well known that one chief end of his politics was to add the Belgian
provinces to his dominions. England was bound by a treaty which had
been concluded with Spain when Danby was Lord Treasurer, to resist any
attempt which France might make on those provinces. The three ministers
informed Barillon that their master considered that treaty as no longer
obligatory. It had been made, they said, by Charles: it might, perhaps,
have been binding on him; but his brother did not think himself bound
by it. The most Christian King might, therefore, without any fear of
opposition from England, proceed to annex Brabant and Hainault to his
empire. 
%[238]
\footnote{ Barillon, Feb. 18-28, 1685.}


It was at the same time resolved that an extraordinary embassy should be
sent to assure Lewis of the gratitude and affection of James. For this
mission was selected a man who did not as yet occupy a very eminent
position, but whose renown, strangely made up of infamy and glory,
filled at a later period the whole civilized world.

Soon after the Restoration, in the gay and dissolute times which have
been celebrated by the lively pen of Hamilton, James, young and ardent
in the pursuit of pleasure, had been attracted to Arabella Churchill,
one of the maids of honour who waited on his first wife. The young
lady was plain: but the taste of James was not nice: and she became
his avowed mistress. She was the daughter of a poor Cavalier knight who
haunted Whitehall, and made himself ridiculous by publishing a dull and
affected folio, long forgotten, in praise of monarchy and monarchs. The
necessities of the Churchills were pressing: their loyalty was ardent:
and their only feeling about Arabella's seduction seems to have been
joyful surprise that so homely a girl should have attained such high
preferment.

Her interest was indeed of great use to her relations: but none of them
was so fortunate as her eldest brother John, a fine youth, who carried a
pair of colours in the foot guards. He rose fast in the court and in the
army, and was early distinguished as a man of fashion and of pleasure.
His stature was commanding, his face handsome, his address singularly
winning, yet of such dignity that the most impertinent fops never
ventured to take any liberty with him; his temper, even in the most
vexatious and irritating circumstances, always under perfect command.
His education had been so much neglected that he could not spell the
most common words of his own language: but his acute and vigorous
understanding amply supplied the place of book learning. He was not
talkative: but when he was forced to speak in public, his natural
eloquence moved the envy of practiced rhetoricians. 
%[239]
\footnote{ Swift who hated Marlborough, and who was little disposed
to allow any merit to those whom he hated, says, in the famous letter to
Crassus, "You are no ill orator in the Senate."}
 His courage
was singularly cool and imperturbable. During many years of anxiety and
peril, he never, in any emergency, lost even for a moment, the perfect
use of his admirable judgment.

In his twenty-third year he was sent with his regiment to join the
French forces, then engaged in operations against Holland. His serene
intrepidity distinguished him among thousands of brave soldiers. His
professional skill commanded the respect of veteran officers. He was
publicly thanked at the head of the army, and received many marks
of esteem and confidence from Turenne, who was then at the height of
military glory.

Unhappily the splendid qualities of John Churchill were mingled with
alloy of the most sordid kind. Some propensities, which in youth are
singularly ungraceful, began very early to show themselves in him. He
was thrifty in his very vices, and levied ample contributions on ladies
enriched by the spoils of more liberal lovers. He was, during a short
time, the object of the violent but fickle fondness of the Duchess of
Cleveland. On one occasion he was caught with her by the King, and was
forced to leap out of the window. She rewarded this hazardous feat of
gallantry with a present of five thousand pounds. With this sum the
prudent young hero instantly bought an annuity of five hundred a year,
well secured on landed property. 
%[240]
\footnote{ Dartmouth's note on Burnet, i. 264. Chesterfleld's
Letters, Nov., 18, 1748. Chesterfield is an unexceptional witness; for
the annuity was a charge on the estate of his grandfather, Halifax. I
believe that there is no foundation for a disgraceful addition to the
story which may be found in Pope:

    "The gallant too, to whom she paid it down,
    Lived to refuse his mistress half a crown."
    Curll calls this a piece of travelling scandal.}
 Already his private drawer
contained a hoard of broad pieces which, fifty years later, when he
was a Duke, a Prince of the Empire, and the richest subject in Europe,
remained untouched. 
%[241]
\footnote{ Pope in Spence's Anecdotes.}


After the close of the war he was attached to the household of the Duke
of York, accompanied his patron to the Low Countries and to Edinburgh,
and was rewarded for his services with a Scotch peerage and with the
command of the only regiment of dragoons which was then on the English
establishment. 
%[242]
\footnote{ See the Historical Records of the first or Royal
Dragoons. The appointment of Churchill to the command of this regiment
was ridiculed as an instance of absurd partiality. One lampoon of that
time which I do not remember to have seen in print, but of which a
manuscript copy is in the British Museum, contains these lines:

     "Let's cut our meat with spoons:
     The sense is as good
     As that Churchill should
     Be put to command the dragoons."}
 His wife had a post in the family of James's
younger daughter, the Princess of Denmark.

Lord Churchill was now sent as ambassador extraordinary to Versailles.
He had it in charge to express the warm gratitude of the English
government for the money which had been so generously bestowed. It had
been originally intended that he should at the same time ask Lewis for
a much larger sum; but, on full consideration, it was apprehended
that such indelicate greediness might disgust the benefactor whose
spontaneous liberality had been so signally displayed. Churchill was
therefore directed to confine himself to thanks for what was past, and
to say nothing about the future. 
%[243]
\footnote{ Barillon, Feb. 16-26, 1685.}


But James and his ministers, even while protesting that they did not
mean to be importunate, contrived to hint, very intelligibly, what they
wished and expected. In the French ambassador they had a dexterous, a
zealous, and perhaps, not a disinterested intercessor. Lewis made some
difficulties, probably with the design of enhancing the value of his
gifts. In a very few weeks, however, Barillon received from Versailles
fifteen hundred thousand livres more. This sum, equivalent to about a
hundred and twelve thousand pounds sterling, he was instructed to dole
out cautiously. He was authorised to furnish the English government with
thirty thousand pounds, for the purpose of corrupting members of the New
House of Commons. The rest he was directed to keep in reserve for some
extraordinary emergency, such as a dissolution or an insurrection. 
%[244]
\footnote{ Barillon, April 6-16; Lewis to Barillon, April 14-24.}


The turpitude of these transactions is universally acknowledged: but
their real nature seems to be often misunderstood: for though the
foreign policy of the last two Kings of the House of Stuart has never,
since the correspondence of Barillon was exposed to the public eye,
found an apologist among us, there is still a party which labours to
excuse their domestic policy. Yet it is certain that between their
domestic policy and their foreign policy there was a necessary and
indissoluble connection. If they had upheld, during a single year, the
honour of the country abroad, they would have been compelled to change
the whole system of their administration at home. To praise them for
refusing to govern in conformity with the sense of Parliament, and yet
to blame them for submitting to the dictation of Lewis, is inconsistent.
For they had only one choice, to be dependent on Lewis, or to be
dependent on Parliament.

James, to do him justice, would gladly have found out a third way: but
there was none. He became the slave of France: but it would be incorrect
to represent him as a contented slave. He had spirit enough to be at
times angry with himself for submitting to such thraldom, and impatient
to break loose from it; and this disposition was studiously encouraged
by the agents of many foreign powers.

His accession had excited hopes and fears in every continental court:
and the commencement of his administration was watched by strangers
with interest scarcely less deep than that which was felt by his own
subjects. One government alone wished that the troubles which had,
during three generations, distracted England, might be eternal. All
other governments, whether republican or monarchical, whether Protestant
or Roman Catholic, wished to see those troubles happily terminated.

The nature of the long contest between the Stuarts and their Parliaments
was indeed very imperfectly apprehended by foreign statesmen: but no
statesman could fail to perceive the effect which that contest had
produced on the balance of power in Europe. In ordinary circumstances,
the sympathies of the courts of Vienna and Madrid would doubtless have
been with a prince struggling against subjects, and especially with a
Roman Catholic prince struggling against heretical subjects: but all
such sympathies were now overpowered by a stronger feeling. The fear and
hatred inspired by the greatness, the injustice, and the arrogance of
the French King were at the height. His neighbours might well doubt
whether it were more dangerous to be at war or at peace with him. For
in peace he continued to plunder and to outrage them; and they had tried
the chances of war against him in vain. In this perplexity they looked
with intense anxiety towards England. Would she act on the principles of
the Triple Alliance or on the principles of the treaty of Dover? On that
issue depended the fate of all her neighbours. With her help Lewis might
yet be withstood: but no help could be expected from her till she was
at unity with herself. Before the strife between the throne and the
Parliament began, she had been a power of the first rank: on the day on
which that strife terminated she became a power of the first rank again:
but while the dispute remained undecided, she was condemned to inaction
and to vassalage. She had been great under the Plantagenets and Tudors:
she was again great under the princes who reigned after the Revolution:
but, under the Kings of the House of Stuart, she was a blank in the map
of Europe. She had lost one class of energies, and had not yet acquired
another. That species of force, which, in the fourteenth century had
enabled her to humble France and Spain, had ceased to exist. That
species of force, which, in the eighteenth century, humbled France and
Spain once more, had not yet been called into action. The government was
no longer a limited monarchy after the fashion of the middle ages. It
had not yet become a limited monarchy after the modern fashion. With
the vices of two different systems it had the strength of neither. The
elements of our polity, instead of combining in harmony, counteracted
and neutralised each other All was transition, conflict, and disorder.
The chief business of the sovereign was to infringe the privileges of
the legislature. The chief business of the legislature was to encroach
on the prerogatives of the sovereign. The King readily accepted foreign
aid, which relieved him from the misery of being dependent on a mutinous
Parliament. The Parliament refused to the King the means of supporting
the national honor abroad, from an apprehension, too well founded, that
those means might be employed in order to establish despotism at home.
The effect of these jealousies was that our country, with all her vast
resources, was of as little weight in Christendom as the duchy of Savoy
or the duchy of Lorraine, and certainly of far less weight than the
small province of Holland.

France was deeply interested in prolonging this state of things. 
%[245]
\footnote{ I might transcribe half Barillon's correspondence in
proof of this proposition, but I will quote only one passage, in
which the policy of the French government towards England is exhibited
concisely and with perfect clearness.---- "On peut tenir pour un maxime
indubitable que l'accord du Roy d'Angleterre avec son parlement, en
quelque maniere qu'il se fasse, n'est pas conforme aux interets de V. M.
Je me contente de penser cela sane m'en ouvrir a personne, et je cache
avec soin mes sentimens a cet egard."--Barillon to Lewis, Feb. 28,/Mar.
1687. That this was the real secret of the whole policy of Lewis towards
our country was perfectly understood at Vienna. The Emperor Leopold
wrote thus to James, March 30,/April 9, 1689: "Galli id unum agebant,
ut, perpetuas inter Serenitatem vestram et ejusdem populos fovendo
simultates, reliquæ Christianæ Europe tanto securius insultarent."}

All other powers were deeply interested in bringing it to a close. The
general wish of Europe was that James would govern in conformity with
law and with public opinion. From the Escurial itself came letters,
expressing an earnest hope that the new King of England would be on good
terms with his Parliament and his people. 
%[246]
\footnote{ "Que sea unido con su reyno, yen todo buena intelligencia
con el parlamenyo." Despatch from the King of Spain to Don Pedro
Ronquillo, March 16-26, 1685. This despatch is in the archives of
Samancas, which contain a great mass of papers relating to English
affairs. Copies of the most interesting of those papers are in the
possession of M. Guizot, and were by him lent to me. It is with peculiar
pleasure that at this time, I acknowledge this mark of the friendship of
so great a man. (1848.)}
 From the Vatican itself
came cautions against immoderate zeal for the Roman Catholic faith.
Benedict Odescalchi, who filled the papal chair under the name of
Innocent the Eleventh, felt, in his character of temporal sovereign, all
those apprehensions with which other princes watched the progress of the
French power. He had also grounds of uneasiness which were peculiar to
himself. It was a happy circumstance for the Protestant religion that,
at the moment when the last Roman Catholic King of England mounted the
throne, the Roman Catholic Church was torn by dissension, and threatened
with a new schism. A quarrel similar to that which had raged in the
eleventh century between the Emperors and the Supreme Pontiffs had
arisen between Lewis and Innocent. Lewis, zealous even to bigotry
for the doctrines of the Church of Rome, but tenacious of his regal
authority, accused the Pope of encroaching on the secular rights of the
French Crown, and was in turn accused by the Pope of encroaching on the
spiritual power of the keys. The King, haughty as he was, encountered a
spirit even more determined than his own. Innocent was, in all private
relations, the meekest and gentlest of men: but when he spoke officially
from the chair of St. Peter, he spoke in the tones of Gregory the
Seventh and of Sixtus the Fifth. The dispute became serious. Agents of
the King were excommunicated. Adherents of the Pope were banished. The
King made the champions of his authority Bishops. The Pope refused them
institution. They took possession of the Episcopal palaces and revenues:
but they were incompetent to perform the Episcopal functions. Before the
struggle terminated, there were in France thirty prelates who could not
confirm or ordain. 
%[247]
\footnote{ Few English readers will be desirous to go deep into the
history of this quarrel. Summaries will be found in Cardinal Bausset's
Life of Bossuet, and in Voltaire's Age of Lewis XIV.}


Had any prince then living, except Lewis, been engaged in such a dispute
with the Vatican, he would have had all Protestant governments on his
side. But the fear and resentment which the ambition and insolence of
the French King had inspired were such that whoever had the courage
manfully to oppose him was sure of public sympathy. Even Lutherans and
Calvinists, who had always detested the Pope, could not refrain from
wishing him success against a tyrant who aimed at universal monarchy.
It was thus that, in the present century, many who regarded Pius the
Seventh as Antichrist were well pleased to see Antichrist confront the
gigantic power of Napoleon.

The resentment which Innocent felt towards France disposed him to take
a mild and liberal view of the affairs of England. The return of
the English people to the fold of which he was the shepherd would
undoubtedly have rejoiced his soul. But he was too wise a man to believe
that a nation so bold and stubborn, could be brought back to the Church
of Rome by the violent and unconstitutional exercise of royal authority.
It was not difficult to foresee that, if James attempted to promote the
interests of his religion by illegal and unpopular means, the attempt
would fail; the hatred with which the heretical islanders regarded
the true faith would become fiercer and stronger than ever; and an
indissoluble association would be created in their minds between
Protestantism and civil freedom, between Popery and arbitrary power. In
the meantime the King would be an object of aversion and suspicion to
his people. England would still be, as she had been under James the
First, under Charles the First, and under Charles the Second, a power of
the third rank; and France would domineer unchecked beyond the Alps and
the Rhine. On the other hand, it was probable that James, by acting with
prudence and moderation, by strictly observing the laws and by exerting
himself to win the confidence of his Parliament, might be able to
obtain, for the professors of his religion, a large measure of relief.
Penal statutes would go first. Statutes imposing civil incapacities
would soon follow. In the meantime, the English King and the English
nation united might head the European coalition, and might oppose an
insuperable barrier to the cupidity of Lewis.

Innocent was confirmed in his judgment by the principal Englishmen who
resided at his court. Of these the most illustrious was Philip Howard,
sprung from the noblest houses of Britain, grandson, on one side, of an
Earl of Arundel, on the other, of a Duke of Lennox. Philip had long
been a member of the sacred college: he was commonly designated as the
Cardinal of England; and he was the chief counsellor of the Holy See in
matters relating to his country. He had been driven into exile by the
outcry of Protestant bigots; and a member of his family, the unfortunate
Stafford, had fallen a victim to their rage. But neither the Cardinal's
own wrongs, nor those of his house, had so heated his mind as to make
him a rash adviser. Every letter, therefore, which went from the Vatican
to Whitehall, recommended patience, moderation, and respect for the
prejudices of the English people. 
%[248]
\footnote{ Burnet, i. 661, and Letter from Rome, Dodd's Church
History, part viii. book i. art. 1.}


In the mind of James there was a great conflict. We should do him
injustice if we supposed that a state of vassalage was agreeable to his
temper. He loved authority and business. He had a high sense of his own
personal dignity. Nay, he was not altogether destitute of a sentiment
which bore some affinity to patriotism. It galled his soul to think that
the kingdom which he ruled was of far less account in the world than
many states which possessed smaller natural advantages; and he listened
eagerly to foreign ministers when they urged him to assert the dignity
of his rank, to place himself at the head of a great confederacy, to
become the protector of injured nations, and to tame the pride of that
power which held the Continent in awe. Such exhortations made his heart
swell with emotions unknown to his careless and effeminate brother.
But those emotions were soon subdued by a stronger feeling. A vigorous
foreign policy necessarily implied a conciliatory domestic policy. It
was impossible at once to confront the might of France and to trample
on the liberties of England. The executive government could undertake
nothing great without the support of the Commons, and could obtain their
support only by acting in conformity with their opinion. Thus James
found that the two things which he most desired could not be enjoyed
together. His second wish was to be feared and respected abroad. But his
first wish was to be absolute master at home. Between the incompatible
objects on which his heart was set he, for a time, went irresolutely
to and fro. The conflict in his own breast gave to his public acts a
strange appearance of indecision and insincerity. Those who, without
the clue, attempted to explore the maze of his politics were unable to
understand how the same man could be, in the same week, so haughty and
so mean. Even Lewis was perplexed by the vagaries of an ally who passed,
in a few hours, from homage to defiance, and from defiance to
homage. Yet, now that the whole conduct of James is before us, this
inconsistency seems to admit of a simple explanation.

At the moment of his accession he was in doubt whether the kingdom
would peaceably submit to his authority. The Exclusionists, lately so
powerful, might rise in arms against him. He might be in great need
of French money and French troops. He was therefore, during some days,
content to be a sycophant and a mendicant. He humbly apologised for
daring to call his Parliament together without the consent of the French
government. He begged hard for a French subsidy. He wept with joy over
the French bills of exchange. He sent to Versailles a special embassy
charged with assurances of his gratitude, attachment, and submission.
But scarcely had the embassy departed when his feelings underwent a
change. He had been everywhere proclaimed without one riot, without
one seditions outcry. From all corners of the island he received
intelligence that his subjects were tranquil and obedient. His spirit
rose. The degrading relation in which he stood to a foreign power seemed
intolerable. He became proud, punctilious, boastful, quarrelsome. He
held such high language about the dignity of his crown and the balance
of power that his whole court fully expected a complete revolution in
the foreign politics of the realm. He commanded Churchill to send home a
minute report of the ceremonial of Versailles, in order that the honours
with which the English embassy was received there might be repaid, and
not more than repaid, to the representative of France at Whitehall. The
news of this change was received with delight at Madrid, Vienna, and
the Hague. 
%[249]
\footnote{ Consultations of the Spanish Council of State on April
2-12 and April 16-26, In the Archives of Simancas.}
 Lewis was at first merely diverted. "My good ally talks
big," he said; "but he is as fond of my pistoles as ever his brother
was." Soon, however, the altered demeanour of James, and the hopes with
which that demeanour inspired both the branches of the House of Austria,
began to call for more serious notice. A remarkable letter is still
extant, in which the French King intimated a strong suspicion that he
had been duped, and that the very money which he had sent to Westminster
would be employed against him. 
%[250]
\footnote{ Lewis to Barillon, May 22,/June 1, 1685; Burnet, i. 623.}


By this time England had recovered from the sadness and anxiety caused
by the death of the goodnatured Charles. The Tories were loud in
professions of attachment to their new master. The hatred of the Whigs
was kept down by fear. That great mass which is not steadily Whig or
Tory, but which inclines alternately to Whiggism and to Toryism, was
still on the Tory side. The reaction which had followed the dissolution
of the Oxford parliament had not yet spent its force.

The King early put the loyalty of his Protestant friends to the proof.
While he was a subject, he had been in the habit of hearing mass with
closed doors in a small oratory which had been fitted up for his wife.
He now ordered the doors to be thrown open, in order that all who came
to pay their duty to him might see the ceremony. When the host was
elevated there was a strange confusion in the antechamber. The Roman
Catholics fell on their knees: the Protestants hurried out of the room.
Soon a new pulpit was erected in the palace; and, during Lent, a
series of sermons was preached there by Popish divines, to the great
discomposure of zealous churchmen. 
%[251]
\footnote{ Life of James the Second, i. 5. Barillon, Feb. 19,/Mar.
1, 1685; Evelyn's Diary, March 5, 1685.}


A more serious innovation followed. Passion week came; and the King
determined to hear mass with the same pomp with which his predecessors
had been surrounded when they repaired to the temples of the established
religion. He announced his intention to the three members of the
interior cabinet, and requested them to attend him. Sunderland, to
whom all religions were the same, readily consented. Godolphin, as
Chamberlain of the Queen, had already been in the habit of giving her
his hand when she repaired to her oratory, and felt no scruple about
bowing himself officially in the house of Rimmon. But Rochester was
greatly disturbed. His influence in the country arose chiefly from the
opinion entertained by the clergy and by the Tory gentry, that he was a
zealous and uncompromising friend of the Church. His orthodoxy had been
considered as fully atoning for faults which would otherwise have made
him the most unpopular man in the kingdom, for boundless arrogance,
for extreme violence of temper, and for manners almost brutal. 
%[252]
\footnote{

     "To those that ask boons
     He swears by God's oons
     And chides them as if they came there to steal spoons."
          Lamentable Lory, a ballad, 1684.}
 He
feared that, by complying with the royal wishes, he should greatly
lower himself in the estimation of his party. After some altercation
he obtained permission to pass the holidays out of town. All the other
great civil dignitaries were ordered to be at their posts on Easter
Sunday. The rites of the Church of Rome were once more, after an
interval of a hundred and twenty-seven years, performed at Westminster
with regal splendour. The Guards were drawn out. The Knights of the
Garter wore their collars. The Duke of Somerset, second in rank among
the temporal nobles of the realm, carried the sword of state. A long
train of great lords accompanied the King to his seat. But it was
remarked that Ormond and Halifax remained in the antechamber. A few
years before they had gallantly defended the cause of James against some
of those who now pressed past them. Ormond had borne no share in the
slaughter of Roman Catholics. Halifax had courageously pronounced
Stafford not guilty. As the timeservers who had pretended to shudder at
the thought of a Popish king, and who had shed without pity the innocent
blood of a Popish peer, now elbowed each other to get near a Popish
altar, the accomplished Trimmer might, with some justice, indulge his
solitary pride in that unpopular nickname. 
%[253]
\footnote{ Barillon, April 20-30. 1685.}


Within a week after this ceremony James made a far greater sacrifice
of his own religious prejudices than he had yet called on any of his
Protestant subjects to make. He was crowned on the twenty-third of
April, the feast of the patron saint of the realm. The Abbey and the
Hall were splendidly decorated. The presence of the Queen and of the
peeresses gave to the solemnity a charm which had been wanting to the
magnificent inauguration of the late King. Yet those who remembered that
inauguration pronounced that there was a great falling off. The ancient
usage was that, before a coronation, the sovereign, with all his
heralds, judges, councillors, lords, and great dignitaries, should ride
in state from the Tower of Westminster. Of these cavalcades the last and
the most glorious was that which passed through the capital while the
feelings excited by the Restoration were still in full vigour. Arches of
triumph overhung the road. All Cornhill, Cheapside, Saint Paul's Church
Yard, Fleet Street, and the Strand, were lined with scaffolding.
The whole city had thus been admitted to gaze on royalty in the most
splendid and solemn form that royalty could wear. James ordered an
estimate to be made of the cost of such a procession, and found that it
would amount to about half as much as he proposed to expend in covering
his wife with trinkets. He accordingly determined to be profuse where he
ought to have been frugal, and niggardly where he might pardonably
have been profuse. More than a hundred thousand pounds were laid out in
dressing the Queen, and the procession from the Tower was omitted. The
folly of this course is obvious. If pageantry be of any use in politics,
it is of use as a means of striking the imagination of the multitude. It
is surely the height of absurdity to shut out the populace from a show
of which the main object is to make an impression on the populace.
James would have shown a more judicious munificence and a more judicious
parsimony, if he had traversed London from east to west with the
accustomed pomp, and had ordered the robes of his wife to be somewhat
less thickly set with pearls and diamonds. His example was, however,
long followed by his successors; and sums, which, well employed, would
have afforded exquisite gratification to a large part of the nation,
were squandered on an exhibition to which only three or four thousand
privileged persons were admitted. At length the old practice was
partially revived. On the day of the coronation of Queen Victoria there
was a procession in which many deficiencies might be noted, but which
was seen with interest and delight by half a million of her subjects,
and which undoubtedly gave far greater pleasure, and called forth far
greater enthusiasm, than the more costly display which was witnessed by
a select circle within the Abbey.

James had ordered Sancroft to abridge the ritual. The reason publicly
assigned was that the day was too short for all that was to be done.
But whoever examines the changes which were made will see that the
real object was to remove some things highly offensive to the religious
feelings of a zealous Roman Catholic. The Communion Service was not
read. The ceremony of presenting the sovereign with a richly bound copy
of the English Bible, and of exhorting him to prize above all earthly
treasures a volume which he had been taught to regard as adulterated
with false doctrine, was omitted. What remained, however, after all this
curtailment, might well have raised scruples in the mind of a man who
sincerely believed the Church of England to be a heretical society,
within the pale of which salvation was not to be found. The King made
an oblation on the altar. He appeared to join in the petitions of the
Litany which was chaunted by the Bishops. He received from those false
prophets the unction typical of a divine influence, and knelt with the
semblance of devotion, while they called down upon him that Holy Spirit
of which they were, in his estimation, the malignant and obdurate foes.
Such are the inconsistencies of human nature that this man, who, from a
fanatical zeal for his religion, threw away three kingdoms, yet chose to
commit what was little short of an act of apostasy, rather than forego
the childish pleasure of being invested with the gewgaws symbolical of
kingly power. 
%[254]
\footnote{ From Adda's despatch of Jan. 22,/Feb. 1, 1686, and
from the expressions of the Pere d'Orleans (Histoire des Revolutions
d'Angleterre, liv. xi.), it is clear that rigid Catholics thought the
King's conduct indefensible.}


Francis Turner, Bishop of Ely, preached. He was one of those writers
who still affected the obsolete style of Archbishop Williams and Bishop
Andrews. The sermon was made up of quaint conceits, such as seventy
years earlier might have been admired, but such as moved the scorn of a
generation accustomed to the purer eloquence of Sprat, of South, and of
Tillotson. King Solomon was King James. Adonijah was Monmouth. Joab was
a Rye House conspirator; Shimei, a Whig libeller; Abiathar, an honest
but misguided old Cavalier. One phrase in the Book of Chronicles was
construed to mean that the King was above the Parliament; and another
was cited to prove that he alone ought to command the militia. Towards
the close of the discourse the orator very timidly alluded to the new
and embarrassing position in which the Church stood with reference to
the sovereign, and reminded his hearers that the Emperor Constantius
Chlorus, though not himself a Christian, had held in honour those
Christians who remained true to their religion, and had treated with
scorn those who sought to earn his favour by apostasy. The service in
the Abbey was followed by a stately banquet in the Hall, the banquet by
brilliant fireworks, and the fireworks by much bad poetry. 
%[255]
\footnote{ London Gazette, Gazette de France; Life of James the
Second, ii. 10; History of the Coronation of King James the Second and
Queen Mary, by Francis Sandford, Lancaster Herald, fol. 1687; Evelyn's
Diary, May, 21, 1685; Despatch of the Dutch Ambassadors, April 10-20,
1685; Burnet, i. 628; Eachard, iii. 734; A sermon preached before their
Majesties King James the Second and Queen Mary at their Coronation in
Westminster Abbey, April 23, 1695, by Francis Lord Bishop of Ely, and
Lord Almoner. I have seen an Italian account of the Coronation which was
published at Modena, and which is chiefly remarkable for the skill with
which the writer sinks the fact that the prayers and psalms were in
English, and that the Bishops were heretics.}


This may be fixed upon as the moment at which the enthusiasm of the
Tory party reached the zenith. Ever since the accession of the new King,
addresses had been pouring in which expressed profound veneration for
his person and office, and bitter detestation of the vanquished Whigs.
The magistrates of Middlesex thanked God for having confounded the
designs of those regicides and exclusionists who, not content with
having murdered one blessed monarch, were bent on destroying the
foundations of monarchy. The city of Gloucester execrated the
bloodthirsty villains who had tried to deprive His Majesty of his just
inheritance. The burgesses of Wigan assured their sovereign that
they would defend him against all plotting Achitophels and rebellions
Absaloms. The grand jury of Suffolk expressed a hope that the Parliament
would proscribe all the exclusionists. Many corporations pledged
themselves never to return to the House of Commons any person who had
voted for taking away the birthright of James. Even the capital was
profoundly obsequious. The lawyers and the traders vied with each
other in servility. Inns of Court and Inns of Chancery sent up fervent
professions of attachment and submission. All the great commercial
societies, the East India Company, the African Company, the Turkey
Company, the Muscovy Company, the Hudson's Bay Company, the Maryland
Merchants, the Jamaica Merchants, the Merchant Adventurers, declared
that they most cheerfully complied with the royal edict which required
them still to pay custom. Bristol, the second city of the island, echoed
the voice of London. But nowhere was the spirit of loyalty stronger than
in the two Universities. Oxford declared that she would never swerve
from those religious principles which bound her to obey the King without
any restrictions or limitations. Cambridge condemned, in severe terms,
the violence and treachery of those turbulent men who had maliciously
endeavoured to turn the stream of succession out of the ancient channel.

%[256]
\footnote{ See the London Gazette during the months of February,
March, and April, 1685.}


Such addresses as these filled, during a considerable time, every number
of the London Gazette. But it was not only by addressing that the Tories
showed their zeal. The writs for the new Parliament had gone forth,
and the country was agitated by the tumult of a general election. No
election had ever taken place under circumstances so favourable to
the Court. Hundreds of thousands whom the Popish plot had scared into
Whiggism had been scared back by the Rye House plot into Toryism. In the
counties the government could depend on an overwhelming majority of the
gentlemen of three hundred a year and upwards, and on the clergy almost
to a man. Those boroughs which had once been the citadels of Whiggism
had recently been deprived of their charters by legal sentence, or
had prevented the sentence by voluntary surrender. They had now been
reconstituted in such a manner that they were certain to return members
devoted to the crown. Where the townsmen could not be trusted, the
freedom had been bestowed on the neighbouring squires. In some of the
small western corporations, the constituent bodies were in great part
composed of Captains and Lieutenants of the Guards. The returning
officers were almost everywhere in the interest of the court. In every
shire the Lord Lieutenant and his deputies formed a powerful, active,
and vigilant committee, for the purpose of cajoling and intimidating the
freeholders. The people were solemnly warned from thousands of pulpits
not to vote for any Whig candidate, as they should answer it to Him who
had ordained the powers that be, and who had pronounced rebellion a sin
not less deadly than witchcraft. All these advantages the predominant
party not only used to the utmost, but abused in so shameless a manner
that grave and reflecting men, who had been true to the monarchy in
peril, and who bore no love to republicans and schismatics, stood
aghast, and augured from such beginnings the approach of evil times.

%[257]
\footnote{ It would be easy to fill a volume with what Whig
historians and pamphleteers have written on this subject. I will cite
only one witness, a churchman and a Tory. "Elections," says Evelyn,
"were thought to be very indecently carried on in most places. God give
a better issue of it than some expect!" May 10, 1685. Again he says,
"The truth is there were many of the new members whose elections and
returns were universally condemned." May 22.}


Yet the Whigs, though suffering the just punishment of their errors,
though defeated, disheartened, and disorganized, did not yield without
an effort. They were still numerous among the traders and artisans of
the towns, and among the yeomanry and peasantry of the open country. In
some districts, in Dorsetshire for example, and in Somersetshire, they
were the great majority of the population. In the remodelled boroughs
they could do nothing: but, in every county where they had a chance,
they struggled desperately. In Bedfordshire, which had lately been
represented by the virtuous and unfortunate Russell, they were
victorious on the show of hands, but were beaten at the poll. 
%[258]
\footnote{ This fact I learned from a newsletter in the library of
the Royal Institution. Van Citters mentions the strength of the Whig
party in Bedfordshire.}
 In
Essex they polled thirteen hundred votes to eighteen hundred. 
%[259]
\footnote{ Bramston's Memoirs.}
 At
the election for Northamptonshire the common people were so violent in
their hostility to the court candidate that a body of troops was drawn
out in the marketplace of the county town, and was ordered to load with
ball. 
%[260]
\footnote{ Reflections on a Remonstrance and Protestation of all the
good Protestants of this Kingdom, 1689; Dialogue between Two Friends,
1689.}
 The history of the contest for Buckinghamshire is still more
remarkable. The whig candidate, Thomas Wharton, eldest son of Philip
Lord Wharton, was a man distinguished alike by dexterity and by
audacity, and destined to play a conspicuous, though not always a
respectable, part in the politics of several reigns. He had been one of
those members of the House of Commons who had carried up the Exclusion
Bill to the bar of the Lords. The court was therefore bent on throwing
him out by fair or foul means. The Lord Chief Justice Jeffreys himself
came down into Buckinghamshire, for the purpose of assisting a gentleman
named Hacket, who stood on the high Tory interest. A stratagem was
devised which, it was thought, could not fail of success. It was given
out that the polling would take place at Ailesbury; and Wharton,
whose skill in all the arts of electioneering was unrivalled, made his
arrangements on that supposition. At a moment's warning the Sheriff
adjourned the poll to Newport Pagnell. Wharton and his friends hurried
thither, and found that Hacket, who was in the secret, had already
secured every inn and lodging. The Whig freeholders were compelled to
tie their horses to the hedges, and to sleep under the open sky in
the meadows which surround the little town. It was with the greatest
difficulty that refreshments could be procured at such short notice for
so large a number of men and beasts, though Wharton, who was utterly
regardless of money when his ambition and party spirit were roused,
disbursed fifteen hundred pounds in one day, an immense outlay for those
times. Injustice seems, however, to have animated the courage of the
stouthearted yeomen of Bucks, the sons of the constituents of John
Hampden. Not only was Wharton at the head of the poll; but he was able
to spare his second votes to a man of moderate opinions, and to throw
out the Chief Justice's candidate. 
%[261]
\footnote{ Memoirs of the Life of Thomas Marquess of Wharton, 1715.}


In Cheshire the contest lasted six days. The Whigs polled about
seventeen hundred votes, the Tories about two thousand. The common
people were vehement on the Whig side, raised the cry of "Down with the
Bishops," insulted the clergy in the streets of Chester, knocked
down one gentleman of the Tory party, broke the windows and beat the
constables. The militia was called out to quell the riot, and was kept
assembled, in order to protect the festivities of the conquerors. When
the poll closed, a salute of five great guns from the castle proclaimed
the triumph of the Church and the Crown to the surrounding country. The
bells rang. The newly elected members went in state to the City Cross,
accompanied by a band of music, and by a long train of knights and
squires. The procession, as it marched, sang "Joy to Great Caesar," a
loyal ode, which had lately been written by Durfey, and which, though
like all Durfey's writings, utterly contemptible, was, at that time,
almost as popular as Lillibullero became a few years later. 
%[262]
\footnote{ See the Guardian, No. 67; an exquisite specimen of
Addison's peculiar manner. It would be difficult to find in the works
of any other writer such an instance of benevolence delicately flavoured
with contempt.}
 Round
the Cross the trainbands were drawn up in order: a bonfire was lighted:
the Exclusion Bill was burned: and the health of King James was drunk
with loud acclamations. The following day was Sunday. In the morning the
militia lined the streets leading to the Cathedral. The two knights of
the shire were escorted with great pomp to their choir by the magistracy
of the city, heard the Dean preach a sermon, probably on the duty of
passive obedience, and were afterwards feasted by the Mayor. 
%[263]
\footnote{ The Observator, April 4, 1685.}


In Northumberland the triumph of Sir John Fenwick, a courtier whose
name afterwards obtained a melancholy celebrity, was attended by
circumstances which excited interest in London, and which were thought
not unworthy of being mentioned in the despatches of foreign ministers.
Newcastle was lighted up with great piles of coal. The steeples sent
forth a joyous peal. A copy of the Exclusion Bill, and a black box,
resembling that which, according to the popular fable, contained the
contract between Charles the Second and Lucy Walters, were publicly
committed to the flames, with loud acclamations. 
%[264]
\footnote{ Despatch of the Dutch Ambasadors, April 10-20, 1685.}


The general result of the elections exceeded the most sanguine
expectations of the court. James found with delight that it would be
unnecessary for him to expend a farthing in buying votes. He Said that,
with the exception of about forty members, the House of Commons was just
such as he should himself have named. 
%[265]
\footnote{ Burnet, i. 626.}
 And this House of Commons
it was in his power, as the law then stood, to keep to the end of his
reign.

Secure of parliamentary support, he might now indulge in the luxury of
revenge. His nature was not placable; and, while still a subject, he had
suffered some injuries and indignities which might move even a placable
nature to fierce and lasting resentment. One set of men in particular
had, with a baseness and cruelty beyond all example and all description,
attacked his honour and his life, the witnesses of the plot. He may
well be excused for hating them; since, even at this day, the mention of
their names excites the disgust and horror of all sects and parties.

Some of these wretches were already beyond the reach of human justice.
Bedloe had died in his wickedness, without one sign of remorse or shame.

%[266]
\footnote{ A faithful account of the Sickness, Death, and Burial of
Captain Bedlow, 1680; Narrative of Lord Chief Justice North.}
 Dugdale had followed, driven mad, men said, by the Furies of an
evil conscience, and with loud shrieks imploring those who stood round
his bed to take away Lord Stafford. 
%[267]
\footnote{ Smith's Intrigues of the Popish Plot, 1685.}
 Carstairs, too, was gone. His
end had been all horror and despair; and, with his last breath, he had
told his attendants to throw him into a ditch like a dog, for that he
was not fit to sleep in a Christian burial ground. 
%[268]
\footnote{ Burnet, i. 439.}
 But Oates and
Dangerfield were still within the reach of the stern prince whom they
had wronged. James, a short time before his accession, had instituted
a civil suit against Oates for defamatory words; and a jury had given
damages to the enormous amount of a hundred thousand pounds. 
%[269]
\footnote{ See the proceedings in the Collection of State Trials.}
 The
defendant had been taken in execution, and was lying in prison as a
debtor, without hope of release. Two bills of indictment against him
for perjury had been found by the grand jury of Middlesex, a few weeks
before the death of Charles. Soon after the close of the elections the
trial came on.

Among the upper and middle classes Oates had few friends left. The most
respectable Whigs were now convinced that, even if his narrative had
some foundation in fact, he had erected on that foundation a vast
superstructure of romance. A considerable number of low fanatics,
however, still regarded him as a public benefactor. These people well
knew that, if he were convicted, his sentence would be one of extreme
severity, and were therefore indefatigable in their endeavours to manage
an escape. Though he was as yet in confinement only for debt, he was put
into irons by the authorities of the King's Bench prison; and even so he
was with difficulty kept in safe custody. The mastiff that guarded his
door was poisoned; and, on the very night preceding the trial, a ladder
of ropes was introduced into the cell.

On the day in which Titus was brought to the bar, Westminster Hall was
crowded with spectators, among whom were many Roman Catholics, eager to
see the misery and humiliation of their persecutor. 
%[270]
\footnote{ Evelyn's Diary, May 7, 1685.}
 A few years
earlier his short neck, his legs uneven, the vulgar said, as those of a
badger, his forehead low as that of a baboon, his purple cheeks, and his
monstrous length of chin, had been familiar to all who frequented the
courts of law. He had then been the idol of the nation. Wherever he had
appeared, men had uncovered their heads to him. The lives and estates of
the magnates of the realm had been at his mercy. Times had now changed;
and many, who had formerly regarded him as the deliverer of his country,
shuddered at the sight of those hideous features on which villany seemed
to be written by the hand of God. 
%[271]
\footnote{ There remain many pictures of Oates. The most striking
descriptions of his person are in North's Examen, 225, in Dryden's
Absalom and Achitophel, and In a broadside entitled, A Hue and Cry after
T. O.}


It was proved, beyond all possibility of doubt, that this man had by
false testimony deliberately murdered several guiltless persons. He
called in vain on the most eminent members of the Parliaments which had
rewarded and extolled him to give evidence in his favour. Some of those
whom he had summoned absented themselves. None of them said anything
tending to his vindication. One of them, the Earl of Huntingdon,
bitterly reproached him with having deceived the Houses and drawn on
them the guilt of shedding innocent blood. The Judges browbeat and
reviled the prisoner with an intemperance which, even in the most
atrocious cases, ill becomes the judicial character. He betrayed,
however, no sign of fear or of shame, and faced the storm of invective
which burst upon him from bar, bench, and witness box, with the
insolence of despair. He was convicted on both indictments. His offence,
though, in a moral light, murder of the most aggravated kind, was, in
the eye of the law, merely a misdemeanour. The tribunal, however, was
desirous to make his punishment more severe than that of felons or
traitors, and not merely to put him to death, but to put him to death
by frightful torments. He was sentenced to be stripped of his clerical
habit, to be pilloried in Palace Yard, to be led round Westminster Hall
with an inscription declaring his infamy over his head, to be pilloried
again in front of the Royal Exchange, to be whipped from Aldgate to
Newgate, and, after an interval of two days, to be whipped from Newgate
to Tyburn. If, against all probability, he should happen to survive this
horrible infliction, he was to be kept close prisoner during life. Five
times every year he was to be brought forth from his dungeon and exposed
on the pillory in different parts of the capital. 
%[272]
\footnote{ The proceedings will be found at length in the Collection
of State Trials.}
 This rigorous
sentence was rigorously executed. On the day on which Oates was
pilloried in Palace Yard he was mercilessly pelted and ran some risk of
being pulled in pieces. 
%[273]
\footnote{ Gazette de France May 29,/June 9, 1685.}
 But in the City his partisans mustered
in great force, raised a riot, and upset the pillory. 
%[274]
\footnote{ Despatch of the Dutch Ambassadors, May 19-29, 1685.}
 They were,
however, unable to rescue their favourite. It was supposed that he would
try to escape the horrible doom which awaited him by swallowing poison.
All that he ate and drank was therefore carefully inspected. On the
following morning he was brought forth to undergo his first flogging.
At an early hour an innumerable multitude filled all the streets from
Aldgate to the Old Bailey. The hangman laid on the lash with such
unusual severity as showed that he had received special instructions.
The blood ran down in rivulets. For a time the criminal showed a strange
constancy: but at last his stubborn fortitude gave way. His bellowings
were frightful to hear. He swooned several times; but the scourge still
continued to descend. When he was unbound, it seemed that he had borne
as much as the human frame can bear without dissolution. James was
entreated to remit the second flogging. His answer was short and clear:
"He shall go through with it, if he has breath in his body." An attempt
was made to obtain the Queen's intercession; but she indignantly refused
to say a word in favour of such a wretch. After an interval of only
forty-eight hours, Oates was again brought out of his dungeon. He was
unable to stand, and it was necessary to drag him to Tyburn on a
sledge. He seemed quite insensible; and the Tories reported that he had
stupified himself with strong drink. A person who counted the stripes
on the second day said that they were seventeen hundred. The bad
man escaped with life, but so narrowly that his ignorant and bigoted
admirers thought his recovery miraculous, and appealed to it as a proof
of his innocence. The doors of the prison closed upon him. During many
months he remained ironed in the darkest hole of Newgate. It was said
that in his cell he gave himself up to melancholy, and sate whole days
uttering deep groans, his arms folded, and his hat pulled over his eyes.
It was not in England alone that these events excited strong interest.
Millions of Roman Catholics, who knew nothing of our institutions or
of our factions, had heard that a persecution of singular barbarity had
raged in our island against the professors of the true faith, that many
pious men had suffered martyrdom, and that Titus Oates had been the
chief murderer. There was, therefore, great joy in distant countries
when it was known that the divine justice had overtaken him. Engravings
of him, looking out from the pillory, and writhing at the cart's tail,
were circulated all over Europe; and epigrammatists, in many languages,
made merry with the doctoral title which he pretended to have received
from the University of Salamanca, and remarked that, since his forehead
could not be made to blush, it was but reasonable that his back should
do so. 
%[275]
\footnote{ Evelyn's Diary, May 22, 1685; Eachard, iii. 741; Burnet,
i. 637; Observator, May 27, 1685; Oates's Eikvn, 89; Eikwn Brotoloigon,
1697; Commons' Journals of May, June, and July, 1689; Tom Brown's
advice to Dr. Oates. Some interesting circumstances are mentioned in
a broadside, printed for A. Brooks, Charing Cross, 1685. I have seen
contemporary French and Italian pamphlets containing the history of the
trial and execution. A print of Titus in the pillory was published at
Milan, with the following curious inscription: "Questo e il naturale
ritratto di Tito Otez, o vero Oatz, Inglese, posto in berlina, uno de'
principali professor della religion protestante, acerrimo persecutore
de' Cattolici, e gran spergiuro." I have also seen a Dutch engraving
of his punishment, with some Latin verses, of which the following are a
specimen:

     "At Doctor fictus non fictos pertulit ictus
     A tortore datos haud molli in corpore gratos,
     Disceret ut vere scelera ob commissa rubere."

The anagram of his name, "Testis Ovat," may be found on many prints
published in different countries.}


Horrible as were the sufferings of Oates, they did not equal his crimes.
The old law of England, which had been suffered to become obsolete,
treated the false witness, who had caused death by means of perjury, as
a murderer. 
%[276]
\footnote{ Blackstone's Commentaries, Chapter of Homicide.}
 This was wise and righteous; for such a witness is, in
truth, the worst of murderers. To the guilt of shedding innocent blood
he has added the guilt of violating the most solemn engagement into
which man can enter with his fellow men, and of making institutions,
to which it is desirable that the public should look with respect
and confidence, instruments of frightful wrong and objects of general
distrust. The pain produced by ordinary murder bears no proportion to
the pain produced by murder of which the courts of justice are made the
agents. The mere extinction of life is a very small part of what makes
an execution horrible. The prolonged mental agony of the sufferer, the
shame and misery of all connected with him, the stain abiding even to
the third and fourth generation, are things far more dreadful than death
itself. In general it may be safely affirmed that the father of a large
family would rather be bereaved of all his children by accident or by
disease than lose one of them by the hands of the hangman. Murder by
false testimony is therefore the most aggravated species of murder; and
Oates had been guilty of many such murders. Nevertheless the punishment
which was inflicted upon him cannot be justified. In sentencing him to
be stripped of his ecclesiastical habit and imprisoned for life, the
judges exceeded their legal power. They were undoubtedly competent to
inflict whipping; nor had the law assigned a limit to the number of
stripes. But the spirit of the law clearly was that no misdemeanour
should be punished more severely than the most atrocious felonies.
The worst felon could only be hanged. The judges, as they believed,
sentenced Oates to be scourged to death. That the law was defective is
not a sufficient excuse: for defective laws should be altered by the
legislature, and not strained by the tribunals; and least of all should
the law be strained for the purpose of inflicting torture and destroying
life. That Oates was a bad man is not a sufficient excuse; for the
guilty are almost always the first to suffer those hardships which are
afterwards used as precedents against the innocent. Thus it was in the
present case. Merciless flogging soon became an ordinary punishment for
political misdemeanours of no very aggravated kind. Men were sentenced,
for words spoken against the government, to pains so excruciating that
they, with unfeigned earnestness, begged to be brought to trial on
capital charges, and sent to the gallows. Happily the progress of this
great evil was speedily stopped by the Revolution, and by that article
of the Bill of Rights which condemns all cruel and unusual punishments.

The villany of Dangerfield had not, like that of Oates, destroyed
many innocent victims; for Dangerfield had not taken up the trade of
a witness till the plot had been blown upon and till juries had become
incredulous. 
%[277]
\footnote{ According to Roger North the judges decided that
Dangerfield, having been previously convicted of perjury, was
incompetent to be a witness of the plot. But this is one among many
instances of Roger's inaccuracy. It appears, from the report of the
trial of Lord Castlemaine in June 1680, that, after much altercation
between counsel, and much consultation among the judges of the different
courts in Westminster Hall, Dangerfield was sworn and suffered to tell
his story; but the jury very properly gave no credit to his testimony.}
 He was brought to trial, not for perjury, but for the
less heinous offense of libel. He had, during the agitation caused by
the Exclusion Bill, put forth a narrative containing some false and
odious imputations on the late and on the present King. For this
publication he was now, after the lapse of five years, suddenly taken
up, brought before the Privy Council, committed, tried, convicted, and
sentenced to be whipped from Aldgate to Newgate and from Newgate to
Tyburn. The wretched man behaved with great effrontery during the trial;
but, when he heard his doom, he went into agonies of despair, gave
himself up for dead, and chose a text for his funeral sermon. His
forebodings were just. He was not, indeed, scourged quite so severely as
Oates had been; but he had not Oates's iron strength of body and mind.
After the execution Dangerfield was put into a hackney coach and was
taken back to prison. As he passed the corner of Hatton Garden, a Tory
gentleman of Gray's Inn, named Francis, stopped the carriage, and cried
out with brutal levity, "Well, friend, have you had your heat this
morning?" The bleeding prisoner, maddened by this insult, answered with
a curse. Francis instantly struck him in the face with a cane which
injured the eye. Dangerfield was carried dying into Newgate. This
dastardly outrage roused the indignation of the bystanders. They seized
Francis, and were with difficulty restrained from tearing him to
pieces. The appearance of Dangerfield's body, which had been frightfully
lacerated by the whip, inclined many to believe that his death was
chiefly, if not wholly, caused by the stripes which he had received. The
government and the Chief Justice thought it convenient to lay the whole
blame on Francis, who; though he seems to have been at worst guilty only
of aggravated manslaughter, was tried and executed for murder. His dying
speech is one of the most curious monuments of that age. The savage
spirit which had brought him to the gallows remained with him to the
last. Boasts of his loyalty and abuse of the Whigs were mingled with the
parting ejaculations in which he commended his soul to the divine
mercy. An idle rumour had been circulated that his wife was in love with
Dangerfield, who was eminently handsome and renowned for gallantry.
The fatal blow, it was said, had been prompted by jealousy. The dying
husband, with an earnestness, half ridiculous, half pathetic, vindicated
the lady's character. She was, he said, a virtuous woman: she came of
a loyal stock, and, if she had been inclined to break her marriage vow,
would at least have selected a Tory and a churchman for her paramour.

%[278]
\footnote{ Dangerfield's trial was not reported; but I have seen a
concise account of it in a contemporary broadside. An abstract of the
evidence against Francis, and his dying speech, will be found in the
Collection of State Trials. See Eachard, iii. 741. Burnet's narrative
contains more mistakes than lines. See also North's Examen, 256, the
sketch of Dangerfield's life in the Bloody Assizes, the Observator of
July 29, 1685, and the poem entitled "Dangerfield's Ghost to Jeffreys."
In the very rare volume entitled "Succinct Genealogies, by Robert
Halstead," Lord Peterbough says that Dangerfield, with whom he had had
some intercourse, was "a young man who appeared under a decent figure,
a serious behaviour, and with words that did not seem to proceed from a
common understanding."}


About the same time a culprit, who bore very little resemblance to Oates
or Dangerfield, appeared on the floor of the Court of King's Bench. No
eminent chief of a party has ever passed through many years of civil
and religious dissension with more innocence than Richard Baxter. He
belonged to the mildest and most temperate section of the Puritan body.
He was a young man when the civil war broke out. He thought that the
right was on the side of the Houses; and he had no scruple about acting
as chaplain to a regiment in the parliamentary army: but his clear
and somewhat sceptical understanding, and his strong sense of justice,
preserved him from all excesses. He exerted himself to check the
fanatical violence of the soldiery. He condemned the proceedings of
the High Court of Justice. In the days of the Commonwealth he had the
boldness to express, on many occasions, and once even in Cromwell's
presence, love and reverence for the ancient institutions of the
country. While the royal family was in exile, Baxter's life was chiefly
passed at Kidderminster in the assiduous discharge of parochial duties.
He heartily concurred in the Restoration, and was sincerely desirous to
bring about an union between Episcopalians and Presbyterians. For, with
a liberty rare in his time, he considered questions of ecclesiastical
polity as of small account when compared with the great principles of
Christianity, and had never, even when prelacy was most odious to the
ruling powers, joined in the outcry against Bishops. The attempt to
reconcile the contending factions failed. Baxter cast in his lot with
his proscribed friends, refused the mitre of Hereford, quitted the
parsonage of Kidderminster, and gave himself up almost wholly to study.
His theological writings, though too moderate to be pleasing to the
bigots of any party, had an immense reputation. Zealous Churchmen called
him a Roundhead; and many Nonconformists accused him of Erastianism and
Arminianism. But the integrity of his heart, the purity of his life,
the vigour of his faculties, and the extent of his attainments were
acknowledged by the best and wisest men of every persuasion. His
political opinions, in spite of the oppression which he and his brethren
had suffered, were moderate. He was friendly to that small party which
was hated by both Whigs and Tories. He could not, he said, join in
cursing the Trimmers, when he remembered who it was that had blessed the
peacemakers. 
%[279]
\footnote{ Baxter's preface to Sir Mathew Hale's Judgment of the
Nature of True Religion, 1684.}


In a Commentary on the New Testament he had complained, with some
bitterness, of the persecution which the Dissenters suffered. That men
who, for not using the Prayer Book, had been driven from their homes,
stripped of their property, and locked up in dungeons, should dare to
utter a murmur, was then thought a high crime against the State and the
Church. Roger Lestrange, the champion of the government and the oracle
of the clergy, sounded the note of war in the Observator. An information
was filed. Baxter begged that he might be allowed some time to prepare
for his defence. It was on the day on which Oates was pilloried in
Palace Yard that the illustrious chief of the Puritans, oppressed by age
and infirmities, came to Westminster Hall to make this request. Jeffreys
burst into a storm of rage. "Not a minute," he cried, "to save his life.
I can deal with saints as well as with sinners. There stands Oates on
one side of the pillory; and, if Baxter stood on the other, the two
greatest rogues in the kingdom would stand together."

When the trial came on at Guildhall, a crowd of those who loved and
honoured Baxter filled the court. At his side stood Doctor William
Bates, one of the most eminent of the Nonconformist divines. Two
Whig barristers of great note, Pollexfen and Wallop, appeared for the
defendant. Pollexfen had scarcely begun his address to the jury, when
the Chief Justice broke forth: "Pollexfen, I know you well. I will set a
mark on you. You are the patron of the faction. This is an old rogue,
a schismatical knave, a hypocritical villain. He hates the Liturgy.
He would have nothing but longwinded cant without book;" and then
his Lordship turned up his eyes, clasped his hands, and began to sing
through his nose, in imitation of what he supposed to be Baxter's style
of praying "Lord, we are thy people, thy peculiar people, thy dear
people." Pollexfen gently reminded the court that his late Majesty
had thought Baxter deserving of a bishopric. "And what ailed the old
blockhead then," cried Jeffreys, "that he did not take it?" His fury now
rose almost to madness. He called Baxter a dog, and swore that it would
be no more than justice to whip such a villain through the whole City.

Wallop interposed, but fared no better than his leader. "You are in all
these dirty causes, Mr. Wallop," said the Judge. "Gentlemen of the long
robe ought to be ashamed to assist such factious knaves." The advocate
made another attempt to obtain a hearing, but to no purpose. "If you do
not know your duty," said Jeffreys, "I will teach it you."

Wallop sate down; and Baxter himself attempted to put in a word. But
the Chief Justice drowned all expostulation in a torrent of ribaldry and
invective, mingled with scraps of Hudibras. "My Lord," said the old
man, "I have been much blamed by Dissenters for speaking respectfully of
Bishops." "Baxter for Bishops!" cried the Judge, "that's a merry
conceit indeed. I know what you mean by Bishops, rascals like yourself,
Kidderminster Bishops, factious snivelling Presbyterians!" Again Baxter
essayed to speak, and again Jeffreys bellowed "Richard, Richard, dost
thou think we will let thee poison the court? Richard, thou art an old
knave. Thou hast written books enough to load a cart, and every book as
full of sedition as an egg is full of meat. By the grace of God, I'll
look after thee. I see a great many of your brotherhood waiting to know
what will befall their mighty Don. And there," he continued, fixing his
savage eye on Bates, "there is a Doctor of the party at your elbow. But,
by the grace of God Almighty, I will crush you all."

Baxter held his peace. But one of the junior counsel for the defence
made a last effort, and undertook to show that the words of which
complaint was made would not bear the construction put on them by the
information. With this view he began to read the context. In a moment
he was roared down. "You sha'n't turn the court into a conventicle."
The noise of weeping was heard from some of those who surrounded Baxter.
"Snivelling calves!" said the Judge.

Witnesses to character were in attendance, and among them were several
clergymen of the Established Church. But the Chief Justice would hear
nothing. "Does your Lordship think," said Baxter, "that any jury will
convict a man on such a trial as this?" "I warrant you, Mr. Baxter,"
said Jeffreys: "don't trouble yourself about that." Jeffreys was right.
The Sheriffs were the tools of the government. The jurymen, selected
by the Sheriffs from among the fiercest zealots of the Tory party,
conferred for a moment, and returned a verdict of Guilty. "My Lord,"
said Baxter, as he left the court, "there was once a Chief Justice who
would have treated me very differently." He alluded to his learned
and virtuous friend Sir Matthew Hale. "There is not an honest man in
England," answered Jeffreys, "but looks on thee as a knave." 
%[280]
\footnote{ See the Observator of February 28, 1685, the information
in the Collection of State Trials, the account of what passed in court
given by Calamy, Life of Baxter, chap. xiv., and the very curious
extracts from the Baxter MSS. in the Life, by Orme, published in 1830.}


The sentence was, for those times, a lenient one. What passed in
conference among the judges cannot be certainly known. It was believed
among the Nonconformists, and is highly probable, that the Chief Justice
was overruled by his three brethren. He proposed, it is said, that
Baxter should be whipped through London at the cart's tail. The majority
thought that an eminent divine, who, a quarter of a century before, had
been offered a mitre, and who was now in his seventieth year, would be
sufficiently punished for a few sharp words by fine and imprisonment.

%[281]
\footnote{ Baxter MS. cited by Orme.}


The manner in which Baxter was treated by a judge, who was a member of
the cabinet and a favourite of the Sovereign, indicated, in a manner
not to be mistaken, the feeling with which the government at this time
regarded the Protestant Nonconformists. But already that feeling had
been indicated by still stronger and more terrible signs. The Parliament
of Scotland had met. James had purposely hastened the session of this
body, and had postponed the session of the English Houses, in the
hope that the example set at Edinburgh would produce a good effect
at Westminster. For the legislature of his northern kingdom was as
obsequious as those provincial Estates which Lewis the Fourteenth still
suffered to play at some of their ancient functions in Britanny and
Burgundy. None but an Episcopalian could sit in the Scottish Parliament,
or could even vote for a member, and in Scotland an Episcopalian was
always a Tory or a timeserver. From an assembly thus constituted, little
opposition to the royal wishes was to be apprehended; and even
the assembly thus constituted could pass no law which had not been
previously approved by a committee of courtiers.

All that the government asked was readily granted. In a financial point
of view, indeed, the liberality of the Scottish Estates was of little
consequence. They gave, however, what their scanty means permitted. They
annexed in perpetuity to the crown the duties which had been granted
to the late King, and which in his time had been estimated at forty
thousand pounds sterling a year. They also settled on James for life
an additional annual income of two hundred and sixteen thousand pounds
Scots, equivalent to eighteen thousand pounds sterling. The whole Sum
which they were able to bestow was about sixty thousand a year, little
more than what was poured into the English Exchequer every fortnight.

%[282]
\footnote{ Act Parl. Car. II. March 29,1661, Jac. VII. April 28,
1685, and May 13, 1685.}


Having little money to give, the Estates supplied the defect by loyal
professions and barbarous statutes. The King, in a letter which was
read to them at the opening of their session, called on them in vehement
language to provide new penal laws against the refractory Presbyterians,
and expressed his regret that business made it impossible for him to
propose such laws in person from the throne. His commands were obeyed.
A statute framed by his ministers was promptly passed, a statute which
stands forth even among the statutes of that unhappy country at that
unhappy period, preeminent in atrocity. It was enacted, in few but
emphatic words, that whoever should preach in a conventicle under a
roof, or should attend, either as preacher or as hearer, a conventicle
in the open air, should be punished with death and confiscation of
property. 
%[283]
\footnote{ Act Parl. Jac. VII. May 8, 1685, Observator, June 20,
1685; Lestrange evidently wished to see the precedent followed in
England.}


This law, passed at the King's instance by an assembly devoted to his
will, deserves especial notice. For he has been frequently represented
by ignorant writers as a prince rash, indeed, and injudicious in his
choice of means, but intent on one of the noblest ends which a ruler
can pursue, the establishment of entire religious liberty. Nor can it be
denied that some portions of his life, when detached from the rest and
superficially considered, seem to warrant this favourable view of his
character.

While a subject he had been, during many years, a persecuted man; and
persecution had produced its usual effect on him. His mind, dull and
narrow as it was, had profited under that sharp discipline. While he was
excluded from the Court, from the Admiralty, and from the Council, and
was in danger of being also excluded from the throne, only because he
could not help believing in transubstantiation and in the authority
of the see of Rome, he made such rapid progress in the doctrines of
toleration that he left Milton and Locke behind. What, he often said,
could be more unjust, than to visit speculations with penalties which
ought to be reserved for acts? What more impolitic than to reject the
services of good soldiers, seamen, lawyers, diplomatists, financiers,
because they hold unsound opinions about the number of the sacraments or
the pluripresence of saints? He learned by rote those commonplaces which
all sects repeat so fluently when they are enduring oppression, and
forget so easily when they are able to retaliate it. Indeed he rehearsed
his lesson so well, that those who chanced to hear him on this subject
gave him credit for much more sense and much readier elocution than he
really possessed. His professions imposed on some charitable persons,
and perhaps imposed on himself. But his zeal for the rights of
conscience ended with the predominance of the Whig party. When fortune
changed, when he was no longer afraid that others would persecute him,
when he had it in his power to persecute others, his real propensities
began to show themselves. He hated the Puritan sects with a manifold
hatred, theological and political, hereditary and personal. He regarded
them as the foes of Heaven, as the foes of all legitimate authority in
Church and State, as his great-grandmother's foes and his grandfather's,
his father's and his mother's, his brother's and his own. He, who had
complained so fondly of the laws against Papists, now declared himself
unable to conceive how men could have the impudence to propose the
repeal of the laws against Puritans. 
%[284]
\footnote{ His own words reported by himself. Life of James the
Second, i. 666. Orig. Mem.}
 He, whose favourite theme had
been the injustice of requiring civil functionaries to take religious
tests, established in Scotland, when he resided there as Viceroy, the
most rigorous religious test that has ever been known in the empire.

%[285]
\footnote{ Act Parl. Car. II. August 31, 1681.}
 He, who had expressed just indignation when the priests of his own
faith were hanged and quartered, amused himself with hearing Covenanters
shriek and seeing them writhe while their knees were beaten flat in the
boots. 
%[286]
\footnote{ Burnet, i. 583; Wodrow, III. v. 2. Unfortunately the Acta
of the Scottish Privy Council during almost the whole administration of
the Duke of York are wanting. (1848.) This assertion has been met by a
direct contradiction. But the fact is exactly as I have stated it. There
is in he Acta of the Scottish Privy Council a hiatus extending from
August 1678 to August 1682. The Duke of York began to reside in Scotland
in December 1679. He left Scotland, never to return in May 1682.
(1857.)}
 In this mood he became King; and he immediately demanded
and obtained from the obsequious Estates of Scotland as the surest
pledge of their loyalty, the most sanguinary law that has ever in our
island been enacted against Protestant Nonconformists.

With this law the whole spirit of his administration was in perfect
harmony. The fiery persecution, which had raged when he ruled Scotland
as vicegerent, waxed hotter than ever from the day on which he became
sovereign. Those shires in which the Covenanters were most numerous
were given up to the license of the army. With the army was mingled a
militia, composed of the most violent and profligate of those who called
themselves Episcopalians. Preeminent among the bands which oppressed
and wasted these unhappy districts were the dragoons commanded by John
Graham of Claverhouse. The story ran that these wicked men used in their
revels to play at the torments of hell, and to call each other by the
names of devils and damned souls. 
%[287]
\footnote{ Wodrow, III. ix. 6.}
 The chief of this Tophet, a
soldier of distinguished courage and professional skill, but rapacious
and profane, of violent temper and of obdurate heart, has left a name
which, wherever the Scottish race is settled on the face of the globe,
is mentioned with a peculiar energy of hatred. To recapitulate all the
crimes, by which this man, and men like him, goaded the peasantry of the
Western Lowlands into madness, would be an endless task. A few instances
must suffice; and all those instances shall be taken from the history
of a single fortnight, that very fortnight in which the Scottish
Parliament, at the urgent request of James, enacted a new law of
unprecedented severity against Dissenters.

John Brown, a poor carrier of Lanarkshire, was, for his singular piety,
commonly called the Christian carrier. Many years later, when Scotland
enjoyed rest, prosperity, and religious freedom, old men who remembered
the evil days described him as one versed in divine things, blameless
in life, and so peaceable that the tyrants could find no offence in
him except that he absented himself from the public worship of the
Episcopalians. On the first of May he was cutting turf, when he was
seized by Claverhouse's dragoons, rapidly examined, convicted of
nonconformity, and sentenced to death. It is said that, even among the
soldiers, it was not easy to find an executioner. For the wife of the
poor man was present; she led one little child by the hand: it was easy
to see that she was about to give birth to another; and even those wild
and hardhearted men, who nicknamed one another Beelzebub and Apollyon,
shrank from the great wickedness of butchering her husband before her
face. The prisoner, meanwhile, raised above himself by the near
prospect of eternity, prayed loud and fervently as one inspired, till
Claverhouse, in a fury, shot him dead. It was reported by credible
witnesses that the widow cried out in her agony, "Well, sir, well; the
day of reckoning will come;" and that the murderer replied, "To man I
can answer for what I have done; and as for God, I will take him into
mine own hand." Yet it was rumoured that even on his seared conscience
and adamantine heart the dying ejaculations of his victim made an
impression which was never effaced. 
%[288]
\footnote{ Wodrow, III. ix. 6. The editor of the Oxford edition of
Burnet attempts to excuse this act by alleging that Claverhouse was then
employed to intercept all communication between Argyle and Monmouth,
and by supposing that John Brown may have been detected in conveying
intelligence between the rebel camps. Unfortunately for this hypothesis
John Brown was shot on the first of May, when both Argyle and Monmouth
were in Holland, and when there was no insurrection in any part of our
island.}


On the fifth of May two artisans, Peter Gillies and John Bryce, were
tried in Ayrshire by a military tribunal consisting of fifteen soldiers.
The indictment is still extant. The prisoners were charged, not with any
act of rebellion, but with holding the same pernicious doctrines which
had impelled others to rebel, and with wanting only opportunity to act
upon those doctrines. The proceeding was summary. In a few hours the two
culprits were convicted, hanged, and flung together into a hole under
the gallows. 
%[289]
\footnote{ Wodrow, III. ix, 6.}


The eleventh of May was made remarkable by more than one great crime.
Some rigid Calvinists had from the doctrine of reprobation drawn the
consequence that to pray for any person who had been predestined to
perdition was an act of mutiny against the eternal decrees of the
Supreme Being. Three poor labouring men, deeply imbued with this
unamiable divinity, were stopped by an officer in the neighbourhood
of Glasgow. They were asked whether they would pray for King James the
Seventh. They refused to do so except under the condition that he was
one of the elect. A file of musketeers was drawn out. The prisoners
knelt down; they were blindfolded; and within an hour after they had
been arrested, their blood was lapped up by the dogs. 
%[290]
\footnote{ Wodrow, III. ix. 6. It has been confidently asserted, by
persons who have not taken the trouble to look at the authority to which
I have referred, that I have grossly calumniated these unfortunate
men; that I do not understand the Calvinistic theology; and that it is
impossible that members of the Church of Scotland can have refused to
pray for any man on the ground that he was not one of the elect.----
I can only refer to the narrative which Wodrow has inserted in his
history, and which he justly calls plain and natural. That narrative
is signed by two eyewitnesses, and Wodrow, before he published it,
submitted it to a third eyewitness, who pronounced it strictly accurate.
From that narrative I will extract the only words which bear on
the point in question: "When all the three were taken, the officers
consulted among themselves, and, withdrawing to the west side of the
town, questioned the prisoners, particularly if they would pray for King
James VII. They answered, they would pray for all within the election of
grace. Balfour said Do you question the King's election? They answered,
sometimes they questioned their own. Upon which he swore dreadfully, and
said they should die presently, because they would not pray for Christ's
vicegerent, and so without one word more, commanded Thomas Cook to go to
his prayers, for he should die."---- In this narrative Wodrow saw nothing
improbable; and I shall not easily be convinced that any writer now
living understands the feelings and opinions of the Covenanters better
than Wodrow did. (1857.)}


While this was done in Clydesdale, an act not less horrible was
perpetrated in Eskdale. One of the proscribed Covenanters, overcome by
sickness, had found shelter in the house of a respectable widow, and
had died there. The corpse was discovered by the Laird of Westerhall, a
petty tyrant who had, in the days of the Covenant, professed inordinate
zeal for the Presbyterian Church, who had, since the Restoration,
purchased the favour of the government by apostasy, and who felt towards
the party which he had deserted the implacable hatred of an apostate.
This man pulled down the house of the poor woman, carried away her
furniture, and, leaving her and her younger children to wander in the
fields, dragged her son Andrew, who was still a lad, before Claverhouse,
who happened to be marching through that part of the country.
Claverhouse was just then strangely lenient. Some thought that he had
not been quite himself since the death of the Christian carrier, ten
days before. But Westerhall was eager to signalise his loyalty, and
extorted a sullen consent. The guns were loaded, and the youth was told
to pull his bonnet over his face. He refused, and stood confronting his
murderers with the Bible in his hand. "I can look you in the face," he
said; "I have done nothing of which I need be ashamed. But how will you
look in that day when you shall be judged by what is written in this
book?" He fell dead, and was buried in the moor. 
%[291]
\footnote{ Wodrow, III. ix. 6. Cloud of Witnesses.}


On the same day two women, Margaret Maclachlin and Margaret Wilson, the
former an aged widow, the latter a maiden of eighteen, suffered death
for their religion in Wigtonshire. They were offered their lives if they
would consent to abjure the cause of the insurgent Covenanters, and to
attend the Episcopal worship. They refused; and they were sentenced to
be drowned. They were carried to a spot which the Solway overflows twice
a day, and were fastened to stakes fixed in the sand between high and
low water mark. The elder sufferer was placed near to the advancing
flood, in the hope that her last agonies might terrify the younger into
submission. The sight was dreadful. But the courage of the survivor
was sustained by an enthusiasm as lofty as any that is recorded in
martyrology. She saw the sea draw nearer and nearer, but gave no sign
of alarm. She prayed and sang verses of psalms till the waves choked her
voice. After she had tasted the bitterness of death, she was, by a cruel
mercy unbound and restored to life. When she came to herself, pitying
friends and neighbours implored her to yield. "Dear Margaret, only say,
God save the King!" The poor girl, true to her stern theology, gasped
out, "May God save him, if it be God's will!" Her friends crowded round
the presiding officer. "She has said it; indeed, sir, she has said it."
"Will she take the abjuration?" he demanded. "Never!" she exclaimed.
"I am Christ's: let me go!" And the waters closed over her for the last
time. 
%[292]
\footnote{ Wodrow, III. ix. 6. The epitaph of Margaret Wilson, in
the churchyard at Wigton, is printed in the Appendix to the Cloud of
Witnesses;

     "Murdered for owning Christ supreme
     Head of his church, and no more crime,
     But her not owning Prelacy.
     And not abjuring Presbytery,
     Within the sea, tied to a stake,
     She suffered for Christ Jesus' sake."}


Thus was Scotland governed by that prince whom ignorant men have
represented as a friend of religious liberty, whose misfortune it was to
be too wise and too good for the age in which he lived. Nay, even
those laws which authorised him to govern thus were in his judgment
reprehensibly lenient. While his officers were committing the murders
which have just been related, he was urging the Scottish Parliament
to pass a new Act compared with which all former Acts might be called
merciful.

In England his authority, though great, was circumscribed by ancient
and noble laws which even the Tories would not patiently have seen him
infringe. Here he could not hurry Dissenters before military tribunals,
or enjoy at Council the luxury of seeing them swoon in the boots. Here
he could not drown young girls for refusing to take the abjuration, or
shoot poor countrymen for doubting whether he was one of the elect. Yet
even in England he continued to persecute the Puritans as far as his
power extended, till events which will hereafter be related induced him
to form the design of uniting Puritans and Papists in a coalition for
the humiliation and spoliation of the established Church.

One sect of Protestant Dissenters indeed he, even at this early period
of his reign, regarded with some tenderness, the Society of Friends.
His partiality for that singular fraternity cannot be attributed to
religious sympathy; for, of all who acknowledge the divine mission of
Jesus, the Roman Catholic and the Quaker differ most widely. It may seem
paradoxical to say that this very circumstance constituted a tie between
the Roman Catholic and the Quaker; yet such was really the case. For
they deviated in opposite directions so far from what the great body of
the nation regarded as right, that even liberal men generally considered
them both as lying beyond the pale of the largest toleration. Thus the
two extreme sects, precisely because they were extreme sects, had a
common interest distinct from the interest of the intermediate sects.
The Quakers were also guiltless of all offence against James and his
House. They had not been in existence as a community till the war
between his father and the Long Parliament was drawing towards a
close. They had been cruelly persecuted by some of the revolutionary
governments. They had, since the Restoration, in spite of much ill
usage, submitted themselves meekly to the royal authority. For they
had, though reasoning on premises which the Anglican divines regarded as
heterodox, arrived, like the Anglican divines, at the conclusion,
that no excess of tyranny on the part of a prince can justify active
resistance on the part of a subject. No libel on the government had ever
been traced to a Quaker. 
%[293]
\footnote{ See the letter to King Charles II. prefixed to Barclay's
Apology.}
 In no conspiracy against the government
had a Quaker been implicated. The society had not joined in the clamour
for the Exclusion Bill, and had solemnly condemned the Rye House plot as
a hellish design and a work of the devil. 
%[294]
\footnote{ Sewel's History of the Quakers, book x.}
 Indeed, the friends then
took very little part in civil contentions; for they were not, as now,
congregated in large towns, but were generally engaged in agriculture,
a pursuit from which they have been gradually driven by the vexations
consequent on their strange scruple about paying tithe. They were,
therefore, far removed from the scene of political strife. They
also, even in domestic privacy, avoided on principle all political
conversation. For such conversation was, in their opinion, unfavourable
to their spirituality of mind, and tended to disturb the austere
composure of their deportment. The yearly meetings of that age
repeatedly admonished the brethren not to hold discourse touching
affairs of state. 
%[295]
\footnote{ Minutes of Yearly Meetings, 1689, 1690.}
 Even within the memory of persons now living
those grave elders who retained the habits of an earlier generation
systematically discouraged such worldly talk. 
%[296]
\footnote{ Clarkson on Quakerism; Peculiar Customs, chapter v.}
 It was natural that
James should make a wide distinction between these harmless people and
those fierce and reckless sects which considered resistance to tyranny
as a Christian duty which had, in Germany, France, and Holland, made
war on legitimate princes, and which had, during four generations, borne
peculiar enmity to the House of Stuart.

It happened, moreover, that it was possible to grant large relief to the
Roman Catholic and to the Quaker without mitigating the sufferings of
the Puritan sects. A law was in force which imposed severe penalties on
every person who refused to take the oath of supremacy when required to
do so. This law did not affect Presbyterians, Independents, or Baptists;
for they were all ready to call God to witness that they renounced all
spiritual connection with foreign prelates and potentates. But the Roman
Catholic would not swear that the Pope had no jurisdiction in England,
and the Quaker would not swear to anything. On the other hand, neither
the Roman Catholic nor the Quaker was touched by the Five Mile Act,
which, of all the laws in the Statute Book, was perhaps the most
annoying to the Puritan Nonconformists. 
%[297]
\footnote{ After this passage was written, I found in the British
Museum, a manuscript (Harl. MS. 7506) entitled, "An Account of the
Seizures, Sequestrations, great Spoil and Havock made upon the Estates
of the several Protestant Dissenters called Quakers, upon Prosecution of
old Statutes made against Papist and Popish Recusants." The manuscript
is marked as having belonged to James, and appears to have been given
by his confidential servant, Colonel Graham, to Lord Oxford. This
circumstance appears to me to confirm the view which I have taken of the
King's conduct towards the Quakers.}


The Quakers had a powerful and zealous advocate at court. Though, as
a class, they mixed little with the world, and shunned politics as a
pursuit dangerous to their spiritual interests, one of them, widely
distinguished from the rest by station and fortune, lived in the
highest circles, and had constant access to the royal ear. This was the
celebrated William Penn. His father had held great naval commands,
had been a Commissioner of the Admiralty, had sate in Parliament, had
received the honour of knighthood, and had been encouraged to expect a
peerage. The son had been liberally educated, and had been designed
for the profession of arms, but had, while still young, injured his
prospects and disgusted his friends by joining what was then generally
considered as a gang of crazy heretics. He had been sent sometimes to
the Tower, and sometimes to Newgate. He had been tried at the Old Bailey
for preaching in defiance of the law. After a time, however, he had been
reconciled to his family, and had succeeded in obtaining such powerful
protection that, while all the gaols of England were filled with his
brethren, he was permitted, during many years, to profess his opinions
without molestation. Towards the close of the late reign he had
obtained, in satisfaction of an old debt due to him from the crown, the
grant of an immense region in North America. In this tract, then peopled
only by Indian hunters, he had invited his persecuted friends to settle.
His colony was still in its infancy when James mounted the throne.

Between James and Penn there had long been a familiar acquaintance. The
Quaker now became a courtier, and almost a favourite. He was every
day summoned from the gallery into the closet, and sometimes had long
audiences while peers were kept waiting in the antechambers. It was
noised abroad that he had more real power to help and hurt than many
nobles who filled high offices. He was soon surrounded by flatterers and
suppliants. His house at Kensington was sometimes thronged, at his
hour of rising, by more than two hundred suitors. 
%[298]
\footnote{ Penn's visits to Whitehall, and levees at Kensington,
are described with great vivacity, though in very bad Latin, by Gerard
Croese. "Sumebat," he says, "rex sæpe secretum, non horarium, vero
horarum plurium, in quo de variis rebus cum Penno serio sermonem
conferebat, et interim differebat audire præcipuorum nobilium ordinem,
qui hoc interim spatio in proc¦tone, in proximo, regem conventum præsto
erant." Of the crowd of suitors at Penn's house. Croese says, "Visi
quandoquo de hoc genere hominum non minus bis centum."--Historia
Quakeriana, lib. ii. 1695.}
 He paid dear,
however, for this seeming prosperity. Even his own sect looked coldly
on him, and requited his services with obloquy. He was loudly accused of
being a Papist, nay, a Jesuit. Some affirmed that he had been educated
at St. Omers, and others that he had been ordained at Rome. These
calumnies, indeed, could find credit only with the undiscerning
multitude; but with these calumnies were mingled accusations much better
founded.

To speak the whole truth concerning Penn is a task which requires some
courage; for he is rather a mythical than a historical person. Rival
nations and hostile sects have agreed in canonising him. England is
proud of his name. A great commonwealth beyond the Atlantic regards him
with a reverence similar to that which the Athenians felt for Theseus,
and the Romans for Quirinus. The respectable society of which he was a
member honours him as an apostle. By pious men of other persuasions he
is generally regarded as a bright pattern of Christian virtue. Meanwhile
admirers of a very different sort have sounded his praises. The French
philosophers of the eighteenth century pardoned what they regarded as
his superstitious fancies in consideration of his contempt for priests,
and of his cosmopolitan benevolence, impartially extended to all races
and to all creeds. His name has thus become, throughout all civilised
countries, a synonyme for probity and philanthropy.

Nor is this high reputation altogether unmerited. Penn was without doubt
a man of eminent virtues. He had a strong sense of religious duty and a
fervent desire to promote the happiness of mankind. On one or two points
of high importance, he had notions more correct than were, in his day,
common even among men of enlarged minds: and as the proprietor and
legislator of a province which, being almost uninhabited when it came
into his possession, afforded a clear field for moral experiments,
he had the rare good fortune of being able to carry his theories into
practice without any compromise, and yet without any shock to existing
institutions. He will always be mentioned with honour as a founder of
a colony, who did not, in his dealings with a savage people, abuse the
strength derived from civilisation, and as a lawgiver who, in an age of
persecution, made religious liberty the cornerstone of a polity. But his
writings and his life furnish abundant proofs that he was not a man of
strong sense. He had no skill in reading the characters of others. His
confidence in persons less virtuous than himself led him into great
errors and misfortunes. His enthusiasm for one great principle sometimes
impelled him to violate other great principles which he ought to
have held sacred. Nor was his rectitude altogether proof against the
temptations to which it was exposed in that splendid and polite, but
deeply corrupted society, with which he now mingled. The whole court was
in a ferment with intrigues of gallantry and intrigues of ambition. The
traffic in honours, places, and pardons was incessant. It was natural
that a man who was daily seen at the palace, and who was known to have
free access to majesty, should be frequently importuned to use his
influence for purposes which a rigid morality must condemn. The
integrity of Penn had stood firm against obloquy and persecution.
But now, attacked by royal smiles, by female blandishments, by the
insinuating eloquence and delicate flattery of veteran diplomatists and
courtiers, his resolution began to give way. Titles and phrases against
which he had often borne his testimony dropped occasionally from his
lips and his pen. It would be well if he had been guilty of nothing
worse than such compliances with the fashions of the world. Unhappily
it cannot be concealed that he bore a chief part in some transactions
condemned, not merely by the rigid code of the society to which he
belonged, but by the general sense of all honest men. He afterwards
solemnly protested that his hands were pure from illicit gain, and
that he had never received any gratuity from those whom he had obliged,
though he might easily, while his influence at court lasted, have made a
hundred and twenty thousand pounds. 
%[299]
\footnote{ "Twenty thousand into my pocket; and a hundred thousand
into my province." Penn's "Letter to Popple."}
 To this assertion full credit
is due. But bribes may be offered to vanity as well as to cupidity; and
it is impossible to deny that Penn was cajoled into bearing a part in
some unjustifiable transactions of which others enjoyed the profits.

The first use which he made of his credit was highly commendable. He
strongly represented the sufferings of his brethren to the new King,
who saw with pleasure that it was possible to grant indulgence to these
quiet sectaries and to the Roman Catholics, without showing similar
favour to other classes which were then under persecution. A list was
framed of prisoners against whom proceedings had been instituted for
not taking the oaths, or for not going to church, and of whose loyalty
certificates had been produced to the government. These persons were
discharged, and orders were given that no similar proceeding should be
instituted till the royal pleasure should be further signified. In this
way about fifteen hundred Quakers, and a still greater number of Roman
Catholics, regained their liberty. 
%[300]
\footnote{ These orders, signed by Sunderland, will be found in
Sewel's History. They bear date April 18, 1685. They are written in
a style singularly obscure and intricate: but I think that I have
exhibited the meaning correctly. I have not been able to find any proof
that any person, not a Roman Catholic or a Quaker, regained his freedom
under these orders. See Neal's History of the Puritans, vol. ii. chap.
ii.; Gerard Croese, lib. ii. Croese estimates the number of Quakers
liberated at fourteen hundred and sixty.}


And now the time had arrived when the English Parliament was to meet.
The members of the House of Commons who had repaired to the capital were
so numerous that there was much doubt whether their chamber, as it was
then fitted up, would afford sufficient accommodation for them. They
employed the days which immediately preceded the opening of the session
in talking over public affairs with each other and with the agents
of the government. A great meeting of the loyal party was held at the
Fountain Tavern in the Strand; and Roger Lestrange, who had recently
been knighted by the King, and returned to Parliament by the city of
Winchester, took a leading part in their consultations. 
%[301]
\footnote{ Barillon, May 28,/June 7, 1685. Observator, May 27, 1685;
Sir J. Reresby's Memoirs.}


It soon appeared that a large portion of the Commons had views which did
not altogether agree with those of the Court. The Tory country gentlemen
were, with scarcely one exception, desirous to maintain the Test Act and
the Habeas Corpus Act; and some among them talked of voting the revenue
only for a term of years. But they were perfectly ready to enact severe
laws against the Whigs, and would gladly have seen all the supporters
of the Exclusion Bill made incapable of holding office. The King, on the
other hand, desired to obtain from the Parliament a revenue for life,
the admission of Roman Catholics to office, and the repeal of the Habeas
Corpus Act. On these three objects his heart was set; and he was by no
means disposed to accept as a substitute for them a penal law against
Exclusionists. Such a law, indeed, would have been positively unpleasing
to him; for one class of Exclusionists stood high in his favour, that
class of which Sunderland was the representative, that class which had
joined the Whigs in the days of the plot, merely because the Whigs were
predominant, and which had changed with the change of fortune. James
justly regarded these renegades as the most serviceable tools that he
could employ. It was not from the stouthearted Cavaliers, who had
been true to him in his adversity, that he could expect abject and
unscrupulous obedience in his prosperity. The men who, impelled, not
by zeal for liberty or for religion, but merely by selfish cupidity and
selfish fear, had assisted to oppress him when he was weak, were the
very men who, impelled by the same cupidity and the same fear, would
assist him to oppress his people now that he was strong. 
%[302]
\footnote{ Lewis wrote to Barillon about this class of Exclusionists
as follows: "L'interet qu'ils auront a effacer cette tache par des
services considerables les portera, aelon toutes les apparences, a le
servir plus utilement que ne pourraient faire ceux qui ont toujours ete
les plus attaches a sa personne." May 15-25,1685.}
 Though
vindictive, he was not indiscriminately vindictive. Not a single
instance can be mentioned in which he showed a generous compassion
to those who had opposed him honestly and on public grounds. But he
frequently spared and promoted those whom some vile motive had induced
to injure him. For that meanness which marked them out as fit implements
of tyranny was so precious in his estimation that he regarded it with
some indulgence even when it was exhibited at his own expense.

The King's wishes were communicated through several channels to the Tory
members of the Lower House. The majority was easily persuaded to forego
all thoughts of a penal law against the Exclusionists, and to consent
that His Majesty should have the revenue for life. But about the Test
Act and the Habeas Corpus Act the emissaries of the Court could obtain
no satisfactory assurances. 
%[303]
\footnote{ Barillon, May 4-14, 1685; Sir John Reresby's Memoirs.}


On the nineteenth of May the session was opened. The benches of the
Commons presented a singular spectacle. That great party, which, in
the last three Parliaments, had been predominant, had now dwindled to a
pitiable minority, and was indeed little more than a fifteenth part of
the House. Of the five hundred and thirteen knights and burgesses only
a hundred and thirty-five had ever sate in that place before. It is
evident that a body of men so raw and inexperienced must have been, in
some important qualities, far below the average of our representative
assemblies. 
%[304]
\footnote{ Burnet, i. 626; Evelyn's Diary, May, 22, 1685.}


The management of the House was confided by James to two peers of the
kingdom of Scotland. One of them, Charles Middleton, Earl of Middleton,
after holding high office at Edinburgh, had, shortly before the death
of the late King, been sworn of the English Privy Council, and appointed
one of the Secretaries of State. With him was joined Richard Graham,
Viscount Preston, who had long held the post of Envoy at Versailles.

The first business of the Commons was to elect a Speaker. Who should
be the man, was a question which had been much debated in the cabinet.
Guildford had recommended Sir Thomas Meres, who, like himself, ranked
among the Trimmers. Jeffreys, who missed no opportunity of crossing the
Lord Keeper, had pressed the claims of Sir John Trevor. Trevor had been
bred half a pettifogger and half a gambler, had brought to political
life sentiments and principles worthy of both his callings, had become
a parasite of the Chief Justice, and could, on occasion, imitate, not
unsuccessfully, the vituperative style of his patron. The minion of
Jeffreys was, as might have been expected, preferred by James, was
proposed by Middleton, and was chosen without opposition. 
%[305]
\footnote{ Roger North's Life of Guildford, 218; Bramston's
Memoirs.}


Thus far all went smoothly. But an adversary of no common prowess was
watching his time. This was Edward Seymour of Berry Pomeroy Castle,
member for the city of Exeter. Seymour's birth put him on a level with
the noblest subjects in Europe. He was the right heir male of the body
of that Duke of Somerset who had been brother-in-law of King Henry the
Eighth, and Protector of the realm of England. In the limitation of the
dukedom of Somerset, the elder Son of the Protector had been postponed
to the younger son. From the younger son the Dukes of Somerset were
descended. From the elder son was descended the family which dwelt at
Berry Pomeroy. Seymour's fortune was large, and his influence in the
West of England extensive. Nor was the importance derived from descent
and wealth the only importance which belonged to him. He was one of the
most skilful debaters and men of business in the kingdom. He had sate
many years in the House of Commons, had studied all its rules and
usages, and thoroughly understood its peculiar temper. He had been
elected speaker in the late reign under circumstances which made that
distinction peculiarly honourable. During several generations none
but lawyers had been called to the chair; and he was the first country
gentleman whose abilities and acquirements had enabled him to break that
long prescription. He had subsequently held high political office, and
had sate in the Cabinet. But his haughty and unaccommodating temper had
given so much disgust that he had been forced to retire. He was a Tory
and a Churchman: he had strenuously opposed the Exclusion Bill: he had
been persecuted by the Whigs in the day of their prosperity; and he
could therefore safely venture to hold language for which any person
suspected of republicanism would have been sent to the Tower. He had
long been at the head of a strong parliamentary connection, which
was called the Western Alliance, and which included many gentlemen of
Devonshire, Somersetshire, and Cornwall. 
%[306]
\footnote{ North's Life of Guildford, 228; News from Westminster.}


In every House of Commons, a member who unites eloquence, knowledge, and
habits of business, to opulence and illustrious descent, must be highly
considered. But in a House of Commons from which many of the most
eminent orators and parliamentary tacticians of the age were excluded,
and which was crowded with people who had never heard a debate, the
influence of such a man was peculiarly formidable. Weight of moral
character was indeed wanting to Edward Seymour. He was licentious,
profane, corrupt, too proud to behave with common politeness, yet not
too proud to pocket illicit gain. But he was so useful an ally, and so
mischievous an enemy that he was frequently courted even by those who
most detested him. 
%[307]
\footnote{ Burnet, i. 382; Letter from Lord Conway to Sir George
Rawdon, Dec. 28, 1677. in the Rawdon Papers.}


He was now in bad humour with the government. His interest had been
weakened in some places by the remodelling of the western boroughs: his
pride had been wounded by the elevation of Trevor to the chair; and he
took an early opportunity of revenging himself.

On the twenty-second of May the Commons were summoned to the bar of the
Lords; and the King, seated on his throne, made a speech to both Houses.
He declared himself resolved to maintain the established government
in Church and State. But he weakened the effect of this declaration
by addressing an extraordinary admonition to the Commons. He was
apprehensive, he said, that they might be inclined to dole out money to
him from time to time, in the hope that they should thus force him to
call them frequently together. But he must warn them that he was not to
be so dealt with, and that, if they wished him to meet them often they
must use him well. As it was evident that without money the government
could not be carried on, these expressions plainly implied that, if they
did not give him as much money as he wished, he would take it. Strange
to say, this harangue was received with loud cheers by the Tory
gentlemen at the bar. Such acclamations were then usual. It has now
been, during many years, the grave and decorous usage of Parliaments
to hear, in respectful silence, all expressions, acceptable or
unacceptable, which are uttered from the throne. 
%[308]
\footnote{ London Gazette, May 25, 1685; Evelyn's Diary, May 22,
1685.}


It was then the custom that, after the King had concisely explained his
reasons for calling Parliament together, the minister who held the Great
Seal should, at more length, explain to the Houses the state of public
affairs. Guildford, in imitation of his predecessors, Clarendon,
Bridgeman, Shaftesbury, and Nottingham, had prepared an elaborate
oration, but found, to his great mortification, that his services were
not wanted. 
%[309]
\footnote{ North's Life of Guildford, 256.}


As soon as the Commons had returned to their own chamber, it was
proposed that they should resolve themselves into a Committee, for the
purpose of settling a revenue on the King.

Then Seymour stood up. How he stood, looking like what he was, the chief
of a dissolute and high spirited gentry, with the artificial ringlets
clustering in fashionable profusion round his shoulders, and a mingled
expression of voluptuousness and disdain in his eye and on his lip, the
likenesses of him which still remain enable us to imagine. It was not,
the haughty Cavalier said, his wish that the Parliament should withhold
from the crown the means of carrying on the government. But was there
indeed a Parliament? Were there not on the benches many men who had, as
all the world knew, no right to sit there, many men whose elections
were tainted by corruption, many men forced by intimidation on reluctant
voters, and many men returned by corporations which had no legal
existence? Had not constituent bodies been remodelled, in defiance
of royal charters and of immemorial prescription? Had not returning
officers been everywhere the unscrupulous agents of the Court? Seeing
that the very principle of representation had been thus systematically
attacked, he knew not how to call the throng of gentlemen which he saw
around him by the honourable name of a House of Commons. Yet never was
there a time when it more concerned the public weal that the character
of Parliament should stand high. Great dangers impended over the
ecclesiastical and civil constitution of the realm. It was matter of
vulgar notoriety, it was matter which required no proof, that the Test
Act, the rampart of religion, and the Habeas Corpus Act, the rampart
of liberty, were marked out for destruction. "Before we proceed to
legislate on questions so momentous, let us at least ascertain whether
we really are a legislature. Let our first proceeding be to enquire into
the manner in which the elections have been conducted. And let us look
to it that the enquiry be impartial. For, if the nation shall find that
no redress is to be obtained by peaceful methods, we may perhaps ere
long suffer the justice which we refuse to do." He concluded by
moving that, before any supply was granted, the House would take into
consideration petitions against returns, and that no member whose right
to sit was disputed should be allowed to vote.

Not a cheer was heard. Not a member ventured to second the motion.
Indeed, Seymour had said much that no other man could have said with
impunity. The proposition fell to the ground, and was not even entered
on the journals. But a mighty effect had been produced. Barillon
informed his master that many who had not dared to applaud that
remarkable speech had cordially approved of it, that it was the
universal subject of conversation throughout London, and that the
impression made on the public mind seemed likely to be durable. 
%[310]
\footnote{ Burnet, i. 639; Evelyn's Diary, May 22, 1685; Barillon,
May 23,/June 2, and May 25,/June 4, 1685 The silence of the journals
perplexed Mr. Fox; but it is explained by the circumstance that
Seymour's motion was not seconded.}


The Commons went into committee without delay, and voted to the King,
for life, the whole revenue enjoyed by his brother. 
%[311]
\footnote{ Journals, May 22. Stat. Jac. II. i. 1.}


The zealous churchmen who formed the majority of the House seem to have
been of opinion that the promptitude with which they had met the wish of
James, touching the revenue, entitled them to expect some concession
on his part. They said that much had been done to gratify him, and that
they must now do something to gratify the nation. The House, therefore,
resolved itself into a Grand Committee of Religion, in order to consider
the best means of providing for the security of the ecclesiastical
establishment. In that Committee two resolutions were unanimously
adopted. The first expressed fervent attachment to the Church of
England. The second called on the King to put in execution the penal
laws against all persons who were not members of that Church. 
%[312]
\footnote{ Journals, May 26, 27. Sir J. Reresby's Memoirs.}


The Whigs would doubtless have wished to see the Protestant dissenters
tolerated, and the Roman Catholics alone persecuted. But the Whigs were
a small and a disheartened minority. They therefore kept themselves as
much as possible out of sight, dropped their party name, abstained from
obtruding their peculiar opinions on a hostile audience, and steadily
supported every proposition tending to disturb the harmony which as yet
subsisted between the Parliament and the Court.

When the proceedings of the Committee of Religion were known at
Whitehall, the King's anger was great. Nor can we justly blame him for
resenting the conduct of the Tories If they were disposed to require
the rigorous execution of the penal code, they clearly ought to have
supported the Exclusion Bill. For to place a Papist on the throne, and
then to insist on his persecuting to the death the teachers of that
faith in which alone, on his principles, salvation could be found, was
monstrous. In mitigating by a lenient administration the severity of the
bloody laws of Elizabeth, the King violated no constitutional principle.
He only exerted a power which has always belonged to the crown. Nay, he
only did what was afterwards done by a succession of sovereigns zealous
for Protestantism, by William, by Anne, and by the princes of the House
of Brunswick. Had he suffered Roman Catholic priests, whose lives
he could save without infringing any law, to be hanged, drawn, and
quartered for discharging what he considered as their first duty, he
would have drawn on himself the hatred and contempt even of those to
whose prejudices he had made so shameful a concession, and, had he
contented himself with granting to the members of his own Church a
practical toleration by a large exercise of his unquestioned prerogative
of mercy, posterity would have unanimously applauded him.

The Commons probably felt on reflection that they had acted absurdly.
They were also disturbed by learning that the King, to whom they looked
up with superstitious reverence, was greatly provoked. They made haste,
therefore, to atone for their offence. In the House, they unanimously
reversed the decision which, in the Committee, they had unanimously
adopted and passed a resolution importing that they relied with entire
confidence on His Majesty's gracious promise to protect that religion
which was dearer to them than life itself. 
%[313]
\footnote{ Commons' Journals, May 27, 1685.}


Three days later the King informed the House that his brother had left
some debts, and that the stores of the navy and ordnance were nearly
exhausted. It was promptly resolved that new taxes should be imposed.
The person on whom devolved the task of devising ways and means was Sir
Dudley North, younger brother of the Lord Keeper. Dudley North was one
of the ablest men of his time. He had early in life been sent to the
Levant, and had there been long engaged in mercantile pursuits. Most men
would, in such a situation, have allowed their faculties to rust. For
at Smyrna and Constantinople there were few books and few intelligent
companions. But the young factor had one of those vigorous
understandings which are independent of external aids. In his solitude
he meditated deeply on the philosophy of trade, and thought out by
degrees a complete and admirable theory, substantially the same with
that which, a century later, was expounded by Adam Smith. After an exile
of many years, Dudley North returned to England with a large fortune,
and commenced business as a Turkey merchant in the City of London.
His profound knowledge, both speculative and practical, of commercial
matters, and the perspicuity and liveliness with which he explained
his views, speedily introduced him to the notice of statesmen.
The government found in him at once an enlightened adviser and an
unscrupulous slave. For with his rare mental endowments were joined lax
principles and an unfeeling heart. When the Tory reaction was in full
progress, he had consented to be made Sheriff for the express purpose
of assisting the vengeance of the court. His juries had never failed
to find verdicts of Guilty; and, on a day of judicial butchery, carts,
loaded with the legs and arms of quartered Whigs, were, to the great
discomposure of his lady, driven to his fine house in Basinghall
Street for orders. His services had been rewarded with the honour of
knighthood, with an Alderman's gown, and with the office of Commissioner
of the Customs. He had been brought into Parliament for Banbury, and
though a new member, was the person on whom the Lord Treasurer chiefly
relied for the conduct of financial business in the Lower House. 
%[314]
\footnote{ Roger North's Life of Sir Dudley North; Life of Lord
Guilford, 166; Mr M'Cullough's Literature of Political Economy.}


Though the Commons were unanimous in their resolution to grant a further
supply to the crown, they were by no means agreed as to the sources from
which that supply should be drawn. It was speedily determined that part
of the sum which was required should be raised by laying an additional
impost, for a term of eight years, on wine and vinegar: but something
more than this was needed. Several absurd schemes were suggested. Many
country gentlemen were disposed to put a heavy tax on all new buildings
in the capital. Such a tax, it was hoped, would check the growth of
a city which had long been regarded with jealousy and aversion by the
rural aristocracy. Dudley North's plan was that additional duties should
be imposed, for a term of eight years, on sugar and tobacco. A great
clamour was raised Colonial merchants, grocers, sugar bakers and
tobacconists, petitioned the House and besieged the public offices. The
people of Bristol, who were deeply interested in the trade with Virginia
and Jamaica, sent up a deputation which was heard at the bar of the
Commons. Rochester was for a moment staggered; but North's ready wit and
perfect knowledge of trade prevailed, both in the Treasury and in the
Parliament, against all opposition. The old members were amazed at
seeing a man who had not been a fortnight in the House, and whose life
had been chiefly passed in foreign countries, assume with confidence,
and discharge with ability, all the functions of a Chancellor of the
Exchequer. 
%[315]
\footnote{ Life of Dudley North, 176, Lonsdale's Memoirs, Van
Citters, June 12-22, 1685.}


His plan was adopted; and thus the Crown was in possession of a clear
income of about nineteen hundred thousand pounds, derived from England
alone. Such an income was then more than sufficient for the support of
the government in time of peace. 
%[316]
\footnote{ Commons' Journals, March 1, 1689.}


The Lords had, in the meantime, discussed several important questions.
The Tory party had always been strong among the peers. It included the
whole bench of Bishops, and had been reinforced during the four
years which had elapsed since the last dissolution, by several fresh
creations. Of the new nobles, the most conspicuous were the Lord
Treasurer Rochester, the Lord Keeper Guildford, the Lord Chief Justice
Jeffreys, the Lord Godolphin, and the Lord Churchill, who, after his
return from Versailles, had been made a Baron of England.

The peers early took into consideration the case of four members of
their body who had been impeached in the late reign, but had never been
brought to trial, and had, after a long confinement, been admitted to
bail by the Court of King's Bench. Three of the noblemen who were thus
under recognisances were Roman Catholics. The fourth was a Protestant
of great note and influence, the Earl of Danby. Since he had fallen from
power and had been accused of treason by the Commons, four Parliaments
had been dissolved; but he had been neither acquitted nor condemned.
In 1679 the Lords had considered, with reference to his situation,
the question whether an impeachment was or was not terminated by a
dissolution. They had resolved, after long debate and full examination
of precedents, that the impeachment was still pending. That resolution
they now rescinded. A few Whig nobles protested against this step, but
to little purpose. The Commons silently acquiesced in the decision of
the Upper House. Danby again took his seat among his peers, and became
an active and powerful member of the Tory party. 
%[317]
\footnote{ Lords' Journals, March 18, 19, 1679, May 22, 1685.}


The constitutional question on which the Lords thus, in the short space
of six years, pronounced two diametrically opposite decisions, slept
during more than a century, and was at length revived by the dissolution
which took place during the long trial of Warren Hastings. It was
then necessary to determine whether the rule laid down in 1679, or the
opposite rule laid down in 1685, was to be accounted the law of the
land. The point was long debated in both houses; and the best legal and
parliamentary abilities which an age preeminently fertile both in
legal and in parliamentary ability could supply were employed in the
discussion. The lawyers were not unequally divided. Thurlow, Kenyon,
Scott, and Erskine maintained that the dissolution had put an end to
the impeachment. The contrary doctrine was held by Mansfield, Camden,
Loughborough, and Grant. But among those statesmen who grounded their
arguments, not on precedents and technical analogies, but on deep and
broad constitutional principles, there was little difference of opinion.
Pitt and Grenville, as well as Burke and Fox, held that the impeachment
was still pending Both Houses by great majorities set aside the decision
of 1685, and pronounced the decision of 1679 to be in conformity with
the law of Parliament.

Of the national crimes which had been committed during the panic excited
by the fictions of Oates, the most signal had been the judicial murder
of Stafford. The sentence of that unhappy nobleman was now regarded
by all impartial persons as unjust. The principal witness for the
prosecution had been convicted of a series of foul perjuries. It was
the duty of the legislature, in such circumstances, to do justice to the
memory of a guiltless sufferer, and to efface an unmerited stain from a
name long illustrious in our annals. A bill for reversing the attainder
of Stafford was passed by the Upper House, in spite of the murmurs of a
few peers who were unwilling to admit that they had shed innocent blood.
The Commons read the bill twice without a division, and ordered it to
be committed. But, on the day appointed for the committee, arrived news
that a formidable rebellion had broken out in the West of England. It
was consequently necessary to postpone much important business. The
amends due to the memory of Stafford were deferred, as was supposed,
only for a short time. But the misgovernment of James in a few months
completely turned the tide of public feeling. During several generations
the Roman Catholics were in no condition to demand reparation for
injustice, and accounted themselves happy if they were permitted to live
unmolested in obscurity and silence. At length, in the reign of King
George the Fourth, more than a hundred and forty years after the day on
which the blood of Stafford was shed on Tower Hill, the tardy expiation
was accomplished. A law annulling the attainder and restoring the
injured family to its ancient dignities was presented to Parliament by
the ministers of the crown, was eagerly welcomed by public men of all
parties, and was passed without one dissentient voice. 
%[318]
\footnote{ Stat. 5 Geo. IV. c. 46.}


It is now necessary that I should trace the origin and progress of
that rebellion by which the deliberations of the Houses were suddenly
interrupted.




\chapter{CHAPTER V.}

TOWARDS the close of the reign of Charles the Second, some Whigs who had
been deeply implicated in the plot so fatal to their party, and who knew
themselves to be marked out for destruction, had sought an asylum in the
Low Countries.

These refugees were in general men of fiery temper and weak judgment.
They were also under the influence of that peculiar illusion which seems
to belong to their situation. A politician driven into banishment by a
hostile faction generally sees the society which he has quitted through
a false medium. Every object is distorted and discoloured by his
regrets, his longings, and his resentments. Every little discontent
appears to him to portend a revolution. Every riot is a rebellion. He
cannot be convinced that his country does not pine for him as much as
he pines for his country. He imagines that all his old associates, who
still dwell at their homes and enjoy their estates, are tormented by
the same feelings which make life a burden to himself. The longer his
expatriation, the greater does this hallucination become. The lapse of
time, which cools the ardour of the friends whom he has left behind,
inflames his. Every month his impatience to revisit his native land
increases; and every month his native land remembers and misses him
less. This delusion becomes almost a madness when many exiles who
suffer in the same cause herd together in a foreign country. Their chief
employment is to talk of what they once were, and of what they may yet
be, to goad each other into animosity against the common enemy, to feed
each other with extravagant hopes of victory and revenge. Thus they
become ripe for enterprises which would at once be pronounced hopeless
by any man whose passions had not deprived him of the power of
calculating chances.

In this mood were many of the outlaws who had assembled on the
Continent. The correspondence which they kept up with England was, for
the most part, such as tended to excite their feelings and to mislead
their judgment. Their information concerning the temper of the public
mind was chiefly derived from the worst members of the Whig party, from
men who were plotters and libellers by profession, who were pursued by
the officers of justice, who were forced to skulk in disguise through
back streets, and who sometimes lay hid for weeks together in cocklofts
and cellars. The statesmen who had formerly been the ornaments of the
Country Party, the statesmen who afterwards guided the councils of the
Convention, would have given advice very different from that which was
given by such men as John Wildman and Henry Danvers.

Wildman had served forty years before in the parliamentary army, but had
been more distinguished there as an agitator than as a soldier, and had
early quitted the profession of arms for pursuits better suited to
his temper. His hatred of monarchy had induced him to engage in a long
series of conspiracies, first against the Protector, and then against
the Stuarts. But with Wildman's fanaticism was joined a tender care for
his own safety. He had a wonderful skill in grazing the edge of
treason. No man understood better how to instigate others to desperate
enterprises by words which, when repeated to a jury, might seem
innocent, or, at worst, ambiguous. Such was his cunning that, though
always plotting, though always known to be plotting, and though long
malignantly watched by a vindictive government, he eluded every
danger, and died in his bed, after having seen two generations of his
accomplices die on the gallows. 
%[319]
\footnote{ Clarendon's History of the Rebellion, book xiv.; Burnet's
Own Times, i. 546, 625; Wade's and Ireton's Narratives, Lansdowne MS.
1152; West's information in the Appendix to Sprat's True Account.}
 Danvers was a man of the same
class, hotheaded, but fainthearted, constantly urged to the brink of
danger by enthusiasm, and constantly stopped on that brink by cowardice.
He had considerable influence among a portion of the Baptists, had
written largely in defence of their peculiar opinions, and had drawn
down on himself the severe censure of the most respectable Puritans by
attempting to palliate the crimes of Matthias and John of Leyden. It is
probable that, had he possessed a little courage, he would have trodden
in the footsteps of the wretches whom he defended. He was, at this time,
concealing himself from the officers of justice; for warrants were
out against him on account of a grossly calumnious paper of which the
government had discovered him to be the author. 
%[320]
\footnote{ London Gazette, January, 4, 1684-5; Ferguson MS. in
Eachard's History, iii. 764; Grey's Narratives; Sprat's True Account,
Danvers's Treatise on Baptism; Danvers's Innocency and Truth vindicated;
Crosby's History of the English Baptists.}


It is easy to imagine what kind of intelligence and counsel men, such
as have been described, were likely to send to the outlaws in the
Netherlands. Of the general character of those outlaws an estimate may
be formed from a few samples.

One of the most conspicuous among them was John Ayloffe, a lawyer
connected by affinity with the Hydes, and through the Hydes, with James.
Ayloffe had early made himself remarkable by offering a whimsical
insult to the government. At a time when the ascendancy of the court
of Versailles had excited general uneasiness, he had contrived to put a
wooden shoe, the established type, among the English, of French tyranny,
into the chair of the House of Commons. He had subsequently been
concerned in the Whig plot; but there is no reason to believe that he
was a party to the design of assassinating the royal brothers. He was
a man of parts and courage; but his moral character did not stand high.
The Puritan divines whispered that he was a careless Gallio or something
worse, and that, whatever zeal he might profess for civil liberty, the
Saints would do well to avoid all connection with him. 
%[321]
\footnote{ Sprat's True Account; Burnet, i. 634; Wade's Confession,
Earl. MS. 6845.---- Lord Howard of Escrick accused Ayloffe of proposing
to assassinate the Duke of York; but Lord Howard was an abject liar;
and this story was not part of his original confession, but was added
afterwards by way of supplement, and therefore deserves no credit
whatever.}


Nathaniel Wade was, like Ayloffe, a lawyer. He had long resided at
Bristol, and had been celebrated in his own neighbourhood as a vehement
republican. At one time he had formed a project of emigrating to New
Jersey, where he expected to find institutions better suited to
his taste than those of England. His activity in electioneering had
introduced him to the notice of some Whig nobles. They had employed him
professionally, and had, at length, admitted him to their most secret
counsels. He had been deeply concerned in the scheme of insurrection,
and had undertaken to head a rising in his own city. He had also been
privy to the more odious plot against the lives of Charles and James.
But he always declared that, though privy to it, he had abhorred it, and
had attempted to dissuade his associates from carrying their design into
effect. For a man bred to civil pursuits, Wade seems to have had, in an
unusual degree, that sort of ability and that sort of nerve which make a
good soldier. Unhappily his principles and his courage proved to be not
of sufficient force to support him when the fight was over, and when in
a prison, he had to choose between death and infamy. 
%[322]
\footnote{ Wade's Confession, Harl. MS. 6845; Lansdowne MS. 1152;
Holloway's narrative in the Appendix to Sprat's True Account. Wade owned
that Holloway had told nothing but truth.}


Another fugitive was Richard Goodenough, who had formerly been Under
Sheriff of London. On this man his party had long relied for services
of no honourable kind, and especially for the selection of jurymen not
likely to be troubled with scruples in political cases. He had been
deeply concerned in those dark and atrocious parts of the Whig plot
which had been carefully concealed from the most respectable Whigs. Nor
is it possible to plead, in extenuation of his guilt, that he was misled
by inordinate zeal for the public good. For it will be seen that after
having disgraced a noble cause by his crimes, he betrayed it in order to
escape from his well merited punishment. 
%[323]
\footnote{ Sprat's True Account and Appendix, passim.}


Very different was the character of Richard Rumbold. He had held a
commission in Cromwell's own regiment, had guarded the scaffold before
the Banqueting House on the day of the great execution, had fought at
Dunbar and Worcester, and had always shown in the highest degree the
qualities which distinguished the invincible army in which he served,
courage of the truest temper, fiery enthusiasm, both political and
religious, and with that enthusiasm, all the power of selfgovernment
which is characteristic of men trained in well disciplined camps to
command and to obey. When the Republican troops were disbanded, Rumbold
became a maltster, and carried on his trade near Hoddesdon, in that
building from which the Rye House plot derives its name. It had been
suggested, though not absolutely determined, in the conferences of the
most violent and unscrupulous of the malecontents, that armed men should
be stationed in the Rye House to attack the Guards who were to escort
Charles and James from Newmarket to London. In these conferences Rumbold
had borne a part from which he would have shrunk with horror, if his
clear understanding had not been overclouded, and his manly heart
corrupted, by party spirit. 
%[324]
\footnote{ Sprat's True Account and Appendix, Proceedings against
Rumbold in the Collection of State Trials; Burnet's Own Times, i. 633;
Appendix to Fox's History, No. IV.}


A more important exile was Ford Grey, Lord Grey of Wark. He had been a
zealous Exclusionist, had concurred in the design of insurrection, and
had been committed to the Tower, but had succeeded in making his keepers
drunk, and in effecting his escape to the Continent. His parliamentary
abilities were great, and his manners pleasing: but his life had been
sullied by a great domestic crime. His wife was a daughter of the noble
house of Berkeley. Her sister, the Lady Henrietta Berkeley, was allowed
to associate and correspond with him as with a brother by blood. A
fatal attachment sprang up. The high spirit and strong passions of
Lady Henrietta broke through all restraints of virtue and decorum. A
scandalous elopement disclosed to the whole kingdom the shame of two
illustrious families. Grey and some of the agents who had served him
in his amour were brought to trial on a charge of conspiracy. A scene
unparalleled in our legal history was exhibited in the Court of King's
Bench. The seducer appeared with dauntless front, accompanied by his
paramour. Nor did the great Whig lords flinch from their friend's side
even in that extremity. Those whom he had wronged stood over against
him, and were moved to transports of rage by the sight of him. The old
Earl of Berkeley poured forth reproaches and curses on the wretched
Henrietta. The Countess gave evidence broken by many sobs, and at length
fell down in a swoon. The jury found a verdict of Guilty. When the court
rose Lord Berkeley called on all his friends to help him to seize his
daughter. The partisans of Grey rallied round her. Swords were drawn on
both sides; a skirmish took place in Westminster Hall; and it was with
difficulty that the Judges and tipstaves parted the combatants. In our
time such a trial would be fatal to the character of a public man; but
in that age the standard of morality among the great was so low,
and party spirit was so violent, that Grey still continued to have
considerable influence, though the Puritans, who formed a strong section
of the Whig party, looked somewhat coldly on him. 
%[325]
\footnote{ Grey's narrative; his trial in the Collection of State
Trials; Sprat's True Account.}


One part of the character, or rather, it may be, of the fortune, of Grey
deserves notice. It was admitted that everywhere, except on the field
of battle, he showed a high degree of courage. More than once, in
embarrassing circumstances, when his life and liberty were at stake, the
dignity of his deportment and his perfect command of all his faculties
extorted praise from those who neither loved nor esteemed him. But as
a soldier he incurred, less perhaps by his fault than by mischance, the
degrading imputation of personal cowardice.

In this respect he differed widely from his friend the Duke of Monmouth.
Ardent and intrepid on the field of battle, Monmouth was everywhere
else effeminate and irresolute. The accident of his birth, his personal
courage, and his superficial graces, had placed him in a post for which
he was altogether unfitted. After witnessing the ruin of the party
of which he had been the nominal head, he had retired to Holland. The
Prince and Princess of Orange had now ceased to regard him as a rival.
They received him most hospitably; for they hoped that, by treating,
him with kindness, they should establish a claim to the gratitude of his
father. They knew that paternal affection was not yet wearied out, that
letters and supplies of money still came secretly from Whitehall to
Monmouth's retreat, and that Charles frowned on those who sought to pay
their court to him by speaking ill of his banished son. The Duke had
been encouraged to expect that, in a very short time, if he gave no
new cause of displeasure, he would be recalled to his native land,
and restored to all his high honours and commands. Animated by such
expectations he had been the life of the Hague during the late winter.
He had been the most conspicuous figure at a succession of balls in
that splendid Orange Hall, which blazes on every side with the most
ostentatious colouring of Jordæns and Hondthorst. 
%[326]
\footnote{ In the Pepysian Collection is a print representing one
of the balls which About this time William and Mary gave in the Oranje
Zaal.}
 He had taught
the English country dance to the Dutch ladies, and had in his turn
learned from them to skate on the canals. The Princess had accompanied
him in his expeditions on the ice; and the figure which she made there,
poised on one leg, and clad in petticoats shorter than are generally
worn by ladies so strictly decorous, had caused some wonder and mirth to
the foreign ministers. The sullen gravity which had been characteristic
of the Stadtholder's court seemed to have vanished before the influence
of the fascinating Englishman. Even the stern and pensive William
relaxed into good humour when his brilliant guest appeared. 
%[327]
\footnote{ Avaux Neg. January 25, 1685. Letter from James to the
Princess of Orange dated January 1684-5, among Birch's Extracts in the
British Museum.}


Monmouth meanwhile carefully avoided all that could give offence in the
quarter to which he looked for protection. He saw little of any Whigs,
and nothing of those violent men who had been concerned in the worst
part of the Whig plot. He was therefore loudly accused, by his old
associates, of fickleness and ingratitude. 
%[328]
\footnote{ Grey's Narrative; Wade's Confession, Lansdowne MS. 1152.}


By none of the exiles was this accusation urged with more vehemence and
bitterness than by Robert Ferguson, the Judas of Dryden's great satire.
Ferguson was by birth a Scot; but England had long been his residence.
At the time of the Restoration, indeed, he had held a living in Kent.
He had been bred a Presbyterian; but the Presbyterians had cast him out,
and he had become an Independent. He had been master of an academy which
the Dissenters had set up at Islington as a rival to Westminster School
and the Charter House; and he had preached to large congregations at
a meeting house in Moorfields. He had also published some theological
treatises which may still be found in the dusty recesses of a few old
libraries; but, though texts of Scripture were always on his lips, those
who had pecuniary transactions with him soon found him to be a mere
swindler.

At length he turned his attention almost entirely from theology to the
worst part of politics. He belonged to the class whose office it is
to render in troubled times to exasperated parties those services from
which honest men shrink in disgust and prudent men in fear, the class of
fanatical knaves. Violent, malignant, regardless of truth, insensible
to shame, insatiable of notoriety, delighting in intrigue, in tumult,
in mischief for its own sake, he toiled during many years in the darkest
mines of faction. He lived among libellers and false witnesses. He
was the keeper of a secret purse from which agents too vile to be
acknowledged received hire, and the director of a secret press whence
pamphlets, bearing no name, were daily issued. He boasted that he had
contrived to scatter lampoons about the terrace of Windsor, and even to
lay them under the royal pillow. In this way of life he was put to
many shifts, was forced to assume many names, and at one time had four
different lodgings in different corners of London. He was deeply engaged
in the Rye House plot. There is, indeed, reason to believe that he was
the original author of those sanguinary schemes which brought so much
discredit on the whole Whig party. When the conspiracy was detected and
his associates were in dismay, he bade them farewell with a laugh,
and told them that they were novices, that he had been used to flight,
concealment and disguise, and that he should never leave off plotting
while he lived. He escaped to the Continent. But it seemed that even on
the Continent he was not secure. The English envoys at foreign courts
were directed to be on the watch for him. The French government offered
a reward of five hundred pistoles to any who would seize him. Nor was it
easy for him to escape notice; for his broad Scotch accent, his tall and
lean figure, his lantern jaws, the gleam of his sharp eyes which were
always overhung by his wig, his cheeks inflamed by an eruption, his
shoulders deformed by a stoop, and his gait distinguished from that
of other men by a peculiar shuffle, made him remarkable wherever he
appeared. But, though he was, as it seemed, pursued with peculiar
animosity, it was whispered that this animosity was feigned, and that
the officers of justice had secret orders not to see him. That he was
really a bitter malecontent can scarcely be doubted. But there is strong
reason to believe that he provided for his own safety by pretending at
Whitehall to be a spy on the Whigs, and by furnishing the government
with just so much information as sufficed to keep up his credit.
This hypothesis furnishes a simple explanation of what seemed to his
associates to be his unnatural recklessness and audacity. Being himself
out of danger, he always gave his vote for the most violent and perilous
course, and sneered very complacently at the pusillanimity of men who,
not having taken the infamous precautions on which he relied, were
disposed to think twice before they placed life, and objects dearer than
life, on a single hazard. 
%[329]
\footnote{ Burnet, i. 542; Wood, Ath. Ox. under the name of Owen;
Absalom and Achtophel, part ii.; Eachard, iii. 682, 697; Sprat's True
Account, passim; Lond. Gaz. Aug. 6,1683; Nonconformist's Memorial;
North's Examen, 399.}


As soon as he was in the Low Countries he began to form new projects
against the English government, and found among his fellow emigrants
men ready to listen to his evil counsels. Monmouth, however, stood
obstinately aloof; and, without the help of Monmouth's immense
popularity, it was impossible to effect anything. Yet such was the
impatience and rashness of the exiles that they tried to find another
leader. They sent an embassy to that solitary retreat on the shores of
Lake Leman where Edmund Ludlow, once conspicuous among the chiefs of the
parliamentary army and among the members of the High Court of Justice,
had, during many years, hidden himself from the vengeance of the
restored Stuarts. The stern old regicide, however, refused to quit
his hermitage. His work, he said, was done. If England was still to be
saved, she must be saved by younger men. 
%[330]
\footnote{ Wade's Confession, Harl. MS. 6845.}


The unexpected demise of the crown changed the whole aspect of affairs.
Any hope which the proscribed Whigs might have cherished of returning
peaceably to their native land was extinguished by the death of a
careless and goodnatured prince, and by the accession of a prince
obstinate in all things, and especially obstinate in revenge. Ferguson
was in his element. Destitute of the talents both of a writer and of a
statesman, he had in a high degree the unenviable qualifications of a
tempter; and now, with the malevolent activity and dexterity of an
evil spirit, he ran from outlaw to outlaw, chattered in every ear, and
stirred up in every bosom savage animosities and wild desires.

He no longer despaired of being able to seduce Monmouth. The situation
of that unhappy young man was completely changed. While he was dancing
and skating at the Hague, and expecting every day a summons to London,
he was overwhelmed with misery by the tidings of his father's death and
of his uncle's accession. During the night which followed the arrival of
the news, those who lodged near him could distinctly hear his sobs and
his piercing cries. He quitted the Hague the next day, having solemnly
pledged his word both to the Prince and to the Princess of Orange not
to attempt anything against the government of England, and having been
supplied by them with money to meet immediate demands. 
%[331]
\footnote{ Avaux Neg. Feb. 20, 22, 1685; Monmouth's letter to James
from Ringwood.}


The prospect which lay before Monmouth was not a bright one. There was
now no probability that he would be recalled from banishment. On the
Continent his life could no longer be passed amidst the splendour and
festivity of a court. His cousins at the Hague seem to have really
regarded him with kindness; but they could no longer countenance him
openly without serious risk of producing a rupture between England and
Holland. William offered a kind and judicious suggestion. The war which
was then raging in Hungary, between the Emperor and the Turks, was
watched by all Europe with interest almost as great as that which the
Crusades had excited five hundred years earlier. Many gallant gentlemen,
both Protestant and Catholic, were fighting as volunteers in the common
cause of Christendom. The Prince advised Monmouth to repair to the
Imperial camp, and assured him that, if he would do so, he should not
want the means of making an appearance befitting an English nobleman.

%[332]
\footnote{ Boyer's History of King William the Third, 2d edition,
1703, vol. i 160.}
 This counsel was excellent: but the Duke could not make up
his mind. He retired to Brussels accompanied by Henrietta Wentworth,
Baroness Wentworth of Nettlestede, a damsel of high rank and ample
fortune, who loved him passionately, who had sacrificed for his sake her
maiden honour and the hope of a splendid alliance, who had followed him
into exile, and whom he believed to be his wife in the sight of heaven.
Under the soothing influence of female friendship, his lacerated mind
healed fast. He seemed to have found happiness in obscurity and repose,
and to have forgotten that he had been the ornament of a splendid court
and the head of a great party, that he had commanded armies, and that he
had aspired to a throne.

But he was not suffered to remain quiet. Ferguson employed all his
powers of temptation. Grey, who knew not where to turn for a pistole,
and was ready for any undertaking, however desperate, lent his aid.
No art was spared which could draw Monmouth from retreat. To the first
invitations which he received from his old associates he returned
unfavourable answers. He pronounced the difficulties of a descent on
England insuperable, protested that he was sick of public life, and
begged to be left in the enjoyment of his newly found happiness. But he
was little in the habit of resisting skilful and urgent importunity.
It is said, too, that he was induced to quit his retirement by the
same powerful influence which had made that retirement delightful. Lady
Wentworth wished to see him a King. Her rents, her diamonds, her credit
were put at his disposal. Monmouth's judgment was not convinced; but he
had not the firmness to resist such solicitations. 
%[333]
\footnote{ Welwood's Memoirs, App. xv.; Burnet, i. 530. Grey told a
somewhat different story, but he told it to save his life. The Spanish
ambassador at the English court, Don Pedro de Ronquillo, in a letter
to the governor of the Low Countries written about this time, sneers
at Monmouth for living on the bounty of a fond woman, and hints a very
unfounded suspicion that the Duke's passion was altogether interested.
"Hallandose hoy tan falto de medios que ha menester trasformarse en Amor
con Miledi en vista de la ecesidad de poder subsistir."--Ronquillo to
Grana. Mar. 30,/Apr. 9, 1685.}


By the English exiles he was joyfully welcomed, and unanimously
acknowledged as their head. But there was another class of emigrants who
were not disposed to recognise his supremacy. Misgovernment, such as
had never been known in the southern part of our island, had driven
from Scotland to the Continent many fugitives, the intemperance of whose
political and religious zeal was proportioned to the oppression which
they had undergone. These men were not willing to follow an English
leader. Even in destitution and exile they retained their punctilious
national pride, and would not consent that their country should be, in
their persons, degraded into a province. They had a captain of their
own, Archibald, ninth Earl of Argyle, who, as chief of the great tribe
of Campbell, was known among the population of the Highlands by the
proud name of Mac Callum More. His father, the Marquess of Argyle, had
been the head of the Scotch Covenanters, had greatly contributed to the
ruin of Charles the First, and was not thought by the Royalists to have
atoned for this offence by consenting to bestow the empty title of King,
and a state prison in a palace, on Charles the Second. After the return
of the royal family the Marquess was put to death. His marquisate became
extinct; but his son was permitted to inherit the ancient earldom,
and was still among the greatest if not the greatest, of the nobles of
Scotland. The Earl's conduct during the twenty years which followed the
Restoration had been, as he afterwards thought, criminally moderate. He
had, on some occasions, opposed the administration which afflicted
his country: but his opposition had been languid and cautious. His
compliances in ecclesiastical matters had given scandal to rigid
Presbyterians: and so far had he been from showing any inclination
to resistance that, when the Covenanters had been persecuted into
insurrection, he had brought into the field a large body of his
dependents to support the government.

Such had been his political course until the Duke of York came down to
Edinburgh armed with the whole regal authority The despotic viceroy soon
found that he could not expect entire support from Argyle. Since the
most powerful chief in the kingdom could not be gained, it was thought
necessary that he should be destroyed. On grounds so frivolous that even
the spirit of party and the spirit of chicane were ashamed of them, he
was brought to trial for treason, convicted, and sentenced to death. The
partisans of the Stuarts afterwards asserted that it was never meant
to carry this sentence into effect, and that the only object of the
prosecution was to frighten him into ceding his extensive jurisdiction
in the Highlands. Whether James designed, as his enemies suspected, to
commit murder, or only, as his friends affirmed, to commit extortion by
threatening to commit murder, cannot now be ascertained. "I know nothing
of the Scotch law," said Halifax to King Charles; "but this I know, that
we should not hang a dog here on the grounds on which my Lord Argyle has
been sentenced." 
%[334]
\footnote{ Proceedings against Argyle in the Collection of State
Trials, Burnet, i 521; A True and Plain Account of the Discoveries
made in Scotland, 1684, The Scotch Mist Cleared; Sir George Mackenzie's
Vindication, Lord Fountainhall's Chronological Notes.}


Argyle escaped in disguise to England, and thence passed over to
Friesland. In that secluded province his father had bought a small
estate, as a place of refuge for the family in civil troubles. It was
said, among the Scots that this purchase had been made in consequence of
the predictions of a Celtic seer, to whom it had been revealed that Mac
Callum More would one day be driven forth from the ancient mansion of
his race at Inverary. 
%[335]
\footnote{ Information of Robert Smith in the Appendix to Sprat's
True Account.}
 But it is probable that the politic Marquess
had been warned rather by the signs of the times than by the visions
of any prophet. In Friesland Earl Archibald resided during some time so
quietly that it was not generally known whither he had fled. From
his retreat he carried on a correspondence with his friends in Great
Britain, was a party to the Whig conspiracy, and concerted with the
chiefs of that conspiracy a plan for invading Scotland. 
%[336]
\footnote{ True and Plain Account of the Discoveries made in
Scotland.}
 This plan
had been dropped upon the detection of the Rye House plot, but became
again the Subject of his thoughts after the demise of the crown.

He had, during his residence on the Continent, reflected much more
deeply on religious questions than in the preceding years of his life.
In one respect the effect of these reflections on his mind had been
pernicious. His partiality for the synodical form of church government
now amounted to bigotry. When he remembered how long he had conformed to
the established worship, he was overwhelmed with shame and remorse, and
showed too many signs of a disposition to atone for his defection
by violence and intolerance. He had however, in no long time, an
opportunity of proving that the fear and love of a higher Power had
nerved him for the most formidable conflicts by which human nature can
be tried.

To his companions in adversity his assistance was of the highest moment.
Though proscribed and a fugitive, he was still, in some sense, the most
powerful subject in the British dominions. In wealth, even before his
attainder, he was probably inferior, not only to the great English
nobles, but to some of the opulent esquires of Kent and Norfolk. But his
patriarchal authority, an authority which no wealth could give and which
no attainder could take away, made him, as a leader of an insurrection,
truly formidable. No southern lord could feel any confidence that, if he
ventured to resist the government, even his own gamekeepers and huntsmen
would stand by him. An Earl of Bedford, an Earl of Devonshire, could not
engage to bring ten men into the field. Mac Callum More, penniless and
deprived of his earldom, might at any moment, raise a serious civil war.
He bad only to show himself on the coast of Lorn; and an army would,
in a few days, gather round him. The force which, in favourable
circumstances, he could bring into the field, amounted to five thousand
fighting, men, devoted to his service accustomed to the use of target
and broadsword, not afraid to encounter regular troops even in the open
plain, and perhaps superior to regular troops in the qualifications
requisite for the defence of wild mountain passes, hidden in mist,
and torn by headlong torrents. What such a force, well directed, could
effect, even against veteran regiments and skilful commanders, was
proved, a few years later, at Killiecrankie.

But, strong as was the claim of Argyle to the confidence of the exiled
Scots, there was a faction among them which regarded him with no
friendly feeling, and which wished to make use of his name and
influence, without entrusting to him any real power. The chief of this
faction was a lowland gentleman, who had been implicated in the Whig
plot, and had with difficulty eluded the vengeance of the court, Sir
Patrick Hume, of Polwarth, in Berwickshire. Great doubt has been thrown
on his integrity, but without sufficient reason. It must, however, be
admitted that he injured his cause by perverseness as much as he could
have done by treachery. He was a man incapable alike of leading and of
following, conceited, captious, and wrongheaded, an endless talker, a
sluggard in action against the enemy and active only against his own
allies. With Hume was closely connected another Scottish exile of great
note, who had many, of the same faults, Sir John Cochrane, second son of
the Earl of Dundonald.

A far higher character belonged to Andrew Fletcher of Saltoun, a man
distinguished by learning and eloquence, distinguished also by
courage, disinterestedness, and public spirit but of an irritable and
impracticable temper. Like many of his most illustrious contemporaries,
Milton for example, Harrington, Marvel, and Sidney, Fletcher had, from
the misgovernment of several successive princes, conceived a strong
aversion to hereditary monarchy. Yet he was no democrat. He was the head
of an ancient Norman house, and was proud of his descent. He was a
fine speaker and a fine writer, and was proud of his intellectual
superiority. Both in his character of gentleman, and in his character
of scholar, he looked down with disdain on the common people, and was
so little disposed to entrust them with political power that he thought
them unfit even to enjoy personal freedom. It is a curious circumstance
that this man, the most honest, fearless, and uncompromising republican
of his time, should have been the author of a plan for reducing a large
part of the working classes of Scotland to slavery. He bore, in truth,
a lively resemblance to those Roman Senators who, while they hated the
name of King, guarded the privileges of their order with inflexible
pride against the encroachments of the multitude, and governed their
bondmen and bondwomen by means of the stocks and the scourge.

Amsterdam was the place where the leading emigrants, Scotch and English,
assembled. Argyle repaired thither from Friesland, Monmouth from
Brabant. It soon appeared that the fugitives had scarcely anything in
common except hatred of James and impatience to return from banishment.
The Scots were jealous of the English, the English of the Scots.
Monmouth's high pretensions were offensive to Argyle, who, proud of
ancient nobility and of a legitimate descent from kings, was by no means
inclined to do homage to the offspring of a vagrant and ignoble love.
But of all the dissensions by which the little band of outlaws was
distracted the most serious was that which arose between Argyle and a
portion of his own followers. Some of the Scottish exiles had, in a long
course of opposition to tyranny, been excited into a morbid state
of understanding and temper, which made the most just and necessary
restraint insupportable to them. They knew that without Argyle they
could do nothing. They ought to have known that, unless they wished to
run headlong to ruin, they must either repose full confidence in their
leader, or relinquish all thoughts of military enterprise. Experience
has fully proved that in war every operation, from the greatest to the
smallest, ought to be under the absolute direction of one mind, and
that every subordinate agent, in his degree, ought to obey implicitly,
strenuously, and with the show of cheerfulness, orders which he
disapproves, or of which the reasons are kept secret from him.
Representative assemblies, public discussions, and all the other checks
by which, in civil affairs, rulers are restrained from abusing power,
are out of place in a camp. Machiavel justly imputed many of the
disasters of Venice and Florence to the jealousy which led those
republics to interfere with every one of their generals. 
%[337]
\footnote{ Discorsi sopra la prima Deca di Tito Livio, lib. ii. cap.
33.}
 The Dutch
practice of sending to an army deputies, without whose consent no great
blow could be struck, was almost equally pernicious. It is undoubtedly
by no means certain that a captain, who has been entrusted with
dictatorial power in the hour of peril, will quietly surrender that
power in the hour of triumph; and this is one of the many considerations
which ought to make men hesitate long before they resolve to vindicate
public liberty by the sword. But, if they determine to try the chance
of war, they will, if they are wise, entrust to their chief that plenary
authority without which war cannot be well conducted. It is possible
that, if they give him that authority, he may turn out a Cromwell or a
Napoleon. But it is almost certain that, if they withhold from him that
authority, their enterprises will end like the enterprise of Argyle.

Some of the Scottish emigrants, heated with republican enthusiasm,
and utterly destitute of the skill necessary to the conduct of great
affairs, employed all their industry and ingenuity, not in collecting
means for the attack which they were about to make on a formidable
enemy, but in devising restraints on their leader's power and securities
against his ambition. The selfcomplacent stupidity with which they
insisted on Organising an army as if they had been organising a
commonwealth would be incredible if it had not been frankly and even
boastfully recorded by one of themselves. 
%[338]
\footnote{ See Sir Patrick Hume's Narrative, passim.}


At length all differences were compromised. It was determined that an
attempt should be forthwith made on the western coast of Scotland, and
that it should be promptly followed by a descent on England.

Argyle was to hold the nominal command in Scotland: but he was placed
under the control of a Committee which reserved to itself all the most
important parts of the military administration. This committee was
empowered to determine where the expedition should land, to appoint
officers, to superintend the levying of troops, to dole out provisions
and ammunition. All that was left to the general was to direct the
evolutions of the army in the field, and he was forced to promise that
even in the field, except in the case of a surprise, he would do nothing
without the assent of a council of war.

Monmouth was to command in England. His soft mind had as usual, taken
an impress from the society which surrounded him. Ambitious hopes, which
had seemed to be extinguished, revived in his bosom. He remembered the
affection with which he had been constantly greeted by the common people
in town and country, and expected that they would now rise by hundreds
of thousands to welcome him. He remembered the good will which the
soldiers had always borne him, and flattered himself that they would
come over to him by regiments. Encouraging messages reached him in quick
succession from London. He was assured that the violence and injustice
with which the elections had been carried on had driven the nation mad,
that the prudence of the leading Whigs had with difficulty prevented a
sanguinary outbreak on the day of the coronation, and that all the great
Lords who had supported the Exclusion Bill were impatient to rally round
him. Wildman, who loved to talk treason in parables, sent to say that
the Earl of Richmond, just two hundred years before, had landed in
England with a handful of men, and had a few days later been crowned, on
the field of Bosworth, with the diadem taken from the head of Richard.
Danvers undertook to raise the City. The Duke was deceived into
the belief that, as soon as he set up his standard, Bedfordshire,
Buckinghamshire, Hampshire, Cheshire would rise in arms. 
%[339]
\footnote{ Grey's Narrative; Wade's Confession, Harl. MS. 6845.}
 He
consequently became eager for the enterprise from which a few weeks
before he had shrunk. His countrymen did not impose on him restrictions
so elaborately absurd as those which the Scotch emigrants had devised.
All that was required of him was to promise that he would not assume the
regal title till his pretensions has been submitted to the judgment of a
free Parliament.

It was determined that two Englishmen, Ayloffe and Rumbold, should
accompany Argyle to Scotland, and that Fletcher should go with Monmouth
to England. Fletcher, from the beginning, had augured ill of the
enterprise: but his chivalrous spirit would not suffer him to decline
a risk which his friends seemed eager to encounter. When Grey repeated
with approbation what Wildman had said about Richmond and Richard, the
well read and thoughtful Scot justly remarked that there was a great
difference between the fifteenth century and the seventeenth. Richmond
was assured of the support of barons, each of whom could bring an army
of feudal retainers into the field; and Richard had not one regiment of
regular soldiers. 
%[340]
\footnote{ Burnet, i. 631.}


The exiles were able to raise, partly from their own resources and
partly from the contributions of well wishers in Holland, a sum
sufficient for the two expeditions. Very little was obtained from
London. Six thousand pounds had been expected thence. But instead of the
money came excuses from Wildman, which ought to have opened the eyes
of all who were not wilfully blind. The Duke made up the deficiency by
pawning his own jewels and those of Lady Wentworth. Arms, ammunition,
and provisions were bought, and several ships which lay at Amsterdam
were freighted. 
%[341]
\footnote{ Grey's Narrative.}


It is remarkable that the most illustrious and the most grossly injured
man among the British exiles stood far aloof from these rash counsels.
John Locke hated tyranny and persecution as a philosopher; but his
intellect and his temper preserved him from the violence of a partisan.
He had lived on confidential terms with Shaftesbury, and had thus
incurred the displeasure of the court. Locke's prudence had, however,
been such that it would have been to little purpose to bring him even
before the corrupt and partial tribunals of that age. In one point,
however, he was vulnerable. He was a student of Christ Church in the
University of Oxford. It was determined to drive from that celebrated
college the greatest man of whom it could ever boast. But this was not
easy. Locke had, at Oxford, abstained from expressing any opinion on the
politics of the day. Spies had been set about him. Doctors of Divinity
and Masters of Arts had not been ashamed to perform the vilest of all
offices, that of watching the lips of a companion in order to report
his words to his ruin. The conversation in the hall had been purposely
turned to irritating topics, to the Exclusion Bill, and to the character
of the Earl of Shaftesbury, but in vain. Locke neither broke out nor
dissembled, but maintained such steady silence and composure as forced
the tools of power to own with vexation that never man was so complete
a master of his tongue and of his passions. When it was found that
treachery could do nothing, arbitrary power was used. After vainly
trying to inveigle Locke into a fault, the government resolved to punish
him without one. Orders came from Whitehall that he should be ejected;
and those orders the Dean and Canons made haste to obey.

Locke was travelling on the Continent for his health when he learned
that he had been deprived of his home and of his bread without a trial
or even a notice. The injustice with which he had been treated would
have excused him if he had resorted to violent methods of redress. But
he was not to be blinded by personal resentment he augured no good from
the schemes of those who had assembled at Amsterdam; and he quietly
repaired to Utrecht, where, while his partners in misfortune were
planning their own destruction, he employed himself in writing his
celebrated letter on Toleration. 
%[342]
\footnote{ Le Clerc's Life of Locke; Lord King's Life of Locke;
Lord Grenville's Oxford and Locke. Locke must not be confounded with
the Anabapist Nicholas Look, whose name was spelled Locke in Grey's
Confession, and who is mentioned in the Lansdowne MS. 1152, and in the
Buccleuch narrative appended to Mr. Rose's dissertation. I should hardly
think it necessary to make this remark, but that the similarity of
the two names appears to have misled a man so well acquainted with the
history of those times as Speaker Onslow. See his note on Burnet, i,
629.}


The English government was early apprised that something was in
agitation among the outlaws. An invasion of England seems not to have
been at first expected; but it was apprehended that Argyle would shortly
appear in arms among his clansmen. A proclamation was accordingly issued
directing that Scotland should be put into a state of defence. The
militia was ordered to be in readiness. All the clans hostile to the
name of Campbell were set in motion. John Murray, Marquess of Athol, was
appointed Lord Lieutenant of Argyleshire, and, at the head of a great
body of his followers, occupied the castle of Inverary. Some suspected
persons were arrested. Others were compelled to give hostages. Ships of
war were sent to cruise near the isle of Bute; and part of the army of
Ireland was moved to the coast of Ulster. 
%[343]
\footnote{ Wodrow, book iii. chap. ix; London Gazette, May 11, 1685;
Barillon, May 11-21.}


While these preparations were making in Scotland, James called into his
closet Arnold Van Citters, who had long resided in England as Ambassador
from the United Provinces, and Everard Van Dykvelt, who, after the death
of Charles, had been sent by the State General on a special mission of
condolence and congratulation. The King said that he had received
from unquestionable sources intelligence of designs which were forming
against the throne by his banished subjects in Holland. Some of the
exiles were cutthroats, whom nothing but the special providence of God
had prevented from committing a foul murder; and among them was the
owner of the spot which had been fixed for the butchery. "Of all men
living," said the King, "Argyle has the greatest means of annoying
me; and of all places Holland is that whence a blow may be best aimed
against me." The Dutch envoys assured his Majesty that what he had
said should instantly be communicated to the government which they
represented, and expressed their full confidence that every exertion
would be made to satisfy him. 
%[344]
\footnote{ Register of the Proceedings of the States General, May
5-15, 1685.}


They were justified in expressing this confidence. Both the Prince of
Orange and the States General, were, at this time, most desirous that
the hospitality of their country should not be abused for purposes of
which the English government could justly complain. James had lately
held language which encouraged the hope that he would not patiently
submit to the ascendancy of France. It seemed probable that he would
consent to form a close alliance with the United Provinces and the House
of Austria. There was, therefore, at the Hague, an extreme anxiety to
avoid all that could give him offence. The personal interest of William
was also on this occasion identical with the interest of his father in
law.

But the case was one which required rapid and vigorous action; and the
nature of the Batavian institutions made such action almost impossible.
The Union of Utrecht, rudely formed, amidst the agonies of a revolution,
for the purpose of meeting immediate exigencies, had never been
deliberately revised and perfected in a time of tranquillity. Every one
of the seven commonwealths which that Union had bound together retained
almost all the rights of sovereignty, and asserted those rights
punctiliously against the central government. As the federal authorities
had not the means of exacting prompt obedience from the provincial
authorities, so the provincial authorities had not the means of exacting
prompt obedience from the municipal authorities. Holland alone contained
eighteen cities, each of which was, for many purposes, an independent
state, jealous of all interference from without. If the rulers of such a
city received from the Hague an order which was unpleasing to them, they
either neglected it altogether, or executed it languidly and tardily.
In some town councils, indeed, the influence of the Prince of Orange was
all powerful. But unfortunately the place where the British exiles had
congregated, and where their ships had been fitted out, was the rich and
populous Amsterdam; and the magistrates of Amsterdam were the heads
of the faction hostile to the federal government and to the House of
Nassau. The naval administration of the United Provinces was conducted
by five distinct boards of Admiralty. One of those boards sate at
Amsterdam, was partly nominated by the authorities of that city, and
seems to have been entirely animated by their spirit.

All the endeavours of the federal government to effect what James
desired were frustrated by the evasions of the functionaries of
Amsterdam, and by the blunders of Colonel Bevil Skelton, who had just
arrived at the Hague as envoy from England. Skelton had been born in
Holland during the English troubles, and was therefore supposed to be
peculiarly qualified for his post; 
%[345]
\footnote{ This is mentioned in his credentials, dated on the 16th
of March, 1684-5.}
 but he was, in truth, unfit for
that and for every other diplomatic situation. Excellent judges of
character pronounced him to be the most shallow, fickle, passionate,
presumptuous, and garrulous of men. 
%[346]
\footnote{ Bonrepaux to Seignelay, February 4-14, 1686.}
 He took no serious notice
of the proceedings of the refugees till three vessels which had been
equipped for the expedition to Scotland were safe out of the Zuyder Zee,
till the arms, ammunition, and provisions were on board, and till the
passengers had embarked. Then, instead of applying, as he should have
done, to the States General, who sate close to his own door, he sent
a messenger to the magistrates of Amsterdam, with a request that the
suspected ships might be detained. The magistrates of Amsterdam answered
that the entrance of the Zuyder Zee was out of their jurisdiction, and
referred him to the federal government. It was notorious that this was a
mere excuse, and that, if there had been any real wish at the Stadthouse
of Amsterdam to prevent Argyle from sailing, no difficulties would have
been made. Skelton now addressed himself to the States General. They
showed every disposition to comply with his demand, and, as the case was
urgent, departed from the course which they ordinarily observed in
the transaction of business. On the same day on which he made his
application to them, an order, drawn in exact conformity with his
request, was despatched to the Admiralty of Amsterdam. But this order,
in consequence of some misinformation, did not correctly describe the
situation of the ships. They were said to be in the Texel. They were in
the Vlie. The Admiralty of Amsterdam made this error a plea for doing
nothing; and, before the error could be rectified, the three ships had
sailed. 
%[347]
\footnote{ Avaux Neg. April 30,/May 10, May 1-11, May 5-15, 1685;
Sir Patrick Hume's Narrative; Letter from The Admiralty of Amsterdam to
the States General, dated June 20, 1685; Memorial of Skelton, delivered
to the States General, May 10, 1685.}


The last hours which Argyle passed on the coast of Holland were hours of
great anxiety. Near him lay a Dutch man of war whose broadside would
in a moment have put an end to his expedition. Round his little fleet
a boat was rowing, in which were some persons with telescopes whom he
suspected to be spies. But no effectual step was taken for the purpose
of detaining him; and on the afternoon of the second of May he stood out
to sea before a favourable breeze.

The voyage was prosperous. On the sixth the Orkneys were in sight.
Argyle very unwisely anchored off Kirkwall, and allowed two of his
followers to go on shore there. The Bishop ordered them to be arrested.
The refugees proceeded to hold a long and animated debate on this
misadventure: for, from the beginning to the end of their expedition,
however languid and irresolute their conduct might be, they never
in debate wanted spirit or perseverance. Some were for an attack on
Kirkwall. Some were for proceeding without delay to Argyleshire. At last
the Earl seized some gentlemen who lived near the coast of the island,
and proposed to the Bishop an exchange of prisoners. The Bishop returned
no answer; and the fleet, after losing three days, sailed away.

This delay was full of danger. It was speedily known at Edinburgh that
the rebel squadron had touched at the Orkneys. Troops were instantly
put in motion. When the Earl reached his own province, he found that
preparations had been made to repel him. At Dunstaffnage he sent his
second son Charles on Shore to call the Campbells to arms. But Charles
returned with gloomy tidings. The herdsmen and fishermen were indeed
ready to rally round Mac Callum More; but, of the heads of the clan,
some were in confinement, and others had fled. Those gentlemen who
remained at their homes were either well affected to the government or
afraid of moving, and refused even to see the son of their chief. From
Dunstaffnage the small armament proceeded to Campbelltown, near the
southern extremity of the peninsula of Kintyre. Here the Earl published
a manifesto, drawn up in Holland, under the direction of the Committee,
by James Stewart, a Scotch advocate, whose pen was, a few months later,
employed in a very different way. In this paper were set forth, with a
strength of language sometimes approaching to scurrility, many real and
some imaginary grievances. It was hinted that the late King had died by
poison. A chief object of the expedition was declared to be the entire
suppression, not only of Popery, but of Prelacy, which was termed the
most bitter root and offspring of Popery; and all good Scotchmen were
exhorted to do valiantly for the cause of their country and of their
God.

Zealous as Argyle was for what he considered as pure religion, he
did not scruple to practice one rite half Popish and half Pagan. The
mysterious cross of yew, first set on fire, and then quenched in the
blood of a goat, was sent forth to summon all the Campbells, from
sixteen to sixty. The isthmus of Tarbet was appointed for the place of
gathering. The muster, though small indeed when compared with what
it would have been if the spirit and strength of the clan had been
unbroken, was still formidable. The whole force assembled amounted to
about eighteen hundred men. Argyle divided his mountaineers into three
regiments, and proceeded to appoint officers.

The bickerings which had begun in Holland had never been intermitted
during the whole course of the expedition; but at Tarbet they became
more violent than ever. The Committee wished to interfere even with the
patriarchal dominion of the Earl over the Campbells, and would not allow
him to settle the military rank of his kinsmen by his own authority.
While these disputatious meddlers tried to wrest from him his power
over the Highlands, they carried on their own correspondence with the
Lowlands, and received and sent letters which were never communicated
to the nominal General. Hume and his confederates had reserved to
themselves the superintendence of the Stores, and conducted this
important part of the administration of war with a laxity hardly to be
distinguished from dishonesty, suffered the arms to be spoiled, wasted
the provisions, and lived riotously at a time when they ought to have
set to all beneath them an example of abstemiousness.

The great question was whether the Highlands or the Lowlands should be
the seat of war. The Earl's first object was to establish his authority
over his own domains, to drive out the invading clans which had been
poured from Perthshire into Argyleshire, and to take possession of the
ancient seat of his family at Inverary. He might then hope to have four
or five thousand claymores at his command. With such a force he would be
able to defend that wild country against the whole power of the kingdom
of Scotland, and would also have secured an excellent base for offensive
operations. This seems to have been the wisest course open to him.
Rumbold, who had been trained in an excellent military school, and who,
as an Englishman, might be supposed to be an impartial umpire between
the Scottish factions, did all in his power to strengthen the Earl's
hands. But Hume and Cochrane were utterly impracticable. Their jealousy
of Argyle was, in truth, stronger than their wish for the success of the
expedition. They saw that, among his own mountains and lakes, and at the
head of an army chiefly composed of his own tribe, he would be able
to bear down their opposition, and to exercise the full authority of a
General. They muttered that the only men who had the good cause at heart
were the Lowlanders, and that the Campbells took up arms neither for
liberty nor for the Church of God, but for Mac Callum More alone.

Cochrane declared that he would go to Ayrshire if he went by himself,
and with nothing but a pitchfork in his hand. Argyle, after long
resistance, consented, against his better judgment, to divide his little
army. He remained with Rumbold in the Highlands. Cochrane and Hume were
at the head of the force which sailed to invade the Lowlands.

Ayrshire was Cochrane's object: but the coast of Ayrshire was guarded
by English frigates; and the adventurers were under the necessity of
running up the estuary of the Clyde to Greenock, then a small fishing
village consisting of a single row of thatched hovels, now a great and
flourishing port, of which the customs amount to more than five
times the whole revenue which the Stuarts derived from the kingdom of
Scotland. A party of militia lay at Greenock: but Cochrane, who
wanted provisions, was determined to land. Hume objected. Cochrane was
peremptory, and ordered an officer, named Elphinstone, to take twenty
men in a boat to the shore. But the wrangling spirit of the leaders had
infected all ranks. Elphinstone answered that he was bound to obey only
reasonable commands, that he considered this command as unreasonable,
and, in short, that he would not go. Major Fullarton, a brave man,
esteemed by all parties, but peculiarly attached to Argyle, undertook to
land with only twelve men, and did so in spite of a fire from the coast.
A slight skirmish followed. The militia fell back. Cochrane entered
Greenock and procured a supply of meal, but found no disposition to
insurrection among the people.

In fact, the state of public feeling in Scotland was not such as the
exiles, misled by the infatuation common in all ages to exiles, had
supposed it to be. The government was, indeed, hateful and hated. But
the malecontents were divided into parties which were almost as hostile
to one another as to their rulers; nor was any of those parties eager to
join the invaders. Many thought that the insurrection had no chance
of success. The spirit of many had been effectually broken by long and
cruel oppression. There was, indeed, a class of enthusiasts who were
little in the habit of calculating chances, and whom oppression had not
tamed but maddened. But these men saw little difference between Argyle
and James. Their wrath had been heated to such a temperature that what
everybody else would have called boiling zeal seemed to them Laodicean
lukewarmness. The Earl's past life had been stained by what they
regarded as the vilest apostasy. The very Highlanders whom he now
summoned to extirpate Prelacy he had a few years before summoned to
defend it. And were slaves who knew nothing and cared nothing about
religion, who were ready to fight for synodical government, for
Episcopacy, for Popery, just as Mac Callum More might be pleased to
command, fit allies for the people of God? The manifesto, indecent
and intolerant as was its tone, was, in the view of these fanatics, a
cowardly and worldly performance. A settlement such as Argyle would have
made, such as was afterwards made by a mightier and happier deliverer,
seemed to them not worth a struggle. They wanted not only freedom of
conscience for themselves, but absolute dominion over the consciences of
others; not only the Presbyterian doctrine, polity, and worship, but the
Covenant in its utmost rigour. Nothing would content them but that
every end for which civil society exists should be sacrificed to the
ascendency of a theological system. One who believed no form of
church government to be worth a breach of Christian charity, and who
recommended comprehension and toleration, was in their phrase, halting
between Jehovah and Baal. One who condemned such acts as the murder of
Cardinal Beatoun and Archbishop Sharpe fell into the same sin for which
Saul had been rejected from being King over Israel. All the rules,
by which, among civilised and Christian men, the horrors of war are
mitigated, were abominations in the sight of the Lord. Quarter was to be
neither taken nor given. A Malay running a muck, a mad dog pursued by
a crowd, were the models to be imitated by warriors fighting in just
self-defence. To reasons such as guide the conduct of statesmen and
generals the minds of these zealots were absolutely impervious. That a
man should venture to urge such reasons was sufficient evidence that
he was not one of the faithful. If the divine blessing were withheld,
little would be effected by crafty politicians, by veteran captains, by
cases of arms from Holland, or by regiments of unregenerate Celts from
the mountains of Lorn. If, on the other hand, the Lord's time were
indeed come, he could still, as of old, cause the foolish things of the
world to confound the wise, and could save alike by many and by few.
The broadswords of Athol and the bayonets of Claverhouse would be put to
rout by weapons as insignificant as the sling of David or the pitcher of
Gideon. 
%[348]
\footnote{ If any person is inclined to suspect that I have
exaggerated the absurdity and ferocity of these men, I would advise him
to read two books, which will convince him that I have rather softened
than overcharged the portrait, the Hind Let Loose, and Faithful
Contendings Displayed.}


Cochrane, having found it impossible to raise the population on the
south of the Clyde, rejoined Argyle, who was in the island of Bute.
The Earl now again proposed to make an attempt upon Inverary. Again he
encountered a pertinacious opposition. The seamen sided with Hume
and Cochrane. The Highlanders were absolutely at the command of their
chieftain. There was reason to fear that the two parties would come to
blows; and the dread of such a disaster induced the Committee to make
some concession. The castle of Ealan Ghierig, situated at the mouth of
Loch Riddan, was selected to be the chief place of arms. The military
stores were disembarked there. The squadron was moored close to the
walls in a place where it was protected by rocks and shallows such
as, it was thought, no frigate could pass. Outworks were thrown up.
A battery was planted with some small guns taken from the ships. The
command of the fort was most unwisely given to Elphinstone, who had
already proved himself much more disposed to argue with his commanders
than to fight the enemy.

And now, during a few hours, there was some show of vigour. Rumbold took
the castle of Ardkinglass. The Earl skirmished successfully with Athol's
troops, and was about to advance on Inverary, when alarming news from
the ships and factions in the Committee forced him to turn back. The
King's frigates had come nearer to Ealan Ghierig than had been thought
possible. The Lowland gentlemen positively refused to advance further
into the Highlands. Argyle hastened back to Ealan Ghierig. There he
proposed to make an attack on the frigates. His ships, indeed, were ill
fitted for such an encounter. But they would have been supported by
a flotilla of thirty large fishing boats, each well manned with armed
Highlanders. The Committee, however, refused to listen to this plan, and
effectually counteracted it by raising a mutiny among the sailors.

All was now confusion and despondency. The provisions had been so ill
managed by the Committee that there was no longer food for the troops.
The Highlanders consequently deserted by hundreds; and the Earl,
brokenhearted by his misfortunes, yielded to the urgency of those who
still pertinaciously insisted that he should march into the Lowlands.

The little army therefore hastened to the shore of Loch Long, passed
that inlet by night in boats, and landed in Dumbartonshire. Hither, on
the following morning, came news that the frigates had forced a passage,
that all the Earl's ships had been taken, and that Elphinstone had fled
from Ealan Ghierig without a blow, leaving the castle and stores to the
enemy.

All that remained was to invade the Lowlands under every disadvantage.
Argyle resolved to make a bold push for Glasgow. But, as soon as this
resolution was announced, the very men, who had, up to that moment,
been urging him to hasten into the low country, took fright, argued,
remonstrated, and when argument and remonstrance proved vain, laid a
scheme for seizing the boats, making their own escape, and leaving
their General and his clansmen to conquer or perish unaided. This scheme
failed; and the poltroons who had formed it were compelled to share with
braver men the risks of the last venture.

During the march through the country which lies between Loch Long and
Loch Lomond, the insurgents were constantly infested by parties
of militia. Some skirmishes took place, in which the Earl had the
advantage; but the bands which he repelled, falling back before him,
spread the tidings of his approach, and, soon after he had crossed the
river Leven, he found a strong body of regular and irregular troops
prepared to encounter him.

He was for giving battle. Ayloffe was of the same opinion. Hume, on the
other hand, declared that to fight would be madness. He saw one regiment
in scarlet. More might be behind. To attack such a force was to rush on
certain death The best course was to remain quiet till night, and then
to give the enemy the slip.

A sharp altercation followed, which was with difficulty quieted by the
mediation of Rumbold. It was now evening. The hostile armies encamped at
no great distance from each other. The Earl ventured to propose a night
attack, and was again overruled.

Since it was determined not to fight, nothing was left but to take the
step which Hume had recommended. There was a chance that, by decamping
secretly, and hastening all night across heaths and morasses, the Earl
might gain many miles on the enemy, and might reach Glasgow without
further obstruction. The watch fires were left burning; and the march
began. And now disaster followed disaster fast. The guides mistook the
track across the moors, and led the army into boggy ground. Military
order could not be preserved by undisciplined and disheartened soldiers
under a dark sky, and on a treacherous and uneven soil. Panic after
panic spread through the broken ranks. Every sight and sound was thought
to indicate the approach of pursuers. Some of the officers contributed
to spread the terror which it was their duty to calm. The army had
become a mob; and the mob melted fast away. Great numbers fled under
cover of the night. Rumbold and a few other brave men whom no danger
could have scared lost their way, and were unable to rejoin the main
body. When the day broke, only five hundred fugitives, wearied and
dispirited, assembled at Kilpatrick.

All thought of prosecuting the war was at an end: and it was plain
that the chiefs of the expedition would have sufficient difficulty
in escaping with their lives. They fled in different directions. Hume
reached the Continent in safety. Cochrane was taken and sent up to
London. Argyle hoped to find a secure asylum under the roof of one
of his old servants who lived near Kilpatrick. But this hope was
disappointed; and he was forced to cross the Clyde. He assumed the dress
of a peasant and pretended to be the guide of Major Fullarton, whose
courageous fidelity was proof to all danger. The friends journeyed
together through Renfrewshire as far as Inchinnan. At that place the
Black Cart and the White Cart, two streams which now flow through
prosperous towns, and turn the wheels of many factories, but which then
held their quiet course through moors and sheepwalks, mingle before they
join the Clyde. The only ford by which the travellers could cross was
guarded by a party of militia. Some questions were asked. Fullarton
tried to draw suspicion on himself, in order that his companion might
escape unnoticed. But the minds of the questioners misgave them that the
guide was not the rude clown that he seemed. They laid hands on him.
He broke loose and sprang into the water, but was instantly chased. He
stood at bay for a short time against five assailants. But he had no
arms except his pocket pistols, and they were so wet, in consequence of
his plunge, that they would not go off. He was struck to the ground with
a broadsword, and secured.

He owned himself to be the Earl of Argyle, probably in the hope that his
great name would excite the awe and pity of those who had seized him.
And indeed they were much moved. For they were plain Scotchmen of humble
rank, and, though in arms for the crown, probably cherished a preference
for the Calvinistic church government and worship, and had been
accustomed to reverence their captive as the head of an illustrious
house and as a champion of the Protestant religion But, though they
were evidently touched, and though some of them even wept, they were not
disposed to relinquish a large reward and to incur the vengeance of
an implacable government. They therefore conveyed their prisoner
to Renfrew. The man who bore the chief part in the arrest was named
Riddell. On this account the whole race of Riddells was, during more
than a century, held in abhorrence by the great tribe of Campbell.
Within living memory, when a Riddell visited a fair in Argyleshire, he
found it necessary to assume a false name.

And now commenced the brightest part of Argyle's career. His enterprise
had hitherto brought on him nothing but reproach and derision. His great
error was that he did not resolutely refuse to accept the name without
the power of a general. Had he remained quietly at his retreat in
Friesland, he would in a few years have been recalled with honour to
his country, and would have been conspicuous among the ornaments and
the props of constitutional monarchy. Had he conducted his expedition
according to his own views, and carried with him no followers but such
as were prepared implicitly to obey all his orders, he might possibly
have effected something great. For what he wanted as a captain seems to
have been, not courage, nor activity, nor skill, but simply authority.
He should have known that of all wants this is the most fatal.
Armies have triumphed under leaders who possessed no very eminent
qualifications. But what army commanded by a debating club ever escaped
discomfiture and disgrace?

The great calamity which had fallen on Argyle had this advantage, that
it enabled him to show, by proofs not to be mistaken, what manner of
man he was. From the day when he quitted. Friesland to the day when his
followers separated at Kilpatrick, he had never been a free agent. He
had borne the responsibility of a long series of measures which his
judgment disapproved. Now at length he stood alone. Captivity had
restored to him the noblest kind of liberty, the liberty of governing
himself in all his words and actions according to his own sense of the
right and of the becoming. From that moment he became as one inspired
with new wisdom and virtue. His intellect seemed to be strengthened and
concentrated, his moral character to be at once elevated and softened.
The insolence of the conquerors spared nothing that could try the temper
of a man proud of ancient nobility and of patriarchal dominion. The
prisoner was dragged through Edinburgh in triumph. He walked on
foot, bareheaded, up the whole length of that stately street which,
overshadowed by dark and gigantic piles of stone, leads from Holyrood
House to the Castle. Before him marched the hangman, bearing the ghastly
instrument which was to be used at the quartering block. The victorious
party had not forgotten that, thirty-five years before this time, the
father of Argyle had been at the head of the faction which put Montrose
to death. Before that event the houses of Graham and Campbell had borne
no love to each other; and they had ever since been at deadly feud. Care
was taken that the prisoner should pass through the same gate and the
same streets through which Montrose had been led to the same doom. 
%[349]
\footnote{ A few words which were in the first five editions have
been omitted in this place. Here and in another passage I had, as Mr.
Aytoun has observed, mistaken the City Guards, which were commanded by
an officer named Graham, for the Dragoons of Graham of Claverhouse.}

When the Earl reached the Castle his legs were put in irons, and he was
informed that he had but a few days to live. It had been determined not
to bring him to trial for his recent offence, but to put him to death
under the sentence pronounced against him several years before, a
sentence so flagitiously unjust that the most servile and obdurate
lawyers of that bad age could not speak of it without shame.

But neither the ignominious procession up the High Street, nor the near
view of death, had power to disturb the gentle and majestic patience of
Argyle. His fortitude was tried by a still more severe test. A paper of
interrogatories was laid before him by order of the Privy Council. He
replied to those questions to which he could reply without danger to
any of his friends, and refused to say more. He was told that unless he
returned fuller answers he should be put to the torture. James, who was
doubtless sorry that he could not feast his own eyes with the sight of
Argyle in the boots, sent down to Edinburgh positive orders that nothing
should be omitted which could wring out of the traitor information
against all who had been concerned in the treason. But menaces were
vain. With torments and death in immediate prospect Mac Callum More
thought far less of himself than of his poor clansmen. "I was busy this
day," he wrote from his cell, "treating for them, and in some hopes. But
this evening orders came that I must die upon Monday or Tuesday; and I
am to be put to the torture if I answer not all questions upon oath. Yet
I hope God shall support me."

The torture was not inflicted. Perhaps the magnanimity of the victim had
moved the conquerors to unwonted compassion. He himself remarked that at
first they had been very harsh to him, but that they soon began to treat
him with respect and kindness. God, he said, had melted their hearts. It
is certain that he did not, to save himself from the utmost cruelty of
his enemies, betray any of his friends. On the last morning of his life
he wrote these words: "I have named none to their disadvantage. I thank
God he hath supported me wonderfully!"

He composed his own epitaph, a short poem, full of meaning and spirit,
simple and forcible in style, and not contemptible in versification. In
this little piece he complained that, though his enemies had repeatedly
decreed his death, his friends had been still more cruel. A comment on
these expressions is to be found in a letter which he addressed to a
lady residing in Holland. She had furnished him with a large sum
of money for his expedition, and he thought her entitled to a full
explanation of the causes which had led to his failure. He acquitted his
coadjutors of treachery, but described their folly, their ignorance,
and their factious perverseness, in terms which their own testimony has
since proved to have been richly deserved. He afterwards doubted whether
he had not used language too severe to become a dying Christian, and,
in a separate paper, begged his friend to suppress what he had said of
these men "Only this I must acknowledge," he mildly added; "they were
not governable."

Most of his few remaining hours were passed in devotion, and in
affectionate intercourse with some members of his family. He professed
no repentance on account of his last enterprise, but bewailed, with
great emotion, his former compliance in spiritual things with the
pleasure of the government He had, he said, been justly punished. One
who had so long been guilty of cowardice and dissimulation was not
worthy to be the instrument of salvation to the State and Church. Yet
the cause, he frequently repeated, was the cause of God, and would
assuredly triumph. "I do not," he said, "take on myself to be a prophet.
But I have a strong impression on my spirit, that deliverance will come
very suddenly." It is not strange that some zealous Presbyterians should
have laid up his saying in their hearts, and should, at a later period,
have attributed it to divine inspiration.

So effectually had religious faith and hope, co-operating with natural
courage and equanimity, composed his spirits, that, on the very day on
which he was to die, he dined with appetite, conversed with gaiety at
table, and, after his last meal, lay down, as he was wont, to take a
short slumber, in order that his body and mind might be in full vigour
when he should mount the scaffold. At this time one of the Lords of the
Council, who had probably been bred a Presbyterian, and had been seduced
by interest to join in oppressing the Church of which he had once been
a member, came to the Castle with a message from his brethren, and
demanded admittance to the Earl. It was answered that the Earl was
asleep. The Privy Councillor thought that this was a subterfuge, and
insisted on entering. The door of the cell was softly opened; and there
lay Argyle, on the bed, sleeping, in his irons, the placid sleep of
infancy. The conscience of the renegade smote him. He turned away sick
at heart, ran out of the Castle, and took refuge in the dwelling of a
lady of his family who lived hard by. There he flung himself on a couch,
and gave himself up to an agony of remorse and shame. His kinswoman,
alarmed by his looks and groans, thought that he had been taken with
sudden illness, and begged him to drink a cup of sack. "No, no," he
said; "that will do me no good." She prayed him to tell her what had
disturbed him. "I have been," he said, "in Argyle's prison. I have seen
him within an hour of eternity, sleeping as sweetly as ever man did. But
as for me -------"

And now the Earl had risen from his bed, and had prepared himself for
what was yet to be endured. He was first brought down the High Street
to the Council House, where he was to remain during the short interval
which was still to elapse before the execution. During that interval
he asked for pen and ink, and wrote to his wife: "Dear heart, God is
unchangeable: He hath always been good and gracious to me: and no place
alters it. Forgive me all my faults; and now comfort thyself in Him, in
whom only true comfort is to be found. The Lord be with thee, bless and
comfort thee, my dearest. Adieu."

It was now time to leave the Council House. The divines who attended the
prisoner were not of his own persuasion; but he listened to them with
civility, and exhorted them to caution their flocks against those
doctrines which all Protestant churches unite in condemning. He mounted
the scaffold, where the rude old guillotine of Scotland, called the
Maiden, awaited him, and addressed the people in a speech, tinctured
with the peculiar phraseology of his sect, but breathing the spirit
of serene piety. His enemies, he said, he forgave, as he hoped to be
forgiven. Only a single acrimonious expression escaped him. One of the
episcopal clergymen who attended him went to the edge of the scaffold,
and called out in a loud voice, "My Lord dies a Protestant." "Yes," said
the Earl, stepping forward, "and not only a Protestant, but with a heart
hatred of Popery, of Prelacy, and of all superstition." He then embraced
his friends, put into their hands some tokens of remembrance for his
wife and children, kneeled down, laid his head on the block, prayed
during a few minutes, and gave the signal to the executioner. His head
was fixed on the top of the Tolbooth, where the head of Montrose had
formerly decayed. 
%[350]
\footnote{ The authors from whom I have taken the history of
Argyle's expedition are Sir Patrick Hume, who was an eyewitness of what
he related, and Wodrow, who had access to materials of the greatest
value, among which were the Earl's own papers. Wherever there is a
question of veracity between Argyle and Hume, I have no doubt that
Argyle's narrative ought to be followed.---- See also Burnet, i. 631,
and the life of Bresson, published by Dr. Mac Crie. The account of
the Scotch rebellion in the Life of James the Second, is a ridiculous
romance, not written by the King himself, nor derived from his papers,
but composed by a Jacobite who did not even take the trouble to look at
a map of the seat of war.}


The head of the brave and sincere, though not blameless Rumbold, was
already on the West Port of Edinburgh. Surrounded by factious and
cowardly associates, he had, through the whole campaign, behaved himself
like a soldier trained in the school of the great Protector, had in
council strenuously supported the authority of Argyle, and had in the
field been distinguished by tranquil intrepidity. After the dispersion
of the army he was set upon by a party of militia. He defended himself
desperately, and would have cut his way through them, had they not
hamstringed his horse. He was brought to Edinburgh mortally wounded. The
wish of the government was that he should be executed in England. But he
was so near death, that, if he was not hanged in Scotland, he could
not be hanged at all; and the pleasure of hanging him was one which the
conquerors could not bear to forego. It was indeed not to be expected
that they would show much lenity to one who was regarded as the chief
of the Rye House plot, and who was the owner of the building from which
that plot took its name: but the insolence with which they treated the
dying man seems to our more humane age almost incredible. One of the
Scotch Privy Councillors told him that he was a confounded villain.
"I am at peace with God," answered Rumbold, calmly; "how then can I be
confounded?"

He was hastily tried, convicted, and sentenced to be hanged and
quartered within a few hours, near the City Cross in the High Street.
Though unable to stand without the support of two men, he maintained
his fortitude to the last, and under the gibbet raised his feeble voice
against Popery and tyranny with such vehemence that the officers ordered
the drums to strike up, lest the people should hear him. He was a
friend, he said, to limited monarchy. But he never would believe that
Providence had sent a few men into the world ready booted and spurred to
ride, and millions ready saddled and bridled to be ridden. "I desire,"
he cried, "to bless and magnify God's holy name for this, that I stand
here, not for any wrong that I have done, but for adhering to his cause
in an evil day. If every hair of my head were a man, in this quarrel I
would venture them all."

Both at his trial and at his execution he spoke of assassination with
the abhorrence which became a good Christian and a brave soldier. He had
never, he protested, on the faith of a dying man, harboured the thought
of committing such villany. But he frankly owned that, in conversation
with his fellow conspirators, he had mentioned his own house as a place
where Charles and James might with advantage be attacked, and that much
had been said on the subject, though nothing had been determined. It may
at first sight seem that this acknowledgment is inconsistent with his
declaration that he had always regarded assassination with horror. But
the truth appears to be that he was imposed upon by a distinction which
deluded many of his contemporaries. Nothing would have induced him to
put poison into the food of the two princes, or to poinard them in their
sleep. But to make an unexpected onset on the troop of Life Guards which
surrounded the royal coach, to exchange sword cuts and pistol shots,
and to take the chance of slaying or of being slain, was, in his view,
a lawful military operation. Ambuscades and surprises were among the
ordinary incidents of war. Every old soldier, Cavalier or Roundhead,
had been engaged in such enterprises. If in the skirmish the King should
fall, he would fall by fair fighting and not by murder. Precisely the
same reasoning was employed, after the Revolution, by James himself and
by some of his most devoted followers, to justify a wicked attempt on
the life of William the Third. A band of Jacobites was commissioned to
attack the Prince of Orange in his winter quarters. The meaning latent
under this specious phrase was that the Prince's throat was to be cut
as he went in his coach from Richmond to Kensington. It may seem strange
that such fallacies, the dregs of the Jesuitical casuistry, should have
had power to seduce men of heroic spirit, both Whigs and Tories, into a
crime on which divine and human laws have justly set a peculiar note of
infamy. But no sophism is too gross to delude minds distempered by party
spirit. 
%[351]
\footnote{ Wodrow, III. ix 10; Western Martyrology; Burnet, i. 633;
Fox's History, Appendix iv. I can find no way, except that indicated in
the text, of reconciling Rumbold's denial that he had ever admitted into
his mind the thought of assassination with his confession that he had
himself mentioned his own house as a convenient place for an attack on
the royal brothers. The distinction which I suppose him to have taken
was certainly taken by another Rye House conspirator, who was, like him,
an old soldier of the Commonwealth, Captain Walcot. On Walcot's trial,
West, the witness for the crown, said, "Captain, you did agree to be one
of those that were to fight the Guards." "What, then, was the reason."
asked Chief Justice Pemberton, "that he would not kill the King?" "He
said," answered West, "that it was a base thing to kill a naked man, and
he would not do it."}


Argyle, who survived Rumbold a few hours, left a dying testimony to the
virtues of the gallant Englishman. "Poor Rumbold was a great support to
me, and a brave man, and died Christianly." 
%[352]
\footnote{ Wodrow, III. ix. 9.}


Ayloffe showed as much contempt of death as either Argyle or Rumbold:
but his end did not, like theirs, edify pious minds. Though political
sympathy had drawn him towards the Puritans, he had no religious
sympathy with them, and was indeed regarded by them as little better
than an atheist. He belonged to that section of the Whigs which sought
for models rather among the patriots of Greece and Rome than among the
prophets and judges of Israel. He was taken prisoner, and carried to
Glasgow. There he attempted to destroy himself with a small penknife:
but though he gave himself several wounds, none of them proved mortal,
and he had strength enough left to bear a journey to London. He was
brought before the Privy Council, and interrogated by the King, but had
too much elevation of mind to save himself by informing against others.
A story was current among the Whigs that the King said, "You had better
be frank with me, Mr. Ayloffe. You know that it is in my power to pardon
you." Then, it was rumoured, the captive broke his sullen silence, and
answered, "It may be in your power; but it is not in your nature." He
was executed under his old outlawry before the gate of the Temple, and
died with stoical composure. 
%[353]
\footnote{ Wade's narrative, Harl, MS. 6845; Burnet, i. 634; Van
Citters's Despatch of Oct. 30,/Nov. 9, 1685; Luttrell's Diary of the
same date.}


In the meantime the vengeance of the conquerors was mercilessly wreaked
on the people of Argyleshire. Many of the Campbells were hanged by Athol
without a trial; and he was with difficulty restrained by the Privy
Council from taking more lives. The country to the extent of thirty
miles round Inverary was wasted. Houses were burned: the stones of mills
were broken to pieces: fruit trees were cut down, and the very roots
seared with fire. The nets and fishing boats, the sole means by which
many inhabitants of the coast subsisted, were destroyed. More than three
hundred rebels and malecontents were transported to the colonies. Many
of them were also Sentenced to mutilation. On a single day the hangman
of Edinburgh cut off the ears of thirty-five prisoners. Several women
were sent across the Atlantic after being first branded in the cheek
with a hot iron. It was even in contemplation to obtain an act of
Parliament proscribing the name of Campbell, as the name of Macgregor
had been proscribed eighty years before. 
%[354]
\footnote{ Wodrow, III, ix. 4, and III. ix. 10. Wodrow gives from
the Acts of Council the names of all the prisoners who were transported,
mutilated or branded.}


Argyle's expedition appears to have produced little sensation in the
south of the island. The tidings of his landing reached London just
before the English Parliament met. The King mentioned the news from the
throne; and the Houses assured him that they would stand by him against
every enemy. Nothing more was required of them. Over Scotland they had
no authority; and a war of which the theatre was so distant, and of
which the event might, almost from the first, be easily foreseen,
excited only a languid interest in London.

But, a week before the final dispersion of Argyle's army England was
agitated by the news that a more formidable invader had landed on her
own shores. It had been agreed among the refugees that Monmouth should
sail from Holland six days after the departure of the Scots. He had
deferred his expedition a short time, probably in the hope that most
of the troops in the south of the island would be moved to the north as
soon as war broke out in the Highlands, and that he should find no force
ready to oppose him. When at length he was desirous to proceed, the wind
had become adverse and violent.

While his small fleet lay tossing in the Texel, a contest was going on
among the Dutch authorities. The States General and the Prince of Orange
were on one side, the Town Council and Admiralty of Amsterdam on the
other.

Skelton had delivered to the States General a list of the refugees whose
residence in the United Provinces caused uneasiness to his master. The
States General, anxious to grant every reasonable request which James
could make, sent copies of the list to the provincial authorities. The
provincial authorities sent copies to the municipal authorities. The
magistrates of all the towns were directed to take such measures
as might prevent the proscribed Whigs from molesting the English
government. In general those directions were obeyed. At Rotterdam
in particular, where the influence of William was all powerful, such
activity was shown as called forth warm acknowledgments from James. But
Amsterdam was the chief seat of the emigrants; and the governing body
of Amsterdam would see nothing, hear nothing, know of nothing. The
High Bailiff of the city, who was himself in daily communication with
Ferguson, reported to the Hague that he did not know where to find a
single one of the refugees; and with this excuse the federal government
was forced to be content. The truth was that the English exiles were
as well known at Amsterdam, and as much stared at in the streets, as if
they had been Chinese. 
%[355]
\footnote{ Skelton's letter is dated the 7-17th of May 1686. It
will be found, together with a letter of the Schout or High Bailiff
of Amsterdam, in a little volume published a few months later, and
entitled, "Histoire des Evenemens Tragiques d'Angleterre." The documents
inserted in that work are, as far as I have examined them, given exactly
from the Dutch archives, except that Skelton's French, which was not
the purest, is slightly corrected. See also Grey's Narrative.----
Goodenough, on his examination after the battle of Sedgemoor, said,
"The Schout of Amsterdam was a particular friend to this last design."
Lansdowne MS. 1152.---- It is not worth while to refute those writers
who represent the Prince of Orange as an accomplice in Monmouth's
enterprise. The circumstance on which they chiefly rely is that the
authorities of Amsterdam took no effectual steps for preventing the
expedition from sailing. This circumstance is in truth the strongest
proof that the expedition was not favoured by William. No person, not
profoundly ignorant of the institutions and politics of Holland, would
hold the Stadtholder answerable for the proceedings of the heads of the
Loevestein party.}


A few days later, Skelton received orders from his Court to request
that, in consequence of the dangers which threatened his master's
throne, the three Scotch regiments in the service of the United
Provinces might be sent to Great Britain without delay. He applied to
the Prince of Orange; and the prince undertook to manage the matter,
but predicted that Amsterdam would raise some difficulty. The prediction
proved correct. The deputies of Amsterdam refused to consent, and
succeeded in causing some delay. But the question was not one of those
on which, by the constitution of the republic, a single city could
prevent the wish of the majority from being carried into effect. The
influence of William prevailed; and the troops were embarked with great
expedition. 
%[356]
\footnote{ Avaux Neg. June 7-17, 8-18, 14-24, 1685, Letter of the
Prince of Orange to Lord Rochester, June 9, 1685.}


Skelton was at the same time exerting himself, not indeed very
judiciously or temperately, to stop the ships which the English refugees
had fitted out. He expostulated in warm terms with the Admiralty of
Amsterdam. The negligence of that board, he said, had already enabled
one band of rebels to invade Britain. For a second error of the same
kind there could be no excuse. He peremptorily demanded that a large
vessel, named the Helderenbergh, might be detained. It was pretended
that this vessel was bound for the Canaries. But in truth, she had been
freighted by Monmouth, carried twenty-six guns, and was loaded with arms
and ammunition. The Admiralty of Amsterdam replied that the liberty of
trade and navigation was not to be restrained for light reasons, and
that the Helderenbergh could not be stopped without an order from the
States General. Skelton, whose uniform practice seems to have been to
begin at the wrong end, now had recourse to the States General.
The States General gave the necessary orders. Then the Admiralty of
Amsterdam pretended that there was not a sufficient naval force in
the Texel to seize so large a ship as the Helderenbergh, and suffered
Monmouth to sail unmolested. 
%[357]
\footnote{ Van Citters, June 9-19, June 12-22,1685. The
correspondence of Skelton with the States General and with the Admiralty
of Amsterdam is in the archives at the Hague. Some pieces will be found
in the Evenemens Tragiques d'Angleterre. See also Burnet, i. 640.}


The weather was bad: the voyage was long; and several English men-of-war
were cruising in the channel. But Monmouth escaped both the sea and
the enemy. As he passed by the cliffs of Dorsetshire, it was thought
desirable to send a boat to the beach with one of the refugees named
Thomas Dare. This man, though of low mind and manners, had great
influence at Taunton. He was directed to hasten thither across the
country, and to apprise his friends that Monmouth would soon be on
English ground. 
%[358]
\footnote{ Wade's Confession in the Hardwicke Papers; Harl. MS.
6845.}


On the morning of the eleventh of June the Helderenbergh, accompanied by
two smaller vessels, appeared off the port of Lyme. That town is a
small knot of steep and narrow alleys, lying on a coast wild, rocky, and
beaten by a stormy sea. The place was then chiefly remarkable for a pier
which, in the days of the Plantagenets, had been constructed of stones,
unhewn and uncemented. This ancient work, known by the name of the Cob,
enclosed the only haven where, in a space of many miles, the fishermen
could take refuge from the tempests of the Channel.

The appearance of the three ships, foreign built and without colours,
perplexed the inhabitants of Lyme; and the uneasiness increased when it
was found that the Customhouse officers, who had gone on board according
to usage, did not return. The town's people repaired to the cliffs, and
gazed long and anxiously, but could find no solution of the mystery. At
length seven boats put off from the largest of the strange vessels, and
rowed to the shore. From these boats landed about eighty men, well armed
and appointed. Among them were Monmouth, Grey, Fletcher, Ferguson, Wade,
and Anthony Buyse, an officer who had been in the service of the Elector
of Brandenburg. 
%[359]
\footnote{ See Buyse's evidence against Monmouth and Fletcher in the
Collection of State Trials.}


Monmouth commanded silence, kneeled down on the shore, thanked God
for having preserved the friends of liberty and pure religion from the
perils of the sea, and implored the divine blessing on what was yet to
be done by land. He then drew his sword, and led his men over the cliffs
into the town.

As soon as it was known under what leader and for what purpose the
expedition came, the enthusiasm of the populace burst through all
restraints. The little town was in an uproar with men running to and
fro, and shouting "A Monmouth! a Monmouth! the Protestant religion!"
Meanwhile the ensign of the adventurers, a blue flag, was set up in the
marketplace. The military stores were deposited in the town hall; and
a Declaration setting forth the objects of the expedition was read from
the Cross. 
%[360]
\footnote{ Journals of the House of Commons, June 13, 1685; Harl.
MS. 6845; Lansdowne MS. 1152.}


This Declaration, the masterpiece of Ferguson's genius, was not a grave
manifesto such as ought to be put forth by a leader drawing the sword
for a great public cause, but a libel of the lowest class, both in
sentiment and language. 
%[361]
\footnote{ Burnet, i. 641, Goodenough's confession in the Lansdowne
MS. 1152. Copies of the Declaration, as originally printed, are very
rare; but there is one in the British Museum.}
 It contained undoubtedly many just charges
against the government. But these charges were set forth in the prolix
and inflated style of a bad pamphlet; and the paper contained other
charges of which the whole disgrace falls on those who made them. The
Duke of York, it was positively affirmed, had burned down London, had
strangled Godfrey, had cut the throat of Essex, and had poisoned the
late King. On account of those villanous and unnatural crimes, but
chiefly of that execrable fact, the late horrible and barbarous
parricide,--such was the copiousness and such the felicity of Ferguson's
diction,--James was declared a mortal and bloody enemy, a tyrant, a
murderer, and an usurper. No treaty should be made with him. The sword
should not be sheathed till he had been brought to condign punishment as
a traitor. The government should be settled on principles favourable
to liberty. All Protestant sects should be tolerated. The forfeited
charters should be restored. Parliament should be held annually, and
should no longer be prorogued or dissolved by royal caprice. The only
standing force should be the militia: the militia should be commanded
by the Sheriffs; and the Sheriffs should be chosen by the freeholders.
Finally Monmouth declared that he could prove himself to have been born
in lawful wedlock, and to be, by right of blood, King of England, but
that, for the present, he waived his claims, that he would leave them to
the judgment of a free Parliament, and that, in the meantime, he desired
to be considered only as the Captain General of the English Protestants,
who were in arms against tyranny and Popery.

Disgraceful as this manifesto was to those who put it forth, it was not
unskilfully framed for the purpose of stimulating the passions of the
vulgar. In the West the effect was great. The gentry and clergy of
that part of England were indeed, with few exceptions, Tories. But the
yeomen, the traders of the towns, the peasants, and the artisans were
generally animated by the old Roundhead spirit. Many of them were
Dissenters, and had been goaded by petty persecution into a temper fit
for desperate enterprise. The great mass of the population abhorred
Popery and adored Monmouth. He was no stranger to them. His progress
through Somersetshire and Devonshire in the summer of 1680 was still
fresh in the memory of all men.

He was on that occasion sumptuously entertained by Thomas Thynne at
Longleat Hall, then, and perhaps still, the most magnificent country
house in England. From Longleat to Exeter the hedges were lined with
shouting spectators. The roads were strewn with boughs and flowers. The
multitude, in their eagerness to see and touch their favourite, broke
down the palings of parks, and besieged the mansions where he was
feasted. When he reached Chard his escort consisted of five thousand
horsemen. At Exeter all Devonshire had been gathered together to welcome
him. One striking part of the show was a company of nine hundred young
men who, clad in a white uniform, marched before him into the city.

%[362]
\footnote{ Historical Account of the Life and magnanimous Actions of
the most illustrious Protestant Prince James, Duke of Monmouth, 1683.}
 The turn of fortune which had alienated the gentry from his cause
had produced no effect on the common people. To them he was still the
good Duke, the Protestant Duke, the rightful heir whom a vile conspiracy
kept out of his own. They came to his standard in crowds. All the
clerks whom he could employ were too few to take down the names of the
recruits. Before he had been twenty-four hours on English ground he was
at the head of fifteen hundred men. Dare arrived from Taunton with
forty horsemen of no very martial appearance, and brought encouraging
intelligence as to the state of public feeling in Somersetshire. As Yet
all seemed to promise well. 
%[363]
\footnote{ Wade's Confession, Hardwicke Papers; Axe Papers; Harl.
MS. 6845.}


But a force was collecting at Bridport to oppose the insurgents. On the
thirteenth of June the red regiment of Dorsetshire militia came pouring
into that town. The Somersetshire, or yellow regiment, of which Sir
William Portman, a Tory gentleman of great note, was Colonel, was
expected to arrive on the following day. 
%[364]
\footnote{ Harl. MS. 6845.}
 The Duke determined to
strike an immediate blow. A detachment of his troops was preparing to
march to Bridport when a disastrous event threw the whole camp into
confusion.

Fletcher of Saltoun had been appointed to command the cavalry under
Grey. Fletcher was ill mounted; and indeed there were few chargers in
the camp which had not been taken from the plough. When he was ordered
to Bridport, he thought that the exigency of the case warranted him in
borrowing, without asking permission, a fine horse belonging to Dare.
Dare resented this liberty, and assailed Fletcher with gross abuse.
Fletcher kept his temper better than any one who knew him expected. At
last Dare, presuming on the patience with which his insolence had been
endured, ventured to shake a switch at the high born and high spirited
Scot Fletcher's blood boiled. He drew a pistol and shot Dare dead.
Such sudden and violent revenge would not have been thought strange in
Scotland, where the law had always been weak, where he who did not right
himself by the strong hand was not likely to be righted at all, and
where, consequently, human life was held almost as cheap as in the worst
governed provinces of Italy. But the people of the southern part of the
island were not accustomed to see deadly weapons used and blood spilled
on account of a rude word or gesture, except in duel between gentlemen
with equal arms. There was a general cry for vengeance on the foreigner
who had murdered an Englishman. Monmouth could not resist the clamour.
Fletcher, who, when his first burst of rage had spent itself, was
overwhelmed with remorse and sorrow, took refuge on board of the
Helderenbergh, escaped to the Continent, and repaired to Hungary, where
he fought bravely against the common enemy of Christendom. 
%[365]
\footnote{ Buyse's evidence in the Collection of State Trials;
Burnet i 642; Ferguson's MS. quoted by Eachard.}


Situated as the insurgents were, the loss of a man of parts and energy
was not easily to be repaired. Early on the morning of the following
day, the fourteenth of June, Grey, accompanied by Wade, marched with
about five hundred men to attack Bridport. A confused and indecisive
action took place, such as was to be expected when two bands of
ploughmen, officered by country gentlemen and barristers, were opposed
to each other. For a time Monmouth's men drove the militia before them.
Then the militia made a stand, and Monmouth's men retreated in some
confusion. Grey and his cavalry never stopped till they were safe at
Lyme again: but Wade rallied the infantry and brought them off in good
order. 
%[366]
\footnote{ London Gazette, June 18, 1685; Wade's Confession,
Hardwicke Papers.}


There was a violent outcry against Grey; and some of the adventurers
pressed Monmouth to take a severe course. Monmouth, however, would not
listen to this advice. His lenity has been attributed by some writers
to his good nature, which undoubtedly often amounted to weakness. Others
have supposed that he was unwilling to deal harshly with the only peer
who served in his army. It is probable, however, that the Duke, who,
though not a general of the highest order, understood war very much
better than the preachers and lawyers who were always obtruding their
advice on him, made allowances which people altogether inexpert in
military affairs never thought of making. In justice to a man who has
had few defenders, it must be observed that the task, which, throughout
this campaign, was assigned to Grey, was one which, if he had been the
boldest and most skilful of soldiers, he would scarcely have performed
in such a manner as to gain credit. He was at the head of the cavalry.
It is notorious that a horse soldier requires a longer training than a
foot soldier, and that the war horse requires a longer training than his
rider. Something may be done with a raw infantry which has enthusiasm
and animal courage: but nothing can be more helpless than a raw cavalry,
consisting of yeomen and tradesmen mounted on cart horses and post
horses; and such was the cavalry which Grey commanded. The wonder is,
not that his men did not stand fire with resolution, not that they did
not use their weapons with vigour, but that they were able to keep their
seats.

Still recruits came in by hundreds. Arming and drilling went on all day.
Meantime the news of the insurrection had spread fast and wide. On
the evening on which the Duke landed, Gregory Alford, Mayor of Lyme, a
zealous Tory, and a bitter persecutor of Nonconformists, sent off
his servants to give the alarm to the gentry of Somersetshire and
Dorsetshire, and himself took horse for the West. Late at night he
stopped at Honiton, and thence despatched a few hurried lines to London
with the ill tidings. 
%[367]
\footnote{ Lords' Journals, June 13,1685.}
 He then pushed on to Exeter, where he found
Christopher Monk, Duke of Albemarle. This nobleman, the son and heir
of George Monk, the restorer of the Stuarts, was Lord Lieutenant of
Devonshire, and was then holding a muster of militia. Four thousand men
of the trainbands were actually assembled under his command. He seems to
have thought that, with this force, he should be able at once to crush
the rebellion. He therefore marched towards Lyme.

But when, on the afternoon of Monday the fifteenth of June, he reached
Axminster, he found the insurgents drawn up there to encounter him. They
presented a resolute front. Four field pieces were pointed against the
royal troops. The thick hedges, which on each side overhung the narrow
lanes, were lined with musketeers. Albemarle, however, was less alarmed
by the preparations of the enemy than by the spirit which appeared in
his own ranks. Such was Monmouth's popularity among the common people
of Devonshire that, if once the trainbands had caught sight of his well
known face and figure, they would have probably gone over to him in a
body.

Albemarle, therefore, though he had a great superiority of force,
thought it advisable to retreat. The retreat soon became a rout. The
whole country was strewn with the arms and uniforms which the fugitives
had thrown away; and, had Monmouth urged the pursuit with vigour, he
would probably have taken Exeter without a blow. But he was satisfied
with the advantage which he had gained, and thought it desirable that
his recruits should be better trained before they were employed in
any hazardous service. He therefore marched towards Taunton, where he
arrived on the eighteenth of June, exactly a week after his landing.

%[368]
\footnote{ Wade's Confession; Ferguson MS.; Axe Papers, Harl. MS.
6845, Oldmixon, 701, 702. Oldmixon, who was then a boy, lived very near
the scene of these events.}


The Court and the Parliament had been greatly moved by the news from
the West. At five in the morning of Saturday the thirteenth of June, the
King had received the letter which the Mayor of Lyme had despatched from
Honiton. The Privy Council was instantly called together. Orders were
given that the strength of every company of infantry and of every troop
of cavalry should be increased. Commissions were issued for the levying
of new regiments. Alford's communication was laid before the Lords; and
its substance was communicated to the Commons by a message. The Commons
examined the couriers who had arrived from the West, and instantly
ordered a bill to be brought in for attainting Monmouth of high treason.
Addresses were voted assuring the King that both his peers and his
people were determined to stand by him with life and fortune against
all his enemies. At the next meeting of the Houses they ordered the
Declaration of the rebels to be burned by the hangman, and passed the
bill of attainder through all its stages. That bill received the
royal assent on the same day; and a reward of five thousand pounds was
promised for the apprehension of Monmouth. 
%[369]
\footnote{ London Gazette, June 18, 1685; Lords' and Commons'
Journals, June 13 and 15; Dutch Despatch, 16-26.}


The fact that Monmouth was in arms against the government was so
notorious that the bill of attainder became a law with only a faint
show of opposition from one or two peers, and has seldom been severely
censured even by Whig historians. Yet, when we consider how important it
is that legislative and judicial functions should be kept distinct, how
important it is that common fame, however strong and general, should not
be received as a legal proof of guilt, how important it is to maintain
the rule that no man shall be condemned to death without an opportunity
of defending himself, and how easily and speedily breaches in great
principles, when once made, are widened, we shall probably be disposed
to think that the course taken by the Parliament was open to some
objection. Neither House had before it anything which even so corrupt
a judge as Jeffreys could have directed a jury to consider as proof of
Monmouth's crime. The messengers examined by the Commons were not on
oath, and might therefore have related mere fictions without incurring
the penalties of perjury. The Lords, who might have administered an
oath, appeared not to have examined any witness, and to have had no
evidence before them except the letter of the Mayor of Lyme, which, in
the eye of the law, was no evidence at all. Extreme danger, it is true,
justifies extreme remedies. But the Act of Attainder was a remedy which
could not operate till all danger was over, and which would become
superfluous at the very moment at which it ceased to be null. While
Monmouth was in arms it was impossible to execute him. If he should
be vanquished and taken, there would be no hazard and no difficulty in
trying him. It was afterwards remembered as a curious circumstance that,
among zealous Tories who went up with the bill from the House of
Commons to the bar of the Lords, was Sir John Fenwick, member for
Northumberland. This gentleman, a few years later, had occasion to
reconsider the whole subject, and then came to the conclusion that acts
of attainder are altogether unjustifiable. 
%[370]
\footnote{ Oldmixon is wrong in saying that Fenwick carried up the
bill. It was carried up, as appears from the Journals, by Lord Ancram.
See Delamere's Observations on the Attainder of the Late Duke of
Monmouth.}


The Parliament gave other proofs of loyalty in this hour of peril.
The Commons authorised the King to raise an extraordinary sum of four
hundred thousand pounds for his present necessities, and that he
might have no difficulty in finding the money, proceeded to devise new
imposts. The scheme of taxing houses lately built in the capital was
revived and strenuously supported by the country gentlemen. It was
resolved not only that such houses should be taxed, but that a bill
should be brought in prohibiting the laying of any new foundations
within the bills of mortality. The resolution, however, was not carried
into effect. Powerful men who had land in the suburbs and who hoped to
see new streets and squares rise on their estates, exerted all their
influence against the project. It was found that to adjust the details
would be a work of time; and the King's wants were so pressing that he
thought it necessary to quicken the movements of the House by a gentle
exhortation to speed. The plan of taxing buildings was therefore
relinquished; and new duties were imposed for a term of five years on
foreign silks, linens, and spirits. 
%[371]
\footnote{ Commons' Journals of June 17, 18, and 19, 1685; Reresby's
Memoirs.}


The Tories of the Lower House proceeded to introduce what they called
a bill for the preservation of the King's person and government.
They proposed that it should be high treason to say that Monmouth was
legitimate, to utter any words tending to bring the person or government
of the sovereign into hatred or contempt, or to make any motion
in Parliament for changing the order of succession. Some of these
provisions excited general disgust and alarm. The Whigs, few and weak
as they were, attempted to rally, and found themselves reinforced by a
considerable number of moderate and sensible Cavaliers. Words, it was
said, may easily be misunderstood by a dull man. They may be easily
misconstrued by a knave. What was spoken metaphorically may be
apprehended literally. What was spoken ludicrously may be apprehended
seriously. A particle, a tense, a mood, an emphasis, may make the whole
difference between guilt and innocence. The Saviour of mankind himself,
in whose blameless life malice could find no acts to impeach, had been
called in question for words spoken. False witnesses had suppressed
a syllable which would have made it clear that those words were
figurative, and had thus furnished the Sanhedrim with a pretext under
which the foulest of all judicial murders had been perpetrated. With
such an example on record, who could affirm that, if mere talk were
made a substantive treason, the most loyal subject would be safe? These
arguments produced so great an effect that in the committee amendments
were introduced which greatly mitigated the severity of the bill. But
the clause which made it high treason in a member of Parliament to
propose the exclusion of a prince of the blood seems to have raised no
debate, and was retained. That clause was indeed altogether unimportant,
except as a proof of the ignorance and inexperience of the hotheaded
Royalists who thronged the House of Commons. Had they learned the first
rudiments of legislation, they would have known that the enactment
to which they attached so much value would be superfluous while the
Parliament was disposed to maintain the order of succession, and would
be repealed as soon as there was a Parliament bent on changing the order
of succession. 
%[372]
\footnote{ Commons' Journals, June 19, 29, 1685; Lord Lonsdale's
Memoirs, 8, 9, Burnet, i. 639. The bill, as amended by the committee,
will be found in Mr. Fox's historical work. Appendix iii. If Burnet's
account be correct, the offences which, by the amended bill, were made
punishable only with civil incapacities were, by the original bill, made
capital.}


The bill, as amended, was passed and carried up to the Lords, but
did not become law. The King had obtained from the Parliament all the
pecuniary assistance that he could expect; and he conceived that, while
rebellion was actually raging, the loyal nobility and gentry would be
of more use in their counties than at Westminster. He therefore hurried
their deliberations to a close, and, on the second of July, dismissed
them. On the same day the royal assent was given to a law reviving that
censorship of the press which had terminated in 1679. This object was
affected by a few words at the end of a miscellaneous statute which
continued several expiring acts. The courtiers did not think that they
had gained a triumph. The Whigs did not utter a murmur. Neither in the
Lords nor in the Commons was there any division, or even, as far as
can now be learned, any debate on a question which would, in our age,
convulse the whole frame of society. In truth, the change was slight and
almost imperceptible; for, since the detection of the Rye House plot,
the liberty of unlicensed printing had existed only in name. During many
months scarcely one Whig pamphlet had been published except by stealth;
and by stealth such pamphlets might be published still. 
%[373]
\footnote{ 1 Jac. II. c. 7; Lords' Journals, July 2, 1685.}


The Houses then rose. They were not prorogued, but only adjourned,
in order that, when they should reassemble, they might take up their
business in the exact state in which they had left it. 
%[374]
\footnote{ Lords' and Commons' Journals, July 2, 1685.}


While the Parliament was devising sharp laws against Monmouth and his
partisans, he found at Taunton a reception which might well encourage
him to hope that his enterprise would have a prosperous issue. Taunton,
like most other towns in the south of England, was, in that age, more
important than at present. Those towns have not indeed declined. On the
contrary, they are, with very few exceptions, larger and richer, better
built and better peopled, than in the seventeenth century. But, though
they have positively advanced, they have relatively gone back. They have
been far outstripped in wealth and population by the great manufacturing
and commercial cities of the north, cities which, in the time of the
Stuarts, were but beginning to be known as seats of industry. When
Monmouth marched into Taunton it was an eminently prosperous place.
Its markets were plentifully supplied. It was a celebrated seat of
the woollen manufacture. The people boasted that they lived in a land
flowing with milk and honey. Nor was this language held only by partial
natives; for every stranger who climbed the graceful tower of St. Mary
Magdalene owned that he saw beneath him the most fertile of English
valleys. It was a country rich with orchards and green pastures, among
which were scattered, in gay abundance, manor houses, cottages, and
village spires. The townsmen had long leaned towards Presbyterian
divinity and Whig politics. In the great civil war Taunton had, through
all vicissitudes, adhered to the Parliament, had been twice closely
besieged by Goring, and had been twice defended with heroic valour by
Robert Blake, afterwards the renowned Admiral of the Commonwealth.
Whole streets had been burned down by the mortars and grenades of
the Cavaliers. Food had been so scarce that the resolute governor had
announced his intention of putting the garrison on rations of horse
flesh. But the spirit of the town had never been subdued either by fire
or by hunger. 
%[375]
\footnote{ Savage's edition of Toulmin's History of Taunton.}


The Restoration had produced no effect on the temper of the Taunton men.
They had still continued to celebrate the anniversary of the happy day
on which the siege laid to their town by the royal army had been raised;
and their stubborn attachment to the old cause had excited so much fear
and resentment at Whitehall that, by a royal order, their moat had
been filled up, and their wall demolished to the foundation. 
%[376]
\footnote{ Sprat's true Account; Toulmin's History of Taunton.}
 The
puritanical spirit had been kept up to the height among them by the
precepts and example of one of the most celebrated of the dissenting
clergy, Joseph Alleine. Alleine was the author of a tract, entitled, An
Alarm to the Unconverted, which is still popular both in England and
in America. From the gaol to which he was consigned by the victorious
Cavaliers, he addressed to his loving friends at Taunton many epistles
breathing the spirit of a truly heroic piety. His frame soon sank under
the effects of study, toil, and persecution: but his memory was long
cherished with exceeding love and reverence by those whom he had
exhorted and catechised. 
%[377]
\footnote{ Life and Death of Joseph Alleine, 1672; Nonconformists'
Memorial.}


The children of the men who, forty years before, had manned the ramparts
of Taunton against the Royalists, now welcomed Monmouth with transports
of joy and affection. Every door and window was adorned with wreaths
of flowers. No man appeared in the streets without wearing in his hat
a green bough, the badge of the popular cause. Damsels of the best
families in the town wove colours for the insurgents. One flag in
particular was embroidered gorgeously with emblems of royal dignity, and
was offered to Monmouth by a train of young girls. He received the gift
with the winning courtesy which distinguished him. The lady who headed
the procession presented him also with a small Bible of great price.
He took it with a show of reverence. "I come," he said, "to defend the
truths contained in this book, and to seal them, if it must be so, with
my blood." 
%[378]
\footnote{ Harl. MS. 7006; Oldmixon. 702; Eachard, iii. 763.}


But while Monmouth enjoyed the applause of the multitude, he could not
but perceive, with concern and apprehension, that the higher classes
were, with scarcely an exception, hostile to his undertaking, and that
no rising had taken place except in the counties where he had himself
appeared. He had been assured by agents, who professed to have derived
their information from Wildman, that the whole Whig aristocracy was
eager to take arms. Nevertheless more than a week had now elapsed since
the blue standard had been set up at Lyme. Day labourers, small farmers,
shopkeepers, apprentices, dissenting preachers, had flocked to the rebel
camp: but not a single peer, baronet, or knight, not a single member
of the House of Commons, and scarcely any esquire of sufficient note to
have ever been in the commission of the peace, had joined the invaders.
Ferguson, who, ever since the death of Charles, had been Monmouth's evil
angel, had a suggestion ready. The Duke had put himself into a false
position by declining the royal title. Had he declared himself sovereign
of England, his cause would have worn a show of legality. At present it
was impossible to reconcile his Declaration with the principles of
the constitution. It was clear that either Monmouth or his uncle
was rightful King. Monmouth did not venture to pronounce himself the
rightful King, and yet denied that his uncle was so. Those who fought
for James fought for the only person who ventured to claim the throne,
and were therefore clearly in their duty, according to the laws of the
realm. Those who fought for Monmouth fought for some unknown polity,
which was to be set up by a convention not yet in existence. None could
wonder that men of high rank and ample fortune stood aloof from
an enterprise which threatened with destruction that system in the
permanence of which they were deeply interested. If the Duke would
assert his legitimacy and assume the crown, he would at once remove this
objection. The question would cease to be a question between the old
constitution and a new constitution. It would be merely a question of
hereditary right between two princes.

On such grounds as these Ferguson, almost immediately after the landing,
had earnestly pressed the Duke to proclaim himself King; and Grey had
seconded Ferguson. Monmouth had been very willing to take this advice;
but Wade and other republicans had been refractory; and their chief,
with his usual pliability, had yielded to their arguments. At
Taunton the subject was revived. Monmouth talked in private with the
dissentients, assured them that he saw no other way of obtaining the
support of any portion of the aristocracy, and succeeded in extorting
their reluctant consent. On the morning of the twentieth of June he was
proclaimed in the market place of Taunton. His followers repeated his
new title with affectionate delight. But, as some confusion might have
arisen if he had been called King James the Second, they commonly used
the strange appellation of King Monmouth: and by this name their unhappy
favourite was often mentioned in the western counties, within the memory
of persons still living. 
%[379]
\footnote{ Wade's Confession; Goodenough's Confession, Harl. MS.
1152, Oldmixon, 702. Ferguson's denial is quite undeserving of credit. A
copy of the proclamation is in the Harl. MS. 7006.}


Within twenty-four hours after he had assumed the regal title, he put
forth several proclamations headed with his sign manual. By one of these
he set a price on the head of his rival. Another declared the Parliament
then sitting at Westminster an unlawful assembly, and commanded the
members to disperse. A third forbade the people to pay taxes to the
usurper. A fourth pronounced Albemarle a traitor. 
%[380]
\footnote{ Copies of the last three proclamations are in the British
Museum; Harl. MS. 7006. The first I have never seen; but it is mentioned
by Wado.}


Albemarle transmitted these proclamations to London merely as specimens
of folly and impertinence. They produced no effect, except wonder and
contempt; nor had Monmouth any reason to think that the assumption of
royalty had improved his position. Only a week had elapsed since he
had solemnly bound himself not to take the crown till a free Parliament
should have acknowledged his rights. By breaking that engagement he had
incurred the imputation of levity, if not of perfidy. The class which he
had hoped to conciliate still stood aloof. The reasons which prevented
the great Whig lords and gentlemen from recognising him as their King
were at least as strong as those which had prevented them from rallying
round him as their Captain General. They disliked indeed the person, the
religion, and the politics of James. But James was no longer young. His
eldest daughter was justly popular. She was attached to the reformed
faith. She was married to a prince who was the hereditary chief of
the Protestants of the Continent, to a prince who had been bred in a
republic, and whose sentiments were supposed to be such as became a
constitutional King. Was it wise to incur the horrors of civil war, for
the mere chance of being able to effect immediately what nature
would, without bloodshed, without any violation of law, effect, in all
probability, before many years should have expired? Perhaps there might
be reasons for pulling down James. But what reason could be given for
setting up Monmouth? To exclude a prince from the throne on account of
unfitness was a course agreeable to Whig principles. But on no principle
could it be proper to exclude rightful heirs, who were admitted to
be, not only blameless, but eminently qualified for the highest public
trust. That Monmouth was legitimate, nay, that he thought himself
legitimate, intelligent men could not believe. He was therefore not
merely an usurper, but an usurper of the worst sort, an impostor. If
he made out any semblance of a case, he could do so only by means of
forgery and perjury. All honest and sensible persons were unwilling to
see a fraud which, if practiced to obtain an estate, would have been
punished with the scourge and the pillory, rewarded with the English
crown. To the old nobility of the realm it seemed insupportable that
the bastard of Lucy Walters should be set up high above the lawful
descendants of the Fitzalans and De Veres. Those who were capable of
looking forward must have seen that, if Monmouth should succeed in
overpowering the existing government, there would still remain a war
between him and the House of Orange, a war which might last longer
and produce more misery than the war of the Roses, a war which might
probably break up the Protestants of Europe into hostile parties, might
arm England and Holland against each other, and might make both those
countries an easy prey to France. The opinion, therefore, of almost all
the leading Whigs seems to have been that Monmouth's enterprise could
not fail to end in some great disaster to the nation, but that, on the
whole, his defeat would be a less disaster than his victory.

It was not only by the inaction of the Whig aristocracy that the
invaders were disappointed. The wealth and power of London had sufficed
in the preceding generation, and might again suffice, to turn the scale
in a civil conflict. The Londoners had formerly given many proofs of
their hatred of Popery and of their affection for the Protestant Duke.
He had too readily believed that, as soon as he landed, there would be
a rising in the capital. But, though advices came down to him that many
thousands of the citizens had been enrolled as volunteers for the good
cause, nothing was done. The plain truth was that the agitators who
had urged him to invade England, who had promised to rise on the first
signal, and who had perhaps imagined, while the danger was remote, that
they should have the courage to keep their promise, lost heart when the
critical time drew near. Wildman's fright was such that he seemed to
have lost his understanding. The craven Danvers at first excused his
inaction by saying that he would not take up arms till Monmouth was
proclaimed King, and when Monmouth had been proclaimed King, turned
round and declared that good republicans were absolved from all
engagements to a leader who had so shamefully broken faith. In every age
the vilest specimens of human nature are to be found among demagogues.

%[381]
\footnote{ Grey's Narrative; Ferguson's MS., Eachard, iii. 754.}


On the day following that on which Monmouth had assumed the regal
title he marched from Taunton to Bridgewater. His own spirits, it was
remarked, were not high. The acclamations of the devoted thousands who
surrounded him wherever he turned could not dispel the gloom which
sate on his brow. Those who had seen him during his progress through
Somersetshire five years before could not now observe without pity the
traces of distress and anxiety on those soft and pleasing features which
had won so many hearts. 
%[382]
\footnote{ Persecution Exposed, by John Whiting.}


Ferguson was in a very different temper. With this man's knavery was
strangely mingled an eccentric vanity which resembled madness. The
thought that he had raised a rebellion and bestowed a crown had turned
his head. He swaggered about, brandishing his naked sword, and crying to
the crowd of spectators who had assembled to see the army march out of
Taunton, "Look at me! You have heard of me. I am Ferguson, the famous
Ferguson, the Ferguson for whose head so many hundred pounds have been
offered." And this man, at once unprincipled and brainsick, had in his
keeping the understanding and the conscience of the unhappy Monmouth.

%[383]
\footnote{ Harl. MS. 6845.}


Bridgewater was one of the few towns which still had some Whig
magistrates. The Mayor and Aldermen came in their robes to welcome
the Duke, walked before him in procession to the high cross, and there
proclaimed him King. His troops found excellent quarters, and were
furnished with necessaries at little or no cost by the people of the
town and neighbourhood. He took up his residence in the Castle, a
building which had been honoured by several royal visits. In the Castle
Field his army was encamped. It now consisted of about six thousand men,
and might easily have been increased to double the number, but for the
want of arms. The Duke had brought with him from the Continent but
a scanty supply of pikes and muskets. Many of his followers had,
therefore, no other weapons than such as could be fashioned out of
the tools which they had used in husbandry or mining. Of these rude
implements of war the most formidable was made by fastening the blade
of a scythe erect on a strong pole. 
%[384]
\footnote{ One of these weapons may still be seen in the tower.}
 The tithing men of the country
round Taunton and Bridgewater received orders to search everywhere
for scythes and to bring all that could be found to the camp. It was
impossible, however, even with the help of these contrivances, to supply
the demand; and great numbers who were desirous to enlist were sent
away. 
%[385]
\footnote{ Grey's Narrative; Paschall's Narrative in the Appendix to
Heywood's Vindication.}


The foot were divided into six regiments. Many of the men had been in
the militia, and still wore their uniforms, red and yellow. The cavalry
were about a thousand in number; but most of them had only large colts,
such as were then bred in great herds on the marshes of Somersetshire
for the purpose of supplying London with coach horses and cart horses.
These animals were so far from being fit for any military purpose that
they had not yet learned to obey the bridle, and became ungovernable as
soon as they heard a gun fired or a drum beaten. A small body guard of
forty young men, well armed, and mounted at their own charge, attended
Monmouth. The people of Bridgewater, who were enriched by a thriving
coast trade, furnished him with a small sum of money. 
%[386]
\footnote{ Oldmixon, 702.}


All this time the forces of the government were fast assembling. On the
west of the rebel army, Albemarle still kept together a large body
of Devonshire militia. On the east, the trainbands of Wiltshire had
mustered under the command of Thomas Herbert, Earl of Pembroke. On the
north east, Henry Somerset, Duke of Beaufort, was in arms. The power of
Beaufort bore some faint resemblance to that of the great barons of the
fifteenth century. He was President of Wales and Lord Lieutenant of four
English counties. His official tours through the extensive region in
which he represented the majesty of the throne were scarcely inferior in
pomp to royal progresses. His household at Badminton was regulated after
the fashion of an earlier generation. The land to a great extent
round his pleasure grounds was in his own hands; and the labourers who
cultivated it formed part of his family. Nine tables were every day
spread under his roof for two hundred persons. A crowd of gentlemen and
pages were under the orders of the steward. A whole troop of cavalry
obeyed the master of the horse. The fame of the kitchen, the cellar, the
kennel, and the stables was spread over all England. The gentry, many
miles round, were proud of the magnificence of their great neighbour,
and were at the same time charmed by his affability and good nature. He
was a zealous Cavalier of the old school. At this crisis, therefore,
he used his whole influence and authority in support of the crown, and
occupied Bristol with the trainbands of Gloucestershire, who seem to
have been better disciplined than most other troops of that description.

%[387]
\footnote{ North's Life of Guildford, 132. Accounts of Beaufort's
progress through Wales and the neighbouring counties are in the London
Gazettes of July 1684. Letter of Beaufort to Clarendon, June 19, 1685.}


In the counties more remote from Somersetshire the supporters of the
throne were on the alert. The militia of Sussex began to march westward,
under the command of Richard, Lord Lumley, who, though he had lately
been converted from the Roman Catholic religion, was still firm in his
allegiance to a Roman Catholic King. James Bertie, Earl of Abingdon,
called out the array of Oxfordshire. John Fell, Bishop of Oxford,
who was also Dean of Christchurch, summoned the undergraduates of his
University to take arms for the crown. The gownsmen crowded to give in
their names. Christchurch alone furnished near a hundred pikemen and
musketeers. Young noblemen and gentlemen commoners acted as officers;
and the eldest son of the Lord Lieutenant was Colonel. 
%[388]
\footnote{ Bishop Fell to Clarendon, June 20; Abingdon to Clarendon,
June 20, 25, 26, 1685; Lansdowne MS. 846.}


But it was chiefly on the regular troops that the King relied. Churchill
had been sent westward with the Blues; and Feversham was following with
all the forces that could be spared from the neighbourhood of London.
A courier had started for Holland with a letter directing Skelton
instantly to request that the three English regiments in the Dutch
service might be sent to the Thames. When the request was made,
the party hostile to the House of Orange, headed by the deputies of
Amsterdam, again tried to cause delay. But the energy of William, who
had almost as much at stake as James, and who saw Monmouth's progress
with serious uneasiness, bore down opposition, and in a few days the
troops sailed. 
%[389]
\footnote{ Avaux, July 5-15, 6-16, 1685.}
 The three Scotch regiments were already in England.
They had arrived at Gravesend in excellent condition, and James had
reviewed them on Blackheath. He repeatedly declared to the Dutch
Ambassador that he had never in his life seen finer or better
disciplined soldiers, and expressed the warmest gratitude to the Prince
of Orange and the States for so valuable and seasonable a reinforcement
This satisfaction, however, was not unmixed. Excellently as the men went
through their drill, they were not untainted with Dutch politics and
Dutch divinity. One of them was shot and another flogged for drinking
the Duke of Monmouth's health. It was therefore not thought advisable to
place them in the post of danger. They were kept in the neighbourhood of
London till the end of the campaign. But their arrival enabled the King
to send to the West some infantry which would otherwise have been wanted
in the capital. 
%[390]
\footnote{ Van Citters, June 30,/July 10, July 3-13, 21-31,1685;
Avaux Neg. July 5-15, London Gazette, July 6.}


While the government was thus preparing for a conflict with the rebels
in the field, precautions of a different kind were not neglected. In
London alone two hundred of those persons who were thought most likely
to be at the head of a Whig movement were arrested. Among the prisoners
were some merchants of great note. Every man who was obnoxious to the
Court went in fear. A general gloom overhung the capital. Business
languished on the Exchange; and the theatres were so generally deserted
that a new opera, written by Dryden, and set off by decorations of
unprecedented magnificence, was withdrawn, because the receipts would
not cover the expenses of the performance. 
%[391]
\footnote{ Barillon, July 6-16, 1685; Scott's preface to Albion and
Albanius.}
 The magistrates and
clergy were everywhere active, the Dissenters were everywhere closely
observed. In Cheshire and Shropshire a fierce persecution raged; in
Northamptonshire arrests were numerous; and the gaol of Oxford was
crowded with prisoners. No Puritan divine, however moderate his
opinions, however guarded his conduct, could feel any confidence that he
should not be torn from his family and flung into a dungeon. 
%[392]
\footnote{ Abingdon to Clarendon, June 29,1685; Life of Philip
Henry, by Bates.}


Meanwhile Monmouth advanced from Bridgewater harassed through the whole
march by Churchill, who appears to have done all that, with a handful
of men, it was possible for a brave and skilful officer to effect. The
rebel army, much annoyed, both by the enemy and by a heavy fall of rain,
halted in the evening of the twenty-second of June at Glastonbury. The
houses of the little town did not afford shelter for so large a force.
Some of the troops were therefore quartered in the churches, and others
lighted their fires among the venerable ruins of the Abbey, once the
wealthiest religious house in our island. From Glastonbury the Duke
marched to Wells, and from Wells to Shepton Mallet. 
%[393]
\footnote{ London Gazette, June 22, and June 25,1685; Wade's
Confession; Oldmixon, 703; Harl. MS. 6845.}


Hitherto he seems to have wandered from place to place with no other
object than that of collecting troops. It was now necessary for him to
form some plan of military operations. His first scheme was to seize
Bristol. Many of the chief inhabitants of that important place were
Whigs. One of the ramifications of the Whig plot had extended thither.
The garrison consisted only of the Gloucestershire trainbands. If
Beaufort and his rustic followers could be overpowered before the
regular troops arrived, the rebels would at once find themselves
possessed of ample pecuniary resources; the credit of Monmouth's
arms would be raised; and his friends throughout the kingdom would be
encouraged to declare themselves. Bristol had fortifications which, on
the north of the Avon towards Gloucestershire, were weak, but on
the south towards Somersetshire were much stronger. It was therefore
determined that the attack should be made on the Gloucestershire side.
But for this purpose it was necessary to take a circuitous route, and
to cross the Avon at Keynsham. The bridge at Keynsham had been partly
demolished by the militia, and was at present impassable. A detachment
was therefore sent forward to make the necessary repairs. The other
troops followed more slowly, and on the evening of the twenty-fourth
of June halted for repose at Pensford. At Pensford they were only five
miles from the Somersetshire side of Bristol; but the Gloucestershire
side, which could be reached only by going round through Keynsham, was
distant a long day's march. 
%[394]
\footnote{ Wade's Confession.}


That night was one of great tumult and expectation in Bristol. The
partisans of Monmouth knew that he was almost within sight of their
city, and imagined that he would be among them before daybreak. About
an hour after sunset a merchantman lying at the quay took fire. Such an
occurrence, in a port crowded with shipping, could not but excite great
alarm. The whole river was in commotion. The streets were crowded.
Seditious cries were heard amidst the darkness and confusion. It was
afterwards asserted, both by Whigs and by Tories, that the fire had
been kindled by the friends of Monmouth, in the hope that the trainbands
would be busied in preventing the conflagration from spreading, and that
in the meantime the rebel army would make a bold push, and would enter
the city on the Somersetshire side. If such was the design of the
incendiaries, it completely failed. Beaufort, instead of sending his men
to the quay, kept them all night drawn up under arms round the beautiful
church of Saint Mary Redcliff, on the south of the Avon. He would see
Bristol burnt down, he said, nay, he would burn it down himself, rather
than that it should be occupied by traitors. He was able, with the help
of some regular cavalry which had joined him from Chippenham a few hours
before, to prevent an insurrection. It might perhaps have been beyond
his power at once to overawe the malecontents within the walls and to
repel an attack from without: but no such attack was made. The fire,
which caused so much commotion at Bristol, was distinctly seen at
Pensford. Monmouth, however, did not think it expedient to change his
plan. He remained quiet till sunrise, and then marched to Keynsham.
There he found the bridge repaired. He determined to let his army rest
during the afternoon, and, as soon as night came, to proceed to Bristol.

%[395]
\footnote{ Wade's Confession; Oldmixon, 703; Harl. MS. 6845; Charge
of Jeffreys to the grand jury of Bristol, Sept. 21, 1685.}


But it was too late. The King's forces were now near at hand. Colonel
Oglethorpe, at the head of about a hundred men of the Life Guards,
dashed into Keynsham, scattered two troops of rebel horse which ventured
to oppose him, and retired after inflicting much injury and suffering
little. In these circumstances it was thought necessary to relinquish
the design on Bristol. 
%[396]
\footnote{ London Gazette, June 29, 1685; Wade's Confession.}


But what was to be done? Several schemes were proposed and discussed. It
was suggested that Monmouth might hasten to Gloucester, might cross
the Severn there, might break down the bridge behind him, and, with his
right flank protected by the river, might march through Worcestershire
into Shropshire and Cheshire. He had formerly made a progress through
those counties, and had been received there with as much enthusiasm as
in Somersetshire and Devonshire. His presence might revive the zeal of
his old friends; and his army might in a few days be swollen to double
its present numbers.

On full consideration, however, it appeared that this plan, though
specious, was impracticable. The rebels were ill shod for such work as
they had lately undergone, and were exhausted by toiling, day after day,
through deep mud under heavy rain. Harassed and impeded as they would
be at every stage by the enemy's cavalry, they could not hope to reach
Gloucester without being overtaken by the main body of the royal troops,
and forced to a general action under every disadvantage.

Then it was proposed to enter Wiltshire. Persons who professed to know
that county well assured the Duke that he would be joined there by such
strong reinforcements as would make it safe for him to give battle.

%[397]
\footnote{ Wade's Confession.}


He took this advice, and turned towards Wiltshire. He first summoned
Bath. But Bath was strongly garrisoned for the King; and Feversham was
fast approaching. The rebels, therefore made no attempt on the walls,
but hastened to Philip's Norton, where they halted on the evening of the
twenty-sixth of June.

Feversham followed them thither. Early on the morning of the
twenty-seventh they were alarmed by tidings that he was close at hand.
They got into order, and lined the hedges leading to the town.

The advanced guard of the royal army soon appeared. It consisted of
about five hundred men, commanded by the Duke of Grafton, a youth of
bold spirit and rough manners, who was probably eager to show that he
had no share in the disloyal schemes of his half brother. Grafton soon
found himself in a deep lane with fences on both sides of him, from
which a galling fire of musketry was kept up. Still he pushed boldly
on till he came to the entrance of Philip's Norton. There his way was
crossed by a barricade, from which a third fire met him full in front.
His men now lost heart, and made the best of their way back. Before
they got out of the lane more than a hundred of them had been killed or
wounded. Grafton's retreat was intercepted by some of the rebel cavalry:
but he cut his way gallantly through them, and came off safe. 
%[398]
\footnote{ London Gazette, July 2,1685; Barillon, July 6-16; Wade's
Confession.}


The advanced guard, thus repulsed, fell back on the main body of the
royal forces. The two armies were now face to face; and a few shots were
exchanged that did little or no execution. Neither side was impatient to
come to action. Feversham did not wish to fight till his artillery came
up, and fell back to Bradford. Monmouth, as soon as the night closed
in, quitted his position, marched southward, and by daybreak arrived at
Frome, where he hoped to find reinforcements.

Frome was as zealous in his cause as either Taunton or Bridgewater,
but could do nothing to serve him. There had been a rising a few days
before; and Monmouth's declaration had been posted up in the market
place. But the news of this movement had been carried to the Earl of
Pembroke, who lay at no great distance with the Wiltshire militia. He
had instantly marched to Frome, had routed a mob of rustics who, with
scythes and pitchforks, attempted to oppose him, had entered the town
and had disarmed the inhabitants. No weapons, therefore, were left
there; nor was Monmouth able to furnish any. 
%[399]
\footnote{ London Gazette, June 29,1685; Van Citters, June 30,/July
10,}


The rebel army was in evil case. The march of the preceding night had
been wearisome. The rain had fallen in torrents; and the roads had
become mere quagmires. Nothing was heard of the promised succours from
Wiltshire. One messenger brought news that Argyle's forces had been
dispersed in Scotland. Another reported that Feversham, having been
joined by his artillery, was about to advance. Monmouth understood war
too well not to know that his followers, with all their courage and
all their zeal, were no match for regular soldiers. He had till lately
flattered himself with the hope that some of those regiments which he
had formerly commanded would pass over to his standard: but that hope he
was now compelled to relinquish. His heart failed him. He could scarcely
muster firmness enough to give orders. In his misery he complained
bitterly of the evil counsellors who had induced him to quit his happy
retreat in Brabant. Against Wildman in particular he broke forth into
violent imprecations. 
%[400]
\footnote{ Harl. MS. 6845; Wade's Confession.}
 And now an ignominious thought rose in his
weak and agitated mind. He would leave to the mercy of the government
the thousands who had, at his call and for his sake, abandoned their
quiet fields and dwellings. He would steal away with his chief officers,
would gain some seaport before his flight was suspected, would escape to
the Continent, and would forget his ambition and his shame in the arms
of Lady Wentworth. He seriously discussed this scheme with his leading
advisers. Some of them, trembling for their necks, listened to it with
approbation; but Grey, who, by the admission of his detractors, was
intrepid everywhere except where swords were clashing and guns going
off around him, opposed the dastardly proposition with great ardour,
and implored the Duke to face every danger rather than requite with
ingratitude and treachery the devoted attachment of the Western
peasantry. 
%[401]
\footnote{ Wade's Confession; Eachard, iii. 766.}


The scheme of flight was abandoned: but it was not now easy to form any
plan for a campaign. To advance towards London would have been madness;
for the road lay right across Salisbury Plain; and on that vast open
space regular troops, and above all regular cavalry, would have acted
with every advantage against undisciplined men. At this juncture a
report reached the camp that the rustics of the marshes near Axbridge
had risen in defence of the Protestant religion, had armed themselves
with flails, bludgeons, and pitchforks, and were assembling by thousands
at Bridgewater. Monmouth determined to return thither, and to strengthen
himself with these new allies. 
%[402]
\footnote{ Wade's Confession.}


The rebels accordingly proceeded to Wells, and arrived there in no
amiable temper. They were, with few exceptions, hostile to Prelacy; and
they showed their hostility in a way very little to their honour. They
not only tore the lead from the roof of the magnificent Cathedral to
make bullets, an act for which they might fairly plead the necessities
of war, but wantonly defaced the ornaments of the building. Grey with
difficulty preserved the altar from the insults of some ruffians who
wished to carouse round it, by taking his stand before it with his sword
drawn. 
%[403]
\footnote{ London Gazette, July 6, 1685; Van Citters, July 3-13,
Oldmixon, 703.}


On Thursday, the second of July, Monmouth again entered Bridgewater,
In circumstances far less cheering than those in which he had marched
thence ten days before. The reinforcement which he found there was
inconsiderable. The royal army was close upon him. At one moment he
thought of fortifying the town; and hundreds of labourers were summoned
to dig trenches and throw up mounds. Then his mind recurred to the plan
of marching into Cheshire, a plan which he had rejected as impracticable
when he was at Keynsham, and which assuredly was not more practicable
now that he was at Bridgewater. 
%[404]
\footnote{ Wade's Confession.}


While he was thus wavering between projects equally hopeless, the King's
forces came in sight. They consisted of about two thousand five hundred
regular troops, and of about fifteen hundred of the Wiltshire militia.
Early on the morning of Sunday, the fifth of July, they left Somerton,
and pitched their tents that day about three miles from Bridgewater, on
the plain of Sedgemoor.

Dr. Peter Mew, Bishop of Winchester, accompanied them. This prelate had
in his youth borne arms for Charles the First against the Parliament.
Neither his years nor his profession had wholly extinguished his martial
ardour; and he probably thought that the appearance of a father of the
Protestant Church in the King's camp might confirm the loyalty of some
honest men who were wavering between their horror of Popery and their
horror of rebellion.

The steeple of the parish church of Bridgewater is said to be the
loftiest of Somersetshire, and commands a wide view over the surrounding
country. Monmouth, accompanied by some of his officers, went up to
the top of the square tower from which the spire ascends, and observed
through a telescope the position of the enemy. Beneath him lay a flat
expanse, now rich with cornfields and apple trees, but then, as its name
imports, for the most part a dreary morass. When the rains were heavy,
and the Parret and its tributary streams rose above their banks, this
tract was often flooded. It was indeed anciently part of that great
swamp which is renowned in our early chronicles as having arrested the
progress of two successive races of invaders, which long protected
the Celts against the aggressions of the kings of Wessex, and which
sheltered Alfred from the pursuit of the Danes. In those remote times
this region could be traversed only in boats. It was a vast pool,
wherein were scattered many islets of shifting and treacherous soil,
overhung with rank jungle, and swarming with deer and wild swine.
Even in the days of the Tudors, the traveller whose journey lay from
Ilchester to Bridgewater was forced to make a circuit of several miles
in order to avoid the waters. When Monmouth looked upon Sedgemoor, it
had been partially reclaimed by art, and was intersected by many deep
and wide trenches which, in that country, are called rhines. In the
midst of the moor rose, clustering round the towers of churches, a
few villages of which the names seem to indicate that they once were
surrounded by waves. In one of these villages, called Weston Zoyland,
the royal cavalry lay; and Feversham had fixed his headquarters there.
Many persons still living have seen the daughter of the servant girl who
waited on him that day at table; and a large dish of Persian ware, which
was set before him, is still carefully preserved in the neighbourhood.
It is to be observed that the population of Somersetshire does not, like
that of the manufacturing districts, consist of emigrants from distant
places. It is by no means unusual to find farmers who cultivate the same
land which their ancestors cultivated when the Plantagenets reigned in
England. The Somersetshire traditions are therefore, of no small value
to a historian. 
%[405]
\footnote{ Matt. West. Flor. Hist., A. D. 788; MS. Chronicle quoted
by Mr. Sharon Turner in the History of the Anglo-Saxons, book IV. chap.
xix; Drayton's Polyolbion, iii; Leland's Itinerary; Oldmixon, 703.
Oldmixon was then at Bridgewater, and probably saw the Duke on the
church tower. The dish mentioned in the text is the property of Mr.
Stradling, who has taken laudable pain's to preserve the relics and
traditions of the Western insurrection.}


At a greater distance from Bridgewater lies the village of Middlezoy.
In that village and its neighbourhood, the Wiltshire militia were
quartered, under the command of Pembroke. On the open moor, not far from
Chedzoy, were encamped several battalions of regular infantry. Monmouth
looked gloomily on them. He could not but remember how, a few years
before, he had, at the head of a column composed of some of those very
men, driven before him in confusion the fierce enthusiasts who defended
Bothwell Bridge He could distinguish among the hostile ranks that
gallant band which was then called from the name of its Colonel,
Dumbarton's regiment, but which has long been known as the first of
the line, and which, in all the four quarters of the world, has nobly
supported its early reputation. "I know those men," said Monmouth; "they
will fight. If I had but them, all would go well." 
%[406]
\footnote{ Oldmixon, 703.}


Yet the aspect of the enemy was not altogether discouraging. The three
divisions of the royal army lay far apart from one another. There was
all appearance of negligence and of relaxed discipline in all their
movements. It was reported that they were drinking themselves drunk with
the Zoyland cider. The incapacity of Feversham, who commanded in chief,
was notorious. Even at this momentous crisis he thought only of eating
and sleeping. Churchill was indeed a captain equal to tasks far more
arduous than that of scattering a crowd of ill armed and ill trained
peasants. But the genius, which, at a later period, humbled six Marshals
of France, was not now in its proper place. Feversham told Churchill
little, and gave him no encouragement to offer any suggestion. The
lieutenant, conscious of superior abilities and science, impatient of
the control of a chief whom he despised, and trembling for the fate of
the army, nevertheless preserved his characteristic self-command, and
dissembled his feelings so well that Feversham praised his submissive
alacrity, and promised to report it to the King. 
%[407]
\footnote{ Churchill to Clarendon, July 4, 1685.}


Monmouth, having observed the disposition of the royal forces, and
having been apprised of the state in which they were, conceived that
a night attack might be attended with success. He resolved to run the
hazard; and preparations were instantly made.

It was Sunday; and his followers, who had, for the most part, been
brought up after the Puritan fashion, passed a great part of the day in
religious exercises. The Castle Field, in which the army was encamped,
presented a spectacle such as, since the disbanding of Cromwell's
soldiers, England had never seen. The dissenting preachers who had taken
arms against Popery, and some of whom had probably fought in the great
civil war, prayed and preached in red coats and huge jackboots, with
swords by their sides. Ferguson was one of those who harangued. He took
for his text the awful imprecation by which the Israelites who dwelt
beyond Jordan cleared themselves from the charge ignorantly brought
against them by their brethren on the other side of the river. "The Lord
God of Gods, the Lord God of Gods, he knoweth; and Israel he shall know.
If it be in rebellion, or if in transgression against the Lord, save us
not this day." 
%[408]
\footnote{ Oldmixon, 703; Observator, Aug. 1, 1685.}


That an attack was to be made under cover of the night was no secret
in Bridgewater. The town was full of women, who had repaired thither
by hundreds from the surrounding region, to see their husbands, sons,
lovers, and brothers once more. There were many sad partings that day;
and many parted never to meet again. 
%[409]
\footnote{ Paschall's Narrative in Heywood's Appendix.}
 The report of the intended
attack came to the ears of a young girl who was zealous for the King.
Though of modest character, she had the courage to resolve that she
would herself bear the intelligence to Feversham. She stole out of
Bridgewater, and made her way to the royal camp. But that camp was not a
place where female innocence could be safe. Even the officers, despising
alike the irregular force to which they were opposed, and the negligent
general who commanded them, had indulged largely in wine, and were ready
for any excess of licentiousness and cruelty. One of them seized the
unhappy maiden, refused to listen to her errand, and brutally outraged
her. She fled in agonies of rage and shame, leaving the wicked army to
its doom. 
%[410]
\footnote{ Kennet, ed. 1719, iii. 432. I am forced to believe
that this lamentable story is true. The Bishop declares that it was
communicated to him in the year 1718 by a brave officer of the Blues,
who had fought at Sedgemoor, and who had himself seen the poor girl
depart in an agony of distress.}


And now the time for the great hazard drew near. The night was not ill
suited for such an enterprise. The moon was indeed at the full, and the
northern streamers were shining brilliantly. But the marsh fog lay
so thick on Sedgemoor that no object could be discerned there at the
distance of fifty paces. 
%[411]
\footnote{ Narrative of an officer of the Horse Guards in Kennet,
ed. 1718, iii. 432; MS. Journal of the Western Rebellion, kept by Mr.
Edward Dummer, Dryden's Hind and Panther, part II. The lines of Dryden
are remarkable:

     "Such were the pleasing triumphs of the sky
     For James's late nocturnal victory.
     The fireworks which his angels made above.
     The pledge of his almighty patron's love,
     I saw myself the lambent easy light
     Gild the brown horror and dispel the night.
     The messenger with speed the tidings bore.
     News which three labouring nations did restore;
     But heaven's own Nuntius was arrived before."}


The clock struck eleven; and the Duke with his body guard rode out of
the Castle. He was not in the frame of mind which befits one who is
about to strike a decisive blow. The very children who pressed to see
him pass observed, and long remembered, that his look was sad and full
of evil augury. His army marched by a circuitous path, near six miles in
length, towards the royal encampment on Sedgemoor. Part of the route is
to this day called War Lane. The foot were led by Monmouth himself. The
horse were confided to Grey, in spite of the remonstrances of some who
remembered the mishap at Bridport. Orders were given that strict silence
should be preserved, that no drum should be beaten, and no shot fired.
The word by which the insurgents were to recognise one another in the
darkness was Soho. It had doubtless been selected in allusion to Soho
Fields in London, where their leader's palace stood. 
%[412]
\footnote{ It has been said by several writers, and among them by
Pennant, that the district in London called Soho derived its name from
the watchword of Monmouth's army at Sedgemoor. Mention of Soho Fields
will be found in many books printed before the Western insurrection; for
example, in Chamberlayne's State of England, 1684.}


At about one in the morning of Monday the sixth of July, the rebels were
on the open moor. But between them and the enemy lay three broad rhines
filled with water and soft mud. Two of these, called the Black Ditch
and the Langmoor Rhine, Monmouth knew that he must pass. But, strange
to say, the existence of a trench, called the Bussex Rhine, which
immediately covered the royal encampment, had not been mentioned to him
by any of his scouts.

The wains which carried the ammunition remained at the entrance of the
moor. The horse and foot, in a long narrow column, passed the Black
Ditch by a causeway. There was a similar causeway across the Langmoor
Rhine: but the guide, in the fog, missed his way. There was some delay
and some tumult before the error could be rectified. At length the
passage was effected: but, in the confusion, a pistol went off. Some men
of the Horse Guards, who were on watch, heard the report, and perceived
that a great multitude was advancing through the mist. They fired their
carbines, and galloped off in different directions to give the alarm.
Some hastened to Weston Zoyland, where the cavalry lay. One trooper
spurred to the encampment of the infantry, and cried out vehemently that
the enemy was at hand. The drums of Dumbarton's regiment beat to arms;
and the men got fast into their ranks. It was time; for Monmouth was
already drawing up his army for action. He ordered Grey to lead the way
with the cavalry, and followed himself at the head of the infantry.
Grey pushed on till his progress was unexpectedly arrested by the Bussex
Rhine. On the opposite side of the ditch the King's foot were hastily
forming in order of battle.

"For whom are you?" called out an officer of the Foot Guards. "For the
King," replied a voice from the ranks of the rebel cavalry. "For which
King?" was then demanded. The answer was a shout of "King Monmouth,"
mingled with the war cry, which forty years before had been inscribed
on the colours of the parliamentary regiments, "God with us." The royal
troops instantly fired such a volley of musketry as sent the rebel horse
flying in all directions. The world agreed to ascribe this ignominious
rout to Grey's pusillanimity. Yet it is by no means clear that Churchill
would have succeeded better at the head of men who had never before
handled arms on horseback, and whose horses were unused, not only to
stand fire, but to obey the rein.

A few minutes after the Duke's horse had dispersed themselves over the
moor, his infantry came up running fast, and guided through the gloom by
the lighted matches of Dumbarton's regiment.

Monmouth was startled by finding that a broad and profound trench lay
between him and the camp which he had hoped to surprise. The insurgents
halted on the edge of the rhine, and fired. Part of the royal infantry
on the opposite bank returned the fire. During three quarters of an
hour the roar of the musketry was incessant. The Somersetshire peasants
behaved themselves as if they had been veteran soldiers, save only that
they levelled their pieces too high.

But now the other divisions of the royal army were in motion. The Life
Guards and Blues came pricking fast from Weston Zoyland, and scattered
in an instant some of Grey's horse, who had attempted to rally. The
fugitives spread a panic among their comrades in the rear, who had
charge of the ammunition. The waggoners drove off at full speed, and
never stopped till they were many miles from the field of battle.
Monmouth had hitherto done his part like a stout and able warrior. He
had been seen on foot, pike in hand, encouraging his infantry by voice
and by example. But he was too well acquainted with military affairs not
to know that all was over. His men had lost the advantage which surprise
and darkness had given them. They were deserted by the horse and by the
ammunition waggons. The King's forces were now united and in good order.
Feversham had been awakened by the firing, had got out of bed, had
adjusted his cravat, had looked at himself well in the glass, and had
come to see what his men were doing. Meanwhile, what was of much more
importance, Churchill had rapidly made an entirely new disposition of
the royal infantry. The day was about to break. The event of a conflict
on an open plain, by broad sunlight, could not be doubtful. Yet Monmouth
should have felt that it was not for him to fly, while thousands
whom affection for him had hurried to destruction were still fighting
manfully in his cause. But vain hopes and the intense love of life
prevailed. He saw that if he tarried the royal cavalry would soon
intercept his retreat. He mounted and rode from the field.

Yet his foot, though deserted, made a gallant stand. The Life Guards
attacked them on the right, the Blues on the left; but the Somersetshire
clowns, with their scythes and the butt ends of their muskets, faced
the royal horse like old soldiers. Oglethorpe made a vigorous attempt to
break them and was manfully repulsed. Sarsfield, a brave Irish officer,
whose name afterwards obtained a melancholy celebrity, charged on the
other flank. His men were beaten back. He was himself struck to the
ground, and lay for a time as one dead. But the struggle of the hardy
rustics could not last. Their powder and ball were spent. Cries were
heard of "Ammunition! For God's sake ammunition!" But no ammunition was
at hand. And now the King's artillery came up. It had been posted half
a mile off, on the high road from Weston Zoyland to Bridgewater. So
defective were then the appointments of an English army that there would
have been much difficulty in dragging the great guns to the place where
the battle was raging, had not the Bishop of Winchester offered
his coach horses and traces for the purpose. This interference of a
Christian prelate in a matter of blood has, with strange inconsistency,
been condemned by some Whig writers who can see nothing criminal in
the conduct of the numerous Puritan ministers then in arms against the
government. Even when the guns had arrived, there was such a want of
gunners that a serjeant of Dumbarton's regiment was forced to take on
himself the management of several pieces. 
%[413]
\footnote{ There is a warrant of James directing that forty pounds
should be paid to Sergeant Weems, of Dumbarton's regiment, "for good
service in the action at Sedgemoor in firing the great guns against the
rebels." Historical Record of the First or Royal Regiment of Foot.}
 The cannon, however,
though ill served, brought the engagement to a speedy close. The pikes
of the rebel battalions began to shake: the ranks broke; the King's
cavalry charged again, and bore down everything before them; the King's
infantry came pouring across the ditch. Even in that extremity the
Mendip miners stood bravely to their arms, and sold their lives dearly.
But the rout was in a few minutes complete. Three hundred of the
soldiers had been killed or wounded. Of the rebels more than a thousand
lay dead on the moor. 
%[414]
\footnote{ James the Second's account of the battle of Sedgemoor
in Lord Hardwicke's State Papers; Wade's Confession; Ferguson's MS.
Narrative in Eachard, iii. 768; Narrative of an Officer of the Horse
Guards in Kennet, ed. 1719, iii. 432, London Gazette, July 9, 1685;
Oldmixon, 703; Paschall's Narrative; Burnet, i. 643; Evelyn's Diary,
July 8; Van Citters,.July 7-17; Barillon, July 9-19; Reresby's Memoirs;
the Duke of Buckingham's battle of Sedgemoor, a Farce; MS. Journal of
the Western Rebellion, kept by Mr. Edward Dummer, then serving in the
train of artillery employed by His Majesty for the suppression of the
same. The last mentioned manuscript is in the Pepysian library, and is
of the greatest value, not on account of the narrative, which contains
little that is remarkable, but on account of the plans, which exhibit
the battle in four or five different stages.}


So ended the last fight deserving the name of battle, that has been
fought on English ground. The impression left on the simple inhabitants
of the neighbourhood was deep and lasting. That impression, indeed, has
been frequently renewed. For even in our own time the plough and the
spade have not seldom turned up ghastly memorials of the slaughter,
skulls, and thigh bones, and strange weapons made out of implements of
husbandry. Old peasants related very recently that, in their childhood,
they were accustomed to play on the moor at the fight between King
James's men and King Monmouth's men, and that King Monmouth's men always
raised the cry of Soho. 
%[415]
\footnote{ I learned these things from persons living close to
Sedgemoor.}


What seems most extraordinary in the battle of Sedgemoor is that the
event should have been for a moment doubtful, and that the rebels should
have resisted so long. That five or six thousand colliers and ploughmen
should contend during an hour with half that number of regular cavalry
and infantry would now be thought a miracle. Our wonder will, perhaps,
be diminished when we remember that, in the time of James the Second,
the discipline of the regular army was extremely lax, and that, on the
other hand, the peasantry were accustomed to serve in the militia.
The difference, therefore, between a regiment of the Foot Guards and a
regiment of clowns just enrolled, though doubtless considerable, was by
no means what it now is. Monmouth did not lead a mere mob to attack good
soldiers. For his followers were not altogether without a tincture of
soldiership; and Feversham's troops, when compared with English troops
of our time, might almost be called a mob.

It was four o'clock: the sun was rising; and the routed army came
pouring into the streets of Bridgewater. The uproar, the blood, the
gashes, the ghastly figures which sank down and never rose again,
spread horror and dismay through the town. The pursuers, too, were close
behind. Those inhabitants who had favoured the insurrection expected
sack and massacre, and implored the protection of their neighbours
who professed the Roman Catholic religion, or had made themselves
conspicuous by Tory politics; and it is acknowledged by the bitterest
of Whig historians that this protection was kindly and generously given.

%[416]
\footnote{ Oldmixon, 704.}


During that day the conquerors continued to chase the fugitives. The
neighbouring villagers long remembered with what a clatter of horsehoofs
and what a storm of curses the whirlwind of cavalry swept by. Before
evening five hundred prisoners had been crowded into the parish church
of Weston Zoyland. Eighty of them were wounded; and five expired within
the consecrated walls. Great numbers of labourers were impressed for the
purpose of burying the slain. A few, who were notoriously partial to the
vanquished side, were set apart for the hideous office of quartering the
captives. The tithing men of the neighbouring parishes were busied in
setting up gibbets and providing chains. All this while the bells of
Weston Zoyland and Chedzoy rang joyously; and the soldiers sang
and rioted on the moor amidst the corpses. For the farmers of the
neighbourhood had made haste, as soon as the event of the fight was
known to send hogsheads of their best cider as peace offerings to the
victors. 
%[417]
\footnote{ Locke's Western Rebellion Stradling's Chilton Priory.}


Feversham passed for a goodnatured man: but he was a foreigner,
ignorant of the laws and careless of the feelings of the English. He was
accustomed to the military license of France, and had learned from
his great kinsman, the conqueror and devastator of the Palatinate, not
indeed how to conquer, but how to devastate. A considerable number of
prisoners were immediately selected for execution. Among them was a
youth famous for his speed. Hopes were held out to him that his life
would be spared If he could run a race with one of the colts of the
marsh. The space through which the man kept up with the horse is still
marked by well known bounds on the moor, and is about three quarters of
a mile. Feversham was not ashamed, after seeing the performance, to
send the wretched performer to the gallows. The next day a long line of
gibbets appeared on the road leading from Bridgewater to Weston Zoyland.
On each gibbet a prisoner was suspended. Four of the sufferers were left
to rot in irons. 
%[418]
\footnote{ Locke's Western Rebellion Stradling's Chilton Priory;
Oldmixon, 704.}


Meanwhile Monmouth, accompanied by Grey, by Buyse, and by a few other
friends, was flying from the field of battle. At Chedzoy he stopped
a moment to mount a fresh horse and to hide his blue riband and his
George. He then hastened towards the Bristol Channel. From the rising
ground on the north of the field of battle he saw the flash and the
smoke of the last volley fired by his deserted followers. Before six
o'clock he was twenty miles from Sedgemoor. Some of his companions
advised him to cross the water, and seek refuge in Wales; and this would
undoubtedly have been his wisest course. He would have been in Wales
many hours before the news of his defeat was known there; and in a
country so wild and so remote from the seat of government, he might
have remained long undiscovered. He determined, however, to push for
Hampshire, in the hope that he might lurk in the cabins of deerstealers
among the oaks of the New Forest, till means of conveyance to the
Continent could be procured. He therefore, with Grey and the German,
turned to the southeast. But the way was beset with dangers. The three
fugitives had to traverse a country in which every one already knew the
event of the battle, and in which no traveller of suspicious appearance
could escape a close scrutiny. They rode on all day, shunning towns and
villages. Nor was this so difficult as it may now appear. For men then
living could remember the time when the wild deer ranged freely through
a succession of forests from the banks of the Avon in Wiltshire to the
southern coast of Hampshire. 
%[419]
\footnote{ Aubrey's Natural History of Wiltshire, 1691.}
 At length, on Cranbourne Chase, the
strength of the horses failed. They were therefore turned loose. The
bridles and saddles were concealed. Monmouth and his friends procured
rustic attire, disguised themselves, and proceeded on foot towards the
New Forest. They passed the night in the open air: but before morning
they were Surrounded on every side by toils. Lord Lumley, who lay
at Ringwood with a strong body of the Sussex militia, had sent forth
parties in every direction. Sir William Portman, with the Somerset
militia, had formed a chain of posts from the sea to the northern
extremity of Dorset. At five in the morning of the seventh, Grey, who
had wandered from his friends, was seized by two of the Sussex scouts.
He submitted to his fate with the calmness of one to whom suspense was
more intolerable than despair. "Since we landed," he said, "I have not
had one comfortable meal or one quiet night." It could hardly be doubted
that the chief rebel was not far off. The pursuers redoubled their
vigilance and activity. The cottages scattered over the heathy country
on the boundaries of Dorsetshire and Hampshire were strictly examined
by Lumley; and the clown with whom Monmouth had changed clothes was
discovered. Portman came with a strong body of horse and foot to assist
in the search. Attention was soon drawn to a place well fitted to
shelter fugitives. It was an extensive tract of land separated by an
enclosure from the open country, and divided by numerous hedges into
small fields. In some of these fields the rye, the pease, and the oats
were high enough to conceal a man. Others were overgrown with fern and
brambles. A poor woman reported that she had seen two strangers lurking
in this covert. The near prospect of reward animated the zeal of the
troops. It was agreed that every man who did his duty in the search
should have a share of the promised five thousand pounds. The outer
fence was strictly guarded: the space within was examined with
indefatigable diligence; and several dogs of quick scent were turned out
among the bushes. The day closed before the work could be completed: but
careful watch was kept all night. Thirty times the fugitives ventured
to look through the outer hedge: but everywhere they found a sentinel
on the alert: once they were seen and fired at; they then separated and
concealed themselves in different hiding places.

At sunrise the next morning the search recommenced, and Buyse was found.
He owned that he had parted from the Duke only a few hours before. The
corn and copsewood were now beaten with more care than ever. At length
a gaunt figure was discovered hidden in a ditch. The pursuers sprang
on their prey. Some of them were about to fire: but Portman forbade
all violence. The prisoner's dress was that of a shepherd; his beard,
prematurely grey, was of several days' growth. He trembled greatly, and
was unable to speak. Even those who had often seen him were at first in
doubt whether this were truly the brilliant and graceful Monmouth. His
pockets were searched by Portman, and in them were found, among some raw
pease gathered in the rage of hunger, a watch, a purse of gold, a
small treatise on fortification, an album filled with songs, receipts,
prayers, and charms, and the George with which, many years before, King
Charles the Second had decorated his favourite son. Messengers were
instantly despatched to Whitehall with the good news, and with the
George as a token that the news was true. The prisoner was conveyed
under a strong guard to Ringwood. 
%[420]
\footnote{ Account of the manner of taking the late Duke of
Monmouth, published by his Majesty's command; Gazette de France, July
18-28, 1688; Eachard, iii. 770; Burnet, i. 664, and Dartmouth's note:
Van Citters, July 10-20,1688.}


And all was lost; and nothing remained but that he should prepare to
meet death as became one who had thought himself not unworthy to wear
the crown of William the Conqueror and of Richard the Lionhearted,
of the hero of Cressy and of the hero of Agincourt. The captive might
easily have called to mind other domestic examples, still better suited
to his condition. Within a hundred years, two sovereigns whose blood ran
in his veins, one of them a delicate woman, had been placed in the same
situation in which he now stood. They had shown, in the prison and on
the scaffold, virtue of which, in the season of prosperity, they had
seemed incapable, and had half redeemed great crimes and errors
by enduring with Christian meekness and princely dignity all that
victorious enemies could inflict. Of cowardice Monmouth had never been
accused; and, even had he been wanting in constitutional courage, it
might have been expected that the defect would be supplied by pride
and by despair. The eyes of the whole world were upon him. The latest
generations would know how, in that extremity, he had borne himself.
To the brave peasants of the West he owed it to show that they had not
poured forth their blood for a leader unworthy of their attachment. To
her who had sacrificed everything for his sake he owed it so to bear
himself that, though she might weep for him, she should not blush for
him. It was not for him to lament and supplicate. His reason, too,
should have told him that lamentation and supplication would be
unavailing. He had done that which could never be forgiven. He was in
the grasp of one who never forgave.

But the fortitude of Monmouth was not that highest sort of fortitude
which is derived from reflection and from selfrespect; nor had nature
given him one of those stout hearts from which neither adversity nor
peril can extort any sign of weakness. His courage rose and fell with
his animal spirits. It was sustained on the field of battle by the
excitement of action. By the hope of victory, by the strange influence
of sympathy. All such aids were now taken away. The spoiled darling of
the court and of the populace, accustomed to be loved and worshipped
wherever he appeared, was now surrounded by stern gaolers in whose eyes
he read his doom. Yet a few hours of gloomy seclusion, and he must die a
violent and shameful death. His heart sank within him. Life seemed worth
purchasing by any humiliation; nor could his mind, always feeble, and
now distracted by terror, perceive that humiliation must degrade, but
could not save him.

As soon as he reached Ringwood he wrote to the King. The letter was that
of a man whom a craven fear had made insensible to shame. He professed
in vehement terms his remorse for his treason. He affirmed that, when he
promised his cousins at the Hague not to raise troubles in England,
he had fully meant to keep his word. Unhappily he had afterwards been
seduced from his allegiance by some horrid people who had heated his
mind by calumnies and misled him by sophistry; but now he abhorred
them: he abhorred himself. He begged in piteous terms that he might be
admitted to the royal presence. There was a secret which he could not
trust to paper, a secret which lay in a single word, and which, if he
spoke that word, would secure the throne against all danger. On the
following day he despatched letters, imploring the Queen Dowager and the
Lord Treasurer to intercede in his behalf. 
%[421]
\footnote{ The letter to the King was printed at the time by
authority; that to the Queen Dowager will be found in Sir H. Ellis's
Original Letters; that to Rochester in the Clarendon Correspondence.}


When it was known in London how he had abased himself the general
surprise was great; and no man was more amazed than Barillon, who
had resided in England during two bloody proscriptions, and had seen
numerous victims, both of the Opposition and of the Court, submit to
their fate without womanish entreaties and lamentations. 
%[422]
\footnote{ "On trouve," he wrote, "fort a redire icy qu'il ayt fait
une chose si peu ordinaire aux Anglois." July 13-23, 1685.}


Monmouth and Grey remained at Ringwood two days. They were then carried
up to London, under the guard of a large body of regular troops and
militia. In the coach with the Duke was an officer whose orders were to
stab the prisoner if a rescue were attempted. At every town along the
road the trainbands of the neighbourhood had been mustered under the
command of the principal gentry. The march lasted three days, and
terminated at Vauxhall, where a regiment, commanded by George Legge,
Lord Dartmouth, was in readiness to receive the prisoners. They were
put on board of a state barge, and carried down the river to Whitehall
Stairs. Lumley and Portman had alternately watched the Duke day and
night till they had brought him within the walls of the palace. 
%[423]
\footnote{ Account of the manner of taking the Duke of Monmouth;
Gazette, July 16, 1685; Van Citters, July 14-24,}


Both the demeanour of Monmouth and that of Grey, during the journey,
filled all observers with surprise. Monmouth was altogether unnerved.
Grey was not only calm but cheerful, talked pleasantly of horses,
dogs, and field sports, and even made jocose allusions to the perilous
situation in which he stood.

The King cannot be blamed for determining that Monmouth should suffer
death. Every man who heads a rebellion against an established government
stakes his life on the event; and rebellion was the smallest part
of Monmouth's crime. He had declared against his uncle a war without
quarter. In the manifesto put forth at Lyme, James had been held up
to execration as an incendiary, as an assassin who had strangled one
innocent man and cut the throat of another, and, lastly, as the poisoner
of his own brother. To spare an enemy who had not scrupled to resort
to such extremities would have been an act of rare, perhaps of blamable
generosity. But to see him and not to spare him was an outrage on
humanity and decency. 
%[424]
\footnote{ Barillon was evidently much shocked. "Ill se vient," he
says, "de passer icy, une chose bien extraordinaire et fort opposee a
l'usage ordinaire des autres nations" 13-23, 1685.}
 This outrage the King resolved to commit.
The arms of the prisoner were bound behind him with a silken cord; and,
thus secured, he was ushered into the presence of the implacable kinsman
whom he had wronged.

Then Monmouth threw himself on the ground, and crawled to the King's
feet. He wept. He tried to embrace his uncle's knees with his pinioned
arms. He begged for life, only life, life at any price. He owned that
he had been guilty of a great crime, but tried to throw the blame on
others, particularly on Argyle, who would rather have put his legs into
the boots than have saved his own life by such baseness. By the ties
of kindred, by the memory of the late King, who had been the best and
truest of brothers, the unhappy man adjured James to show some mercy.
James gravely replied that this repentance was of the latest, that he
was sorry for the misery which the prisoner had brought on himself,
but that the case was not one for lenity. A Declaration, filled with
atrocious calumnies, had been put forth. The regal title had been
assumed. For treasons so aggravated there could be no pardon on this
side of the grave. The poor terrified Duke vowed that he had never
wished to take the crown, but had been led into that fatal error by
others. As to the Declaration, he had not written it: he had not read
it: he had signed it without looking at it: it was all the work of
Ferguson, that bloody villain Ferguson. "Do you expect me to believe,"
said James, with contempt but too well merited, "that you set your hand
to a paper of such moment without knowing what it contained?" One depth
of infamy only remained; and even to that the prisoner descended. He was
preeminently the champion of the Protestant religion. The interest of
that religion had been his plea for conspiring against the government of
his father, and for bringing on his country the miseries of civil war;
yet he was not ashamed to hint that he was inclined to be reconciled to
the Church of Rome. The King eagerly offered him spiritual assistance,
but said nothing of pardon or respite. "Is there then no hope?" asked
Monmouth. James turned away in silence. Then Monmouth strove to rally
his courage, rose from his knees, and retired with a firmness which he
had not shown since his overthrow. 
%[425]
\footnote{ Burnet. i. 644; Evelyn's Diary, July 15; Sir J.
Bramston's Memoirs; Reresby's Memoirs; James to the Prince of Orange,
July 14, 1685; Barillon, July 16-26; Bucclench MS.}


Grey was introduced next. He behaved with a propriety and fortitude
which moved even the stern and resentful King, frankly owned himself
guilty, made no excuses, and did not once stoop to ask his life. Both
the prisoners were sent to the Tower by water. There was no tumult; but
many thousands of people, with anxiety and sorrow in their faces, tried
to catch a glimpse of the captives. The Duke's resolution failed as soon
as he had left the royal presence. On his way to his prison he bemoaned
himself, accused his followers, and abjectly implored the intercession
of Dartmouth. "I know, my Lord, that you loved my father. For his sake,
for God's sake, try if there be any room for mercy." Dartmouth replied
that the King had spoken the truth, and that a subject who assumed the
regal title excluded himself from all hope of pardon. 
%[426]
\footnote{ James to the Prince of Orange, July 14, 1685, Dutch
Despatch of the same date, Dartmouth's note on Burnet, i. 646; Narcissus
Luttrell's Diary, (1848) a copy of this diary, from July 1685 to Sept.
1690, is among the Mackintosh papers. To the rest I was allowed access
by the kindness of the Warden of All Souls' College, where the original
MS. is deposited. The delegates of the Press of the University of Oxford
have since published the whole in six substantial volumes, which will, I
am afraid, find little favour with readers who seek only for amusement,
but which will always be useful as materials for history. (1857.)}


Soon after Monmouth had been lodged in the Tower, he was informed
that his wife had, by the royal command, been sent to see him. She was
accompanied by the Earl of Clarendon, Keeper of the Privy Seal. Her
husband received her very coldly, and addressed almost all his discourse
to Clarendon whose intercession he earnestly implored. Clarendon held
out no hopes; and that same evening two prelates, Turner, Bishop of Ely,
and Ken, Bishop of Bath and Wells, arrived at the Tower with a solemn
message from the King. It was Monday night. On Wednesday morning
Monmouth was to die.

He was greatly agitated. The blood left his cheeks; and it was some time
before he could speak. Most of the short time which remained to him he
wasted in vain attempts to obtain, if not a pardon, at least a respite.
He wrote piteous letters to the King and to several courtiers, but in
vain. Some Roman Catholic divines were sent to him from Whitehall. But
they soon discovered that, though he would gladly have purchased his
life by renouncing the religion of which he had professed himself in an
especial manner the defender, yet, if he was to die, he would as soon
die without their absolution as with it. 
%[427]
\footnote{ Buccleuch MS; Life of James the Second, ii. 37, Orig.
Mem., Van Citters, July 14-24, 1685; Gazette de France, August 1-11.}


Nor were Ken and Turner much better pleased with his frame of mind. The
doctrine of nonresistance was, in their view, as in the view of most of
their brethren, the distinguishing badge of the Anglican Church. The two
Bishops insisted on Monmouth's owning that, in drawing the sword against
the government, he had committed a great sin; and, on this point,
they found him obstinately heterodox. Nor was this his only heresy. He
maintained that his connection with Lady Wentworth was blameless in the
sight of God. He had been married, he said, when a child. He had never
cared for his Duchess. The happiness which he had not found at home
he had sought in a round of loose amours, condemned by religion and
morality. Henrietta had reclaimed him from a life of vice. To her he had
been strictly constant. They had, by common consent, offered up fervent
prayers for the divine guidance. After those prayers they had found
their affection for each other strengthened; and they could then no
longer doubt that, in the sight of God, they were a wedded pair. The
Bishops were so much scandalised by this view of the conjugal relation
that they refused to administer the sacrament to the prisoner. All that
they could obtain from him was a promise that, during the single night
which still remained to him, he would pray to be enlightened if he were
in error.

On the Wednesday morning, at his particular request, Doctor Thomas
Tenison, who then held the vicarage of Saint Martin's, and, in that
important cure, had obtained the high esteem of the public, came to the
Tower. From Tenison, whose opinions were known to be moderate, the Duke
expected more indulgence than Ken and Turner were disposed to show. But
Tenison, whatever might be his sentiments concerning nonresistance in
the abstract, thought the late rebellion rash and wicked, and considered
Monmouth's notion respecting marriage as a most dangerous delusion.
Monmouth was obstinate. He had prayed, he said, for the divine
direction. His sentiments remained unchanged; and he could not doubt
that they were correct. Tenison's exhortations were in milder tone than
those of the Bishops. But he, like them, thought that he should not be
justified in administering the Eucharist to one whose penitence was of
so unsatisfactory a nature. 
%[428]
\footnote{ Buccleuch MS.; Life of James the Second, ii. 37, 38,
Orig. Mem., Burnet, i. 645; Tenison's account in Kennet, iii. 432, ed.
1719.}


The hour drew near: all hope was over; and Monmouth had passed from
pusillanimous fear to the apathy of despair. His children were brought
to his room that he might take leave of them, and were followed by his
wife. He spoke to her kindly, but without emotion. Though she was a
woman of great strength of mind, and had little cause to love him, her
misery was such that none of the bystanders could refrain from weeping.
He alone was unmoved. 
%[429]
\footnote{ Buccleuch MS.}


It was ten o'clock. The coach of the Lieutenant of the Tower was ready.
Monmouth requested his spiritual advisers to accompany him to the place
of execution; and they consented: but they told him that, in their
judgment, he was about to die in a perilous state of mind, and that, if
they attended him it would be their duty to exhort him to the last. As
he passed along the ranks of the guards he saluted them with a smile;
and he mounted the scaffold with a firm tread. Tower Hill was covered
up to the chimney tops with an innumerable multitude of gazers, who, in
awful silence, broken only by sighs and the noise of weeping, listened
for the last accents of the darling of the people. "I shall say little,"
he began. "I come here, not to speak, but to die. I die a Protestant of
the Church of England." The Bishops interrupted him, and told him that,
unless he acknowledged resistance to be sinful, he was no member of
their church He went on to speak of his Henrietta. She was, he said, a
young lady of virtue and honour. He loved her to the last, and he could
not die without giving utterance to his feelings The Bishops again
interfered, and begged him not to use such language. Some altercation
followed. The divines have been accused of dealing harshly with the
dying man. But they appear to have only discharged what, in their view,
was a sacred duty. Monmouth knew their principles, and, if he wished to
avoid their importunity, should have dispensed with their attendance.
Their general arguments against resistance had no effect on him. But
when they reminded him of the ruin which he had brought on his brave and
loving followers, of the blood which had been shed, of the souls which
had been sent unprepared to the great account, he was touched, and said,
in a softened voice, "I do own that. I am sorry that it ever happened."
They prayed with him long and fervently; and he joined in their
petitions till they invoked a blessing on the King. He remained silent.
"Sir," said one of the Bishops, "do you not pray for the King with us?"
Monmouth paused some time, and, after an internal struggle, exclaimed
"Amen." But it was in vain that the prelates implored him to address to
the soldiers and to the people a few words on the duty of obedience
to the government. "I will make no speeches," he exclaimed. "Only ten
words, my Lord." He turned away, called his servant, and put into the
man's hand a toothpick case, the last token of ill starred love.
"Give it," he said, "to that person." He then accosted John Ketch the
executioner, a wretch who had butchered many brave and noble victims,
and whose name has, during a century and a half, been vulgarly given to
all who have succeeded him in his odious office. 
%[430]
\footnote{ The name of Ketch was often associated with that of
Jeffreys in the lampoons of those days.

     "While Jeffreys on the bench,
     Ketch on the gibbet sits,"

says one poet. In the year which followed Monmouth's execution Ketch
was turned out of his office for insulting one of the Sheriffs, and was
succeeded by a butcher named Rose. But in four months Rose himself was
hanged at Tyburn, and Ketch was reinstated. Luttrell's Diary, January
20, and May 28, 1686. See a curious note by Dr. Grey, on Hudibras, part
iii. canto ii. line 1534.}
 "Here," said
the Duke, "are six guineas for you. Do not hack me as you did my Lord
Russell. I have heard that you struck him three or four times. My
servant will give you some more gold if you do the work well." He then
undressed, felt the edge of the axe, expressed some fear that it was
not sharp enough, and laid his head on the block. The divines in the
meantime continued to ejaculate with great energy: "God accept your
repentance! God accept your imperfect repentance!"

The hangman addressed himself to his office. But he had been
disconcerted by what the Duke had said. The first blow inflicted only
a slight wound. The Duke struggled, rose from the block, and looked
reproachfully at the executioner. The head sunk down once more. The
stroke was repeated again and again; but still the neck was not severed,
and the body continued to move. Yells of rage and horror rose from the
crowd. Ketch flung down the axe with a curse. "I cannot do it," he said;
"my heart fails me." "Take up the axe, man," cried the sheriff. "Fling
him over the rails," roared the mob. At length the axe was taken up. Two
more blows extinguished the last remains of life; but a knife was used
to separate the head from the shoulders. The crowd was wrought up to
such an ecstasy of rage that the executioner was in danger of being torn
in pieces, and was conveyed away under a strong guard. 
%[431]
\footnote{ Account of the execution of Monmouth, signed by the
divines who attended him; Buccleuch MS; Burnet, i. 646; Van Citters,
July 17-27,1685, Luttrell's Diary; Evelyn's Diary, July 15; Barillon,
July 19-29.}


In the meantime many handkerchiefs were dipped in the Duke's blood; for
by a large part of the multitude he was regarded as a martyr who had
died for the Protestant religion. The head and body were placed in a
coffin covered with black velvet, and were laid privately under the
communion table of Saint Peter's Chapel in the Tower. Within four years
the pavement of the chancel was again disturbed, and hard by the remains
of Monmouth were laid the remains of Jeffreys. In truth there is no
sadder spot on the earth than that little cemetery. Death is there
associated, not, as in Westminster Abbey and St. Paul's, with genius and
virtue, with public veneration and imperishable renown; not, as in
our humblest churches and churchyards, with everything that is most
endearing in social and domestic charities; but with whatever is
darkest in human nature and in human destiny, with the savage triumph of
implacable enemies, with the inconstancy, the ingratitude, the cowardice
of friends, with all the miseries of fallen greatness and of blighted
fame. Thither have been carried, through successive ages, by the rude
hands of gaolers, without one mourner following, the bleeding relics
of men who had been the captains of armies, the leaders of parties,
the oracles of senates, and the ornaments of courts. Thither was borne,
before the window where Jane Grey was praying, the mangled corpse of
Guilford Dudley. Edward Seymour, Duke of Somerset, and Protector of
the realm, reposes there by the brother whom he murdered. There has
mouldered away the headless trunk of John Fisher, Bishop of Rochester
and Cardinal of Saint Vitalis, a man worthy to have lived in a better
age and to have died in a better cause. There are laid John Dudley,
Duke of Northumberland, Lord High Admiral, and Thomas Cromwell, Earl of
Essex, Lord High Treasurer. There, too, is another Essex, on whom nature
and fortune had lavished all their bounties in vain, and whom valour,
grace, genius, royal favour, popular applause, conducted to an early
and ignominious doom. Not far off sleep two chiefs of the great house
of Howard, Thomas, fourth Duke of Norfolk, and Philip, eleventh Earl of
Arundel. Here and there, among the thick graves of unquiet and aspiring
statesmen, lie more delicate sufferers; Margaret of Salisbury, the last
of the proud name of Plantagenet; and those two fair Queens who perished
by the jealous rage of Henry. Such was the dust with which the dust of
Monmouth mingled. 
%[432]
\footnote{ I cannot refrain from expressing my disgust at the
barbarous stupidity which has transformed this most interesting little
church into the likeness of a meetinghouse in a manufacturing town.}


Yet a few months, and the quiet village of Toddington, in Bedfordshire,
witnessed a still sadder funeral. Near that village stood an ancient
and stately hall, the seat of the Wentworths. The transept of the parish
church had long been their burial place. To that burial place, in the
spring which followed the death of Monmouth, was borne the coffin of the
young Baroness Wentworth of Nettlestede. Her family reared a sumptuous
mausoleum over her remains: but a less costly memorial of her was long
contemplated with far deeper interest. Her name, carved by the hand of
him whom she loved too well, was, a few years ago, still discernible on
a tree in the adjoining park.

It was not by Lady Wentworth alone that the memory of Monmouth was
cherished with idolatrous fondness. His hold on the hearts of the people
lasted till the generation which had seen him had passed away. Ribands,
buckles, and other trifling articles of apparel which he had worn, were
treasured up as precious relics by those who had fought under him at
Sedgemoor. Old men who long survived him desired, when they were dying,
that these trinkets might be buried with them. One button of gold thread
which narrowly escaped this fate may still be seen at a house which
overlooks the field of battle. Nay, such was the devotion of the people
to their unhappy favourite that, in the face of the strongest evidence
by which the fact of a death was ever verified, many continued to
cherish a hope that he was still living, and that he would again appear
in arms. A person, it was said, who was remarkably like Monmouth,
had sacrificed himself to save the Protestant hero. The vulgar long
continued, at every important crisis, to whisper that the time was at
hand, and that King Monmouth would soon show himself. In 1686, a knave
who had pretended to be the Duke, and had levied contributions in
several villages of Wiltshire, was apprehended, and whipped from Newgate
to Tyburn. In 1698, when England had long enjoyed constitutional freedom
under a new dynasty, the son of an innkeeper passed himself on the
yeomanry of Sussex as their beloved Monmouth, and defrauded many who
were by no means of the lowest class. Five hundred pounds were collected
for him. The farmers provided him with a horse. Their wives sent him
baskets of chickens and ducks, and were lavish, it was said, of favours
of a more tender kind; for in gallantry at least, the counterfeit was
a not unworthy representative of the original. When this impostor
was thrown into prison for his fraud, his followers maintained him in
luxury. Several of them appeared at the bar to countenance him when he
was tried at the Horsham assizes. So long did this delusion last
that, when George the Third had been some years on the English throne,
Voltaire thought it necessary gravely to confute the hypothesis that the
man in the iron mask was the Duke of Monmouth. 
%[433]
\footnote{ Observator, August 1, 1685; Gazette de France, Nov. 2,
1686; Letter from Humphrey Wanley, dated Aug. 25, 1698, in the Aubrey
Collection; Voltaire, Dict. Phil. There are, in the Pepysian Collection,
several ballads written after Monmouth's death which represent him as
living, and predict his speedy return. I will give two specimens.

     "Though this is a dismal story
     Of the fall of my design,
     Yet I'll come again in glory,
     If I live till eighty-nine:
     For I'll have a stronger army
     And of ammunition store."

     Again;

     "Then shall Monmouth in his glories
     Unto his English friends appear,
     And will stifle all such stories
     As are vended everywhere.
     They'll see I was not so degraded,
     To be taken gathering pease,
     Or in a cock of hay up braided.
     What strange stories now are these!"}


It is, perhaps, a fact scarcely less remarkable that, to this day,
the inhabitants of some parts of the West of England, when any bill
affecting their interest is before the House of Lords, think themselves
entitled to claim the help of the Duke of Buccleuch, the descendant of
the unfortunate leader for whom their ancestors bled.

The history of Monmouth would alone suffice to refute the Imputation
of inconstancy which is so frequently thrown on the common people. The
common people are sometimes inconstant; for they are human beings. But
that they are inconstant as compared with the educated classes, with
aristocracies, or with princes, may be confidently denied. It would be
easy to name demagogues whose popularity has remained undiminished while
sovereigns and parliaments have withdrawn their confidence from a long
succession of statesmen. When Swift had survived his faculties many
years, the Irish populace still continued to light bonfires on his
birthday, in commemoration of the services which they fancied that he
had rendered to his country when his mind was in full vigour. While
seven administrations were raised to power and hurled from it in
consequence of court intrigues or of changes in the sentiments of the
higher classes of society, the profligate Wilkes retained his hold on
the selections of a rabble whom he pillaged and ridiculed. Politicians,
who, in 1807, had sought to curry favour with George the Third by
defending Caroline of Brunswick, were not ashamed, in 1820, to curry
favour with George the Fourth by persecuting her. But in 1820, as in
1807, the whole body of working men was fanatically devoted to her
cause. So it was with Monmouth. In 1680, he had been adored alike by the
gentry and by the peasantry of the West. In 1685 he came again. To the
gentry he had become an object of aversion: but by the peasantry he
was still loved with a love strong as death, with a love not to be
extinguished by misfortunes or faults, by the flight from Sedgemoor, by
the letter from Ringwood, or by the tears and abject supplications at
Whitehall. The charge which may with justice be brought against the
common people is, not that they are inconstant, but that they almost
invariably choose their favourite so ill that their constancy is a vice
and not a virtue.

While the execution of Monmouth occupied the thoughts of the Londoners,
the counties which had risen against the government were enduring all
that a ferocious soldiery could inflict. Feversham had been summoned to
the court, where honours and rewards which he little deserved awaited
him. He was made a Knight of the Garter and Captain of the first and
most lucrative troop of Life Guards: but Court and City laughed at his
military exploits; and the wit of Buckingham gave forth its last feeble
flash at the expense of the general who had won a battle in bed. 
%[434]
\footnote{ London Gazette, August 3, 1685; the Battle of Sedgemoor,
a Farce.}

Feversham left in command at Bridgewater Colonel Percy Kirke, a military
adventurer whose vices had been developed by the worst of all schools,
Tangier. Kirke had during some years commanded the garrison of that
town, and had been constantly employed in hostilities against tribes of
foreign barbarians, ignorant of the laws which regulate the warfare of
civilized and Christian nations. Within the ramparts of his fortress
he was a despotic prince. The only check on his tyranny was the fear of
being called to account by a distant and a careless government. He might
therefore safely proceed to the most audacious excesses of rapacity,
licentiousness, and cruelty. He lived with boundless dissoluteness, and
procured by extortion the means of indulgence. No goods could be sold
till Kirke had had the refusal of them. No question of right could be
decided till Kirke had been bribed. Once, merely from a malignant whim,
he staved all the wine in a vintner's cellar. On another occasion he
drove all the Jews from Tangier. Two of them he sent to the Spanish
Inquisition, which forthwith burned them. Under this iron domination
scarce a complaint was heard; for hatred was effectually kept down by
terror. Two persons who had been refractory were found murdered; and it
was universally believed that they had been slain by Kirke's order. When
his soldiers displeased him he flogged them with merciless severity: but
he indemnified them by permitting them to sleep on watch, to reel
drunk about the streets, to rob, beat, and insult the merchants and the
labourers.

When Tangier was abandoned, Kirke returned to England. He still
continued to command his old soldiers, who were designated sometimes as
the First Tangier Regiment, and sometimes as Queen Catharine's Regiment.
As they had been levied for the purpose of waging war on an infidel
nation, they bore on their flag a Christian emblem, the Paschal Lamb.
In allusion to this device, and with a bitterly ironical meaning, these
men, the rudest and most ferocious in the English army, were called
Kirke's Lambs. The regiment, now the second of the line, still
retains this ancient badge, which is however thrown into the shade by
decorations honourably earned in Egypt, in Spain, and in the heart of
Asia. 
%[435]
\footnote{ Pepys's Diary, kept at Tangier; Historical Records of the
Second or Queen's Royal Regiment of Foot.}


Such was the captain and such the soldiers who were now let loose on the
people of Somersetshire. From Bridgewater Kirke marched to Taunton. He
was accompanied by two carts filled with wounded rebels whose gashes
had not been dressed, and by a long drove of prisoners on foot, who were
chained two and two Several of these he hanged as soon as he reached
Taunton, without the form of a trial. They were not suffered even to
take leave of their nearest relations. The signpost of the White Hart
Inn served for a gallows. It is said that the work of death went on in
sight of the windows where the officers of the Tangier regiment were
carousing, and that at every health a wretch was turned off. When the
legs of the dying man quivered in the last agony, the colonel ordered
the drums to strike up. He would give the rebels, he said music to
their dancing. The tradition runs that one of the captives was not even
allowed the indulgence of a speedy death. Twice he was suspended from
the signpost, and twice cut down. Twice he was asked if he repented of
his treason, and twice he replied that, if the thing were to do again,
he would do it. Then he was tied up for the last time. So many dead
bodies were quartered that the executioner stood ankle deep in blood.
He was assisted by a poor man whose loyalty was suspected, and who was
compelled to ransom his own life by seething the remains of his friends
in pitch. The peasant who had consented to perform this hideous office
afterwards returned to his plough. But a mark like that of Cain was
upon him. He was known through his village by the horrible name of Tom
Boilman. The rustics long continued to relate that, though he had, by
his sinful and shameful deed, saved himself from the vengeance of the
Lambs, he had not escaped the vengeance of a higher power. In a great
storm he fled for shelter under an oak, and was there struck dead by
lightning. 
%[436]
\footnote{ Bloody Assizes, Burnet, i. 647; Luttrell's Diary, July
15, 1685; Locke's Western Rebellion; Toulmin's History of Taunton,
edited by Savage.}


The number of those who were thus butchered cannot now be ascertained.
Nine were entered in the parish registers of Taunton: but those
registers contained the names of such only as had Christian burial.
Those who were hanged in chains, and those whose heads and limbs were
sent to the neighbouring villages, must have been much more numerous. It
was believed in London, at the time, that Kirke put a hundred captives
to death during the week which followed the battle. 
%[437]
\footnote{ Luttrell's Diary, July 15, 1685; Toulmin's Hist. of
Taunton.}


Cruelty, however, was not this man's only passion. He loved money; and
was no novice in the arts of extortion. A safe conduct might be bought
of him for thirty or forty pounds; and such a safe conduct, though of
no value in law, enabled the purchaser to pass the post of the Lambs
without molestation, to reach a seaport, and to fly to a foreign
country. The ships which were bound for New England were crowded at
this juncture with so many fugitives from Sedgemoor that there was great
danger lest the water and provisions should fail. 
%[438]
\footnote{ Oldmixon, 705; Life and Errors of John Dunton, chap.
vii.}


Kirke was also, in his own coarse and ferocious way, a man of pleasure;
and nothing is more probable than that he employed his power for the
purpose of gratifying his licentious appetites. It was reported that he
conquered the virtue of a beautiful woman by promising to spare the
life of one to whom she was strongly attached, and that, after she had
yielded, he showed her suspended on the gallows the lifeless remains of
him for whose sake she had sacrificed her honour. This tale an impartial
judge must reject. It is unsupported by proof. The earliest authority
for it is a poem written by Pomfret. The respectable historians of that
age, while they speak with just severity of the crimes of Kirke, either
omit all mention of this most atrocious crime, or mention it as a thing
rumoured but not proved. Those who tell the story tell it with such
variations as deprive it of all title to credit. Some lay the scene at
Taunton, some at Exeter. Some make the heroine of the tale a maiden,
some a married woman. The relation for whom the shameful ransom was paid
is described by some as her father, by some as her brother, and by some
as her husband. Lastly the story is one which, long before Kirke was
born, had been told of many other oppressors, and had become a favourite
theme of novelists and dramatists. Two politicians of the fifteenth
century, Rhynsault, the favourite of Charles the Bold of Burgundy, and
Oliver le Dain, the favourite of Lewis the Eleventh of France, had been
accused of the same crime. Cintio had taken it for the subject of a
romance. Whetstone had made out of Cintio's narrative the rude play of
Promos and Cassandra; and Shakspeare had borrowed from Whetstone the
plot of the noble tragicomedy of Measure for Measure. As Kirke was not
the first so he was not the last, to whom this excess of wickedness
was popularly imputed. During the reaction which followed the Jacobin
tyranny in France, a very similar charge was brought against Joseph
Lebon, one of the most odious agents of the Committee of Public
Safety, and, after enquiry, was admitted even by his prosecutors to be
unfounded. 
%[439]
\footnote{ The silence of Whig writers so credulous and so
malevolent as Oldmixon and the compilers of the Western Martyrology
would alone seem to me to settle the question. It also deserves to be
remarked that the story of Rhynsault is told by Steele in the Spectator,
No. 491. Surely it is hardly possible to believe that, if a crime
exactly resembling that of Rhynsault had been committed within living
memory in England by an officer of James the Second, Steele, who was
indiscreetly and unseasonably forward to display his Whiggism, would
have made no allusion to that fact. For the case of Lebon, see the
Moniteur, 4 Messidor, l'an 3.}


The government was dissatisfied with Kirke, not on account of the
barbarity with which he had treated his needy prisoners, but on account
of the interested lenity which he had shown to rich delinquents. 
%[440]
\footnote{ Sunderland to Kirke, July 14 and 28, 1685. "His Majesty,"
says Sunderland, "commands me to signify to you his dislike of these
proceedings, and desires you to take care that no person concerned in
the rebellion be at large." It is but just to add that, in the same
letter, Kirke is blamed for allowing his soldiers to live at free
quarter.}

He was soon recalled from the West. A less irregular and more cruel
massacre was about to be perpetrated. The vengeance was deferred during
some weeks. It was thought desirable that the Western Circuit should not
begin till the other circuits had terminated. In the meantime the gaols
of Somersetshire and Dorsetshire were filled with thousands of captives.
The chief friend and protector of these unhappy men in their extremity
was one who abhorred their religious and political opinions, one whose
order they hated, and to whom they had done unprovoked wrong, Bishop
Ken. That good prelate used all his influence to soften the gaolers, and
retrenched from his own episcopal state that he might be able to make
some addition to the coarse and scanty fare of those who had defaced his
beloved Cathedral. His conduct on this occasion was of a piece with his
whole life. His intellect was indeed darkened by many superstitions and
prejudices: but his moral character, when impartially reviewed, sustains
a comparison with any in ecclesiastical history, and seems to approach,
as near as human infirmity permits, to the ideal perfection of Christian
virtue. 
%[441]
\footnote{ I should be very glad if I could give credit to the
popular story that Ken, immediately after the battle of Sedgemoor,
represented to the chiefs of the royal army the illegality of military
executions. He would, I doubt not, have exerted all his influence on
the side of law and of mercy, if he had been present. But there is no
trustworthy evidence that he was then in the West at all. Indeed what we
know about his proceedings at this time amounts very nearly to proof of
an alibi. It is certain from the Journals of the House of Lords that,
on the Thursday before the battle, he was at Westminster, it is equally
certain that, on the Monday after the battle, he was with Monmouth in
the Tower; and, in that age, a journey from London to Bridgewater and
back again was no light thing.}


His labour of love was of no long duration. A rapid and effectual gaol
delivery was at hand. Early in September, Jeffreys, accompanied by four
other judges, set out on that circuit of which the memory will last as
long as our race and language. The officers who commanded the troops
in the districts through which his course lay had orders to furnish him
with whatever military aid he might require. His ferocious temper needed
no spur; yet a spur was applied. The health and spirits of the Lord
Keeper had given way. He had been deeply mortified by the coldness
of the King and by the insolence of the Chief Justice, and could find
little consolation in looking back on a life, not indeed blackened
by any atrocious crime, but sullied by cowardice, selfishness, and
servility. So deeply was the unhappy man humbled that, when he appeared
for the last time in Westminster Hall he took with him a nosegay to hide
his face, because, as he afterwards owned, he could not bear the eyes of
the bar and of the audience. The prospect of his approaching end seems
to have inspired him with unwonted courage. He determined to discharge
his conscience, requested an audience of the King, spoke earnestly
of the dangers inseparable from violent and arbitrary counsels, and
condemned the lawless cruelties which the soldiers had committed in
Somersetshire. He soon after retired from London to die. He breathed
his last a few days after the Judges set out for the West. It was
immediately notified to Jeffreys that he might expect the Great Seal as
the reward of faithful and vigorous service. 
%[442]
\footnote{ North's Life of Guildford, 260, 263, 273; Mackintosh's
View of the Reign of James the Second, page 16, note; Letter of Jeffreys
to Sunderland, Sept. 5, 1685.}


At Winchester the Chief Justice first opened his commission. Hampshire
had not been the theatre of war; but many of the vanquished rebels
had, like their leader, fled thither. Two of them, John Hickes, a
Nonconformist divine, and Richard Nelthorpe, a lawyer who had been
outlawed for taking part in the Rye House plot, had sought refuge at
the house of Alice, widow of John Lisle. John Lisle had sate in the Long
Parliament and in the High Court of Justice, had been a commissioner of
the Great Seal in the days of the Commonwealth and had been created
a Lord by Cromwell. The titles given by the Protector had not been
recognised by any government which had ruled England since the downfall
of his house; but they appear to have been often used in conversation
even by Royalists. John Lisle's widow was therefore commonly known as
the Lady Alice. She was related to many respectable, and to some noble,
families; and she was generally esteemed even by the Tory gentlemen of
her country. For it was well known to them that she had deeply regretted
some violent acts in which her husband had borne a part, that she had
shed bitter tears for Charles the First, and that she had protected and
relieved many Cavaliers in their distress. The same womanly kindness,
which had led her to befriend the Royalists in their time of trouble,
would not suffer her to refuse a meal and a hiding place to the wretched
men who now entreated her to protect them. She took them into her house,
set meat and drink before them, and showed them where they might take
rest. The next morning her dwelling was surrounded by soldiers. Strict
search was made. Hickes was found concealed in the malthouse, and
Nelthorpe in the chimney. If Lady Alice knew her guests to have been
concerned in the insurrection, she was undoubtedly guilty of what in
strictness was a capital crime. For the law of principal and accessory,
as respects high treason, then was, and is to this day, in a state
disgraceful to English jurisprudence. In cases of felony, a distinction
founded on justice and reason, is made between the principal and the
accessory after the fact. He who conceals from justice one whom he knows
to be a murderer is liable to punishment, but not to the punishment of
murder. He, on the other hand, who shelters one whom he knows to be a
traitor is, according to all our jurists, guilty of high treason. It
is unnecessary to point out the absurdity and cruelty of a law which
includes under the same definition, and visits with the same penalty,
offences lying at the opposite extremes of the scale of guilt. The
feeling which makes the most loyal subject shrink from the thought of
giving up to a shameful death the rebel who, vanquished, hunted down,
and in mortal agony, begs for a morsel of bread and a cup of water, may
be a weakness; but it is surely a weakness very nearly allied to
virtue, a weakness which, constituted as human beings are, we can hardly
eradicate from the mind without eradicating many noble and benevolent
sentiments. A wise and good ruler may not think it right to sanction
this weakness; but he will generally connive at it, or punish it very
tenderly. In no case will he treat it as a crime of the blackest dye.
Whether Flora Macdonald was justified in concealing the attainted heir
of the Stuarts, whether a brave soldier of our own time was justified in
assisting the escape of Lavalette, are questions on which casuists
may differ: but to class such actions with the crimes of Guy Faux and
Fieschi is an outrage to humanity and common sense. Such, however, is
the classification of our law. It is evident that nothing but a lenient
administration could make such a state of the law endurable. And it is
just to say that, during many generations, no English government,
save one, has treated with rigour persons guilty merely of harbouring
defeated and flying insurgents. To women especially has been granted,
by a kind of tacit prescription, the right of indulging in the midst of
havoc and vengeance, that compassion which is the most endearing of
all their charms. Since the beginning of the great civil war, numerous
rebels, some of them far more important than Hickes or Nelthorpe, have
been protected from the severity of victorious governments by female
adroitness and generosity. But no English ruler who has been thus
baffled, the savage and implacable James alone excepted, has had the
barbarity even to think of putting a lady to a cruel and shameful death
for so venial and amiable a transgression.

Odious as the law was, it was strained for the purpose of destroying
Alice Lisle. She could not, according to the doctrine laid down by the
highest authority, be convicted till after the conviction of the rebels
whom she had harboured. 
%[443]
\footnote{ See the preamble of the Act of Parliament reversing her
attainder.}
 She was, however, set to the bar before
either Hickes or Nelthorpe had been tried. It was no easy matter in such
a case to obtain a verdict for the crown. The witnesses prevaricated.
The jury, consisting of the principal gentlemen of Hampshire, shrank
from the thought of sending a fellow creature to the stake for conduct
which seemed deserving rather of praise than of blame. Jeffreys was
beside himself with fury. This was the first case of treason on the
circuit; and there seemed to be a strong probability that his prey would
escape him. He stormed, cursed, and swore in language which no wellbred
man would have used at a race or a cockfight. One witness named Dunne,
partly from concern for Lady Alice, and partly from fright at the
threats and maledictions of the Chief Justice, entirely lost his head,
and at last stood silent. "Oh how hard the truth is," said Jeffreys, "to
come out of a lying Presbyterian knave." The witness, after a pause
of some minutes, stammered a few unmeaning words. "Was there ever,"
exclaimed the judge, with an oath, "was there ever such a villain on
the face of the earth? Dost thou believe that there is a God? Dost thou
believe in hell fire. Of all the witnesses that I ever met with I never
saw thy fellow." Still the poor man, scared out of his senses, remained
mute; and again Jeffreys burst forth. "I hope, gentlemen of the jury,
that you take notice of the horrible carriage of this fellow. How can
one help abhorring both these men and their religion? A Turk is a saint
to such a fellow as this. A Pagan would be ashamed of such villany. Oh
blessed Jesus! What a generation of vipers do we live among!" "I cannot
tell what to say, my Lord," faltered Dunne. The judge again broke forth
into a volley of oaths. "Was there ever," he cried, "such an impudent
rascal? Hold the candle to him that we may see his brazen face. You,
gentlemen, that are of counsel for the crown, see that an information
for perjury be preferred against this fellow." After the witnesses had
been thus handled, the Lady Alice was called on for her defence. She
began by saying, what may possibly have been true, that though she
knew Hickes to be in trouble when she took him in, she did not know or
suspect that he had been concerned in the rebellion. He was a divine,
a man of peace. It had, therefore, never occurred to her that he could
have borne arms against the government; and she had supposed that he
wished to conceal himself because warrants were out against him for
field preaching. The Chief Justice began to storm. "But I will tell you.
There is not one of those lying, snivelling, canting Presbyterians but,
one way or another, had a hand in the rebellion. Presbytery has all
manner of villany in it. Nothing but Presbytery could have made Dunne
such a rogue. Show me a Presbyterian; and I'll show thee a lying knave."
He summed up in the same style, declaimed during an hour against Whigs
and Dissenters, and reminded the jury that the prisoner's husband had
borne a part in the death of Charles the First, a fact which had not
been proved by any testimony, and which, if it had been proved, would
have been utterly irrelevant to the issue. The jury retired, and
remained long in consultation. The judge grew impatient. He could not
conceive, he said, how, in so plain a case, they should even have
left the box. He sent a messenger to tell them that, if they did not
instantly return, he would adjourn the court and lock them up all night.
Thus put to the torture, they came, but came to say that they doubted
whether the charge had been made out. Jeffreys expostulated with them
vehemently, and, after another consultation, they gave a reluctant
verdict of Guilty.

On the following morning sentence was pronounced. Jeffreys gave
directions that Alice Lisle should be burned alive that very afternoon.
This excess of barbarity moved the pity and indignation even of the
class which was most devoted to the crown. The clergy of Winchester
Cathedral remonstrated with the Chief Justice, who, brutal as he was,
was not mad enough to risk a quarrel on such a subject with a body so
much respected by the Tory party. He consented to put off the execution
five days. During that time the friends of the prisoner besought James
to be merciful. Ladies of high rank interceded for her. Feversham, whose
recent victory had increased his influence at court, and who, it is
said, had been bribed to take the compassionate side, spoke in her
favour. Clarendon, the King's brother in law, pleaded her cause. But all
was vain. The utmost that could be obtained was that her sentence
should be commuted from burning to beheading. She was put to death on a
scaffold in the marketplace of Winchester, and underwent her fate with
serene courage. 
%[444]
\footnote{ Trial of Alice Lisle in the Collection of State Trials;
Act of the First of William and Mary for annulling and making void
the Attainder of Alice Lisle widow; Burnet, i. 649; Caveat against the
Whigs.}


In Hampshire Alice Lisle was the only victim: but, on the day following
her execution, Jeffreys reached Dorchester, the principal town of the
county in which Monmouth had landed; and the judicial massacre began.
The court was hung, by order of the Chief Justice, with scarlet; and
this innovation seemed to the multitude to indicate a bloody purpose.
It was also rumoured that, when the clergyman who preached the assize
sermon enforced the duty of mercy, the ferocious mouth of the Judge was
distorted by an ominous grin. These things made men augur ill of what
was to follow. 
%[445]
\footnote{ Bloody Assizes.}


More than three hundred prisoners were to be tried. The work seemed
heavy; but Jeffreys had a contrivance for making it light. He let it be
understood that the only chance of obtaining pardon or respite was to
plead guilty. Twenty-nine persons, who put themselves on their country
and were convicted, were ordered to be tied up without delay. The
remaining prisoners pleaded guilty by scores. Two hundred and ninety-two
received sentence of death. The whole number hanged in Dorsetshire
amounted to seventy-four.

From Dorchester Jeffreys proceeded to Exeter. The civil war had barely
grazed the frontier of Devonshire. Here, therefore, comparatively few
persons were capitally punished. Somersetshire, the chief seat of the
rebellion, had been reserved for the last and most fearful vengeance.
In this county two hundred and thirty-three prisoners were in a few
days hanged, drawn, and quartered. At every spot where two roads met,
on every marketplace, on the green of every large village which had
furnished Monmouth with soldiers, ironed corpses clattering in the wind,
or heads and quarters stuck on poles, poisoned the air, and made the
traveller sick with horror. In many parishes the peasantry could not
assemble in the house of God without seeing the ghastly face of a
neighbour grinning at them over the porch. The Chief Justice was all
himself. His spirits rose higher and higher as the work went on. He
laughed, shouted, joked, and swore in such a way that many thought him
drunk from morning to night. But in him it was not easy to distinguish
the madness produced by evil passions from the madness produced by
brandy. A prisoner affirmed that the witnesses who appeared against him
were not entitled to credit. One of them, he said, was a Papist, and
another a prostitute. "Thou impudent rebel," exclaimed the Judge, "to
reflect on the King's evidence! I see thee, villain, I see thee already
with the halter round thy neck." Another produced testimony that he was
a good Protestant. "Protestant!" said Jeffreys; "you mean Presbyterian.
I'll hold you a wager of it. I can smell a Presbyterian forty miles."
One wretched man moved the pity even of bitter Tories. "My Lord,"
they said, "this poor creature is on the parish." "Do not trouble
yourselves," said the Judge, "I will ease the parish of the burden." It
was not only against the prisoners that his fury broke forth. Gentlemen
and noblemen of high consideration and stainless loyalty, who ventured
to bring to his notice any extenuating circumstance, were almost sure
to receive what he called, in the coarse dialect which he had learned in
the pothouses of Whitechapel, a lick with the rough side of his tongue.
Lord Stawell, a Tory peer, who could not conceal his horror at the
remorseless manner in which his poor neighbours were butchered, was
punished by having a corpse suspended in chains at his park gate. 
%[446]
\footnote{ Locke's Western Rebellion.}

In such spectacles originated many tales of terror, which were long told
over the cider by the Christmas fires of the farmers of Somersetshire.
Within the last forty years, peasants, in some districts, well knew the
accursed spots, and passed them unwillingly after sunset. 
%[447]
\footnote{ This I can attest from my own childish recollections.}


Jeffreys boasted that he had hanged more traitors than all his
predecessors together since the Conquest. It is certain that the number
of persons whom he put to death in one month, and in one shire, very
much exceeded the number of all the political offenders who have been
put to death in our island since the Revolution. The rebellions of
1715 and 1745 were of longer duration, of wider extent, and of more
formidable aspect than that which was put down at Sedgemoor. It has
not been generally thought that, either after the rebellion of 1715, or
after the rebellion of 1745, the House of Hanover erred on the side of
clemency. Yet all the executions of 1715 and 1745 added together will
appear to have been few indeed when compared with those which disgraced
the Bloody Assizes. The number of the rebels whom Jeffreys hanged on
this circuit was three hundred and twenty. 
%[448]
\footnote{ Lord Lonsdale says seven hundred; Burnet six hundred. I
have followed the list which the Judges sent to the Treasury, and which
may still be seen there in the letter book of 1685. See the Bloody
Assizes, Locke's Western Rebellion; the Panegyric on Lord Jeffreys;
Burnet, i. 648; Eachard, iii. 775; Oldmixon, 705.}


Such havoc must have excited disgust even if the sufferers had been
generally odious. But they were, for the most part, men of blameless
life, and of high religious profession. They were regarded by
themselves, and by a large proportion of their neighbours, not as
wrongdoers, but as martyrs who sealed with blood the truth of the
Protestant religion. Very few of the convicts professed any repentance
for what they had done. Many, animated by the old Puritan spirit, met
death, not merely with fortitude, but with exultation. It was in vain
that the ministers of the Established Church lectured them on the guilt
of rebellion and on the importance of priestly absolution. The claim of
the King to unbounded authority in things temporal, and the claim of the
clergy to the spiritual power of binding and loosing, moved the bitter
scorn of the intrepid sectaries. Some of them composed hymns in the
dungeon, and chaunted them on the fatal sledge. Christ, they sang while
they were undressing for the butchery, would soon come to rescue Zion
and to make war on Babylon, would set up his standard, would blow his
trumpet, and would requite his foes tenfold for all the evil which had
been inflicted on his servants. The dying words of these men were noted
down: their farewell letters were kept as treasures; and, in this way,
with the help of some invention and exaggeration, was formed a copious
supplement to the Marian martyrology. 
%[449]
\footnote{ Some of the prayers, exhortations, and hymns of the
sufferers will be found in the Bloody Assizes.}


A few eases deserve special mention. Abraham Holmes, a retired officer
of the parliamentary army, and one of those zealots who would own no
king but King Jesus, had been taken at Sedgemoor. His arm had been
frightfully mangled and shattered in the battle; and, as no surgeon was
at hand, the stout old soldier amputated it himself. He was carried
up to London, and examined by the King in Council, but would make no
submission. "I am an aged man," he said, "and what remains to me of life
is not worth a falsehood or a baseness. I have always been a republican;
and I am so still." He was sent back to the West and hanged. The people
remarked with awe and wonder that the beasts which were to drag him to
the gallows became restive and went back. Holmes himself doubted not
that the Angel of the Lord, as in the old time, stood in the way sword
in hand, invisible to human eyes, but visible to the inferior animals.
"Stop, gentlemen," he cried: "let me go on foot. There is more in this
than you think. Remember how the ass saw him whom the prophet could not
see." He walked manfully to the gallows, harangued the people with a
smile, prayed fervently that God would hasten the downfall of Antichrist
and the deliverance of England, and went up the ladder with an apology
for mounting so awkwardly. "You see," he said, "I have but one arm."

%[450]
\footnote{ Bloody Assizes; Locke's Western Rebellion; Lord
Lonsdale's Memoirs; Account of the Battle of Sedgemoor in the Hardwicke
Papers. The story in the Life of James the Second, ii. 43; is not taken
from the King's manuscripts, and sufficiently refutes itself.}


Not less courageously died Christopher Balttiscombe, a young Templar
of good family and fortune, who, at Dorchester, an agreeable provincial
town proud of its taste and refinement, was regarded by all as the
model of a fine gentleman. Great interest was made to save him. It was
believed through the West of England that he was engaged to a young lady
of gentle blood, the sister of the Sheriff, that she threw herself at
the feet of Jeffreys to beg for mercy, and that Jeffreys drove her from
him with a jest so hideous that to repeat it would be an offence
against decency and humanity. Her lover suffered at Lyme piously and
courageously. 
%[451]
\footnote{ Bloody Assizes; Locke's Western Rebellion, Humble
Petition of Widows and Fatherless Children in the West of England;
Panegyric on Lord Jeffreys.}


A still deeper interest was excited by the fate of two gallant brothers,
William and Benjamin Hewling. They were young, handsome, accomplished,
and well connected. Their maternal grandfather was named Kiffin. He was
one of the first merchants in London, and was generally considered as
the head of the Baptists. The Chief Justice behaved to William Hewling
on the trial with characteristic brutality. "You have a grandfather,"
he said, "who deserves to be hanged as richly as you." The poor lad, who
was only nineteen, suffered death with so much meekness and fortitude,
that an officer of the army who attended the execution, and who had made
himself remarkable by rudeness and severity, was strangely melted, and
said, "I do not believe that my Lord Chief Justice himself could be
proof against this." Hopes were entertained that Benjamin would be
pardoned. One victim of tender years was surely enough for one house to
furnish. Even Jeffreys was, or pretended to be, inclined to lenity. The
truth was that one of his kinsmen, from whom he had large expectations,
and whom, therefore, he could not treat as he generally treated
intercessors pleaded strongly for the afflicted family. Time was allowed
for a reference to London. The sister of the prisoner went to Whitehall
with a petition. Many courtiers wished her success; and Churchill, among
whose numerous faults cruelty had no place, obtained admittance for her.
"I wish well to your suit with all my heart," he said, as they stood
together in the antechamber; "but do not flatter yourself with hopes.
This marble,"--and he laid his hand on the chimneypiece,--"is not
harder than the King." The prediction proved true. James was inexorable.
Benjamin Hewling died with dauntless courage, amidst lamentations in
which the soldiers who kept guard round the gallows could not refrain
from joining. 
%[452]
\footnote{ As to the Hewlings, I have followed Kiffin's Memoirs, and
Mr. Hewling Luson's narrative, which will be found in the second edition
of the Hughes Correspondence, vol. ii. Appendix. The accounts in Locke's
Western Rebellion and in the Panegyric on Jeffreys are full of errors.
Great part of the account in the Bloody Assizes was written by Kiffin,
and agrees word for word with his Memoirs.}


Yet those rebels who were doomed to death were less to be pitied than
some of the survivors. Several prisoners to whom Jeffreys was unable to
bring home the charge of high treason were convicted of misdemeanours,
and were sentenced to scourging not less terrible than that which Oates
had undergone. A woman for some idle words, such as had been uttered by
half the women in the districts where the war had raged, was condemned
to be whipped through all the market towns in the county of Dorset. She
suffered part of her punishment before Jeffreys returned to London;
but, when he was no longer in the West, the gaolers, with the humane
connivance of the magistrates, took on themselves the responsibility
of sparing her any further torture. A still more frightful sentence was
passed on a lad named Tutchin, who was tried for seditious words. He
was, as usual, interrupted in his defence by ribaldry and scurrility
from the judgment seat. "You are a rebel; and all your family have been
rebels Since Adam. They tell me that you are a poet. I'll cap verses
with you." The sentence was that the boy should be imprisoned seven
years, and should, during that period, be flogged through every market
town in Dorsetshire every year. The women in the galleries burst into
tears. The clerk of the arraigns stood up in great disorder. "My Lord,"
said he, "the prisoner is very young. There are many market towns in
our county. The sentence amounts to whipping once a fortnight for seven
years." "If he is a young man," said Jeffreys, "he is an old rogue.
Ladies, you do not know the villain as well as I do. The punishment is
not half bad enough for him. All the interest in England shall not alter
it." Tutchin in his despair petitioned, and probably with sincerity,
that he might be hanged. Fortunately for him he was, just at this
conjuncture, taken ill of the smallpox and given over. As it seemed
highly improbable that the sentence would ever be executed, the Chief
Justice consented to remit it, in return for a bribe which reduced the
prisoner to poverty. The temper of Tutchin, not originally very mild,
was exasperated to madness by what he had undergone. He lived to be
known as one of the most acrimonious and pertinacious enemies of the
House of Stuart and of the Tory party. 
%[453]
\footnote{ See Tutchin's account of his own case in the Bloody
Assizes.}


The number of prisoners whom Jeffreys transported was eight hundred and
forty-one. These men, more wretched than their associates who suffered
death, were distributed into gangs, and bestowed on persons who enjoyed
favour at court. The conditions of the gift were that the convicts
should be carried beyond sea as slaves, that they should not be
emancipated for ten years, and that the place of their banishment should
be some West Indian island. This last article was studiously framed for
the purpose of aggravating the misery of the exiles. In New England or
New Jersey they would have found a population kindly disposed to them
and a climate not unfavourable to their health and vigour. It was
therefore determined that they should be sent to colonies where a
Puritan could hope to inspire little sympathy, and where a labourer born
in the temperate zone could hope to enjoy little health. Such was the
state of the slave market that these bondmen, long as was the passage,
and sickly as they were likely to prove, were still very valuable. It
was estimated by Jeffreys that, on an average, each of them, after all
charges were paid, would be worth from ten to fifteen pounds. There was
therefore much angry competition for grants. Some Tories in the West
conceived that they had, by their exertions and sufferings during the
insurrection, earned a right to share in the profits which had been
eagerly snatched up by the sycophants of Whitehall. The courtiers,
however, were victorious. 
%[454]
\footnote{ Sunderland to Jeffreys, Sept. 14, 1685; Jeffreys to the
King, Sept. 19, 1685, in the State Paper Office.}


The misery of the exiles fully equalled that of the negroes who are now
carried from Congo to Brazil. It appears from the best information which
is at present accessible that more than one fifth of those who were
shipped were flung to the sharks before the end of the voyage. The human
cargoes were stowed close in the holds of small vessels. So little space
was allowed that the wretches, many of whom were still tormented by
unhealed wounds, could not all lie down at once without lying on one
another. They were never suffered to go on deck. The hatchway was
constantly watched by sentinels armed with hangers and blunderbusses.
In the dungeon below all was darkness, stench, lamentation, disease
and death. Of ninety-nine convicts who were carried out in one vessel,
twenty-two died before they reached Jamaica, although the voyage was
performed with unusual speed. The survivors when they arrived at their
house of bondage were mere skeletons. During some weeks coarse biscuit
and fetid water had been doled out to them in such scanty measure that
any one of them could easily have consumed the ration which was assigned
to five. They were, therefore, in such a state that the merchant to whom
they had been consigned found it expedient to fatten them before selling
them. 
%[455]
\footnote{ The best account of the sufferings of those rebels
who were sentenced to transportation is to be found in a very curious
narrative written by John Coad, an honest, Godfearing carpenter who
joined Monmouth, was badly wounded at Philip's Norton, was tried by
Jeffreys, and was sent to Jamaica. The original manuscript was kindly
lent to me by Mr. Phippard, to whom it belongs.}


Meanwhile the property both of the rebels who had suffered death, and
of those more unfortunate men who were withering under the tropical sun,
was fought for and torn in pieces by a crowd of greedy informers. By law
a subject attainted of treason forfeits all his substance; and this law
was enforced after the Bloody Assizes with a rigour at once cruel
and ludicrous. The brokenhearted widows and destitute orphans of the
labouring men whose corpses hung at the cross roads were called upon by
the agents of the Treasury to explain what had become of a basket, of a
goose, of a flitch of bacon, of a keg of cider, of a sack of beans, of
a truss of hay. 
%[456]
\footnote{ In the Treasury records of the autumn of 1685 are several
letters directing search to be made for trifles of this sort.}
 While the humbler retainers of the government were
pillaging the families of the slaughtered peasants, the Chief Justice
was fast accumulating a fortune out of the plunder of a higher class of
Whigs. He traded largely in pardons. His most lucrative transaction of
this kind was with a gentleman named Edmund Prideaux. It is certain that
Prideaux had not been in arms against the government; and it is probable
that his only crime was the wealth which he had inherited from his
father, an eminent lawyer who had been high in office under the
Protector. No exertions were spared to make out a case for the crown.
Mercy was offered to some prisoners on condition that they would bear
evidence against Prideaux. The unfortunate man lay long in gaol and
at length, overcome by fear of the gallows, consented to pay fifteen
thousand pounds for his liberation. This great sum was received by
Jeffreys. He bought with it an estate, to which the people gave the name
of Aceldama, from that accursed field which was purchased with the price
of innocent blood. 
%[457]
\footnote{ Commons' Journals, Oct. 9, Nov. 10, Dec 26, 1690;
Oldmixon, 706. Panegyrie on Jeffreys.}


He was ably assisted in the work of extortion by the crew of parasites
who were in the habit of drinking and laughing with him. The office
of these men was to drive hard bargains with convicts under the strong
terrors of death, and with parents trembling for the lives of children.
A portion of the spoil was abandoned by Jeffreys to his agents. To one
of his boon companions, it is said he tossed a pardon for a rich traitor
across the table during a revel. It was not safe to have recourse to any
intercession except that of his creatures, for he guarded his profitable
monopoly of mercy with jealous care. It was even suspected that he sent
some persons to the gibbet solely because they had applied for the royal
clemency through channels independent of him. 
%[458]
\footnote{ Life and Death of Lord Jeffreys; Panegyric on Jeffreys;
Kiffin's Memoirs.}


Some courtiers nevertheless contrived to obtain a small share of this
traffic. The ladies of the Queen's household distinguished themselves
preeminently by rapacity and hardheartedness. Part of the disgrace which
they incurred falls on their mistress: for it was solely on account of
the relation in which they stood to her that they were able to enrich
themselves by so odious a trade; and there can be no question that
she might with a word or a look have restrained them. But in truth she
encouraged them by her evil example, if not by her express approbation.
She seems to have been one of that large class of persons who bear
adversity better than prosperity. While her husband was a subject and an
exile, shut out from public employment, and in imminent danger of being
deprived of his birthright, the suavity and humility of her manners
conciliated the kindness even of those who most abhorred her religion.
But when her good fortune came her good nature disappeared. The meek and
affable Duchess turned out an ungracious and haughty Queen. 
%[459]
\footnote{ Burnet, i 368; Evelyn's Diary, Feb. 4, 1684-5, July 13,
1686. In one of the satires of that time are these lines:

     "When Duchess, she was gentle, mild, and civil;
     When Queen, she proved a raging furious devil."}
 The
misfortunes which she subsequently endured have made her an object of
some interest; but that interest would be not a little heightened if it
could be shown that, in the season of her greatness, she saved, or even
tried to save, one single victim from the most frightful proscription
that England has ever seen. Unhappily the only request that she is known
to have preferred touching the rebels was that a hundred of those who
were sentenced to transportation might be given to her. 
%[460]
\footnote{ Sunderland to Jeffreys, Sept. 14, 1685.}
 The profit
which she cleared on the cargo, after making large allowance for those
who died of hunger and fever during the passage, cannot be estimated
at less than a thousand guineas. We cannot wonder that her attendants
should have imitated her unprincely greediness and her unwomanly
cruelty. They exacted a thousand pounds from Roger Hoare, a merchant
of Bridgewater; who had contributed to the military chest of the rebel
army. But the prey on which they pounced most eagerly was one which it
might have been thought that even the most ungentle natures would have
spared. Already some of the girls who had presented the standard to
Monmouth at Taunton had cruelly expiated their offence. One of them had
been thrown into prison where an infectious malady was raging. She had
sickened and died there. Another had presented herself at the bar before
Jeffreys to beg for mercy. "Take her, gaoler," vociferated the Judge,
with one of those frowns which had often struck terror into stouter
hearts than hers. She burst into tears, drew her hood over her face,
followed the gaoler out of the court, fell ill of fright, and in a few
hours was a corpse. Most of the young ladies, however, who had walked
in the procession were still alive. Some of them were under ten years
of age. All had acted under the orders of their schoolmistress, without
knowing that they were committing a crime. The Queen's maids of honour
asked the royal permission to wring money out of the parents of the
poor children; and the permission was granted. An order was sent down to
Taunton that all these little girls should be seized and imprisoned.
Sir Francis Warre of Hestercombe, the Tory member for Bridgewater, was
requested to undertake the office of exacting the ransom. He was charged
to declare in strong language that the maids of honour would not endure
delay, that they were determined to prosecute to outlawry, unless a
reasonable sum were forthcoming, and that by a reasonable sum was meant
seven thousand pounds. Warre excused himself from taking any part in a
transaction so scandalous. The maids of honour then requested William
Penn to act for them; and Penn accepted the commission. Yet it should
seem that a little of the pertinacious scrupulosity which he had often
shown about taking off his hat would not have been altogether out of
place on this occasion. He probably silenced the remonstrances of his
conscience by repeating to himself that none of the money which he
extorted would go into his own pocket; that if he refused to be
the agent of the ladies they would find agents less humane; that by
complying he should increase his influence at the court, and that his
influence at the court had already enabled him, and still might enable
him, to render great services to his oppressed brethren. The maids of
honour were at last forced to content themselves with less than a third
part of what they had demanded. 
%[461]
\footnote{ Locke's Western Rebellion; Toulmin's History of Taunton,
edited by Savage, Letter of the Duke of Somerset to Sir F. Warre; Letter
of Sunderland to Penn, Feb. 13, 1685-6, from the State Paper Office, in
the Mackintosh Collection. (1848.)---- The letter of Sunderland is as
follows:--

    "Whitehall, Feb. 13, 1685-6.

    "Mr. Penne,

    "Her Majesty's Maids of Honour having acquainted me that they
    design to employ you and Mr. Walden in making a composition with
    the Relations of the Maids of Taunton for the high Misdemeanour
    they have been guilty of, I do at their request hereby let you
    know that His Majesty has been pleased to give their Fines to the
    said Maids of Honour, and therefore recommend it to Mr. Walden
    and you to make the most advantageous composition you can in
    their behalf."

    I am, Sir,

    "Your humble servant,

    "SUNDERLAND."

That the person to whom this letter was addressed was William Penn the
Quaker was not doubted by Sir James Mackintosh who first brought it to
light, or, as far as I am aware, by any other person, till after
the publication of the first part of this History. It has since been
confidently asserted that the letter was addressed to a certain George
Penne, who appears from an old accountbook lately discovered to have
been concerned in a negotiation for the ransom of one of Monmouth's
followers, named Azariah Pinney.---- If I thought that I had committed
an error, I should, I hope, have the honesty to acknowledge it. But,
after full consideration, I am satisfied that Sunderland's letter was
addressed to William Penn.---- Much has been said about the way in which
the name is spelt. The Quaker, we are told, was not Mr. Penne, but
Mr. Penn. I feel assured that no person conversant with the books and
manuscripts of the seventeenth century will attach any importance to
this argument. It is notorious that a proper name was then thought to
be well spelt if the sound were preserved. To go no further than the
persons, who, in Penn's time, held the Great Seal, one of them is
sometimes Hyde and sometimes Hide: another is Jefferies, Jeffries,
Jeffereys, and Jeffreys: a third is Somers, Sommers, and Summers: a
fourth is Wright and Wrighte; and a fifth is Cowper and Cooper. The
Quaker's name was spelt in three ways. He, and his father the Admiral
before him, invariably, as far as I have observed, spelt it Penn; but
most people spelt it Pen; and there were some who adhered to the ancient
form, Penne. For example. William the father is Penne in a letter from
Disbrowe to Thurloe, dated on the 7th of December, 1654; and William the
son is Penne in a newsletter of the 22nd of September, 1688, printed in
the Ellis Correspondence. In Richard Ward's Life and Letters of Henry
More, printed in 1710, the name of the Quaker will be found spelt in all
the three ways, Penn in the index, Pen in page 197, and Penne in page
311. The name is Penne in the Commission which the Admiral carried out
with him on his expedition to the West Indies. Burchett, who became
Secretary to the Admiralty soon after the Revolution, and remained in
office long after the accession of the House of Hannover, always, in his
Naval History, wrote the name Penne. Surely it cannot be thought strange
that an old-fashioned spelling, in which the Secretary of the Admiralty
persisted so late as 1720, should have been used at the office of the
Secretary of State in 1686. I am quite confident that, if the letter
which we are considering had been of a different kind, if Mr. Penne had
been informed that, in consequence of his earnest intercession, the King
had been graciously pleased to grant a free pardon to the Taunton girls,
and if I had attempted to deprive the Quaker of the credit of that
intercession on the ground that his name was not Penne, the very persons
who now complain so bitterly that I am unjust to his memory would
have complained quite as bitterly, and, I must say, with much more
reason.---- I think myself, therefore perfectly justified in considering
the names, Penn and Penne, as the same. To which, then, of the two
persons who bore that name George or William, is it probable that the
letter of the Secretary of State was addressed?---- George was evidently
an adventurer of a very low class. All that we learn about him from the
papers of the Pinney family is that he was employed in the purchase of a
pardon for the younger son of a dissenting minister. The whole sum
which appears to have passed through George's hands on this occasion was
sixty-five pounds. His commission on the transaction must therefore have
been small. The only other information which we have about him, is that
he, some time later, applied to the government for a favour which was
very far from being an honour. In England the Groom Porter of the Palace
had a jurisdiction over games of chance, and made some very dirty gain
by issuing lottery tickets and licensing hazard tables. George appears
to have petitioned for a similar privilege in the American colonies.----
William Penn was, during the reign of James the Second, the most active
and powerful solicitor about the Court. I will quote the words of his
admirer Crose. "Quum autem Pennus tanta gratia plurinum apud regem
valeret, et per id perplures sibi amicos acquireret, illum omnes,
etiam qui modo aliqua notitia erant conjuncti, quoties aliquid a rege
postulandum agendumve apud regem esset, adire, ambire, orare, ut eos
apud regem adjuvaret." He was overwhelmed by business of this kind,
"obrutus negotiationibus curationibusque." His house and the approaches
to it were every day blocked up by crowds of persons who came to request
his good offices; "domus ac vestibula quotidie referta clientium et
suppliccantium." From the Fountainhall papers it appears that his
influence was felt even in the highlands of Scotland. We learn from
himself that, at this time, he was always toiling for others, that he
was a daily suitor at Whitehall, and that, if he had chosen to sell his
influence, he could, in little more than three, years, have put twenty
thousand pounds into his pocket, and obtained a hundred thousand more
for the improvement of the colony of which he was proprietor.---- Such
was the position of these two men. Which of them, then, was the more
likely to be employed in the matter to which Sunderland's letter
related? Was it George or William, an agent of the lowest or of the
highest class? The persons interested were ladies of rank and fashion,
resident at the palace. where George would hardly have been admitted
into an outer room, but where William was every day in the presence
chamber and was frequently called into the closet. The greatest nobles
in the kingdom were zealous and active in the cause of their fair
friends, nobles with whom William lived in habits of familiar
intercourse, but who would hardly have thought George fit company for
their grooms. The sum in question was seven thousand pounds, a sum not
large when compared with the masses of wealth with which William had
constantly to deal, but more than a hundred times as large as the only
ransom which is known to have passed through the hands of George.
These considerations would suffice to raise a strong presumption that
Sunderland's letter was addressed to William, and not to George: but
there is a still stronger argument behind.---- It is most important to
observe that the person to whom this letter was addressed was not the
first person whom the Maids of Honour had requested to act for them.
They applied to him because another person to whom they had previously
applied, had, after some correspondence, declined the office. From
their first application we learn with certainty what sort of person
they wished to employ. If their first application had been made to
some obscure pettifogger or needy gambler, we should be warranted in
believing that the Penne to whom their second application was made was
George. If, on the other hand, their first application was made to a
gentleman of the highest consideration, we can hardly be wrong in saying
that the Penne to whom their second application was made must have been
William. To whom, then, was their first application made? It was to Sir
Francis Warre of Hestercombe, a Baronet and a Member of Parliament. The
letters are still extant in which the Duke of Somerset, the proud Duke,
not a man very likely to have corresponded with George Penne, pressed
Sir Francis to undertake the commission. The latest of those letters
is dated about three weeks before Sunderland's letter to Mr. Penne.
Somerset tells Sir Francis that the town clerk of Bridgewater, whose
name, I may remark in passing, is spelt sometimes Bird and sometimes
Birde, had offered his services, but that those services had been
declined. It is clear, therefore, that the Maids of Honour were desirous
to have an agent of high station and character. And they were right. For
the sum which they demanded was so large that no ordinary jobber could
safely be entrusted with the care of their interests.---- As Sir Francis
Warre excused himself from undertaking the negotiation, it became
necessary for the Maids of Honour and their advisers to choose somebody
who might supply his place; and they chose Penne. Which of the two
Pennes, then, must have been their choice, George, a petty broker to
whom a percentage on sixty-five pounds was an object, and whose highest
ambition was to derive an infamous livelihood from cards and dice, or
William, not inferior in social position to any commoner in the kingdom?
Is it possible to believe that the ladies, who, in January, employed the
Duke of Somerset to procure for them an agent in the first rank of the
English gentry, and who did not think an attorney, though occupying a
respectable post in a respectable corporation, good enough for their
purpose, would, in February, have resolved to trust everything to a
fellow who was as much below Bird as Bird was below Warre?---- But, it
is said, Sunderland's letter is dry and distant; and he never would have
written in such a style to William Penn with whom he was on friendly
terms. Can it be necessary for me to reply that the official
communications which a Minister of State makes to his dearest friends
and nearest relations are as cold and formal as those which he makes to
strangers? Will it be contended that the General Wellesley to whom the
Marquis Wellesley, when Governor of India, addressed so many letters
beginning with "Sir," and ending with "I have the honour to be your
obedient servant,'' cannot possibly have been his Lordship's brother
Arthur?---- But, it is said, Oldmixon tells a different story. According
to him, a Popish lawyer named Brent, and a subordinate jobber, named
Crane, were the agents in the matter of the Taunton girls. Now it is
notorious that of all our historians Oldmixon is the least trustworthy.
His most positive assertion would be of no value when opposed to such
evidence as is furnished by Sunderland's letter, But Oldmixon asserts
nothing positively. Not only does he not assert positively that Brent
and Crane acted for the Maids of Honour; but he does not even assert
positively that the Maids of Honour were at all concerned. He goes
no further than "It was said," and "It was reported." It is plain,
therefore, that he was very imperfectly informed. I do not think it
impossible, however, that there may have been some foundation for the
rumour which he mentions. We have seen that one busy lawyer, named Bird,
volunteered to look after the interest of the Maids of Honour, and that
they were forced to tell him that they did not want his services. Other
persons, and among them the two whom Oldmixon names, may have tried to
thrust themselves into so lucrative a job, and may, by pretending to
interest at Court, have succeeded in obtaining a little money from
terrified families. But nothing can be more clear than that the
authorised agent of the Maids of Honour was the Mr. Penne, to whom the
Secretary of State wrote; and I firmly believe that Mr. Penne to have
been William the Quaker---- If it be said that it is incredible that
so good a man would have been concerned in so bad an affair, I can only
answer that this affair was very far indeed from being the worst in
which he was concerned.---- For those reasons I leave the text, and
shall leave it exactly as it originally stood. (1857.)}


No English sovereign has ever given stronger proof of a cruel nature
than James the Second. Yet his cruelty was not more odious than his
mercy. Or perhaps it may be more correct to say that his mercy and his
cruelty were such that each reflects infamy on the other. Our horror at
the fate of the simple clowns, the young lads, the delicate women, to
whom he was inexorably severe, is increased when we find to whom and for
what considerations he granted his pardon.

The rule by which a prince ought, after a rebellion, to be guided in
selecting rebels for punishment is perfectly obvious. The ringleaders,
the men of rank, fortune, and education, whose power and whose artifices
have led the multitude into error, are the proper objects of severity.
The deluded populace, when once the slaughter on the field of battle
is over, can scarcely be treated too leniently. This rule, so evidently
agreeable to justice and humanity, was not only not observed: it was
inverted. While those who ought to have been spared were slaughtered by
hundreds, the few who might with propriety have been left to the utmost
rigour of the law were spared. This eccentric clemency has perplexed
some writers, and has drawn forth ludicrous eulogies from others. It was
neither at all mysterious nor at all praiseworthy. It may be distinctly
traced in every case either to a sordid or to a malignant motive, either
to thirst for money or to thirst for blood.

In the case of Grey there was no mitigating circumstance. His parts and
knowledge, the rank which he had inherited in the state, and the high
command which he had borne in the rebel army, would have pointed him out
to a just government as a much fitter object of punishment than Alice
Lisle, than William Hewling, than any of the hundreds of ignorant
peasants whose skulls and quarters were exposed in Somersetshire. But
Grey's estate was large and was strictly entailed. He had only a life
interest in his property; and he could forfeit no more interest than he
had. If he died, his lands at once devolved on the next heir. If he
were pardoned, he would be able to pay a large ransom. He was therefore
suffered to redeem himself by giving a bond for forty thousand pounds to
the Lord Treasurer, and smaller sums to other courtiers. 
%[462]
\footnote{ Burnet, i. 646, and Speaker Onslow's note; Clarendon to
Rochester, May 8, 1686.}


Sir John Cochrane had held among the Scotch rebels the same rank which
had been held by Grey in the West of England. That Cochrane should be
forgiven by a prince vindictive beyond all example, seemed incredible.
But Cochrane was the younger son of a rich family; it was therefore only
by sparing him that money could be made out of him. His father, Lord
Dundonald, offered a bribe of five thousand pounds to the priests of the
royal household; and a pardon was granted. 
%[463]
\footnote{ Burnet, i. 634.}


Samuel Storey, a noted sower of sedition, who had been Commissary to the
rebel army, and who had inflamed the ignorant populace of Somersetshire
by vehement harangues in which James had been described as an incendiary
and a poisoner, was admitted to mercy. For Storey was able to give
important assistance to Jeffreys in wringing fifteen thousand pounds out
of Prideaux. 
%[464]
\footnote{ Calamy's Memoirs; Commons' Journals, December 26,1690;
Sunderland to Jeffreys, September 14, 1685; Privy Council Book, February
26, 1685-6.}


None of the traitors had less right to expect favour than Wade,
Goodenough, and Ferguson. These three chiefs of the rebellion had fled
together from the field of Sedgemoor, and had reached the coast in
safety. But they had found a frigate cruising near the spot where they
had hoped to embark. They had then separated. Wade and Goodenough
were soon discovered and brought up to London. Deeply as they had been
implicated in the Rye House plot, conspicuous as they had been among the
chiefs of the Western insurrection, they were suffered to live, because
they had it in their power to give information which enabled the King
to slaughter and plunder some persons whom he hated, but to whom he had
never yet been able to bring home any crime. 
%[465]
\footnote{ Lansdowne MS. 1152; Harl. MS. 6845; London Gazette, July
20, 1685.}


How Ferguson escaped was, and still is, a mystery. Of all the enemies of
the government he was, without doubt, the most deeply criminal. He was
the original author of the plot for assassinating the royal brothers.
He had written that Declaration which, for insolence, malignity, and
mendacity, stands unrivalled even among the libels of those stormy
times. He had instigated Monmouth first to invade the kingdom, and then
to usurp the crown. It was reasonable to expect that a strict search
would be made for the archtraitor, as he was often called; and such a
search a man of so singular an aspect and dialect could scarcely have
eluded. It was confidently reported in the coffee houses of London
that Ferguson was taken, and this report found credit with men who had
excellent opportunities of knowing the truth. The next thing that was
heard of him was that he was safe on the Continent. It was strongly
suspected that he had been in constant communication with the government
against which he was constantly plotting, that he had, while urging his
associates to every excess of rashness sent to Whitehall just so much
information about their proceedings as might suffice to save his own
neck, and that therefore orders had been given to let him escape. 
%[466]
\footnote{ Many writers have asserted, without the slightest
foundation, that a pardon was granted to Ferguson by James. Some have
been so absurd as to cite this imaginary pardon, which, if it were
real would prove only that Ferguson was a court spy, in proof of the
magnanimity and benignity of the prince who beheaded Alice Lisle and
burned Elizabeth Gaunt. Ferguson was not only not specially pardoned,
but was excluded by name from the general pardon published in the
following spring. (London Gazette, March 15, 1685-6.) If, as the public
suspected and as seems probable, indulgence was shown to him; it was
indulgence of which James was, not without reason, ashamed, and which
was, as far as possible, kept secret. The reports which were current in
London at the time are mentioned in the Observator, Aug. 1,1685.---- Sir
John Reresby, who ought to have been well informed, positively affirms
that Ferguson was taken three days after the battle of Sedgemoor. But
Sir John was certainly wrong as to the date, and may therefore have
been wrong as to the whole story. From the London Gazette, and from
Goodenough's confession (Lansdowne MS. 1152), it is clear that, a
fortnight after the battle, Ferguson had not been caught, and was
supposed to be still lurking in England.}


And now Jeffreys had done his work, and returned to claim his reward. He
arrived at Windsor from the West, leaving carnage, mourning, and terror
behind him. The hatred with which he was regarded by the people of
Somersetshire has no parallel in our history. It was not to be quenched
by time or by political changes, was long transmitted from generation to
generation, and raged fiercely against his innocent progeny. When he
had been many years dead, when his name and title were extinct, his
granddaughter, the Countess of Pomfret, travelling along the western
road, was insulted by the populace, and found that she could not safely
venture herself among the descendants of those who had witnessed the
Bloody Assizes. 
%[467]
\footnote{ Granger's Biographical History.}


But at the Court Jeffreys was cordially welcomed. He was a judge after
his master's own heart. James had watched the circuit with interest and
delight. In his drawingroom and at his table he had frequently talked of
the havoc which was making among his disaffected subjects with a glee
at which the foreign ministers stood aghast. With his own hand he had
penned accounts of what he facetiously called his Lord Chief Justice's
campaign in the West. Some hundreds of rebels, His Majesty wrote to the
Hague, had been condemned. Some of them had been hanged: more should
be hanged: and the rest should be sent to the plantations. It was to no
purpose that Ken wrote to implore mercy for the misguided people, and
described with pathetic eloquence the frightful state of his diocese.
He complained that it was impossible to walk along the highways without
seeing some terrible spectacle, and that the whole air of Somersetshire
was tainted with death. The King read, and remained, according to the
saying of Churchill, hard as the marble chimneypieces of Whitehall. At
Windsor the great seal of England was put into the hands of Jeffreys and
in the next London Gazette it was solemnly notified that this honour
was the reward of the many eminent and faithful services which he had
rendered to the crown. 
%[468]
\footnote{ Burnet, i. 648; James to the Prince of Orange, Sept. 10,
and 24, 1685; Lord Lonadale's Memoirs; London Gazette, Oct. 1, 1685.}


At a later period, when all men of all parties spoke with horror of
the Bloody Assizes, the wicked Judge and the wicked King attempted to
vindicate themselves by throwing the blame on each other. Jeffreys, in
the Tower, protested that, in his utmost cruelty, he had not gone beyond
his master's express orders, nay, that he had fallen short of them.
James, at Saint Germain's would willingly have had it believed that his
own inclinations had been on the side of clemency, and that unmerited
obloquy had been brought on him by the violence of his minister. But
neither of these hardhearted men must be absolved at the expense of the
other. The plea set up for James can be proved under his own hand to
be false in fact. The plea of Jeffreys, even if it be true in fact, is
utterly worthless.

The slaughter in the West was over. The slaughter in London was about to
begin. The government was peculiarly desirous to find victims among the
great Whig merchants of the City. They had, in the last reign, been a
formidable part of the strength of the opposition. They were wealthy;
and their wealth was not, like that of many noblemen and country
gentlemen, protected by entail against forfeiture. In the case of Grey
and of men situated like him, it was impossible to gratify cruelty and
rapacity at once; but a rich trader might be both hanged and plundered.
The commercial grandees, however, though in general hostile to Popery
and to arbitrary power, had yet been too scrupulous or too timid to
incur the guilt of high treason. One of the most considerable among them
was Henry Cornish. He had been an Alderman under the old charter of
the City, and had filled the office of Sheriff when the question of the
Exclusion Bill occupied the public mind. In politics he was a Whig: his
religious opinions leaned towards Presbyterianism: but his temper was
cautious and moderate. It is not proved by trustworthy evidence that he
ever approached the verge of treason. He had, indeed, when Sheriff, been
very unwilling to employ as his deputy a man so violent and unprincipled
as Goodenough. When the Rye House plot was discovered, great hopes
were entertained at Whitehall that Cornish would appear to have been
concerned: but these hopes were disappointed. One of the conspirators,
indeed, John Rumsey, was ready to swear anything: but a single witness
was not sufficient; and no second witness could be found. More than two
years had since elapsed. Cornish thought himself safe; but the eye of
the tyrant was upon him. Goodenough, terrified by the near prospect
of death, and still harbouring malice on account of the unfavourable
opinion which had always been entertained of him by his old master,
consented to supply the testimony which had hitherto been wanting.
Cornish was arrested while transacting business on the Exchange, was
hurried to gaol, was kept there some days in solitary confinement, and
was brought altogether unprepared to the bar of the Old Bailey. The case
against him rested wholly on the evidence of Rumsey and Goodenough. Both
were, by their own confession accomplices in the plot with which they
charged the prisoner. Both were impelled by the strongest pressure of
hope end fear to criminate him. Evidence was produced which proved that
Goodenough was also under the influence of personal enmity. Rumsey's
story was inconsistent with the story which he had told when he appeared
as a witness against Lord Russell. But these things were urged in vain.
On the bench sate three judges who had been with Jeffreys in the West;
and it was remarked by those who watched their deportment that they had
come back from the carnage of Taunton in a fierce and excited state. It
is indeed but too true that the taste for blood is a taste which even
men not naturally cruel may, by habit, speedily acquire. The bar and
the bench united to browbeat the unfortunate Whig. The jury, named by a
courtly Sheriff, readily found a verdict of Guilty; and, in spite of the
indignant murmurs of the public, Cornish suffered death within ten days
after he had been arrested. That no circumstance of degradation might
be wanting, the gibbet was set up where King Street meets Cheapside, in
sight of the house where he had long lived in general respect, of the
Exchange where his credit had always stood high, and of the Guildhall
where he had distinguished himself as a popular leader. He died with
courage and with many pious expressions, but showed, by look and
gesture, such strong resentment at the barbarity and injustice with
which he had been treated, that his enemies spread a calumnious report
concerning him. He was drunk, they said, or out of his mind, when he was
turned off. William Penn, however, who stood near the gallows, and whose
prejudice were all on the side of the government, afterwards said that
he could see in Cornish's deportment nothing but the natural indignation
of an innocent man slain under the forms of law. The head of the
murdered magistrate was placed over the Guildhall. 
%[469]
\footnote{ Trial of Cornish in the Collection of State Trials,
Sir J. Hawles's Remarks on Mr. Cornish's Trial; Burnet, i. 651; Bloody
Assizes; Stat. 1 Gul. and Mar.}


Black as this case was, it was not the blackest which disgraced the
sessions of that autumn at the Old Bailey. Among the persons concerned
in the Rye House plot was a man named James Burton. By his own
confession he had been present when the design of assassination was
discussed by his accomplices. When the conspiracy was detected, a reward
was offered for his apprehension. He was saved from death by an ancient
matron of the Baptist persuasion, named Elizabeth Gaunt. This woman,
with the peculiar manners and phraseology which then distinguished her
sect, had a large charity. Her life was passed in relieving the unhappy
of all religious denominations, and she was well known as a constant
visitor of the gaols. Her political and theological opinions, as well as
her compassionate disposition, led her to do everything in her power for
Burton. She procured a boat which took him to Gravesend, where he got
on board of a ship bound for Amsterdam. At the moment of parting she
put into his hand a sum of money which, for her means, was very large.
Burton, after living some time in exile, returned to England with
Monmouth, fought at Sedgemoor, fled to London, and took refuge in the
house of John Fernley, a barber in Whitechapel. Fernley was very poor.
He was besieged by creditors. He knew that a reward of a hundred pounds
had been offered by the government for the apprehension of Burton. But
the honest man was incapable of betraying one who, in extreme peril, had
come under the shadow of his roof. Unhappily it was soon noised abroad
that the anger of James was more strongly excited against those who
harboured rebels than against the rebels themselves. He had publicly
declared that of all forms of treason the hiding of traitors from his
vengeance was the most unpardonable. Burton knew this. He delivered
himself up to the government; and he gave information against Fernley
and Elizabeth Gaunt. They were brought to trial. The villain whose
life they had preserved had the heart and the forehead to appear as
the principal witness against them. They were convicted. Fernley was
sentenced to the gallows, Elizabeth Gaunt to the stake. Even after
all the horrors of that year, many thought it impossible that these
judgments should be carried into execution. But the King was without
pity. Fernley was hanged. Elizabeth Gaunt was burned alive at Tyburn on
the same day on which Cornish suffered death in Cheapside. She left a
paper written, indeed, in no graceful style, yet such as was read by
many thousands with compassion and horror. "My fault," she said, "was
one which a prince might well have forgiven. I did but relieve a poor
family; and lo! I must die for it." She complained of the insolence of
the judges, of the ferocity of the gaoler, and of the tyranny of him,
the great one of all, to whose pleasure she and so many other victims
had been sacrificed. In so far as they had injured herself, she forgave
them: but, in that they were implacable enemies of that good cause which
would yet revive and flourish, she left them to the judgment of the King
of Kings. To the last she preserved a tranquil courage, which reminded
the spectators of the most heroic deaths of which they had read in Fox.
William Penn, for whom exhibitions which humane men generally avoid seem
to have had a strong attraction, hastened from Cheapside, where he had
seen Cornish hanged, to Tyburn, in order to see Elizabeth Gaunt burned.
He afterwards related that, when she calmly disposed the straw about her
in such a manner as to shorten her sufferings, all the bystanders burst
into tears. It was much noticed that, while the foulest judicial murder
which had disgraced even those times was perpetrating, a tempest burst
forth, such as had not been known since that great hurricane which had
raged round the deathbed of Oliver. The oppressed Puritans reckoned up,
not without a gloomy satisfaction the houses which had been blown down,
and the ships which had been cast away, and derived some consolation
from thinking that heaven was bearing awful testimony against the
iniquity which afflicted the earth. Since that terrible day no woman has
suffered death in England for any political offence. 
%[470]
\footnote{ Trials of Fernley and Elizabeth Gaunt, in the Collection
of State Trials Burnet, i. 649; Bloody Assizes; Sir J. Bramston's
Memoirs; Luttrell's Diary, Oct. 23, 1685.}


It was not thought that Goodenough had yet earned his pardon. The
government was bent on destroying a victim of no high rank, a surgeon in
the City, named Bateman. He had attended Shaftesbury professionally, and
had been a zealous Exclusionist. He may possibly have been privy to the
Whig plot; but it is certain that he had not been one of the leading
conspirators; for, in the great mass of depositions published by the
government, his name occurs only once, and then not in connection with
any crime bordering on high treason. From his indictment, and from the
scanty account which remains of his trial, it seems clear that he was
not even accused of participating in the design of murdering the royal
brothers. The malignity with which so obscure a man, guilty of so slight
an offence, was hunted down, while traitors far more criminal and
far more eminent were allowed to ransom themselves by giving evidence
against him, seemed to require explanation; and a disgraceful
explanation was found. When Oates, after his scourging, was carried into
Newgate insensible, and, as all thought, in the last agony, he had been
bled and his wounds had been dressed by Bateman. This was an offence not
to be forgiven. Bateman was arrested and indicted. The witnesses against
him were men of infamous character, men, too, who were swearing for
their own lives. None of them had yet got his pardon; and it was a
popular saying, that they fished for prey, like tame cormorants, with
ropes round their necks. The prisoner, stupefied by illness, was unable
to articulate, or to understand what passed. His son and daughter stood
by him at the bar. They read as well as they could some notes which he
had set down, and examined his witnesses. It was to little purpose. He
was convicted, hanged, and quartered. 
%[471]
\footnote{ Bateman's Trial in the Collection of State Trials;
Sir John Hawles's Remarks. It is worth while to compare Thomas Lee's
evidence on this occasion with his confession previously published by
authority.}


Never, not even under the tyranny of Laud, had the condition of the
Puritans been so deplorable as at that time. Never had spies been so
actively employed in detecting congregations. Never had magistrates,
grand jurors, rectors and churchwardens been so much on the alert. Many
Dissenters were cited before the ecclesiastical courts. Others found it
necessary to purchase the connivance of the agents of the government by
presents of hogsheads of wine, and of gloves stuffed with guineas. It
was impossible for the separatists to pray together without precautions
such as are employed by coiners and receivers of stolen goods. The
places of meeting were frequently changed. Worship was performed
sometimes just before break of day and sometimes at dead of night. Round
the building where the little flock was gathered sentinels were posted
to give the alarm if a stranger drew near. The minister in disguise was
introduced through the garden and the back yard. In some houses there
were trap doors through which, in case of danger, he might descend.
Where Nonconformists lived next door to each other, the walls were often
broken open, and secret passages were made from dwelling to dwelling. No
psalm was sung; and many contrivances were used to prevent the voice
of the preacher, in his moments of fervour, from being heard beyond the
walls. Yet, with all this care, it was often found impossible to elude
the vigilance of informers. In the suburbs of London, especially, the
law was enforced with the utmost rigour. Several opulent gentlemen were
accused of holding conventicles. Their houses were strictly searched,
and distresses were levied to the amount of many thousands of pounds.
The fiercer and bolder sectaries, thus driven from the shelter of roofs,
met in the open air, and determined to repel force by force. A Middlesex
justice who had learned that a nightly prayer meeting was held in a
gravel pit about two miles from London, took with him a strong body of
constables, broke in upon the assembly, and seized the preacher. But
the congregation, which consisted of about two hundred men, soon rescued
their pastor and put the magistrate and his officers to flight. 
%[472]
\footnote{ Van Citters, Oct. 13-23, 1685.}

This, however, was no ordinary occurrence. In general the Puritan spirit
seemed to be more effectually cowed at this conjuncture than at any
moment before or since. The Tory pamphleteers boasted that not one
fanatic dared to move tongue or pen in defence of his religious
opinions. Dissenting ministers, however blameless in life, however
eminent for learning and abilities, could not venture to walk the
streets for fear of outrages, which were not only not repressed, but
encouraged, by those whose duty it was to preserve the peace. Some
divines of great fame were in prison. Among these was Richard Baxter.
Others, who had, during a quarter of a century, borne up against
oppression, now lost heart, and quitted the kingdom. Among these was
John Howe. Great numbers of persons who had been accustomed to frequent
conventicles repaired to the parish churches. It was remarked that the
schismatics who had been terrified into this show of conformity might
easily be distinguished by the difficulty which they had in finding out
the collect, and by the awkward manner in which they bowed at the name
of Jesus. 
%[473]
\footnote{ Neal's History of the Puritans, Calamy's Account of the
ejected Ministers and the Nonconformists' Memorial contain abundant
proofs of the severity of this persecution. Howe's farewell letter to
his flock will be found in the interesting life of that great man, by
Mr. Rogers. Howe complains that he could not venture to show himself in
the streets of London, and that his health had suffered from want of
air and exercise. But the most vivid picture of the distress of the
Nonconformists is furnished by their deadly enemy, Lestrange, in the
Observators of September and October, 1685.}


Through many years the autumn of 1685 was remembered by the
Nonconformists as a time of misery and terror. Yet in that autumn might
be discerned the first faint indications of a great turn of fortune;
and before eighteen months had elapsed, the intolerant King and the
intolerant Church were eagerly bidding against each other for the
support of the party which both had so deeply injured.

END OF VOL. I.

*****



% [Footnote 1: In this, and in the next chapter, I have very seldom
% thought it necessary to cite authorities: for, in these chapters, I have
% not detailed events minutely, or used recondite materials; and the facts
% which I mention are for the most part such that a person tolerably well
% read in English history, if not already apprised of them, will at least
% know where to look for evidence of them. In the subsequent chapters I
% shall carefully indicate the sources of my information.]

% [Footnote 2: This is excellently put by Mr. Hallam in the first chapter
% of his Constitutional History.]

% [Footnote 3: See a very curious paper which Strype believed to be in
% Gardiner's handwriting. Ecclesiastical Memorials, Book 1., Chap. xvii.]

% [Footnote 4: These are Cranmer's own words. See the Appendix to Burnet's
% History of the Reformation, Part 1. Book III. No. 21. Question 9.]

% [Footnote 5: The Puritan historian, Neal, after censuring the cruelty
% with which she treated the sect to which he belonged, concludes thus:
% "However, notwithstanding all these blemishes, Queen Elizabeth stands
% upon record as a wise and politic princess, for delivering her kingdom
% from the difficulties in which it was involved at her accession, for
% preserving the Protestant reformation against the potent attempts of the
% Pope, the Emperor, and King of Spain abroad, and the Queen of Scots and
% her Popish subjects at home.... She was the glory of the age in which
% she lived, and will be the admiration of posterity."--History of the
% Puritans, Part I. Chap. viii.]

% [Footnote 6: On this subject, Bishop Cooper's language is remarkably
% clear and strong. He maintains, in his Answer to Martin Marprelate,
% printed in 1589, "that no form of church government is divinely
% ordained; that Protestant communities, in establishing different forms,
% have only made a legitimate use of their Christian liberty; and
% that episcopacy is peculiarly suited to England, because the English
% constitution is monarchical." All those Churches," says the Bishop,
% "in which the Gospell, in these daies, after great darknesse, was first
% renewed, and the learned men whom God sent to instruct them, I doubt not
% but have been directed by the Spirite of God to retaine this liberty,
% that, in external government and other outward orders; they might choose
% such as they thought in wisedome and godlinesse to be most convenient
% for the state of their countrey and disposition of their people. Why
% then should this liberty that other countreys have used under anie
% colour be wrested from us? I think it therefore great presumption and
% boldnesse that some of our nation, and those, whatever they may think
% of themselves, not of the greatest wisedome and skill, should take upon
% them to controlle the whole realme, and to binde both prince and people
% in respect of conscience to alter the present state, and tie themselves
% to a certain platforme devised by some of our neighbours, which, in the
% judgment of many wise and godly persons, is most unfit for the state of
% a Kingdome."]

% [Footnote 7: Strype's Life of Grindal, Appendix to Book II. No. xvii.]

% [Footnote 8: Canon 55, of 1603.]

% [Footnote 9: Joseph Hall, then dean of Worcester, and afterwards bishop
% of Norwich, was one of the commissioners. In his life of himself, he
% says: "My unworthiness was named for one of the assistants of that
% honourable, grave, and reverend meeting." To high churchmen this
% humility will seem not a little out of place.]

% [Footnote 10: It was by the Act of Uniformity, passed after the
% Restoration, that persons not episcopally ordained were, for the first
% time, made incapable of holding benefices. No man was more zealous for
% this law than Clarendon. Yet he says: "This was new; for there had been
% many, and at present there were some, who possessed benefices with cure
% of souls and other ecclesiastical promotions, who had never received
% orders but in France or Holland; and these men must now receive new
% ordination, which had been always held unlawful in the Church, or by
% this act of parliament must be deprived of their livelihood which they
% enjoyed in the most flourishing and peaceable time of the Church."]

% [Footnote 11: Peckard's Life of Ferrar; The Arminian Nunnery, or a Brief
% Description of the late erected monastical Place called the Arminian
% Nunnery, at Little Gidding in Huntingdonshire, 1641.]

% [Footnote 12: The correspondence of Wentworth seems to me fully to bear
% out what I have said in the text. To transcribe all the passages
% which have led me to the conclusion at which I have arrived, would be
% impossible, nor would it be easy to make a better selection than has
% already been made by Mr. Hallam. I may, however direct the attention of
% the reader particularly to the very able paper which Wentworth drew up
% respecting the affairs of the Palatinate. The date is March 31, 1637.]

% [Footnote 13: These are Wentworth's own words. See his letter to Laud,
% dated Dec. 16, 1634.]

% [Footnote 14: See his report to Charles for the year 1639.]

% [Footnote 15: See his letter to the Earl of Northumberland, dated July
% 30, 1638.]

% [Footnote 16: How little compassion for the bear had to do with the
% matter is sufficiently proved by the following extract from a paper
% entitled A perfect Diurnal of some Passages of Parliament, and from
% other Parts of the Kingdom, from Monday July 24th, to Monday July 31st,
% 1643. "Upon the Queen's coming from Holland, she brought with her,
% besides a company of savage-like ruffians, a company of savage bears,
% to what purpose you may judge by the sequel. Those bears were left about
% Newark, and were brought into country towns constantly on the Lord's
% day to be baited, such is the religion those here related would settle
% amongst us; and, if any went about to hinder or but speak against their
% damnable profanations, they were presently noted as Roundheads
% and Puritans, and sure to be plundered for it. But some of Colonel
% Cromwell's forces coming by accident into Uppingham town, in Rutland,
% on the Lord's day, found these bears playing there in the usual manner,
% and, in the height of their sport, caused them to be seized upon, tied
% to a tree and shot." This was by no means a solitary instance. Colonel
% Pride, when Sheriff of Surrey, ordered the beasts in the bear garden
% of Southwark to be killed. He is represented by a loyal satirist as
% defending the act thus: "The first thing that is upon my spirits is the
% killing of the bears, for which the people hate me, and call me all the
% names in the rainbow. But did not David kill a bear? Did not the Lord
% Deputy Ireton kill a bear? Did not another lord of ours kill five
% bears?"-Last Speech and Dying Words of Thomas pride.]

% [Footnote 17: See Penn's New Witnesses proved Old Heretics, and
% Muggleton's works, passim.]

% [Footnote 18: I am happy to say, that, since this passage was written,
% the territories both of the Rajah of Nagpore and of the King of Oude
% have been added to the British dominions. (1857.)]

% [Footnote 19: The most sensible thing said in the House of Commons, on
% this subject, came from Sir William Coventry: "Our ancestors never did
% draw a line to circumscribe prerogative and liberty."]

% [Footnote 20: Halifax was undoubtedly the real author of the Character
% of a Trimmer, which, for a time, went under the name of his kinsman, Sir
% William Coventry.]

% [Footnote 21: North's Examen, 231, 574.]

% [Footnote 22: A peer who was present has described the effect of
% Halifax's oratory in words which I will quote, because, though they
% have been long in print, they are probably known to few even of the
% most curious and diligent readers of history. "Of powerful eloquence and
% great parts were the Duke's enemies who did assert the Bill; but a noble
% Lord appeared against it who, that day, in all the force of speech, in
% reason, in arguments of what could concern the public or the private
% interests of men, in honour, in conscience, in estate, did outdo himself
% and every other man; and in fine his conduct and his parts were
% both victorious, and by him all the wit and malice of that party was
% overthrown." This passage is taken from a memoir of Henry Earl of
% Peterborough, in a volume entitled "Succinct Genealogies, by Robert
% Halstead," fol. 1685. The name of Halstead is fictitious. The real
% authors were the Earl of Peterborough himself and his chaplain. The book
% is extremely rare. Only twenty-four copies were printed, two of which
% are now in the British Museum. Of these two one belonged to George the
% Fourth, and the other to Mr. Grenville.]

% [Footnote 23: This is mentioned in the curious work entitled "Ragguaglio
% della solenne Comparsa fatta in Roma gli otto di Gennaio, 1687, dall'
% illustrissimo et eccellentissimo signor Conte di Castlemaine."]

% [Footnote 24: North's Examen, 69.]

% [Footnote 25: Lord Preston, who was envoy at Paris, wrote thence to
% Halifax as follows: "I find that your Lordship lies still under the same
% misfortune of being no favourite to this court; and Monsieur Barillon
% dare not do you the honor to shine upon you, since his master frowneth.
% They know very well your lordship's qualifications which make them
% fear and consequently hate you; and be assured, my lord, if all their
% strength can send you to Rufford, it shall be employed for that end. Two
% things, I hear, they particularly object against you, your secrecy, and
% your being incapable of being corrupted. Against these two things I know
% they have declared." The date of the letter is October 5, N. S. 1683]

% [Footnote 26: During the interval which has elapsed since this chapter
% was written, England has continued to advance rapidly in material
% prosperity, I have left my text nearly as it originally stood; but I
% have added a few notes which may enable the reader to form some notion
% of the progress which has been made during the last nine years; and,
% in general, I would desire him to remember that there is scarcely a
% district which is not more populous, or a source of wealth which is not
% more productive, at present than in 1848. (1857.)]

% [Footnote 27: Observations on the Bills of Mortality, by Captain John
% Graunt (Sir William Petty), chap. xi.]

% [Footnote 28:
% 
%     "She doth comprehend
%      Full fifteen hundred thousand which do spend
%      Their days within."
% 
%      --Great Britain's Beauty, 1671.]

% [Footnote 29: Isaac Vossius, De Magnitudine Urbium Sinarum, 1685.
% Vossius, as we learn from Saint Evremond, talked on this subject oftener
% and longer than fashionable circles cared to listen.]

% [Footnote 30: King's Natural and Political Observations, 1696 This
% valuable treatise, which ought to be read as the author wrote it, and
% not as garbled by Davenant, will be found in some editions of Chalmers's
% Estimate.]

% [Footnote 31: Dalrymple's Appendix to Part II. Book I, The practice of
% reckoning the population by sects was long fashionable. Gulliver says
% of the King of Brobdignag; "He laughed at my odd arithmetic, as he
% was pleased to call it, in reckoning the numbers of our people by
% a computation drawn from the several sects among us in religion and
% politics."]

% [Footnote 32: Preface to the Population Returns of 1831.]

% [Footnote 33: Statutes 14 Car. II. c. 22.; 18 \& 19 Car. II. c. 3., 29 \&
% 30 Car. II. c. 2.]

% [Footnote 34: Nicholson and Bourne, Discourse on the Ancient State of
% the Border, 1777.]

% [Footnote 35: Gray's Journal of a Tour in the Lakes, Oct. 3, 1769.]

% [Footnote 36: North's Life of Guildford; Hutchinson's History of
% Cumberland, Parish of Brampton.]

% [Footnote 37: See Sir Walter Scott's Journal, Oct. 7, 1827, in his Life
% by Mr. Lockhart.]

% [Footnote 38: Dalrymple, Appendix to Part II. Book I. The returns of
% the hearth money lead to nearly the same conclusion. The hearths in the
% province of York were not a sixth of the hearths of England.]

% [Footnote 39: I do not, of course, pretend to strict accuracy here; but
% I believe that whoever will take the trouble to compare the last returns
% of hearth money in the reign of William the Third with the census of
% 1841, will come to a conclusion not very different from mine.]

% [Footnote 40: There are in the Pepysian Library some ballads of that age
% on the chimney money. I will give a specimen or two:
% 
%    "The good old dames whenever they the chimney man espied,
%    Unto their nooks they haste away, their pots and pipkins hide.
%    There is not one old dame in ten, and search the nation through,
%    But, if you talk of chimney men, will spare a curse or two."
% 
%    Again:
% 
%    "Like plundering soldiers they'd enter the door,
%    And make a distress on the goods of the poor.
%    While frighted poor children distractedly cried;
%    This nothing abated their insolent pride."
% 
%    In the British Museum there are doggrel verses composed on the
%    same subject and in the same spirit:
% 
%    "Or, if through poverty it be not paid
%    For cruelty to tear away the single bed,
%    On which the poor man rests his weary head,
%    At once deprives him of his rest and bread."
% 
% I take this opportunity the first which occurs, of acknowledging most
% grateful the kind and liberal manner in which the Master and Vicemaster
% of Magdalei College, Cambridge, gave me access to the valuable
% collections of Pepys.]

% [Footnote 41: My chief authorities for this financial statement will be
% found in the Commons' Journal, March 1, and March 20, 1688-9.]

% [Footnote 42: See, for example, the picture of the mound at Marlborough,
% in Stukeley's Dinerarium Curiosum.]

% [Footnote 43: Chamberlayne's State of England, 1684.]

% [Footnote 44: 13 and 14 Car. II. c. 3; 15 Car. II. c. 4. Chamberlayne's
% State of England, 1684.]

% [Footnote 45: Dryden, in his Cymon and Iphigenia, expressed, with his
% usual keenness and energy, the sentiments which had been fashionable
% among the sycophants of James the Second:--
% 
%      "The country rings around with loud alarms,
%      And raw in fields the rude militia swarms;
%      Mouths without hands, maintained at vast expense,
%      Stout once a month they march, a blustering band,
%      And ever, but in time of need at hand.
%      This was the morn when, issuing on the guard,
%      Drawn up in rank and file, they stood prepared
%      Of seeming arms to make a short essay.
%      Then hasten to be drunk, the business of the day."]

% [Footnote 46: Most of the materials which I have used for this
% account of the regular army will be found in the Historical Records of
% Regiments, published by command of King William the Fourth, and under
% the direction of the Adjutant General. See also Chamberlayne's State of
% England, 1684; Abridgment of the English Military Discipline, printed by
% especial command, 1688; Exercise of Foot, by their Majesties' command,
% 1690.]

% [Footnote 47: I refer to a despatch of Bonrepaux to Seignelay, dated
% Feb. 8/18. 1686. It was transcribed for Mr. Fox from the French
% archives, during the peace of Amiens, and, with the other materials
% brought together by that great man, was entrusted to me by the kindness
% of the late Lady Holland, and of the present Lord Holland. I ought to
% add that, even in the midst of the troubles which have lately agitated
% Paris, I found no difficulty in obtaining, from the liberality of
% the functionaries there, extracts supplying some chasms in Mr. Fox's
% collection. (1848.)]

% [Footnote 48: My information respecting the condition of the navy,
% at this time, is chiefly derived from Pepys. His report, presented to
% Charles the Second in May, 1684, has never, I believe, been printed. The
% manuscript is at Magdalene College Cambridge. At Magdalene College is
% also a valuable manuscript containing a detailed account of the maritime
% establishments of the country in December 1684. Pepys's "Memoirs
% relating to the State of the Royal Navy for Ten Years determined
% December, 1688," and his diary and correspondence during his mission
% to Tangier, are in print. I have made large use of them. See also
% Sheffield's Memoirs, Teonge's Diary, Aubrey's Life of Monk, the Life of
% Sir Cloudesley Shovel, 1708, Commons' Journals, March 1 and March 20.
% 1688-9.]

% [Footnote 49: Chamberlayne's State of England, 1684; Commons' Journals,
% March 1, and March 20, 1688-9. In 1833, it was determined, after full
% enquiry, that a hundred and seventy thousand barrels of gunpowder should
% constantly be kept in store.]

% [Footnote 50: It appears from the records of the Admiralty, that Flag
% officers were allowed half pay in 1668, Captains of first and second
% rates not till 1674.]

% [Footnote 51: Warrant in the War Office Records; dated March 26, 1678.]

% [Footnote 52: Evelyn's Diary. Jan. 27, 1682. I have seen a privy seal,
% dated May 17. 1683, which confirms Evelyn's testimony.]

% [Footnote 53: James the Second sent Envoys to Spain, Sweden, and
% Denmark; yet in his reign the diplomatic expenditure was little more
% than 30,000£. a year. See the Commons' Journals, March 20, 1688-9.
% Chamberlayne's State of England, 1684.]

% [Footnote 54: Carte's Life of Ormond.]

% [Footnote 55: Pepys's Diary, Feb. 14, 1668-9.]

% [Footnote 56: See the Report of the Bath and Montague case, which was
% decided by Lord Keeper Somers, in December, 1693.]

% [Footnote 57: During three quarters of a year, beginning from Christmas,
% 1689, the revenues of the see of Canterbury were received by an officer
% appointed by the crown. That officer's accounts are now in the British
% Museum. (Lansdowne MSS. 885.) The gross revenue for the three quarters
% was not quite four thousand pounds; and the difference between the gross
% and the net revenue was evidently something considerable.]

% [Footnote 58: King's Natural and Political Conclusions. Davenant on
% the Balance of Trade. Sir W. Temple says, "The revenues of a House of
% Commons have seldom exceeded four hundred thousand pounds." Memoirs,
% Third Part.]

% [Footnote 59: Langton's Conversations with Chief Justice Hale, 1672.]

% [Footnote 60: Commons' Journals, April 27,1689; Chamberlayne's State of
% England, 1684.]

% [Footnote 61: See the Travels of the Grand Duke Cosmo.]

% [Footnote 62: King's Natural and Political Conclusions. Davenant on the
% Balance of Trade.]

% [Footnote 63: See the Itinerarium Angliae, 1675, by John Ogilby,
% Cosmographer Royal. He describes great part of the land as wood, fen,
% heath on both sides, marsh on both sides. In some of his maps the roads
% through enclosed country are marked by lines, and the roads through
% unenclosed country by dots. The proportion of unenclosed country, which,
% if cultivated, must have been wretchedly cultivated, seems to have been
% very great. From Abingdon to Gloucester, for example, a distance of
% forty or fifty miles, there was not a single enclosure, and scarcely one
% enclosure between Biggleswade and Lincoln.]

% [Footnote 64: Large copies of these highly interesting drawings are in
% the noble collection bequeathed by Mr. Grenville to the British Museum.
% See particularly the drawings of Exeter and Northampton.]

% [Footnote 65: Evelyn's Diary, June 2, 1675.]

% [Footnote 66: See White's Selborne; Bell's History of British
% Quadrupeds, Gentleman's Recreation, 1686; Aubrey's Natural History
% of Wiltshire, 1685; Morton's History of Northamptonshire, 1712;
% Willoughby's Ornithology, by Ray, 1678; Latham's General Synopsis of
% Birds; and Sir Thomas Browne's Account of Birds found in Norfolk.]

% [Footnote 67: King's Natural and Political Conclusions. Davenant on the
% Balance of Trade.]

% [Footnote 68: See the Almanacks of 1684 and 1685.]

% [Footnote 69: See Mr. M'Culloch's Statistical Account of the British
% Empire, Part III. chap. i. sec. 6.]

% [Footnote 70: King and Davenant as before The Duke of Newcastle on
% Horsemanship; Gentleman's Recreation, 1686. The "dappled Flanders mares"
% were marks of greatness in the time of Pope, and even later. The vulgar
% proverb, that the grey mare is the better horse, originated, I suspect,
% in the preference generally given to the grey mares of Flanders over the
% finest coach horses of England.]

% [Footnote 71: See a curious note by Tonkin, in Lord De Dunstanville's
% edition of Carew's Survey of Cornwall.]

% [Footnote 72: Borlase's Natural History of Cornwall, 1758. The quantity
% of copper now produced, I have taken from parliamentary returns.
% Davenant, in 1700, estimated the annual produce of all the mines of
% England at between seven and eight hundred thousand pounds]

% [Footnote 73: Philosophical Transactions, No. 53. Nov. 1669, No. 66.
% Dec. 1670, No. 103. May 1674, No 156. Feb. 1683-4]

% [Footnote 74: Yarranton, England's Improvement by Sea and Land, 1677;
% Porter's Progress of the Nation. See also a remarkably perspicnous
% history, in small compass of the English iron works, in Mr. M'Culloch's
% Statistical Account of the British Empire.]

% [Footnote 75: See Chamberlayne's State of England, 1684, 1687, Angliae,
% Metropolis, 1691; M'Culloch's Statistical Account of the British Empire
% Part III. chap. ii. (edition of 1847). In 1845 the quantity of coal
% brought into London appeared, by the Parliamentary returns, to be
% 3,460,000 tons. (1848.) In 1854 the quantity of coal brought into London
% amounted to 4,378,000 tons. (1857.)]

% [Footnote 76: My notion of the country gentleman of the seventeenth
% century has been derived from sources too numerous to be recapitulated.
% I must leave my description to the judgment of those who have studied
% the history and the lighter literature of that age.]

% [Footnote 77: In the eighteenth century the great increase in the value
% of benefices produced a change. The younger sons of the nobility were
% allured back to the clerical profession. Warburton in a letter to
% Hurd, dated the 6th of July, 1762, mentions this change, which was then
% recent. "Our grandees have at last found their way back into the Church.
% I only wonder they have been so long about it. But be assured that
% nothing but a new religious revolution, to sweep away the fragments that
% Henry the Eighth left after banqueting his courtiers, will drive them
% out again."]

% [Footnote 78: See Heylin's Cyprianus Anglicus.]

% [Footnote 79: Eachard, Causes of the Contempt of the Clergy; Oldham,
% Satire addressed to a Friend about to leave the University; Tatler, 255,
% 258. That the English clergy were a lowborn class, is remarked in the
% Travels of the Grand Duke Cosmo, Appendix A.]

% [Footnote 80: "A causidico, medicastro, ipsaque artificum farragine,
% ecclesiae rector aut vicarius contemnitur et fit ludibrio. Gentis et
% familiae nitor sacris ordinibus pollutus censetur: foeminisque natalitio
% insignibus unicum inculcatur saepius praeceptum, ne modestiae naufragium
% faciant, aut, (quod idem auribus tam delicatulis sonat,) ne clerico se
% nuptas dari patiantur."--Angliae Notitia, by T. Wood, of New College
% Oxford 1686.]

% [Footnote 81: Clarendon's Life, ii. 21.]

% [Footnote 82: See the injunctions of 1559, In Bishop Sparrow's
% Collection. Jeremy Collier, in his Essay on Pride, speaks of this
% injunction with a bitterness which proves that his own pride had not
% been effectually tamed.]

% [Footnote 83: Roger and Abigail in Fletcher's Scornful Lady, Bull
% and the Nurse in Vanbrugh's Relapse, Smirk and Susan in Shadwell's
% Lancashire Witches, are instances.]

% [Footnote 84: Swift's Directions to Servants. In Swift's Remarks on the
% Clerical Residence Bill, he describes the family of an English vicar
% thus: "His wife is little better than a Goody, in her birth, education,
% or dress..... His daughters shall go to service, or be sent apprentice
% to the sempstress of the next town."]

% [Footnote 85: Even in Tom Jones, published two generations later. Mrs.
% Seagrim, the wife of a gamekeeper, and Mrs. Honour, a waitingwoman,
% boast of their descent from clergymen, "It is to be hoped," says
% Fielding, "such instances will in future ages, when some provision is
% made for the families of the inferior clergy, appear stranger than they
% can be thought at present."]

% [Footnote 86: This distinction between country clergy and town clergy is
% strongly marked by Eachard, and cannot but be observed by every person
% who has studied the ecclesiastical history of that age.]

% [Footnote 87: Nelson's Life of Bull. As to the extreme difficulty which
% the country clergy found in procuring books, see the Life of Thomas
% Bray, the founder of the Society for the Propagation of the Gospel.]

% [Footnote 88: "I have frequently heard him (Dryden) own with pleasure,
% that if he had any talent for English prose it was owing to his having
% often read the writings of the great Archbishop Tillotson."--Congreve's
% Dedication of Dryden's Plays.]

% [Footnote 89: I have taken Davenant's estimate, which is a little lower
% than King's.]

% [Footnote 90: Evelvn's Diary, June 27. 1654; Pepys's Diary, June 13.
% 1668; Roger North's Lives of Lord Keeper Guildford, and of Sir Dudley
% North; Petty's Political Arithmetic. I have taken Petty's facts, but, in
% drawing inferences from them, I have been guided by King and Davenant,
% who, though not abler men than he, had the advantage of coming after
% him. As to the kidnapping for which Bristol was infamous, see North's
% Life of Guildford, 121, 216, and the harangue of Jeffreys on the
% subject, in the Impartial History of his Life and Death, printed with
% the Bloody Assizes. His style was, as usual, coarse, but I cannot reckon
% the reprimand which he gave to the magistrates of Bristol among his
% crimes.]

% [Footnote 91: Fuller's Worthies; Evelyn's Diary, Oct. 17,1671; Journal
% of T. Browne, son of Sir Thomas Browne, Jan. 1663-4; Blomefield's
% History of Norfolk; History of the City and County of Norwich, 2 vols.
% 1768.]

% [Footnote 92: The population of York appears, from the return of
% baptisms and burials in Drake's History, to have been about 13,000 in
% 1730. Exeter had only 17,000 inhabitants in 1801. The population of
% Worcester was numbered just before the siege in 1646. See Nash's History
% of Worcestershire. I have made allowance for the increase which must be
% supposed to have taken place in forty years. In 1740, the population of
% Nottingham was found, by enumeration, to be just 10,000. See Dering's
% History. The population of Gloucester may readily be inferred from the
% number of houses which King found in the returns of hearth money,
% and from the number of births and burials which is given in Atkyns's
% History. The population of Derby was 4,000 in 1712. See Wolley's MS.
% History, quoted in Lyson's Magna Britannia. The population of Shrewsbury
% was ascertained, in 1695, by actual enumeration. As to the gaieties of
% Shrewsbury, see Farquhar's Recruiting Officer. Farquhar's description
% is borne out by a ballad in the Pepysian Library, of which the burden is
% "Shrewsbury for me."]

% [Footnote 93: Blome's Britannia, 1673; Aikin's Country round Manchester;
% Manchester Directory, 1845: Baines, History of the Cotton Manufacture.
% The best information which I have been able to find, touching the
% population of Manchester in the seventeenth century is contained in
% a paper drawn up by the Reverend R. Parkinson, and published in the
% Journal of the Statistical Society for October 1842.]

% [Footnote 94: Thoresby's Ducatus Leodensis; Whitaker's Loidis and
% Elmete; Wardell's Municipal History of the Borough of Leeds. (1848.) In
% 1851 Leeds had 172,000 Inhabitants. (1857.)]

% [Footnote 95: Hunter's History of Hallamshire. (1848.) In 1851 the
% population of Sheffield had increased to 135,000. (1857.)]

% [Footnote 96: Blome's Britannia, 1673; Dugdale's Warwickshire, North's
% Examen, 321; Preface to Absalom and Achitophel; Hutton's History of
% Birmingham; Boswell's Life of Johnson. In 1690 the burials at Birmingham
% were 150, the baptisms 125. I think it probable that the annual
% mortality was little less than one in twenty-five. In London it was
% considerably greater. A historian of Nottingham, half a century later,
% boasted of the extraordinary salubrity of his town, where the annual
% mortality was one in thirty. See Doring's History of Nottingham. (1848.)
% In 1851 the population of Birmingham had increased to 222,000. (1857.)]

% [Footnote 97: Blome's Britannia; Gregson's Antiquities of the County
% Palatine and Duchy of Lancaster, Part II.; Petition from Liverpool in
% the Privy Council Book, May 10, 1686. In 1690 the burials at Liverpool
% were 151, the baptisms 120. In 1844 the net receipt of the customs at
% Liverpool was 4,366,526£. 1s. 8d. (1848.) In 1851 Liverpool contained
% 375,000 inhabitants, (1857.)]

% [Footnote 98: Atkyne's Gloucestershire.]

% [Footnote 99: Magna Britannia; Grose's Antiquities; New Brighthelmstone
% Directory.]

% [Footnote 100: Tour in Derbyshire, by Thomas Browne, son of Sir Thomas.]

% [Footnote 101: Memoires de Grammont; Hasted's History of Kent; Tunbridge
% Wells, a Comedy, 1678; Causton's Tunbridgialia, 1688; Metellus, a poem
% on Tunbridge Wells, 1693.]

% [Footnote 102: See Wood's History of Bath, 1719; Evelyn's Diary, June
% 27,1654; Pepys's Diary, June 12, 1668; Stukeley's Itinerarium Curiosum;
% Collinson's Somersetshire; Dr. Peirce's History and Memoirs of the Bath,
% 1713, Book I. chap. viii. obs. 2, 1684. I have consulted several
% old maps and pictures of Bath, particularly one curious map which is
% surrounded by views of the principal buildings. It Dears the date of
% 1717.]

% [Footnote 103: According to King 530,000. (1848.) In 1851 the population
% of London exceeded, 2,300,000. (1857.)]

% [Footnote 104: Macpherson's History of Commerce; Chalmers's Estimate;
% Chamberlayne's State of England, 1684. The tonnage of the steamers
% belonging to the port of London was, at the end of 1847, about 60,000
% tons. The customs of the port, from 1842 to 1845, very nearly averaged
% 11,000,000£. (1848.) In 1854 the tonnage of the steamers of the port of
% London amounted to 138,000 tons, without reckoning vessels of less than
% fifty tons. (1857.)]

% [Footnote 105: Lyson's Environs of London. The baptisms at Chelsea,
% between 1680 and 1690, were only 42 a year.]

% [Footnote 106: Cowley, Discourse of Solitude.]

% [Footnote 107: The fullest and most trustworthy information about the
% state of the buildings of London at this time is to be derived from the
% maps and drawings in the British Museum and in the Pepysian Library.
% The badness of the bricks in the old buildings of London is particularly
% mentioned in the Travels of the Grand Duke Cosmo. There is an account of
% the works at Saint Paul's in Ward's London Spy. I am almost ashamed to
% quote such nauseous balderdash; but I have been forced to descend even
% lower, if possible, in search of materials.]

% [Footnote 108: Evelyn's Diary, Sept. 20. 1672.]

% [Footnote 109: Roger North's Life of Sir Dudley North.]

% [Footnote 110: North's Examen. This amusing writer has preserved
% a specimen of the sublime raptures in which the Pindar of the City
% indulged:--
% 
%      "The worshipful sir John Moor!
%      After age that name adore!"]

% [Footnote 111: Chamberlayne's State of England, 1684; Anglie Metropolis,
% 1690; Seymour's London, 1734.]

% [Footnote 112: North's Examen, 116; Wood, Ath. Ox. Shaftesbury; The Duke
% of B.'s Litany.]

% [Footnote 113: Travels of the Grand Duke Cosmo.]

% [Footnote 114: Chamberlayne's State of England, 1684; Pennant's London;
% Smith's Life of Nollekens.]

% [Footnote 115: Evelyn's Diary, Oct. 10, 1683, Jan. 19, 1685-6.]

% [Footnote 116: Stat. 1 Jac. II. c. 22; Evelyn's Diary, Dec, 7, 1684.]

% [Footnote 117: Old General Oglethorpe, who died in 1785, used to boast
% that he had shot birds here in Anne's reign. See Pennant's London, and
% the Gentleman's Magazine for July, 1785.]

% [Footnote 118: The pest field will be seen in maps of London as late as
% the end of George the First's reign.]

% [Footnote 119: See a very curious plan of Covent Garden made about 1690,
% and engraved for Smith's History of Westminster. See also Hogarth's
% Morning, painted while some of the houses in the Piazza were still
% occupied by people of fashion.]

% [Footnote 120: London Spy, Tom Brown's comical View of London and
% Westminster; Turner's Propositions for the employing of the Poor, 1678;
% Daily Courant and Daily Journal of June 7, 1733; Case of Michael v.
% Allestree, in 1676, 2 Levinz, p. 172. Michael had been run over by
% two horses which Allestree was breaking in Lincoln's Inn Fields.
% The declaration set forth that the defendant "porta deux chivals
% ungovernable en un coach, et improvide, incante, et absque debita
% consideratione ineptitudinis loci la eux drive pur eux faire tractable
% et apt pur an coach, quels chivals, pur ceo que, per leur ferocite, ne
% poientestre rule, curre sur le plaintiff et le noie."]

% [Footnote 121: Stat. 12 Geo. I. c. 25; Commons' Journals, Feb. 25, March
% 2, 1725-6; London Gardener, 1712; Evening Post, March, 23, 1731. I have
% not been able to find this number of the Evening Post; I therefore
% quote it on the faith of Mr. Malcolm, who mentions it in his History of
% London.]

% [Footnote 122: Lettres sur les Anglois, written early in the reign of
% William the Third; Swift's City Shower; Gay's Trivia. Johnson used to
% relate a curious conversation which he had with his mother about giving
% and taking the wall.]

% [Footnote 123: Oldham's Imitation of the 3d Satire of Juvenal, 1682;
% Shadwell's Scourers, 1690. Many other authorities will readily occur
% to all who are acquainted with the popular literature of that and the
% succeeding generation. It may be suspected that some of the Tityre
% Tus, like good Cavaliers, broke Milton's windows shortly after the
% Restoration. I am confident that he was thinking of those pests of
% London when he dictated the noble lines:
% 
%      "And in luxurious cities, when the noise
%      Of riot ascends above their loftiest towers,
%      And injury and outrage, and when night
%      Darkens the streets, then wander forth the sons
%      Of Belial, flown With innocence and wine."]

% [Footnote 124: Seymour's London.]

% [Footnote 125: Angliae Metropolis, 1690, Sect. 17, entitled, "Of the new
% lights"; Seymour's London.]

% [Footnote 126: Stowe's Survey of London; Shadwell's Squire of Alsatia;
% Ward's London Spy; Stat. 8 \& 9 Gul. III. cap. 27.]

% [Footnote 127: See Sir Roger North's account of the way in which Wright
% was made a judge, and Clarendon's account of the way in which Sir George
% Savile was made a peer.]

% [Footnote 128: The sources from which I have drawn my information about
% the state of the Court are too numerous to recapitulate. Among them
% are the Despatches of Barillon, Van Citters, Ronquillo, and Adda, the
% Travels of the Grand Duke Cosmo, the works of Roger North, the Diares of
% Pepys, Evelyn, and Teonge, and the Memoirs of Grammont and Reresby.]

% [Footnote 129: The chief peculiarity of this dialect was that, in
% a large class of words, the O was pronounced like A. Thus Lord was
% pronounced Lard. See Vanbrugh's Relapse. Lord Sunderland was a great
% master of this court tune, as Roger North calls it; and Titus Oates
% affected it in the hope of passing for a fine gentleman. Examen, 77,
% 254.]

% [Footnote 130: Lettres sur les Anglois; Tom Brown's Tour; Ward's London
% Spy; The Character of a Coffee House, 1673; Rules and Orders of the
% Coffee House, 1674; Coffee Houses vindicated, 1675; A Satyr against
% Coffee; North's Examen, 138; Life of Guildford, 152; Life of Sir Dudley
% North, 149; Life of Dr. Radcliffe, published by Curll in 1715. The
% liveliest description of Will's is in the City and Country Mouse. There
% is a remarkable passage about the influence of the coffee house orators
% in Halstead's Succinct Genealogies, printed in 1685.]

% [Footnote 131: Century of inventions, 1663, No. 68.]

% [Footnote 132: North's Life of Guildford, 136.]

% [Footnote 133: Thoresby's Diary Oct. 21,1680, Aug. 3, 1712.]

% [Footnote 134: Pepys's Diary, June 12 and 16,1668.]

% [Footnote 135: Ibid. Feb. 28, 1660.]

% [Footnote 136: Thoresby's Diary, May 17,1695.]

% [Footnote 137: Ibid. Dec. 27,1708.]

% [Footnote 138: Tour in Derbyshire, by J. Browne, son of Sir Thomas
% Browne, 1662; Cotton's Angler, 1676.]

% [Footnote 139: Correspondence of Henry Earl of Clarendon, Dec. 30, 1685,
% Jan. 1, 1686.]

% [Footnote 140: Postlethwaite's Dictionary, Roads; History of Hawkhurst,
% in the Bibliotheca Topographica Britannica.]

% [Footnote 141: Annals of Queen Anne, 1703, Appendix, No. 3.]

% [Footnote 142: 15 Car. II. c. 1.]

% [Footnote 143: The evils of the old system are strikingly set forth
% in many petitions which appear in the Commons' Journal of 172 5/6. How
% fierce an opposition was offered to the new system may be learned from
% the Gentleman's Magazine of 1749.]

% [Footnote 144: Postlethwaite's Dict., Roads.]

% [Footnote 145: Loidis and Elmete; Marshall's Rural Economy of England,
% In 1739 Roderic Random came from Scotland to Newcastle on a packhorse.]

% [Footnote 146: Cotton's Epistle to J. Bradshaw.]

% [Footnote 147: Anthony a Wood's Life of himself.]

% [Footnote 148: Chamberlayne's State of England, 1684. See also the list
% of stage coaches and waggons at the end of the book, entitled Angliae
% Metropolis, 1690.]

% [Footnote 149: John Cresset's Reasons for suppressing Stage Coaches,
% 1672. These reason were afterwards inserted in a tract, entitled "The
% Grand Concern of England explained, 1673." Cresset's attack on stage
% coaches called forth some answers which I have consulted.]

% [Footnote 150: Chamberlayne's State of England, 1684; North's Examen,
% 105; Evelyn's Diary, Oct. 9,10, 1671.]

% [Footnote 151: See the London Gazette, May 14, 1677, August 4, 1687,
% Dec. 5, 1687. The last confession of Augustin King, who was the son of
% an eminent divine, and had been educated at Cambridge but was hanged at
% Colchester in March, 1688, is highly curious.]

% [Footnote 152: Aimwell. Pray sir, han't I seen your face at Will's
% coffeehouse? Gibbet. Yes sir, and at White's too.--Beaux' Stratagem.]

% [Footnote 153: Gent's History of York. Another marauder of the same
% description, named Biss, was hanged at Salisbury in 1695. In a ballad
% which is in the Pepysian Library, he is represented as defending himself
% thus before the Judge:
% 
%      "What say you now, my honoured Lord
%      What harm was there in this?
%      Rich, wealthy misers were abhorred
%      By brave, freehearted Biss."]

% [Footnote 154: Pope's Memoirs of Duval, published immediately after the
% execution. Oates's Eikwg basilikh, Part I.]

% [Footnote 155: See the prologue to the Canterbury Tales, Harrison's
% Historical Description of the Island of Great Britain, and Pepys's
% account of his tour in the summer of 1668. The excellence of the English
% inns is noticed in the Travels of the Grand Duke Cosmo.]

% [Footnote 156: Stat. 12 Car. II. c. 36; Chamberlayne's State of England,
% 1684; Angliae Metropolis, 1690; London Gazette, June 22, 1685, August
% 15, 1687.]

% [Footnote 157: Lond. Gaz., Sept. 14, 1685.]

% [Footnote 158: Smith's Current intelligence, March 30, and April 3,
% 1680.]

% [Footnote 159: Anglias Metropolis, 1690.]

% [Footnote 160: Commons' Journals, Sept. 4, 1660, March 1, 1688-9;
% Chamberlayne, 1684; Davenant on the Public Revenue, Discourse IV.]

% [Footnote 161: I have left the text as it stood in 1848. In the year
% 1856 the gross receipt of the Post Office was more than 2,800,000£.; and
% the net receipt was about 1,200,000£. The number of letters conveyed by
% post was 478,000,000. (1857).]

% [Footnote 162: London Gazette, May 5, and 17, 1680.]

% [Footnote 163: There is a very curious, and, I should think, unique
% collection of these papers in the British Museum.]

% [Footnote 164: For example, there is not a word in the Gazette about
% the important parliamentary proceedings of November, 1685, or about the
% trial and acquittal of the Seven Bishops.]

% [Footnote 165: Roger North's Life of Dr. John North. On the subject of
% newsletters, see the Examen, 133.]

% [Footnote 166: I take this opportunity of expressing my warm gratitude
% to the family of my dear and honoured friend sir James Mackintosh
% for confiding to me the materials collected by him at a time when he
% meditated a work similar to that which I have undertaken. I have never
% seen, and I do not believe that there anywhere exists, within the same
% compass, so noble a collection of extracts from public and private
% archives The judgment with which sir James in great masses of the
% rudest ore of history, selected what was valuable, and rejected what was
% worthless, can be fully appreciated only by one who has toiled after him
% in the same mine.]

% [Footnote 167: Life of Thomas Gent. A complete list of all printing
% houses in 1724 will be found in Nichols's Literary Anecdotae of the
% eighteenth century. There had then been a great increase within a few
% years in the number of presses, and yet there were thirty-four counties
% in which there was no printer, one of those counties being Lancashire.]

% [Footnote 168: Observator, Jan. 29, and 31, 1685; Calamy's Life of
% Baxter; Nonconformist Memorial.]

% [Footnote 169: Cotton seems, from his Angler, to have found room for his
% whole library in his hall window; and Cotton was a man of letters. Even
% when Franklin first visited London in 1724, circulating libraries were
% unknown there. The crowd at the booksellers' shops in Little Britain is
% mentioned by Roger North in his life of his brother John.]

% [Footnote 170: One instance will suffice. Queen Mary, the daughter of
% James, had excellent natural abilities, had been educated by a Bishop,
% was fond of history and poetry and was regarded by very eminent men as a
% superior woman. There is, in the library at the Hague, a superb English
% Bible which was delivered to her when she was crowned in Westminster
% Abbey. In the titlepage are these words in her own hand, "This book was
% given the King and I, at our crownation. Marie R."]

% [Footnote 171: Roger North tells us that his brother John, who was Greek
% professor at Cambridge, complained bitterly of the general neglect of
% the Greek tongue among the academical clergy.]

% [Footnote 172: Butler, in a satire of great asperity, says,
% 
%      "For, though to smelter words of Greek
%      And Latin be the rhetorique
%      Of pedants counted, and vainglorious,
%      To smatter French is meritorious."]

% [Footnote 173: The most offensive instance which I remember is in a poem
% on the coronation of Charles the Second by Dryden, who certainly could
% not plead poverty as an excuse for borrowing words from any foreign
% tongue:--
% 
%      "Hither in summer evenings you repair
%      To taste the fraicheur of the cooler air."]

% [Footnote 174: Jeremy Collier has censured this odious practice with his
% usual force and keenness.]

% [Footnote 175: The contrast will be found in Sir Walter Scott's edition
% of Dryden.]

% [Footnote 176: See the Life of Southern. by Shiels.]

% [Footnote 177: See Rochester's Trial of the Poets.]

% [Footnote 178: Some Account of the English Stage.]

% [Footnote 179: Life of Southern, by Shiels.]

% [Footnote 180: If any reader thinks my expressions too severe, I would
% advise him to read Dryden's Epilogue to the Duke of Guise, and to
% observe that it was spoken by a woman.]

% [Footnote 181: See particularly Harrington's Oceana.]

% [Footnote 182: See Sprat's History of the Royal Society.]

% [Footnote 183: Cowley's Ode to the Royal Society.]

% [Footnote 184:
% 
%      "Then we upon the globe's last verge shall go,
%      And view the ocean leaning  on the  sky;
%      From  thence our rolling neighbours we shall know,
%      And on the lunar world secretly pry.']
          --Annus Mirabilis, 164]

% [Footnote 185: North's Life of Guildford.]

% [Footnote 186: Pepys's Diary, May 30, 1667.]

% [Footnote 187: Butler was, I think, the only man of real genius who,
% between the Restoration and the Revolution showed a bitter enmity to
% the new philosophy, as it was then called. See the Satire on the Royal
% Society, and the Elephant in the Moon.]

% [Footnote 188: The eagerness with which the agriculturists of that
% age tried experiments and introduced improvements is well described by
% Aubrey. See the Natural history of Wiltshire, 1685.]

% [Footnote 189: Sprat's History of the Royal Society.]

% [Footnote 190: Walpole's Anecdotes of Painting, London Gazette, May 31,
% 1683; North's Life of Guildford.]

% [Footnote 191: The great prices paid to Varelst and Verrio are mentioned
% in Walpole's Anecdotes of Painting.]

% [Footnote 192: Petty's Political Arithmetic.]

% [Footnote 193: Stat 5 Eliz. c. 4; Archaeologia, vol. xi.]

% [Footnote 194: Plain and easy Method showing how the office of Overseer
% of the Poor may be managed, by Richard Dunning; 1st edition, 1685; 2d
% edition, 1686.]

% [Footnote 195: Cullum's History of Hawsted.]

% [Footnote 196: Ruggles on the Poor.]

% [Footnote 197: See, in Thurloe's State Papers, the memorandum of the
% Dutch Deputies dated August 2-12, 1653.]

% [Footnote 198: The orator was Mr. John Basset, member for Barnstaple.
% See Smith's Memoirs of Wool, chapter lxviii.]

% [Footnote 199: This ballad is in the British Museum. The precise year
% is not given; but the Imprimatur of Roger Lestrange fixes the date
% sufficiently for my purpose. I will quote some of the lines. The master
% clothier is introduced speaking as follows:
% 
%      "In former ages we used to give,
%      So that our workfolks like farmers did live;
%      But the times are changed, we will make them know.
% 
%      "We will make them to work hard for sixpence a day,
%      Though a shilling they deserve if they kind their just pay;
%      If at all they murmur and say 'tis too small,
%      We bid them choose whether they'll work at all.
%      And thus we forgain all our wealth and estate,
%      By many poor men that work early and late.
%      Then hey for the clothing trade! It goes on brave;
%      We scorn for to toyl and moyl, nor yet to slave.
%      Our workmen do work hard, but we live at ease,
%      We go when we will, and we come when we please."]

% [Footnote 200: Chamberlayne's State of England; Petty's Political
% Arithmetic, chapter viii.; Dunning's Plain and Easy Method; Firmin's
% Proposition for the Employing of the Poor. It ought to be observed that
% Firmin was an eminent philanthropist.]

% [Footnote 201: King in his Natural and Political Conclusions roughly
% estimated the common people of England at 880,000 families. Of these
% families 440,000, according to him ate animal food twice a week. The
% remaining 440,000, ate it not at all, or at most not oftener than once a
% week.]

% [Footnote 202: Fourteenth Report of the Poor Law Commissioners, Appendix
% B. No. 2, Appendix C. No 1, 1848. Of the two estimates of the poor rate
% mentioned in the text one was formed by Arthur Moore, the other, some
% years later, by Richard Dunning. Moore's estimate will be found in
% Davenant's Essay on Ways and Means; Dunning's in Sir Frederic Eden's
% valuable work on the poor. King and Davenant estimate the paupers
% and beggars in 1696, at the incredible number of 1,330,000 out of a
% population of 5,500,000. In 1846 the number of persons who received
% relief appears from the official returns to have been only 1,332,089 out
% of a population of about 17,000,000. It ought also to be observed that,
% in those returns, a pauper must very often be reckoned more than once.
% I would advise the reader to consult De Foe's pamphlet entitled "Giving
% Alms no Charity," and the Greenwich tables which will be found in Mr.
% M'Culloch's Commercial Dictionary under the head Prices.]

% [Footnote 203: The deaths were 23,222. Petty's Political Arithmetic.]

% [Footnote 204: Burnet, i. 560.]

% [Footnote 205: Muggleton's Acts of the Witnesses of the Spirit.]

% [Footnote 206: Tom Brown describes such a scene in lines which I do not
% venture to quote.]

% [Footnote 207: Ward's London Spy.]

% [Footnote 208: Pepys's Diary, Dec. 28, 1663, Sept. 2, 1667.]

% [Footnote 209: Burnet, i, 606; Spectator, No. 462; Lords' Journals,
% October 28, 1678; Cibber's Apology.]

% [Footnote 210: Burnet, i. 605, 606, Welwood, North's Life of Guildford,
% 251.]

% [Footnote 211: I may take this opportunity of mentioning that whenever
% I give only one date, I follow the old style, which was, in the
% seventeenth century, the style of England; but I reckon the year from
% the first of January.]

% [Footnote 212: Saint Everemond, passim; Saint Real, Memoires de la
% Duchesse de Mazarin; Rochester's Farewell; Evelyn's Diary, Sept. 6,
% 1676, June 11, 1699.]

% [Footnote 213: Evelyn's Diary, Jan. 28, 1684-5, Saint Evremond's Letter
% to Dery.]

% [Footnote 214: Id., February 4, 1684-5.]

% [Footnote 215: Roger North's Life of Sir Dudley North, 170; The true
% Patriot vindicated, or a Justification of his Excellency the E-of
% R-; Burnet, i. 605. The Treasury Books prove that Burnet had good
% intelligence.]

% [Footnote 216: Evelyn's Diary, Jan. 24, 1681-2, Oct. 4, 1683.]

% [Footnote 217: Dugdale's Correspondence.]

% [Footnote 218: Hawkins's Life of Ken, 1713.]

% [Footnote 219: See the London Gazette of Nov. 21, 1678. Barillon
% and Burnet say that Huddleston was excepted out of all the Acts of
% Parliament made against priests; but this is a mistake.]

% [Footnote 220: Clark's Life of James the Second, i, 746. Orig. Mem.;
% Barillon's Despatch of Feb. 1-18, 1685; Van Citters's Despatches of Feb.
% 3-13 and Feb. 1-16. Huddleston's Narrative; Letters of Philip, second
% Earl of Chesterfield, 277; Sir H. Ellis's Original Letters, First
% Series. iii. 333: Second Series, iv 74; Chaillot MS.; Burnet, i. 606:
% Evelyn's Diary, Feb. 4. 1684-5: Welwood's Memoires 140; North's Life of
% Guildford. 252; Examen, 648; Hawkins's Life of Ken; Dryden's Threnodia
% Augustalis; Sir H. Halford's Essay on Deaths of Eminent Persons. See
% also a fragment of a letter written by the Earl of Ailesbury, which is
% printed in the European Magazine for April, 1795. Ailesbury calls Burnet
% an impostor. Yet his own narrative and Burnet's will not, to any candid
% and sensible reader, appear to contradict each other. I have seen in
% the British Museum, and also in the Library of the Royal Institution, a
% curious broadside containing an account of the death of Charles. It will
% be found in the Somers Collections. The author was evidently a zealous
% Roman Catholic, and must have had access to good sources of information.
% I strongly suspect that he had been in communication, directly or
% indirectly, with James himself. No name is given at length; but the
% initials are perfectly intelligible, except in one place. It is said
% that the D. of Y. was reminded of the duty which he owed to his brother
% by P.M.A.C.F. I must own myself quite unable to decipher the last
% five letters. It is some consolation that Sir Walter Scott was
% equally unsuccessful. (1848.) Since the first edition of this work
% was published, several ingenious conjectures touching these mysterious
% letters have been communicated to me, but I am convinced that the true
% solution has not yet been suggested. (1850.) I still greatly
% doubt whether the riddle has been solved. But the most plausible
% interpretation is one which, with some variations, occurred, almost at
% the same time, to myself and to several other persons; I am inclined to
% read "Pere Mansuete A Cordelier Friar." Mansuete, a Cordelier, was
% then James's confessor. To Mansuete therefore it peculiarly belonged
% to remind James of a sacred duty which had been culpably neglected. The
% writer of the broadside must have been unwilling to inform the world
% that a soul which many devout Roman Catholics had left to perish had
% been snatched from destruction by the courageous charity of a woman of
% loose character. It is therefore not unlikely that he would prefer a
% fiction, at once probable and edifying, to a truth which could not
% fail to give scandal. (1856.)----It should seem that no transactions in
% history ought to be more accurately known to us than those which
% took place round the deathbed of Charles the Second. We have several
% relations written by persons who were actually in his room. We have
% several relations written by persons who, though not themselves
% eyewitnesses, had the best opportunity of obtaining information from
% eyewitnesses. Yet whoever attempts to digest this vast mass of materials
% into a consistent narrative will find the task a difficult one. Indeed
% James and his wife, when they told the story to the nuns of Chaillot,
% could not agree as to some circumstances. The Queen said that, after
% Charles had received the last sacraments the Protestant Bishops renewed
% their exhortations. The King said that nothing of the kind took place.
% "Surely," said the Queen, "you told me so yourself." "It is impossible
% that I have told you so," said the King, "for nothing of the sort
% happened."----It is much to be regretted that Sir Henry Halford should
% have taken so little trouble ascertain the facts on which he pronounced
% judgment. He does not seem to have been aware of the existence of the
% narrative of James, Barillon, and Huddleston.----As this is the first
% occasion on which I cite the correspondence of the Dutch ministers
% at the English court, I ought here to mention that a series of their
% despatches, from the accession of James the Second to his flight,
% forms one of the most valuable parts of the Mackintosh collection.
% The subsequent despatches, down to the settlement of the government in
% February, 1689, I procured from the Hague. The Dutch archives have been
% far too little explored. They abound with information interesting in the
% highest degree to every Englishman. They are admirably arranged and they
% are in the charge of gentlemen whose courtesy, liberality and zeal for
% the interests of literature, cannot be too highly praised. I wish to
% acknowledge, in the strongest manner, my own obligations to Mr. De Jonge
% and to Mr. Van Zwanne.]

% [Footnote 221: Clarendon mentions this calumny with just scorn.
% "According to the charity of the time towards Cromwell, very many would
% have it believed to be by poison, of which there was no appearance, nor
% any proof ever after made."--Book xiv.]

% [Footnote 222: Welwood, 139 Burnet, i. 609; Sheffield's Character
% of Charles the Second; North's Life of Guildford, 252; Examen,
% 648; Revolution Politics; Higgons on Burnet. What North says of the
% embarrassment and vacillation of the physicians is confirmed by the
% despatches of Van Citters. I have been much perplexed by the strange
% story about Short's suspicions. I was, at one time, inclined to adopt
% North's solution. But, though I attach little weight to the authority of
% Welwood and Burnet in such a case, I cannot reject the testimony of so
% well informed and so unwilling a witness as Sheffield.]

% [Footnote 223: London Gazette, Feb. 9. 1684-5; Clarke's Life of James
% the Second, ii. 3; Barillon, Feb. 9-19: Evelyn's Diary, Feb. 6.]

% [Footnote 224: See the authorities cited in the last note. See also the
% Examen, 647; Burnet, i. 620; Higgons on Burnet.]

% [Footnote 225: London Gazette, Feb. 14, 1684-5; Evelyn's Diary of the
% same day; Burnet, i. 610: The Hind let loose.]

% [Footnote 226: Burnet, i. 628; Lestrange, Observator, Feb. 11, 1684.]

% [Footnote 227: The letters which passed between Rochester and Ormond on
% this subject will be found in the Clarendon Correspondence.]

% [Footnote 228: The ministerial changes are announced in the London
% Gazette, Feb. 19, 1684-5. See Burnet, i. 621; Barillon, Feb. 9-19,
% 16-26; and Feb. 19,/Mar. 1.]

% [Footnote 229: Carte's Life of Ormond; Secret Consults of the Romish
% Party in Ireland, 1690; Memoirs of Ireland, 1716.]

% [Footnote 230: Christmas Sessions Paper of 1678.]

% [Footnote 231: The Acts of the Witnesses of the Spirit, part v chapter
% v. In this work Lodowick, after his fashion, revenges himself on the
% "bawling devil," as he calls Jeffreys, by a string of curses which
% Ernulphus, or Jeffreys himself, might have envied. The trial was in
% January, 1677.]

% [Footnote 232: This saying is to be found in many contemporary
% pamphlets. Titus Oates was never tired of quoting it. See his Eikwg
% Basilikh.]

% [Footnote 233: The chief sources of information concerning Jeffreys are
% the State Trials and North's Life of Lord Guildford. Some touches of
% minor importance I owe to contemporary pamphlets in verse and prose.
% Such are the Bloody Assizes the life and Death of George Lord Jeffreys,
% the Panegyric on the late Lord Jeffreys, the Letter to the Lord
% Chancellor, Jeffreys's Elegy. See also Evelyn's Diary, Dec. 5, 1683,
% Oct. 31. 1685. I scarcely need advise every reader to consult Lord
% Campbell's excellent Life of Jeffreys.]

% [Footnote 234: London Gazette, Feb. 12, 1684-5. North's Life of
% Guildford, 254.]

% [Footnote 235: The chief authority for these transactions is Barillon's
% despatch of February 9-19, 1685. It will be found in the Appendix to Mr.
% Fox's History. See also Preston's Letter to James, dated April 18-28,
% 1685, in Dalrymple.]

% [Footnote 236: Lewis to Barillon, Feb. 16-26, 1685.]

% [Footnote 237: Barillon, Feb. 16-26, 1685.]

% [Footnote 238: Barillon, Feb. 18-28, 1685.]

% [Footnote 239: Swift who hated Marlborough, and who was little disposed
% to allow any merit to those whom he hated, says, in the famous letter to
% Crassus, "You are no ill orator in the Senate."]

% [Footnote 240: Dartmouth's note on Burnet, i. 264. Chesterfleld's
% Letters, Nov., 18, 1748. Chesterfield is an unexceptional witness; for
% the annuity was a charge on the estate of his grandfather, Halifax. I
% believe that there is no foundation for a disgraceful addition to the
% story which may be found in Pope:
% 
%     "The gallant too, to whom she paid it down,
%     Lived to refuse his mistress half a crown."
%     Curll calls this a piece of travelling scandal.]

% [Footnote 241: Pope in Spence's Anecdotes.]

% [Footnote 242: See the Historical Records of the first or Royal
% Dragoons. The appointment of Churchill to the command of this regiment
% was ridiculed as an instance of absurd partiality. One lampoon of that
% time which I do not remember to have seen in print, but of which a
% manuscript copy is in the British Museum, contains these lines:
% 
%      "Let's cut our meat with spoons:
%      The sense is as good
%      As that Churchill should
%      Be put to command the dragoons."]

% [Footnote 243: Barillon, Feb. 16-26, 1685.]

% [Footnote 244: Barillon, April 6-16; Lewis to Barillon, April 14-24.]

% [Footnote 245: I might transcribe half Barillon's correspondence in
% proof of this proposition, but I will quote only one passage, in
% which the policy of the French government towards England is exhibited
% concisely and with perfect clearness.---- "On peut tenir pour un maxime
% indubitable que l'accord du Roy d'Angleterre avec son parlement, en
% quelque maniere qu'il se fasse, n'est pas conforme aux interets de V. M.
% Je me contente de penser cela sane m'en ouvrir a personne, et je cache
% avec soin mes sentimens a cet egard."--Barillon to Lewis, Feb. 28,/Mar.
% 1687. That this was the real secret of the whole policy of Lewis towards
% our country was perfectly understood at Vienna. The Emperor Leopold
% wrote thus to James, March 30,/April 9, 1689: "Galli id unum agebant,
% ut, perpetuas inter Serenitatem vestram et ejusdem populos fovendo
% simultates, reliquæ Christianæ Europe tanto securius insultarent."]

% [Footnote 246: "Que sea unido con su reyno, yen todo buena intelligencia
% con el parlamenyo." Despatch from the King of Spain to Don Pedro
% Ronquillo, March 16-26, 1685. This despatch is in the archives of
% Samancas, which contain a great mass of papers relating to English
% affairs. Copies of the most interesting of those papers are in the
% possession of M. Guizot, and were by him lent to me. It is with peculiar
% pleasure that at this time, I acknowledge this mark of the friendship of
% so great a man. (1848.)]

% [Footnote 247: Few English readers will be desirous to go deep into the
% history of this quarrel. Summaries will be found in Cardinal Bausset's
% Life of Bossuet, and in Voltaire's Age of Lewis XIV.]

% [Footnote 248: Burnet, i. 661, and Letter from Rome, Dodd's Church
% History, part viii. book i. art. 1.]

% [Footnote 249: Consultations of the Spanish Council of State on April
% 2-12 and April 16-26, In the Archives of Simancas.]

% [Footnote 250: Lewis to Barillon, May 22,/June 1, 1685; Burnet, i. 623.]

% [Footnote 251: Life of James the Second, i. 5. Barillon, Feb. 19,/Mar.
% 1, 1685; Evelyn's Diary, March 5, 1685.]

% [Footnote 252:
% 
%      "To those that ask boons
%      He swears by God's oons
%      And chides them as if they came there to steal spoons."
%           Lamentable Lory, a ballad, 1684.]

% [Footnote 253: Barillon, April 20-30. 1685.]

% [Footnote 254: From Adda's despatch of Jan. 22,/Feb. 1, 1686, and
% from the expressions of the Pere d'Orleans (Histoire des Revolutions
% d'Angleterre, liv. xi.), it is clear that rigid Catholics thought the
% King's conduct indefensible.]

% [Footnote 255: London Gazette, Gazette de France; Life of James the
% Second, ii. 10; History of the Coronation of King James the Second and
% Queen Mary, by Francis Sandford, Lancaster Herald, fol. 1687; Evelyn's
% Diary, May, 21, 1685; Despatch of the Dutch Ambassadors, April 10-20,
% 1685; Burnet, i. 628; Eachard, iii. 734; A sermon preached before their
% Majesties King James the Second and Queen Mary at their Coronation in
% Westminster Abbey, April 23, 1695, by Francis Lord Bishop of Ely, and
% Lord Almoner. I have seen an Italian account of the Coronation which was
% published at Modena, and which is chiefly remarkable for the skill with
% which the writer sinks the fact that the prayers and psalms were in
% English, and that the Bishops were heretics.]

% [Footnote 256: See the London Gazette during the months of February,
% March, and April, 1685.]

% [Footnote 257: It would be easy to fill a volume with what Whig
% historians and pamphleteers have written on this subject. I will cite
% only one witness, a churchman and a Tory. "Elections," says Evelyn,
% "were thought to be very indecently carried on in most places. God give
% a better issue of it than some expect!" May 10, 1685. Again he says,
% "The truth is there were many of the new members whose elections and
% returns were universally condemned." May 22.]

% [Footnote 258: This fact I learned from a newsletter in the library of
% the Royal Institution. Van Citters mentions the strength of the Whig
% party in Bedfordshire.]

% [Footnote 259: Bramston's Memoirs.]

% [Footnote 260: Reflections on a Remonstrance and Protestation of all the
% good Protestants of this Kingdom, 1689; Dialogue between Two Friends,
% 1689.]

% [Footnote 261: Memoirs of the Life of Thomas Marquess of Wharton, 1715.]

% [Footnote 262: See the Guardian, No. 67; an exquisite specimen of
% Addison's peculiar manner. It would be difficult to find in the works
% of any other writer such an instance of benevolence delicately flavoured
% with contempt.]

% [Footnote 263: The Observator, April 4, 1685.]

% [Footnote 264: Despatch of the Dutch Ambasadors, April 10-20, 1685.]

% [Footnote 265: Burnet, i. 626.]

% [Footnote 266: A faithful account of the Sickness, Death, and Burial of
% Captain Bedlow, 1680; Narrative of Lord Chief Justice North.]

% [Footnote 267: Smith's Intrigues of the Popish Plot, 1685.]

% [Footnote 268: Burnet, i. 439.]

% [Footnote 269: See the proceedings in the Collection of State Trials.]

% [Footnote 270: Evelyn's Diary, May 7, 1685.]

% [Footnote 271: There remain many pictures of Oates. The most striking
% descriptions of his person are in North's Examen, 225, in Dryden's
% Absalom and Achitophel, and In a broadside entitled, A Hue and Cry after
% T. O.]

% [Footnote 272: The proceedings will be found at length in the Collection
% of State Trials.]

% [Footnote 273: Gazette de France May 29,/June 9, 1685.]

% [Footnote 274: Despatch of the Dutch Ambassadors, May 19-29, 1685.]

% [Footnote 275: Evelyn's Diary, May 22, 1685; Eachard, iii. 741; Burnet,
% i. 637; Observator, May 27, 1685; Oates's Eikvn, 89; Eikwn Brotoloigon,
% 1697; Commons' Journals of May, June, and July, 1689; Tom Brown's
% advice to Dr. Oates. Some interesting circumstances are mentioned in
% a broadside, printed for A. Brooks, Charing Cross, 1685. I have seen
% contemporary French and Italian pamphlets containing the history of the
% trial and execution. A print of Titus in the pillory was published at
% Milan, with the following curious inscription: "Questo e il naturale
% ritratto di Tito Otez, o vero Oatz, Inglese, posto in berlina, uno de'
% principali professor della religion protestante, acerrimo persecutore
% de' Cattolici, e gran spergiuro." I have also seen a Dutch engraving
% of his punishment, with some Latin verses, of which the following are a
% specimen:
% 
%      "At Doctor fictus non fictos pertulit ictus
%      A tortore datos haud molli in corpore gratos,
%      Disceret ut vere scelera ob commissa rubere."
% 
% The anagram of his name, "Testis Ovat," may be found on many prints
% published in different countries.]

% [Footnote 276: Blackstone's Commentaries, Chapter of Homicide.]

% [Footnote 277: According to Roger North the judges decided that
% Dangerfield, having been previously convicted of perjury, was
% incompetent to be a witness of the plot. But this is one among many
% instances of Roger's inaccuracy. It appears, from the report of the
% trial of Lord Castlemaine in June 1680, that, after much altercation
% between counsel, and much consultation among the judges of the different
% courts in Westminster Hall, Dangerfield was sworn and suffered to tell
% his story; but the jury very properly gave no credit to his testimony.]

% [Footnote 278: Dangerfield's trial was not reported; but I have seen a
% concise account of it in a contemporary broadside. An abstract of the
% evidence against Francis, and his dying speech, will be found in the
% Collection of State Trials. See Eachard, iii. 741. Burnet's narrative
% contains more mistakes than lines. See also North's Examen, 256, the
% sketch of Dangerfield's life in the Bloody Assizes, the Observator of
% July 29, 1685, and the poem entitled "Dangerfield's Ghost to Jeffreys."
% In the very rare volume entitled "Succinct Genealogies, by Robert
% Halstead," Lord Peterbough says that Dangerfield, with whom he had had
% some intercourse, was "a young man who appeared under a decent figure,
% a serious behaviour, and with words that did not seem to proceed from a
% common understanding."]

% [Footnote 279: Baxter's preface to Sir Mathew Hale's Judgment of the
% Nature of True Religion, 1684.]

% [Footnote 280: See the Observator of February 28, 1685, the information
% in the Collection of State Trials, the account of what passed in court
% given by Calamy, Life of Baxter, chap. xiv., and the very curious
% extracts from the Baxter MSS. in the Life, by Orme, published in 1830.]

% [Footnote 281: Baxter MS. cited by Orme.]

% [Footnote 282: Act Parl. Car. II. March 29,1661, Jac. VII. April 28,
% 1685, and May 13, 1685.]

% [Footnote 283: Act Parl. Jac. VII. May 8, 1685, Observator, June 20,
% 1685; Lestrange evidently wished to see the precedent followed in
% England.]

% [Footnote 284: His own words reported by himself. Life of James the
% Second, i. 666. Orig. Mem.]

% [Footnote 285: Act Parl. Car. II. August 31, 1681.]

% [Footnote 286: Burnet, i. 583; Wodrow, III. v. 2. Unfortunately the Acta
% of the Scottish Privy Council during almost the whole administration of
% the Duke of York are wanting. (1848.) This assertion has been met by a
% direct contradiction. But the fact is exactly as I have stated it. There
% is in he Acta of the Scottish Privy Council a hiatus extending from
% August 1678 to August 1682. The Duke of York began to reside in Scotland
% in December 1679. He left Scotland, never to return in May 1682.
% (1857.)]

% [Footnote 287: Wodrow, III. ix. 6.]

% [Footnote 288: Wodrow, III. ix. 6. The editor of the Oxford edition of
% Burnet attempts to excuse this act by alleging that Claverhouse was then
% employed to intercept all communication between Argyle and Monmouth,
% and by supposing that John Brown may have been detected in conveying
% intelligence between the rebel camps. Unfortunately for this hypothesis
% John Brown was shot on the first of May, when both Argyle and Monmouth
% were in Holland, and when there was no insurrection in any part of our
% island.]

% [Footnote 289: Wodrow, III. ix, 6.]

% [Footnote 290: Wodrow, III. ix. 6. It has been confidently asserted, by
% persons who have not taken the trouble to look at the authority to which
% I have referred, that I have grossly calumniated these unfortunate
% men; that I do not understand the Calvinistic theology; and that it is
% impossible that members of the Church of Scotland can have refused to
% pray for any man on the ground that he was not one of the elect.----
% I can only refer to the narrative which Wodrow has inserted in his
% history, and which he justly calls plain and natural. That narrative
% is signed by two eyewitnesses, and Wodrow, before he published it,
% submitted it to a third eyewitness, who pronounced it strictly accurate.
% From that narrative I will extract the only words which bear on
% the point in question: "When all the three were taken, the officers
% consulted among themselves, and, withdrawing to the west side of the
% town, questioned the prisoners, particularly if they would pray for King
% James VII. They answered, they would pray for all within the election of
% grace. Balfour said Do you question the King's election? They answered,
% sometimes they questioned their own. Upon which he swore dreadfully, and
% said they should die presently, because they would not pray for Christ's
% vicegerent, and so without one word more, commanded Thomas Cook to go to
% his prayers, for he should die.---- In this narrative Wodrow saw nothing
% improbable; and I shall not easily be convinced that any writer now
% living understands the feelings and opinions of the Covenanters better
% than Wodrow did. (1857.)]

% [Footnote 291: Wodrow, III. ix. 6. Cloud of Witnesses.]

% [Footnote 292: Wodrow, III. ix. 6. The epitaph of Margaret Wilson, in
% the churchyard at Wigton, is printed in the Appendix to the Cloud of
% Witnesses;
% 
%      "Murdered for owning Christ supreme
%      Head of his church, and no more crime,
%      But her not owning Prelacy.
%      And not abjuring Presbytery,
%      Within the sea, tied to a stake,
%      She suffered for Christ Jesus' sake."]

% [Footnote 293: See the letter to King Charles II. prefixed to Barclay's
% Apology.]

% [Footnote 294: Sewel's History of the Quakers, book x.]

% [Footnote 295: Minutes of Yearly Meetings, 1689, 1690.]

% [Footnote 296: Clarkson on Quakerism; Peculiar Customs, chapter v.]

% [Footnote 297: After this passage was written, I found in the British
% Museum, a manuscript (Harl. MS. 7506) entitled, "An Account of the
% Seizures, Sequestrations, great Spoil and Havock made upon the Estates
% of the several Protestant Dissenters called Quakers, upon Prosecution of
% old Statutes made against Papist and Popish Recusants." The manuscript
% is marked as having belonged to James, and appears to have been given
% by his confidential servant, Colonel Graham, to Lord Oxford. This
% circumstance appears to me to confirm the view which I have taken of the
% King's conduct towards the Quakers.]

% [Footnote 298: Penn's visits to Whitehall, and levees at Kensington,
% are described with great vivacity, though in very bad Latin, by Gerard
% Croese. "Sumebat," he says, "rex sæpe secretum, non horarium, vero
% horarum plurium, in quo de variis rebus cum Penno serio sermonem
% conferebat, et interim differebat audire præcipuorum nobilium ordinem,
% qui hoc interim spatio in proc¦tone, in proximo, regem conventum præsto
% erant." Of the crowd of suitors at Penn's house. Croese says, "Visi
% quandoquo de hoc genere hominum non minus bis centum."--Historia
% Quakeriana, lib. ii. 1695.]

% [Footnote 299: "Twenty thousand into my pocket; and a hundred thousand
% into my province." Penn's "Letter to Popple."]

% [Footnote 300: These orders, signed by Sunderland, will be found in
% Sewel's History. They bear date April 18, 1685. They are written in
% a style singularly obscure and intricate: but I think that I have
% exhibited the meaning correctly. I have not been able to find any proof
% that any person, not a Roman Catholic or a Quaker, regained his freedom
% under these orders. See Neal's History of the Puritans, vol. ii. chap.
% ii.; Gerard Croese, lib. ii. Croese estimates the number of Quakers
% liberated at fourteen hundred and sixty.]

% [Footnote 301: Barillon, May 28,/June 7, 1685. Observator, May 27, 1685;
% Sir J. Reresby's Memoirs.]

% [Footnote 302: Lewis wrote to Barillon about this class of Exclusionists
% as follows: "L'interet qu'ils auront a effacer cette tache par des
% services considerables les portera, aelon toutes les apparences, a le
% servir plus utilement que ne pourraient faire ceux qui ont toujours ete
% les plus attaches a sa personne." May 15-25,1685.]

% [Footnote 303: Barillon, May 4-14, 1685; Sir John Reresby's Memoirs.]

% [Footnote 304: Burnet, i. 626; Evelyn's Diary, May, 22, 1685.]

% [Footnote 305: Roger North's Life of Guildford, 218; Bramston's
% Memoirs.]

% [Footnote 306: North's Life of Guildford, 228; News from Westminster.]

% [Footnote 307: Burnet, i. 382; Letter from Lord Conway to Sir George
% Rawdon, Dec. 28, 1677. in the Rawdon Papers.]

% [Footnote 308: London Gazette, May 25, 1685; Evelyn's Diary, May 22,
% 1685.]

% [Footnote 309: North's Life of Guildford, 256.]

% [Footnote 310: Burnet, i. 639; Evelyn's Diary, May 22, 1685; Barillon,
% May 23,/June 2, and May 25,/June 4, 1685 The silence of the journals
% perplexed Mr. Fox; but it is explained by the circumstance that
% Seymour's motion was not seconded.]

% [Footnote 311: Journals, May 22. Stat. Jac. II. i. 1.]

% [Footnote 312: Journals, May 26, 27. Sir J. Reresby's Memoirs.]

% [Footnote 313: Commons' Journals, May 27, 1685.]

% [Footnote 314: Roger North's Life of Sir Dudley North; Life of Lord
% Guilford, 166; Mr M'Cullough's Literature of Political Economy.]

% [Footnote 315: Life of Dudley North, 176, Lonsdale's Memoirs, Van
% Citters, June 12-22, 1685.]

% [Footnote 316: Commons' Journals, March 1, 1689.]

% [Footnote 317: Lords' Journals, March 18, 19, 1679, May 22, 1685.]

% [Footnote 318: Stat. 5 Geo. IV. c. 46.]

% [Footnote 319: Clarendon's History of the Rebellion, book xiv.; Burnet's
% Own Times, i. 546, 625; Wade's and Ireton's Narratives, Lansdowne MS.
% 1152; West's information in the Appendix to Sprat's True Account.]

% [Footnote 320: London Gazette, January, 4, 1684-5; Ferguson MS. in
% Eachard's History, iii. 764; Grey's Narratives; Sprat's True Account,
% Danvers's Treatise on Baptism; Danvers's Innocency and Truth vindicated;
% Crosby's History of the English Baptists.]

% [Footnote 321: Sprat's True Account; Burnet, i. 634; Wade's Confession,
% Earl. MS. 6845.---- Lord Howard of Escrick accused Ayloffe of proposing
% to assassinate the Duke of York; but Lord Howard was an abject liar;
% and this story was not part of his original confession, but was added
% afterwards by way of supplement, and therefore deserves no credit
% whatever.]

% [Footnote 322: Wade's Confession, Harl. MS. 6845; Lansdowne MS. 1152;
% Holloway's narrative in the Appendix to Sprat's True Account. Wade owned
% that Holloway had told nothing but truth.]

% [Footnote 323: Sprat's True Account and Appendix, passim.]

% [Footnote 324: Sprat's True Account and Appendix, Proceedings against
% Rumbold in the Collection of State Trials; Burnet's Own Times, i. 633;
% Appendix to Fox's History, No. IV.]

% [Footnote 325: Grey's narrative; his trial in the Collection of State
% Trials; Sprat's True Account.]

% [Footnote 326: In the Pepysian Collection is a print representing one
% of the balls which About this time William and Mary gave in the Oranje
% Zaal.]

% [Footnote 327: Avaux Neg. January 25, 1685. Letter from James to the
% Princess of Orange dated January 1684-5, among Birch's Extracts in the
% British Museum.]

% [Footnote 328: Grey's Narrative; Wade's Confession, Lansdowne MS. 1152.]

% [Footnote 329: Burnet, i. 542; Wood, Ath. Ox. under the name of Owen;
% Absalom and Achtophel, part ii.; Eachard, iii. 682, 697; Sprat's True
% Account, passim; Lond. Gaz. Aug. 6,1683; Nonconformist's Memorial;
% North's Examen, 399.]

% [Footnote 330: Wade's Confession, Harl. MS. 6845.]

% [Footnote 331: Avaux Neg. Feb. 20, 22, 1685; Monmouth's letter to James
% from Ringwood.]

% [Footnote 332: Boyer's History of King William the Third, 2d edition,
% 1703, vol. i 160.]

% [Footnote 333: Welwood's Memoirs, App. xv.; Burnet, i. 530. Grey told a
% somewhat different story, but he told it to save his life. The Spanish
% ambassador at the English court, Don Pedro de Ronquillo, in a letter
% to the governor of the Low Countries written about this time, sneers
% at Monmouth for living on the bounty of a fond woman, and hints a very
% unfounded suspicion that the Duke's passion was altogether interested.
% "Hallandose hoy tan falto de medios que ha menester trasformarse en Amor
% con Miledi en vista de la ecesidad de poder subsistir."--Ronquillo to
% Grana. Mar. 30,/Apr. 9, 1685.]

% [Footnote 334: Proceedings against Argyle in the Collection of State
% Trials, Burnet, i 521; A True and Plain Account of the Discoveries
% made in Scotland, 1684, The Scotch Mist Cleared; Sir George Mackenzie's
% Vindication, Lord Fountainhall's Chronological Notes.]

% [Footnote 335: Information of Robert Smith in the Appendix to Sprat's
% True Account.]

% [Footnote 336: True and Plain Account of the Discoveries made in
% Scotland.]

% [Footnote 337: Discorsi sopra la prima Deca di Tito Livio, lib. ii. cap.
% 33.]

% [Footnote 338: See Sir Patrick Hume's Narrative, passim.]

% [Footnote 339: Grey's Narrative; Wade's Confession, Harl. MS. 6845.]

% [Footnote 340: Burnet, i. 631.]

% [Footnote 341: Grey's Narrative.]

% [Footnote 342: Le Clerc's Life of Locke; Lord King's Life of Locke;
% Lord Grenville's Oxford and Locke. Locke must not be confounded with
% the Anabapist Nicholas Look, whose name was spelled Locke in Grey's
% Confession, and who is mentioned in the Lansdowne MS. 1152, and in the
% Buccleuch narrative appended to Mr. Rose's dissertation. I should hardly
% think it necessary to make this remark, but that the similarity of
% the two names appears to have misled a man so well acquainted with the
% history of those times as Speaker Onslow. See his note on Burnet, i,
% 629.]

% [Footnote 343: Wodrow, book iii. chap. ix; London Gazette, May 11, 1685;
% Barillon, May 11-21.]

% [Footnote 344: Register of the Proceedings of the States General, May
% 5-15, 1685.]

% [Footnote 345: This is mentioned in his credentials, dated on the 16th
% of March, 1684-5.]

% [Footnote 346: Bonrepaux to Seignelay, February 4-14, 1686.]

% [Footnote 347: Avaux Neg. April 30,/May 10, May 1-11, May 5-15, 1685;
% Sir Patrick Hume's Narrative; Letter from The Admiralty of Amsterdam to
% the States General, dated June 20, 1685; Memorial of Skelton, delivered
% to the States General, May 10, 1685.]

% [Footnote 348: If any person is inclined to suspect that I have
% exaggerated the absurdity and ferocity of these men, I would advise him
% to read two books, which will convince him that I have rather softened
% than overcharged the portrait, the Hind Let Loose, and Faithful
% Contendings Displayed.]

% [Footnote 349: A few words which were in the first five editions have
% been omitted in this place. Here and in another passage I had, as Mr.
% Aytoun has observed, mistaken the City Guards, which were commanded by
% an officer named Graham, for the Dragoons of Graham of Claverhouse.]

% [Footnote 350: The authors from whom I have taken the history of
% Argyle's expedition are Sir Patrick Hume, who was an eyewitness of what
% he related, and Wodrow, who had access to materials of the greatest
% value, among which were the Earl's own papers. Wherever there is a
% question of veracity between Argyle and Hume, I have no doubt that
% Argyle's narrative ought to be followed.---- See also Burnet, i. 631,
% and the life of Bresson, published by Dr. Mac Crie. The account of
% the Scotch rebellion in the Life of James the Second, is a ridiculous
% romance, not written by the King himself, nor derived from his papers,
% but composed by a Jacobite who did not even take the trouble to look at
% a map of the seat of war.]

% [Footnote 351: Wodrow, III. ix 10; Western Martyrology; Burnet, i. 633;
% Fox's History, Appendix iv. I can find no way, except that indicated in
% the text, of reconciling Rumbold's denial that he had ever admitted into
% his mind the thought of assassination with his confession that he had
% himself mentioned his own house as a convenient place for an attack on
% the royal brothers. The distinction which I suppose him to have taken
% was certainly taken by another Rye House conspirator, who was, like him,
% an old soldier of the Commonwealth, Captain Walcot. On Walcot's trial,
% West, the witness for the crown, said, "Captain, you did agree to be one
% of those that were to fight the Guards." "What, then, was the reason."
% asked Chief Justice Pemberton, "that he would not kill the King?" "He
% said," answered West, "that it was a base thing to kill a naked man, and
% he would not do it."]

% [Footnote 352: Wodrow, III. ix. 9.]

% [Footnote 353: Wade's narrative, Harl, MS. 6845; Burnet, i. 634; Van
% Citters's Despatch of Oct. 30,/Nov. 9, 1685; Luttrell's Diary of the
% same date.]

% [Footnote 354: Wodrow, III, ix. 4, and III. ix. 10. Wodrow gives from
% the Acts of Council the names of all the prisoners who were transported,
% mutilated or branded.]

% [Footnote 355: Skelton's letter is dated the 7-17th of May 1686. It
% will be found, together with a letter of the Schout or High Bailiff
% of Amsterdam, in a little volume published a few months later, and
% entitled, "Histoire des Evenemens Tragiques d'Angleterre." The documents
% inserted in that work are, as far as I have examined them, given exactly
% from the Dutch archives, except that Skelton's French, which was not
% the purest, is slightly corrected. See also Grey's Narrative.----
% Goodenough, on his examination after the battle of Sedgemoor, said,
% "The Schout of Amsterdam was a particular friend to this last design."
% Lansdowne MS. 1152.---- It is not worth while to refute those writers
% who represent the Prince of Orange as an accomplice in Monmouth's
% enterprise. The circumstance on which they chiefly rely is that the
% authorities of Amsterdam took no effectual steps for preventing the
% expedition from sailing. This circumstance is in truth the strongest
% proof that the expedition was not favoured by William. No person, not
% profoundly ignorant of the institutions and politics of Holland, would
% hold the Stadtholder answerable for the proceedings of the heads of the
% Loevestein party.]

% [Footnote 356: Avaux Neg. June 7-17, 8-18, 14-24, 1685, Letter of the
% Prince of Orange to Lord Rochester, June 9, 1685.]

% [Footnote 357: Van Citters, June 9-19, June 12-22,1685. The
% correspondence of Skelton with the States General and with the Admiralty
% of Amsterdam is in the archives at the Hague. Some pieces will be found
% in the Evenemens Tragiques d'Angleterre. See also Burnet, i. 640.]

% [Footnote 358: Wade's Confession in the Hardwicke Papers; Harl. MS.
% 6845.]

% [Footnote 359: See Buyse's evidence against Monmouth and Fletcher in the
% Collection of State Trials.]

% [Footnote 360: Journals of the House of Commons, June 13, 1685; Harl.
% MS. 6845; Lansdowne MS. 1152.]

% [Footnote 361: Burnet, i. 641, Goodenough's confession in the Lansdowne
% MS. 1152. Copies of the Declaration, as originally printed, are very
% rare; but there is one in the British Museum.]

% [Footnote 362: Historical Account of the Life and magnanimous Actions of
% the most illustrious Protestant Prince James, Duke of Monmouth, 1683.]

% [Footnote 363: Wade's Confession, Hardwicke Papers; Axe Papers; Harl.
% MS. 6845.]

% [Footnote 364: Harl. MS. 6845.]

% [Footnote 365: Buyse's evidence in the Collection of State Trials;
% Burnet i 642; Ferguson's MS. quoted by Eachard.]

% [Footnote 366: London Gazette, June 18, 1685; Wade's Confession,
% Hardwicke Papers.]

% [Footnote 367: Lords' Journals, June 13,1685.]

% [Footnote 368: Wade's Confession; Ferguson MS.; Axe Papers, Harl. MS.
% 6845, Oldmixon, 701, 702. Oldmixon, who was then a boy, lived very near
% the scene of these events.]

% [Footnote 369: London Gazette, June 18, 1685; Lords' and Commons'
% Journals, June 13 and 15; Dutch Despatch, 16-26.]

% [Footnote 370: Oldmixon is wrong in saying that Fenwick carried up the
% bill. It was carried up, as appears from the Journals, by Lord Ancram.
% See Delamere's Observations on the Attainder of the Late Duke of
% Monmouth.]

% [Footnote 371: Commons' Journals of June 17, 18, and 19, 1685; Reresby's
% Memoirs.]

% [Footnote 372: Commons' Journals, June 19, 29, 1685; Lord Lonsdale's
% Memoirs, 8, 9, Burnet, i. 639. The bill, as amended by the committee,
% will be found in Mr. Fox's historical work. Appendix iii. If Burnet's
% account be correct, the offences which, by the amended bill, were made
% punishable only with civil incapacities were, by the original bill, made
% capital.]

% [Footnote 373: 1 Jac. II. c. 7; Lords' Journals, July 2, 1685.]

% [Footnote 374: Lords' and Commons' Journals, July 2, 1685.]

% [Footnote 375: Savage's edition of Toulmin's History of Taunton.]

% [Footnote 376: Sprat's true Account; Toulmin's History of Taunton.]

% [Footnote 377: Life and Death of Joseph Alleine, 1672; Nonconformists'
% Memorial.]

% [Footnote 378: Harl. MS. 7006; Oldmixon. 702; Eachard, iii. 763.]

% [Footnote 379: Wade's Confession; Goodenough's Confession, Harl. MS.
% 1152, Oldmixon, 702. Ferguson's denial is quite undeserving of credit. A
% copy of the proclamation is in the Harl. MS. 7006.]

% [Footnote 380: Copies of the last three proclamations are in the British
% Museum; Harl. MS. 7006. The first I have never seen; but it is mentioned
% by Wado.]

% [Footnote 381: Grey's Narrative; Ferguson's MS., Eachard, iii. 754.]

% [Footnote 382: Persecution Exposed, by John Whiting.]

% [Footnote 383: Harl. MS. 6845.]

% [Footnote 384: One of these weapons may still be seen in the tower.]

% [Footnote 385: Grey's Narrative; Paschall's Narrative in the Appendix to
% Heywood's Vindication.]

% [Footnote 386: Oldmixon, 702.]

% [Footnote 387: North's Life of Guildford, 132. Accounts of Beaufort's
% progress through Wales and the neighbouring counties are in the London
% Gazettes of July 1684. Letter of Beaufort to Clarendon, June 19, 1685.]

% [Footnote 388: Bishop Fell to Clarendon, June 20; Abingdon to Clarendon,
% June 20, 25, 26, 1685; Lansdowne MS. 846.]

% [Footnote 389: Avaux, July 5-15, 6-16, 1685.]

% [Footnote 390: Van Citters, June 30,/July 10, July 3-13, 21-31,1685;
% Avaux Neg. July 5-15, London Gazette, July 6.]

% [Footnote 391: Barillon, July 6-16, 1685; Scott's preface to Albion and
% Albanius.]

% [Footnote 392: Abingdon to Clarendon, June 29,1685; Life of Philip
% Henry, by Bates.]

% [Footnote 393: London Gazette, June 22, and June 25,1685; Wade's
% Confession; Oldmixon, 703; Harl. MS. 6845.]

% [Footnote 394: Wade's Confession.]

% [Footnote 395: Wade's Confession; Oldmixon, 703; Harl. MS. 6845; Charge
% of Jeffreys to the grand jury of Bristol, Sept. 21, 1685.]

% [Footnote 396: London Gazette, June 29, 1685; Wade's Confession.]

% [Footnote 397: Wade's Confession.]

% [Footnote 398: London Gazette, July 2,1685; Barillon, July 6-16; Wade's
% Confession.]

% [Footnote 399: London Gazette, June 29,1685; Van Citters, June 30,/July
% 10,]

% [Footnote 400: Harl. MS. 6845; Wade's Confession.]

% [Footnote 401: Wade's Confession; Eachard, iii. 766.]

% [Footnote 402: Wade's Confession.]

% [Footnote 403: London Gazette, July 6, 1685; Van Citters, July 3-13,
% Oldmixon, 703.]

% [Footnote 404: Wade's Confession.]

% [Footnote 405: Matt. West. Flor. Hist., A. D. 788; MS. Chronicle quoted
% by Mr. Sharon Turner in the History of the Anglo-Saxons, book IV. chap.
% xix; Drayton's Polyolbion, iii; Leland's Itinerary; Oldmixon, 703.
% Oldmixon was then at Bridgewater, and probably saw the Duke on the
% church tower. The dish mentioned in the text is the property of Mr.
% Stradling, who has taken laudable pain's to preserve the relics and
% traditions of the Western insurrection.]

% [Footnote 406: Oldmixon, 703.]

% [Footnote 407: Churchill to Clarendon, July 4, 1685.]

% [Footnote 408: Oldmixon, 703; Observator, Aug. 1, 1685.]

% [Footnote 409: Paschall's Narrative in Heywood's Appendix.]

% [Footnote 410: Kennet, ed. 1719, iii. 432. I am forced to believe
% that this lamentable story is true. The Bishop declares that it was
% communicated to him in the year 1718 by a brave officer of the Blues,
% who had fought at Sedgemoor, and who had himself seen the poor girl
% depart in an agony of distress.]

% [Footnote 411: Narrative of an officer of the Horse Guards in Kennet,
% ed. 1718, iii. 432; MS. Journal of the Western Rebellion, kept by Mr.
% Edward Dummer, Dryden's Hind and Panther, part II. The lines of Dryden
% are remarkable:
% 
%      "Such were the pleasing triumphs of the sky
%      For James's late nocturnal victory.
%      The fireworks which his angels made above.
%      The pledge of his almighty patron's love,
%      I saw myself the lambent easy light
%      Gild the brown horror and dispel the night.
%      The messenger with speed the tidings bore.
%      News which three labouring nations did restore;
%      But heaven's own Nuntius was arrived before.']

% [Footnote 412: It has been said by several writers, and among them by
% Pennant, that the district in London called Soho derived its name from
% the watchword of Monmouth's army at Sedgemoor. Mention of Soho Fields
% will be found in many books printed before the Western insurrection; for
% example, in Chamberlayne's State of England, 1684.]

% [Footnote 413: There is a warrant of James directing that forty pounds
% should be paid to Sergeant Weems, of Dumbarton's regiment, "for good
% service in the action at Sedgemoor in firing the great guns against the
% rebels." Historical Record of the First or Royal Regiment of Foot.]

% [Footnote 414: James the Second's account of the battle of Sedgemoor
% in Lord Hardwicke's State Papers; Wade's Confession; Ferguson's MS.
% Narrative in Eachard, iii. 768; Narrative of an Officer of the Horse
% Guards in Kennet, ed. 1719, iii. 432, London Gazette, July 9, 1685;
% Oldmixon, 703; Paschall's Narrative; Burnet, i. 643; Evelyn's Diary,
% July 8; Van Citters,.July 7-17; Barillon, July 9-19; Reresby's Memoirs;
% the Duke of Buckingham's battle of Sedgemoor, a Farce; MS. Journal of
% the Western Rebellion, kept by Mr. Edward Dummer, then serving in the
% train of artillery employed by His Majesty for the suppression of the
% same. The last mentioned manuscript is in the Pepysian library, and is
% of the greatest value, not on account of the narrative, which contains
% little that is remarkable, but on account of the plans, which exhibit
% the battle in four or five different stages.]

"The history of a battle," says the greatest of living generals, "is
not unlike the history of a ball. Some individuals may recollect all the
little events of which the great result is the battle won or lost, but
no individual can recollect the order in which, or the exact moment at
which, they occurred, which makes all the difference as to their value
or importance..... Just to show you how little reliance can be placed
even on what are supposed the best accounts of a battle, I mention that
there are some circumstances mentioned in General--'s account which
did not occur as he relates them. It is impossible to say when each
important occurrence took place, or in what order."--Wellington Papers,
Aug. 8, and 17, 1815.---- The battle concerning which the Duke of
Wellington wrote thus was that of Waterloo, fought only a few weeks
before, by broad day, under his own vigilant and experienced eye.
What then must be the difficulty of compiling from twelve or thirteen
narratives an account of a battle fought more than a hundred and sixty
years ago in such darkness that not a man of those engaged could see
fifty paces before him? The difficulty is aggravated by the circumstance
that those witnesses who had the best opportunity of knowing the truth
were by no means inclined to tell it. The Paper which I have placed at
the head of my list of authorities was evidently drawn up with extreme
partiality to Feversham. Wade was writing under the dread of the halter.
Ferguson, who was seldom scrupulous about the truth of his assertions,
lied on this occasion like Bobadil or Parolles. Oldmixon, who was a boy
at Bridgewater when the battle was fought, and passed a great part of
his subsequent life there, was so much under the influence of local
passions that his local information was useless to him. His desire to
magnify the valour of the Somersetshire peasants, a valour which
their enemies acknowledged and which did not need to be set off by
exaggeration and fiction, led him to compose an absurd romance. The
eulogy which Barillon, a Frenchman accustomed to despise raw levies,
pronounced on the vanquished army, is of much more value, "Son
infanterie fit fort bien. On eut de la peine a les rompre, et les
soldats combattoient avec les crosses de mousquet et les scies qu'ils
avoient au bout de grands bastons au lieu de picques."---- Little is
now to be learned by visiting the field of battle for the face of the
country has been greatly changed; and the old Bussex Rhine on the banks
of which the great struggle took place, has long disappeared. The
rhine now called by that name is of later date, and takes a different
course.---- I have derived much assistance from Mr. Roberts's account of
the battle. Life of Monmouth, chap. xxii. His narrative is in the main
confirmed by Dummer's plans.]

% [Footnote 415: I learned these things from persons living close to
% Sedgemoor.]

% [Footnote 416: Oldmixon, 704.]

% [Footnote 417: Locke's Western Rebellion Stradling's Chilton Priory.]

% [Footnote 418: Locke's Western Rebellion Stradling's Chilton Priory;
% Oldmixon, 704.]

% [Footnote 419: Aubrey's Natural History of Wiltshire, 1691.]

% [Footnote 420: Account of the manner of taking the late Duke of
% Monmouth, published by his Majesty's command; Gazette de France, July
% 18-28, 1688; Eachard, iii. 770; Burnet, i. 664, and Dartmouth's note:
% Van Citters, July 10-20,1688.]

% [Footnote 421: The letter to the King was printed at the time by
% authority; that to the Queen Dowager will be found in Sir H. Ellis's
% Original Letters; that to Rochester in the Clarendon Correspondence.]

% [Footnote 422: "On trouve," he wrote, "fort a redire icy qu'il ayt fait
% une chose si peu ordinaire aux Anglois." July 13-23, 1685.]

% [Footnote 423: Account of the manner of taking the Duke of Monmouth;
% Gazette, July 16, 1685; Van Citters, July 14-24,]

% [Footnote 424: Barillon was evidently much shocked. "Ill se vient," he
% says, "de passer icy, une chose bien extraordinaire et fort opposee a
% l'usage ordinaire des autres nations" 13-23, 1685.]

% [Footnote 425: Burnet. i. 644; Evelyn's Diary, July 15; Sir J.
% Bramston's Memoirs; Reresby's Memoirs; James to the Prince of Orange,
% July 14, 1685; Barillon, July 16-26; Bucclench MS.]

% [Footnote 426: James to the Prince of Orange, July 14, 1685, Dutch
% Despatch of the same date, Dartmouth's note on Burnet, i. 646; Narcissus
% Luttrell's Diary, (1848) a copy of this diary, from July 1685 to Sept.
% 1690, is among the Mackintosh papers. To the rest I was allowed access
% by the kindness of the Warden of All Souls' College, where the original
% MS. is deposited. The delegates of the Press of the University of Oxford
% have since published the whole in six substantial volumes, which will, I
% am afraid, find little favour with readers who seek only for amusement,
% but which will always be useful as materials for history. (1857.)]

% [Footnote 427: Buccleuch MS; Life of James the Second, ii. 37, Orig.
% Mem., Van Citters, July 14-24, 1685; Gazette de France, August 1-11.]

% [Footnote 428: Buccleuch MS.; Life of James the Second, ii. 37, 38,
% Orig. Mem., Burnet, i. 645; Tenison's account in Kennet, iii. 432, ed.
% 1719.]

% [Footnote 429: Buccleuch MS.]

% [Footnote 430: The name of Ketch was often associated with that of
% Jeffreys in the lampoons of those days.
% 
%      "While Jeffreys on the bench,
%      Ketch on the gibbet sits,"
% 
% says one poet. In the year which followed Monmouth's execution Ketch
% was turned out of his office for insulting one of the Sheriffs, and was
% succeeded by a butcher named Rose. But in four months Rose himself was
% hanged at Tyburn, and Ketch was reinstated. Luttrell's Diary, January
% 20, and May 28, 1686. See a curious note by Dr. Grey, on Hudibras, part
% iii. canto ii. line 1534.]

% [Footnote 431: Account of the execution of Monmouth, signed by the
% divines who attended him; Buccleuch MS; Burnet, i. 646; Van Citters,
% July 17-27,1685, Luttrell's Diary; Evelyn's Diary, July 15; Barillon,
% July 19-29.]

% [Footnote 432: I cannot refrain from expressing my disgust at the
% barbarous stupidity which has transformed this most interesting little
% church into the likeness of a meetinghouse in a manufacturing town.]

% [Footnote 433: Observator, August 1, 1685; Gazette de France, Nov. 2,
% 1686; Letter from Humphrey Wanley, dated Aug. 25, 1698, in the Aubrey
% Collection; Voltaire, Dict. Phil. There are, in the Pepysian Collection,
% several ballads written after Monmouth's death which represent him as
% living, and predict his speedy return. I will give two specimens.
% 
%      "Though this is a dismal story
%      Of the fall of my design,
%      Yet I'll come again in glory,
%      If I live till eighty-nine:
%      For I'll have a stronger army
%      And of ammunition store."
% 
%      Again;
% 
%      "Then shall Monmouth in his glories
%      Unto his English friends appear,
%      And will stifle all such stories
%      As are vended everywhere.
%      "They'll see I was not so degraded,
%      To be taken gathering pease,
%      Or in a cock of hay up braided.
%      What strange stories now are these!"]

% [Footnote 434: London Gazette, August 3, 1685; the Battle of Sedgemoor,
% a Farce.]

% [Footnote 435: Pepys's Diary, kept at Tangier; Historical Records of the
% Second or Queen's Royal Regiment of Foot.]

% [Footnote 436: Bloody Assizes, Burnet, i. 647; Luttrell's Diary, July
% 15, 1685; Locke's Western Rebellion; Toulmin's History of Taunton,
% edited by Savage.]

% [Footnote 437: Luttrell's Diary, July 15, 1685; Toulmin's Hist. of
% Taunton.]

% [Footnote 438: Oldmixon, 705; Life and Errors of John Dunton, chap.
% vii.]

% [Footnote 439: The silence of Whig writers so credulous and so
% malevolent as Oldmixon and the compilers of the Western Martyrology
% would alone seem to me to settle the question. It also deserves to be
% remarked that the story of Rhynsault is told by Steele in the Spectator,
% No. 491. Surely it is hardly possible to believe that, if a crime
% exactly resembling that of Rhynsault had been committed within living
% memory in England by an officer of James the Second, Steele, who was
% indiscreetly and unseasonably forward to display his Whiggism, would
% have made no allusion to that fact. For the case of Lebon, see the
% Moniteur, 4 Messidor, l'an 3.]

% [Footnote 440: Sunderland to Kirke, July 14 and 28, 1685. "His Majesty,"
% says Sunderland, "commands me to signify to you his dislike of these
% proceedings, and desires you to take care that no person concerned in
% the rebellion be at large." It is but just to add that, in the same
% letter, Kirke is blamed for allowing his soldiers to live at free
% quarter.]

% [Footnote 441: I should be very glad if I could give credit to the
% popular story that Ken, immediately after the battle of Sedgemoor,
% represented to the chiefs of the royal army the illegality of military
% executions. He would, I doubt not, have exerted all his influence on
% the side of law and of mercy, if he had been present. But there is no
% trustworthy evidence that he was then in the West at all. Indeed what we
% know about his proceedings at this time amounts very nearly to proof of
% an alibi. It is certain from the Journals of the House of Lords that,
% on the Thursday before the battle, he was at Westminster, it is equally
% certain that, on the Monday after the battle, he was with Monmouth in
% the Tower; and, in that age, a journey from London to Bridgewater and
% back again was no light thing.]

% [Footnote 442: North's Life of Guildford, 260, 263, 273; Mackintosh's
% View of the Reign of James the Second, page 16, note; Letter of Jeffreys
% to Sunderland, Sept. 5, 1685.]

% [Footnote 443: See the preamble of the Act of Parliament reversing her
% attainder.]

% [Footnote 444: Trial of Alice Lisle in the Collection of State Trials;
% Act of the First of William and Mary for annulling and making void
% the Attainder of Alice Lisle widow; Burnet, i. 649; Caveat against the
% Whigs.]

% [Footnote 445: Bloody Assizes.]

% [Footnote 446: Locke's Western Rebellion.]

% [Footnote 447: This I can attest from my own childish recollections.]

% [Footnote 448: Lord Lonsdale says seven hundred; Burnet six hundred. I
% have followed the list which the Judges sent to the Treasury, and which
% may still be seen there in the letter book of 1685. See the Bloody
% Assizes, Locke's Western Rebellion; the Panegyric on Lord Jeffreys;
% Burnet, i. 648; Eachard, iii. 775; Oldmixon, 705.]

% [Footnote 449: Some of the prayers, exhortations, and hymns of the
% sufferers will be found in the Bloody Assizes.]

% [Footnote 450: Bloody Assizes; Locke's Western Rebellion; Lord
% Lonsdale's Memoirs; Account of the Battle of Sedgemoor in the Hardwicke
% Papers. The story in the Life of James the Second, ii. 43; is not taken
% from the King's manuscripts, and sufficiently refutes itself.]

% [Footnote 451: Bloody Assizes; Locke's Western Rebellion, Humble
% Petition of Widows and Fatherless Children in the West of England;
% Panegyric on Lord Jeffreys.]

% [Footnote 452: As to the Hewlings, I have followed Kiffin's Memoirs, and
% Mr. Hewling Luson's narrative, which will be found in the second edition
% of the Hughes Correspondence, vol. ii. Appendix. The accounts in Locke's
% Western Rebellion and in the Panegyric on Jeffreys are full of errors.
% Great part of the account in the Bloody Assizes was written by Kiffin,
% and agrees word for word with his Memoirs.]

% [Footnote 453: See Tutchin's account of his own case in the Bloody
% Assizes.]

% [Footnote 454: Sunderland to Jeffreys, Sept. 14, 1685; Jeffreys to the
% King, Sept. 19, 1685, in the State Paper Office.]

% [Footnote 455: The best account of the sufferings of those rebels
% who were sentenced to transportation is to be found in a very curious
% narrative written by John Coad, an honest, Godfearing carpenter who
% joined Monmouth, was badly wounded at Philip's Norton, was tried by
% Jeffreys, and was sent to Jamaica. The original manuscript was kindly
% lent to me by Mr. Phippard, to whom it belongs.]

% [Footnote 456: In the Treasury records of the autumn of 1685 are several
% letters directing search to be made for trifles of this sort.]

% [Footnote 457: Commons' Journals, Oct. 9, Nov. 10, Dec 26, 1690;
% Oldmixon, 706. Panegyrie on Jeffreys.]

% [Footnote 458: Life and Death of Lord Jeffreys; Panegyric on Jeffreys;
% Kiffin's Memoirs.]

% [Footnote 459: Burnet, i 368; Evelyn's Diary, Feb. 4, 1684-5, July 13,
% 1686. In one of the satires of that time are these lines:
% 
%      "When Duchess, she was gentle, mild, and civil;
%      When Queen, she proved a raging furious devil."]

% [Footnote 460: Sunderland to Jeffreys, Sept. 14, 1685.]

% [Footnote 461: Locke's Western Rebellion; Toulmin's History of Taunton,
% edited by Savage, Letter of the Duke of Somerset to Sir F. Warre; Letter
% of Sunderland to Penn, Feb. 13, 1685-6, from the State Paper Office, in
% the Mackintosh Collection. (1848.)---- The letter of Sunderland is as
% follows:--
% 
%     "Whitehall, Feb. 13, 1685-6.
% 
%     "Mr. Penne,
% 
%     "Her Majesty's Maids of Honour having acquainted me that they
%     design to employ you and Mr. Walden in making a composition with
%     the Relations of the Maids of Taunton for the high Misdemeanour
%     they have been guilty of, I do at their request hereby let you
%     know that His Majesty has been pleased to give their Fines to the
%     said Maids of Honour, and therefore recommend it to Mr. Walden
%     and you to make the most advantageous composition you can in
%     their behalf."
% 
%     I am, Sir,
% 
%     "Your humble servant,
% 
%     "SUNDERLAND."
% 
% That the person to whom this letter was addressed was William Penn the
% Quaker was not doubted by Sir James Mackintosh who first brought it to
% light, or, as far as I am aware, by any other person, till after
% the publication of the first part of this History. It has since been
% confidently asserted that the letter was addressed to a certain George
% Penne, who appears from an old accountbook lately discovered to have
% been concerned in a negotiation for the ransom of one of Monmouth's
% followers, named Azariah Pinney.---- If I thought that I had committed
% an error, I should, I hope, have the honesty to acknowledge it. But,
% after full consideration, I am satisfied that Sunderland's letter was
% addressed to William Penn.---- Much has been said about the way in which
% the name is spelt. The Quaker, we are told, was not Mr. Penne, but
% Mr. Penn. I feel assured that no person conversant with the books and
% manuscripts of the seventeenth century will attach any importance to
% this argument. It is notorious that a proper name was then thought to
% be well spelt if the sound were preserved. To go no further than the
% persons, who, in Penn's time, held the Great Seal, one of them is
% sometimes Hyde and sometimes Hide: another is Jefferies, Jeffries,
% Jeffereys, and Jeffreys: a third is Somers, Sommers, and Summers: a
% fourth is Wright and Wrighte; and a fifth is Cowper and Cooper. The
% Quaker's name was spelt in three ways. He, and his father the Admiral
% before him, invariably, as far as I have observed, spelt it Penn; but
% most people spelt it Pen; and there were some who adhered to the ancient
% form, Penne. For example. William the father is Penne in a letter from
% Disbrowe to Thurloe, dated on the 7th of December, 1654; and William the
% son is Penne in a newsletter of the 22nd of September, 1688, printed in
% the Ellis Correspondence. In Richard Ward's Life and Letters of Henry
% More, printed in 1710, the name of the Quaker will be found spelt in all
% the three ways, Penn in the index, Pen in page 197, and Penne in page
% 311. The name is Penne in the Commission which the Admiral carried out
% with him on his expedition to the West Indies. Burchett, who became
% Secretary to the Admiralty soon after the Revolution, and remained in
% office long after the accession of the House of Hannover, always, in his
% Naval History, wrote the name Penne. Surely it cannot be thought strange
% that an old-fashioned spelling, in which the Secretary of the Admiralty
% persisted so late as 1720, should have been used at the office of the
% Secretary of State in 1686. I am quite confident that, if the letter
% which we are considering had been of a different kind, if Mr. Penne had
% been informed that, in consequence of his earnest intercession, the King
% had been graciously pleased to grant a free pardon to the Taunton girls,
% and if I had attempted to deprive the Quaker of the credit of that
% intercession on the ground that his name was not Penne, the very persons
% who now complain so bitterly that I am unjust to his memory would
% have complained quite as bitterly, and, I must say, with much more
% reason.---- I think myself, therefore perfectly justified in considering
% the names, Penn and Penne, as the same. To which, then, of the two
% persons who bore that name George or William, is it probable that the
% letter of the Secretary of State was addressed?---- George was evidently
% an adventurer of a very low class. All that we learn about him from the
% papers of the Pinney family is that he was employed in the purchase of a
% pardon for the younger son of a dissenting minister. The whole sum
% which appears to have passed through George's hands on this occasion was
% sixty-five pounds. His commission on the transaction must therefore have
% been small. The only other information which we have about him, is that
% he, some time later, applied to the government for a favour which was
% very far from being an honour. In England the Groom Porter of the Palace
% had a jurisdiction over games of chance, and made some very dirty gain
% by issuing lottery tickets and licensing hazard tables. George appears
% to have petitioned for a similar privilege in the American colonies.----
% William Penn was, during the reign of James the Second, the most active
% and powerful solicitor about the Court. I will quote the words of his
% admirer Crose. "Quum autem Pennus tanta gratia plurinum apud regem
% valeret, et per id perplures sibi amicos acquireret, illum omnes,
% etiam qui modo aliqua notitia erant conjuncti, quoties aliquid a rege
% postulandum agendumve apud regem esset, adire, ambire, orare, ut eos
% apud regem adjuvaret." He was overwhelmed by business of this kind,
% "obrutus negotiationibus curationibusque." His house and the approaches
% to it were every day blocked up by crowds of persons who came to request
% his good offices; "domus ac vestibula quotidie referta clientium et
% suppliccantium." From the Fountainhall papers it appears that his
% influence was felt even in the highlands of Scotland. We learn from
% himself that, at this time, he was always toiling for others, that he
% was a daily suitor at Whitehall, and that, if he had chosen to sell his
% influence, he could, in little more than three, years, have put twenty
% thousand pounds into his pocket, and obtained a hundred thousand more
% for the improvement of the colony of which he was proprietor.---- Such
% was the position of these two men. Which of them, then, was the more
% likely to be employed in the matter to which Sunderland's letter
% related? Was it George or William, an agent of the lowest or of the
% highest class? The persons interested were ladies of rank and fashion,
% resident at the palace. where George would hardly have been admitted
% into an outer room, but where William was every day in the presence
% chamber and was frequently called into the closet. The greatest nobles
% in the kingdom were zealous and active in the cause of their fair
% friends, nobles with whom William lived in habits of familiar
% intercourse, but who would hardly have thought George fit company for
% their grooms. The sum in question was seven thousand pounds, a sum not
% large when compared with the masses of wealth with which William had
% constantly to deal, but more than a hundred times as large as the only
% ransom which is known to have passed through the hands of George.
% These considerations would suffice to raise a strong presumption that
% Sunderland's letter was addressed to William, and not to George: but
% there is a still stronger argument behind.---- It is most important to
% observe that the person to whom this letter was addressed was not the
% first person whom the Maids of Honour had requested to act for them.
% They applied to him because another person to whom they had previously
% applied, had, after some correspondence, declined the office. From
% their first application we learn with certainty what sort of person
% they wished to employ. If their first application had been made to
% some obscure pettifogger or needy gambler, we should be warranted in
% believing that the Penne to whom their second application was made was
% George. If, on the other hand, their first application was made to a
% gentleman of the highest consideration, we can hardly be wrong in saying
% that the Penne to whom their second application was made must have been
% William. To whom, then, was their first application made? It was to Sir
% Francis Warre of Hestercombe, a Baronet and a Member of Parliament. The
% letters are still extant in which the Duke of Somerset, the proud Duke,
% not a man very likely to have corresponded with George Penne, pressed
% Sir Francis to undertake the commission. The latest of those letters
% is dated about three weeks before Sunderland's letter to Mr. Penne.
% Somerset tells Sir Francis that the town clerk of Bridgewater, whose
% name, I may remark in passing, is spelt sometimes Bird and sometimes
% Birde, had offered his services, but that those services had been
% declined. It is clear, therefore, that the Maids of Honour were desirous
% to have an agent of high station and character. And they were right. For
% the sum which they demanded was so large that no ordinary jobber could
% safely be entrusted with the care of their interests.---- As Sir Francis
% Warre excused himself from undertaking the negotiation, it became
% necessary for the Maids of Honour and their advisers to choose somebody
% who might supply his place; and they chose Penne. Which of the two
% Pennes, then, must have been their choice, George, a petty broker to
% whom a percentage on sixty-five pounds was an object, and whose highest
% ambition was to derive an infamous livelihood from cards and dice, or
% William, not inferior in social position to any commoner in the kingdom?
% Is it possible to believe that the ladies, who, in January, employed the
% Duke of Somerset to procure for them an agent in the first rank of the
% English gentry, and who did not think an attorney, though occupying a
% respectable post in a respectable corporation, good enough for their
% purpose, would, in February, have resolved to trust everything to a
% fellow who was as much below Bird as Bird was below Warre?---- But, it
% is said, Sunderland's letter is dry and distant; and he never would have
% written in such a style to William Penn with whom he was on friendly
% terms. Can it be necessary for me to reply that the official
% communications which a Minister of State makes to his dearest friends
% and nearest relations are as cold and formal as those which he makes to
% strangers? Will it be contended that the General Wellesley to whom the
% Marquis Wellesley, when Governor of India, addressed so many letters
% beginning with "Sir," and ending with "I have the honour to be your
% obedient servant,'' cannot possibly have been his Lordship's brother
% Arthur?---- But, it is said, Oldmixon tells a different story. According
% to him, a Popish lawyer named Brent, and a subordinate jobber, named
% Crane, were the agents in the matter of the Taunton girls. Now it is
% notorious that of all our historians Oldmixon is the least trustworthy.
% His most positive assertion would be of no value when opposed to such
% evidence as is furnished by Sunderland's letter, But Oldmixon asserts
% nothing positively. Not only does he not assert positively that Brent
% and Crane acted for the Maids of Honour; but he does not even assert
% positively that the Maids of Honour were at all concerned. He goes
% no further than "It was said," and "It was reported." It is plain,
% therefore, that he was very imperfectly informed. I do not think it
% impossible, however, that there may have been some foundation for the
% rumour which he mentions. We have seen that one busy lawyer, named Bird,
% volunteered to look after the interest of the Maids of Honour, and that
% they were forced to tell him that they did not want his services. Other
% persons, and among them the two whom Oldmixon names, may have tried to
% thrust themselves into so lucrative a job, and may, by pretending to
% interest at Court, have succeeded in obtaining a little money from
% terrified families. But nothing can be more clear than that the
% authorised agent of the Maids of Honour was the Mr. Penne, to whom the
% Secretary of State wrote; and I firmly believe that Mr. Penne to have
% been William the Quaker---- If it be said that it is incredible that
% so good a man would have been concerned in so bad an affair, I can only
% answer that this affair was very far indeed from being the worst in
% which he was concerned.---- For those reasons I leave the text, and
% shall leave it exactly as it originally stood. (1857.)]

% [Footnote 462: Burnet, i. 646, and Speaker Onslow's note; Clarendon to
% Rochester, May 8, 1686.]

% [Footnote 463: Burnet, i. 634.]

% [Footnote 464: Calamy's Memoirs; Commons' Journals, December 26,1690;
% Sunderland to Jeffreys, September 14, 1685; Privy Council Book, February
% 26, 1685-6.]

% [Footnote 465: Lansdowne MS. 1152; Harl. MS. 6845; London Gazette, July
% 20, 1685.]

% [Footnote 466: Many writers have asserted, without the slightest
% foundation, that a pardon was granted to Ferguson by James. Some have
% been so absurd as to cite this imaginary pardon, which, if it were
% real would prove only that Ferguson was a court spy, in proof of the
% magnanimity and benignity of the prince who beheaded Alice Lisle and
% burned Elizabeth Gaunt. Ferguson was not only not specially pardoned,
% but was excluded by name from the general pardon published in the
% following spring. (London Gazette, March 15, 1685-6.) If, as the public
% suspected and as seems probable, indulgence was shown to him; it was
% indulgence of which James was, not without reason, ashamed, and which
% was, as far as possible, kept secret. The reports which were current in
% London at the time are mentioned in the Observator, Aug. 1,1685.---- Sir
% John Reresby, who ought to have been well informed, positively affirms
% that Ferguson was taken three days after the battle of Sedgemoor. But
% Sir John was certainly wrong as to the date, and may therefore have
% been wrong as to the whole story. From the London Gazette, and from
% Goodenough's confession (Lansdowne MS. 1152), it is clear that, a
% fortnight after the battle, Ferguson had not been caught, and was
% supposed to be still lurking in England.]

% [Footnote 467: Granger's Biographical History.]

% [Footnote 468: Burnet, i. 648; James to the Prince of Orange, Sept. 10,
% and 24, 1685; Lord Lonadale's Memoirs; London Gazette, Oct. 1, 1685.]

% [Footnote 469: Trial of Cornish in the Collection of State Trials,
% Sir J. Hawles's Remarks on Mr. Cornish's Trial; Burnet, i. 651; Bloody
% Assizes; Stat. 1 Gul. and Mar.]

% [Footnote 470: Trials of Fernley and Elizabeth Gaunt, in the Collection
% of State Trials Burnet, i. 649; Bloody Assizes; Sir J. Bramston's
% Memoirs; Luttrell's Diary, Oct. 23, 1685.]

% [Footnote 471: Bateman's Trial in the Collection of State Trials;
% Sir John Hawles's Remarks. It is worth while to compare Thomas Lee's
% evidence on this occasion with his confession previously published by
% authority.]

% [Footnote 472: Van Citters, Oct. 13-23, 1685.]

% [Footnote 473: Neal's History of the Puritans, Calamy's Account of the
% ejected Ministers and the Nonconformists' Memorial contain abundant
% proofs of the severity of this persecution. Howe's farewell letter to
% his flock will be found in the interesting life of that great man, by
% Mr. Rogers. Howe complains that he could not venture to show himself in
% the streets of London, and that his health had suffered from want of
% air and exercise. But the most vivid picture of the distress of the
% Nonconformists is furnished by their deadly enemy, Lestrange, in the
% Observators of September and October, 1685.]





